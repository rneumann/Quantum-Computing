%\motto{Use the template \emph{chapter.tex} to style the various elements of your chapter content.}

\chapter{Anwendung in der Chemie \& Materialforschung}
\label{trends} % Always give a unique label
% use \chaptermark{}
% to alter or adjust the chapter heading in the running head

\chapterauthor{Hüma Yilmaz, Sabine Weigand}

\abstract\\
Dieses Kapitel zeigt zukünftige Potenzialbereiche des Quantencomputings für den Einsatz in der Chemie und Materialwissenschaft. Im Fokus stehen drei zentrale Anwendungsfelder: die Simulation von Molekülen, die Entwicklung und Analyse von Batteriematerialien sowie die Modellierung elektronischer Korrelationen und Defekte in Festkörpern. Dabei werden insbesondere hybride Quantenalgorithmen wie der Variational Quantum Eigensolver (VQE), die Quantum Phase Estimation (QPE) und der Quantum Approximate Optimization Algorithm (QAOA) betrachtet. Darüber hinaus beleuchtet das Kapitel den Stand laufender Forschungsinitiativen sowie die Rolle quantenmechanischer Modellierung in der Materialentwicklung. Den Abschluss bildet eine vergleichende Bewertung der drei Anwendungsfelder hinsichtlich Forschungsstand, Anwendungspotenzial und technischer Herausforderungen.

\section{Einleitung}
\label{Chemie_Einleitung}
Quantencomputing gilt als vielversprechender Ansatz zur Lösung komplexer Probleme in der Chemie und Materialwissenschaft. Dieses Kapitel zeigt Anwendungsgebiete auf, in denen quantenmechanische Algorithmen potenziell neue methodische Möglichkeiten bieten und klassische Simulationsverfahren langfristig ergänzen könnten. Im Mittelpunkt stehen drei Felder: die Simulation von Molekülen, die Entwicklung und Analyse von Batteriematerialien sowie die Modellierung elektronischer Korrelationen und Defekte in Festkörpern. Zusätzlich wird der aktuelle Forschungsstand vorgestellt, einschließlich relevanter Projekte und Technologien. Den Abschluss bildet eine vergleichende Einordnung der drei Anwendungsfelder hinsichtlich Reifegrad, Anwendungspotenzial und technischer Herausforderungen.

\section{Relevanz \& Problemstellung}
\label{Chemie_Problemstellung}
Das Verständnis quantenmechanischer Prozesse ist grundlegend für Fortschritte in der Chemie und Materialwissenschaft (vgl. \cite{hanaor_computational_2024}, \cite{daley_practical_2022}, \cite{bauer_quantum_2020}). Klassische Simulationsmethoden wie die Dichtefunktionaltheorie (DFT), Hartree-Fock oder kraftfeldbasierte Molekulardynamik vereinfachen das Verhalten von Elektronen, um es rechnerisch behandelbar zu machen. Bei Systemen mit stark korrelierten Elektronenzuständen, Übergangsmetallen oder reaktiven Dynamiken stoßen diese Verfahren an ihre methodischen und rechnerischen Grenzen (vgl. \cite{cao_quantum_2019}, \cite{vermaStatusChallengesDensity2020}). Typische Schwächen klassischer Ansätze sind die unzureichende Beschreibung statischer Korrelation, elektronischer Verschränkung und Van-der-Waals-Kräfte sowie systematische Fehler durch Selbstwechselwirkungen oder Elektronendelokalisierung. Auch chemische Reaktionen mit Ladungstransfer oder Bindungsbrüchen lassen sich nur eingeschränkt realitätsnah abbilden. Quantencomputer bieten hier neue Möglichkeiten. Sie beruhen auf quantenmechanischen Prinzipien und ermöglichen eine realistischere Modellierung elektronischer Systeme (vgl. \cite{akromDevelopmentQuantumMachine2024}). Besonders relevant ist ihre Fähigkeit zur Beschreibung von Multireferenz-Zuständen und korrelierten Elektronenkonfigurationen, wie sie in vielen chemischen und materialwissenschaftlichen Fragestellungen auftreten. Hybride Quantenalgorithmen wie der Variational Quantum Eigensolver (VQE) oder die Quantum Phase Estimation (QPE) verbinden klassische Optimierung mit quantenmechanischer Zustandsvorbereitung. Erste Berechnungen elektronischer Grund- und Anregungszustände sind damit bereits gelungen (vgl. \cite{aspuru-guzik_simulated_2005}, \cite{weidman_quantum_2024}). 

\section{Top 3 Anwendungsfelder (Praxis \& Theorie)}
\label{Chemie_Anwendungsfelder}
{In diesem Kapitel stehen drei Anwendungsfelder im Fokus: die Simulation von Molekülen, Materialforschung am Beispiel Batterien und die Modellierung elektronischer Korrelationen und Defekte in Festkörpern.
Diese Auswahl basiert auf drei Kriterien. Erstens besitzen alle drei Bereiche hohe praktische Bedeutung, etwa in der Energiewende, der Entwicklung neuer Katalysatoren und der Halbleitertechnologie. Zweitens stoßen klassische Simulationsmethoden in diesen Bereichen an Grenzen, da viele Prozesse durch stark korrelierte Elektronen, Multireferenzzustände, Defekte oder nichtlineare Dynamiken geprägt sind, die mit herkömmlichen Verfahren nur unzureichend erfasst werden können. Drittens gibt es in allen drei Feldern eine aktive Forschungsgemeinschaft und konkrete Projekte, die Quantenalgorithmen erproben und weiterentwickeln.}

\subsection{Simulation von Molekülen}
\label{Chemie_Simulation_Moleküle}
{Die Simulation von Molekülen und chemischen Reaktionen ist ein zentrales Anwendungsfeld des Quantencomputings in der Chemie. Klassische Computer stoßen hierbei schnell an ihre Grenzen, da der Rechenaufwand zur Lösung der Schrödinger-Gleichung mit der Anzahl der Teilchen exponentiell ansteigt. Besonders bei größeren Molekülen oder Elektronensystemen mit starker Wechselwirkung werden etablierte Verfahren wie Hartree-Fock oder die Dichtefunktionaltheorie (DFT) entweder zu ungenau oder zu rechenintensiv (vgl. \cite{bauer_quantum_2020}).}
\newline\\
{Richard Feynman und Yuri Manin erkannten bereits in den 1980er Jahren, dass Quantencomputer die natürliche Komplexität quantenmechanischer Systeme effizient modellieren können. Quantencomputer werden dafür genutzt, quantenmechanische Phänomene wie Superposition und Verschränkung abzubilden. So lassen sich die komplexen Zustände von Molekülen und chemischen Reaktionen realistischer und effizienter darstellen (vgl. \cite{feynmanSimulatingPhysicsComputers1982}).}
\newline\\
Ein zentrales Ziel in der chemischen Forschung ist das Verständnis von Reaktionsmechanismen. Quantencomputer können mehrdimensionale Potenzialflächen ohne vereinfachte Näherungen simulieren, was präzisere Aussagen über Aktivierungsenergien und Isotopieeffekte erlaubt (vgl. \cite{liu_quantum_2020}, \cite{mcardle_quantum_2020}). Moleküle mit Übergangsmetallen oder ungepaarten Elektronen besitzen komplexe Elektronenstrukturen, die klassische Methoden nur näherungsweise erfassen (vgl. \cite{weidman_quantum_2024}). Quantencomputer ermöglichen die Darstellung solcher Multireferenz-Zustände und damit eine genauere Modellierung von Beispielweise Spin-Kopplungen oder Ladungstransfer-Prozessen (vgl. \cite{bauer_quantum_2020}, \cite{mcardle_quantum_2020}). Auch die zeitliche Entwicklung molekularer Zustände kann quantenmechanisch simuliert werden (vgl. \cite{bauer_quantum_2020}). Das ist etwa in der Photochemie relevant, wenn Licht elektronische und strukturelle Änderungen auslöst. Quantenalgorithmen ermöglichen die Vorhersage ultrakurzer Zwischenzustände und elektronischer Übergänge. Anwendungen reichen von der Effizienzsteigerung organischer Solarzellen durch Singulettspaltung (vgl. \cite{motlagh_quantum_2025}, \cite{baldacchino_singlet_2022}) bis zur Analyse lichtinduzierter Prozesse in der Biochemie (vgl. \cite{macdonell_predicting_2023}). Erste erfolgreiche Simulationen kleiner Moleküle wie H$_2$, LiH oder BeH$_2$ mit VQE wurden auf echter Hardware durchgeführt (vgl. \cite{kandala_hardware-efficient_2017}). Tools wie Qiskit Nature (vgl.  \ref{Chemie_Projekte_IBMQiskit}) und OpenFermion ermöglichen heute modulare quantenchemische Simulationen (vgl. \cite{the_qiskit_nature_development_team_qiskit_2023}, \cite{mcardle_quantum_2020}).


\subsection{Materialforschung anhand von Batterien}
\label{Chemie_Materialforschung_Batterien}

Die Entwicklung und Optimierung von Batteriematerialien stellt eine zentrale Herausforderung der modernen Materialwissenschaft dar. Batterien sind hochkomplexe Vielteilchensysteme, in denen quantenmechanische Effekte überlagert zusammenspielen und unmittelbaren Einfluss auf Abschlussprozesse wie Ionentransport, Lade-/Entlade-Kinetiken und Grenzflächenbildung nehmen (vgl. \cite{bauer_quantum_2020}). Klassische Ansätze stoßen rasch an ihre Grenzen, weil sie die feingliedrigen elektronischen Wechselwirkungen nur näherungsweise abbilden können. Dadurch wird das Verständnis, wie Atome und Moleküle in Elektroden und Elektrolyten „miteinander kommunizieren“ und damit die Batterie-Performance langfristig eingeschränkt (vgl. \cite{demirApplicationQuantumComputing2024}).
Aktuell dominieren lithium-basierte Materialien den Markt, doch ihre Stabilität und Kapazität nehmen mit zunehmender Zyklenzahl ab, daher rücken alternative Konzepte wie Natrium-, Magnesium- oder Calcium-Batterien in den Fokus (vgl. \cite{demirApplicationQuantumComputing2024}). Gerade im Kontext der Energiewende sind leistungsfähige, langlebige und umweltverträgliche Batteriesysteme wichtig für den flächendeckenden Einsatz erneuerbarer Energien. 
\newline\\
Im Folgenden werden drei zentrale Simulationsziele im Bereich der quantencomputergestützten Batteriematerialforschung vorgestellt: die quantenmechanische Berechnung elektrochemischer Kennwerte, die Simulation von Ionenbewegung im Festkörper sowie die zeitliche Modellierung dynamischer Prozesse wie der SEI-Bildung.

\subsubsection{Berechnung von Gleichgewichtszellspannungen und Redoxpotentialen}

Ein grundlegender Baustein in der Materialbewertung ist die quantenmechanische Vorhersage elektrochemischer Kennzahlen wie Zellspannungen und Redoxpotentiale. Die Gleichgewichtszellspannung $E^\circ$
einer elektrochemischen Zelle beschreibt die im Ruhezustand messbare Spannung zwischen Anode und Kathode. Sie ergibt sich aus der Differenz der Redoxpotentiale, die angeben, wie leicht eine Substanz Elektronen aufnimmt oder abgibt. Mit quantenmechanischen Methoden lassen sich Zellspannungen und Redoxpotentiale vorab berechnen, um neue Elektrodenmaterialien gezielt zu bewerten. Ein praktisches Beispiel ist die Berechnung der Zellspannung in Lithium-Ionen-Batterien, etwa für verschiedene Kathodenmaterialien wie LiCoO\textsubscript{2}, um die Energiedichte und Leistungsfähigkeit neuer Batterien schon vor der Synthese vorherzusagen (vgl. \cite{urban_computational_2016}, \cite{hanaor_computational_2024}).

\subsubsection{Simulation von Ionenmobilität und Diffusionskoeffizienten}
Die Kinetik, insbesondere die Ionenmobilität im Festkörper, ist neben den thermodynamischen Eigenschaften von zentraler Bedeutung. Die Beweglichkeit von Ionen ist ein wesentlicher Faktor für die Ladegeschwindigkeit und Effizienz moderner Batterien. Damit Ionen wie Lithium oder Natrium während des Lade- und Entladevorgangs schnell und verlustarm durch das Elektrodenmaterial wandern können, müssen sie Energiebarrieren überwinden. Mithilfe quantenmechanischer Simulationen lassen sich diese Barrieren sowie die Diffusionskoeffizienten, also die Geschwindigkeit der Ionenbewegung im Festkörper, präzise auf atomarer Ebene berechnen (vgl. \cite{hanaor_computational_2024}, \cite{urban_computational_2016}). Quantenalgorithmen ermöglichen hierbei eine realistische Abbildung der elektronischen Struktur und der Wechselwirkungen im Material, wodurch der Einfluss verschiedener Kristallstrukturen quantitativ erfasst werden kann ( vgl. \cite{aspuru-guzik_simulated_2005}, \cite{baker_simulating_2024}).
Diese Simulationen sind in der Materialentwicklung unverzichtbar, um elektrodenspezifische Ionentransportwege zu identifizieren und zu optimieren. Beispielsweise werden Kathodenmaterialien wie LiCoO\textsubscript{2} oder neuartige Festelektrolyte hinsichtlich ihrer Ionenmobilität systematisch untersucht. Solche Erkenntnisse unterstützen die Entwicklung von Batterien, die sowohl schnelle Ladeprozesse als auch eine lange Lebensdauer gewährleisten (vgl. \cite{hanaor_computational_2024}, \cite{urban_computational_2016}).

\subsubsection{Dynamische Prozesse: SEI-Bildung und Elektrolyt-Zersetzung (Time Evolution)}
Quantencomputer können die zeitliche Entwicklung chemischer Reaktionen (time evolution) direkt simulieren. So lassen sich beispielsweise die Ausbildung der SEI-Schicht an der Anode und die schrittweise Zersetzung von Elektrolyten zeitlich detailliert nachvollziehen (vgl. \cite{hanaor_computational_2024}). Der Einfluss des Hamilton-Operators auf solche Reaktionen ist mit klassischen Methoden nur schwer zu erfassen, während Quantenalgorithmen diese Prozesse effizient und atomgenau beschreiben können (vgl. \cite{weidman_quantum_2024}). Auf diese Weise lassen sich wertvolle Einblicke in Mechanismen der Batteriealterung, Ionenmobilität und chemischen Stabilität gewinnen (vgl. \cite{hanaor_computational_2024}, \cite{weidman_quantum_2024}).
\newline\\
Erste Pilotprojekte, wie BASIQ des Deutschen Zentrums für Luft- und Raumfahrt oder Kooperationen zwischen Automobil-OEMs und Quanten-Start-ups, zeigen, dass Quantensimulationen bereits heute Materialentwürfe beschleunigen (vgl. \cite{kaysser-pyzalla_dlr_nodate}). Mit zunehmender Reife fehlertoleranter Quantencomputer wird erwartet, dass solche Methoden vom Labor in die industrielle Batterieentwicklung übergehen und dort sowohl den Innovationszyklus verkürzen als auch Ressourcen sparen.

\subsection{Elektronenkorrelation und Defektmodellierung in der Festkörperchemie}
\label{Chemie_Elektronenkorrelation}
Die Modellierung von Defekten in Festkörpern zählt zu den anspruchsvollsten Aufgaben der Materialwissenschaft. Abweichungen von der idealen Gitterstruktur durch Leerstellen, Zwischengitteratome oder gezielte Dotierungen beeinflussen zentrale Eigenschaften wie Leitfähigkeit, Magnetismus, optische Absorption und mechanische Stabilität. Besonders in Halbleitern, Supraleitern oder Quantenmaterialien wie topologischen Isolatoren sind Defekte entscheidend für die Funktion des gesamten Systems (vgl. \cite{bassett_quantum_2019}).
Klassische Simulationen sind oft unzureichend, da stark lokalisierte Elektronenzustände oder Spin-Zustände rund um Defektstellen nur ungenau beschrieben werden können (vgl. \cite{bauer_quantum_2020}). Vor allem bei Systemen mit mehreren konkurrierenden Elektronenkonfigurationen sind klassische Verfahren oft unzureichend (vgl. \cite{bassett_quantum_2019}).
Quantencomputer bieten hier neue Möglichkeiten, da sie stark korrelierte Vielteilchensysteme direkt auf Basis quantenmechanischer Prinzipien modellieren können (vgl. \cite{daley_practical_2022}).

Dies zeigt sich unter anderem bei der Untersuchung von Hochtemperatur-Supraleitern, in denen kollektive Effekte wie der Mott-Übergang oder antiferromagnetische Ordnungen auftreten (vgl.\cite{daley_practical_2022}). Traditionelle Methoden wie Hartree-Fock oder die Standard-DFT sind hier oft nicht ausreichend, weshalb weiterentwickelte Ansätze erforderlich sind. Inzwischen ermöglichen Quantencomputer die realitätsnahe Modellierung solcher Systeme, da sie komplexe Vielteilchenzustände ab-initio und auf quantenmechanischer Grundlage simulieren können (vgl. \cite{baker_simulating_2024}).

In der Praxis haben Experimentalphysiker mit Hilfe analoger Quantensimulatoren, etwa auf Basis ultrakalter Atome, erstmals den Mott-Übergang im Fermi-Hubbard-Modell nachgestellt (vgl. \cite{daley_practical_2022}). Auch die Quantensimulation von antiferromagnetischen Phasenübergängen in Festkörpern wurde erfolgreich umgesetzt. Ebenso werden heute auf Quantenprozessoren mit optimierten Algorithmen wie dem Variational Quantum Eigensolver (VQE) neue Einsichten in die elektronischen Strukturen korrelierter Systeme gewonnen, was konkrete Impulse für das Materialdesign etwa von Hochtemperatur-Supraleitern liefert (vgl. \cite{weidman_quantum_2024}).

Auch Punktdefekte in Halbleitern beeinflussen Eigenschaften wie Leitfähigkeit und optische Aktivierbarkeit (vgl.\cite{bassett_quantum_2019}). Theoretisch werden die energetische Lage von Defektzuständen im Bandabstand oder die Wechselwirkung mit freien Ladungsträgern untersucht (vgl. \cite{freysoldt_first-principles_2014}). Quantencomputing erlaubt hier die ab-initio-Berechnung von Defektbildungsenergien, Ladungsniveaus und Diffusionsbarrieren und liefert damit deutlich präzisere Vorhersagen als klassische Methoden (vgl. \cite{bassett_quantum_2019}).
Praktisch wurden mit hybriden Quantenalgorithmen bereits Simulationen von Farbzentren in Diamant oder Stickstoff-Leerstellen in Siliziumcarbid durchgeführt. Dies sind beides Systeme, die als Quantensensoren oder als Bauelemente für Quantencomputer zum Einsatz kommen könnten  (vgl. \cite{baker_simulating_2024}, \cite{cao_ab_2023}). Durch Quantensimulation lässt sich beispielsweise der Aufbau von Spin-Zuständen oder die Dynamik von Elektron-Loch-Paaren experimentell überprüfen und gezielt optimieren, was direkt in neue optoelektronische oder quantentechnologische Anwendungen einfließt (vgl. \cite{cao_ab_2023}).Damit schließt sich der Bogen zur eingangs genannten Relevanz quantenmechanischer Modellierung: Insbesondere stark korrelierte Festkörpersysteme könnten mit Hilfe von Quantencomputing künftig realistischer und anwendungsnäher verstanden werden.


\section{Top Technologien \& Algorithmen}
\label{Chemie_Algos}
In der Entwicklung quantenbasierter Verfahren zur Lösung komplexer Probleme haben sich in den letzten Jahren verschiedene vielversprechende Ansätze etabliert. Insbesondere hybride Algorithmen, die Quanten- und klassische Rechenmethoden kombinieren, spielen dabei eine zentrale Rolle. Sie nutzen das Potenzial von Quantencomputern für spezielle Aufgaben, wie die präzise Bestimmung von Energiezuständen in Molekülen oder die effiziente Lösung kombinatorischer Optimierungsprobleme, und ergänzen diese durch bewährte klassische Optimierungsverfahren.
\vspace{1em}
Dieses Kapitel stellt drei zentrale Vertreter solcher Technologien vor: Variational Quantum Algorithms (VQAs), die Quantum Phase Estimation (QPE) sowie den Quantum Approximate Optimization Algorithm (QAOA). Dabei werden sowohl deren Funktionsweise als auch ihre Anwendungsmöglichkeiten in der Chemie und Materialforschung und Grenzen näher beleuchtet.



\subsubsection*{Variational Quantum Algorithms}
\label{Chemie_Algos_VQA}

Variational Quantum Algorithms (VQAs) sind eine Methode, um mit Quantencomputern die Energiezustände eines physikalischen Systems, beispielsweise eines Moleküls, zu berechnen. Es handelt sich hierbei um ein hybrides Verfahren, bei dem sowohl Quantencomputer als auch herkömmliche Computer genutzt werden, um ein klassisches Optimierungsverfahren auszuführen (vgl. \cite{weidman_quantum_2024}).

\vspace{0.5em}

Es gibt dabei zwei Haupttypen von VQAs, zwischen denen zu differenzieren ist: Variational Quantum Simulation (VQS) und Variational Quantum Optimization (VQO). VQS simulieren dynamische Prozesse, indem sie die zeitliche Entwicklung in einem abstrakten Raum möglichst akkurat nachbilden. VQO dagegen streben danach, Zielzustände zu finden, indem sie eine sogenannte Kostenfunktion minimieren (vgl. \cite{mottaEmergingQuantumComputing2022}).

\vspace{0.5em}

Konkret finden VQO Bedeutung in der Chemie unter anderem als Variational Quantum Eigensolver (VQE). Dieser VQE-Algorithmus wurde zur Bestimmung der Grundzustandsenergie elektronischer Moleküle entwickelt. Dafür wird zunächst eine angenommene Form der Elektronenverteilung, der sogenannten Wellenfunktion, für ein Molekül gewählt. Der Quantencomputer berechnet den Erwartungswert der Energie eines Systems für bestimmte Parameter \( E(\theta) \) (vgl. \cite{mottaEmergingQuantumComputing2022}), woraufhin der klassische Computer die Parameter dieser Wellenfunktion so anpasst, dass die berechnete Energie immer kleiner wird. Dieser Ablauf wird so lange wiederholt, bis die geringste mögliche Energie ermittelt wird und somit die Grundzustandsenergie des elektronischen Moleküls geschätzt werden kann (vgl. \cite{weidman_quantum_2024}).
Diese statistische Schätzung ergibt sich als Durchschnittswert mit einer gewissen Streuung. Deshalb müssen die Optimierungsverfahren, die die Parameter schrittweise verbessern, darauf achten, dass diese Messwerte nicht exakt, sondern mit Unsicherheit behaftet sind (vgl. \cite{mottaEmergingQuantumComputing2022}).

\vspace{0.5em}

Dieser VQE-Algorithmus wurde auch auf andere Bereiche ausgeweitet, etwa Lösungsverfahren für lineare Gleichungssysteme, Matrixzerlegung und numerische lineare Algebra (vgl. \cite{weidman_quantum_2024}).

\vspace{0.5em}

Bei dem VQE-Verfahren sind die Messfehler statistisch klar abschätzbar, allerdings bleibt dennoch eine gewisse Unsicherheit in der geschätzten elektronischen Energie bestehen. Um diese Fehler zu verringern, müsste der verwendete Ansatz (die Wellenfunktion) verbessert werden. Dafür wären allerdings deutlich mehr Quantenoperationen nötig, als heutige Quantencomputer ohne Fehlerkorrektur ausführen können. Aus diesem Grund sollte alternativ die Quantum Phase Estimation (QPE) betrachtet werden (vgl. \cite{vonburgQuantumComputingEnhanced2021}).



\subsubsection*{Quantum Phase Estimation}
\label{Chemie_Algos_QPE}

Quantum Phase Estimation (QPE) ist ein grundlegender Quantum-Algorithmus, mit dem sich die Eigenenergie eines Quantensystems präzise bestimmen lässt. In der Quantenchemie wurde bereits 2005 mit dem QPE die Energie eines Moleküls in einer Simulation ermittelt (vgl. \cite{weidman_quantum_2024}). Der Algorithmus kann den möglichst niedrigen Energiezustand direkt identifizieren, wenn ein geeigneter Startzustand verwendet wird. Je öfter die zentrale Quantenoperation wiederholt wird, desto genauer wird das Ergebnis.

\vspace{0.5em}

Die Durchführung der Methode sieht die Vorbereitung von zwei Qubit-Registern vor, eines für das Molekül (Systemregister) und eines für die Messung (Hilfsregister). Das Hilfsregister wird mithilfe sogenannter Hadamard-Operationen in eine Überlagerung gebracht, sodass es viele mögliche Phasen gleichzeitig abbilden kann. Diese Vorbereitung ermöglicht es, die Energieinformation als Phase im späteren Verlauf präzise auszulesen (vgl. \cite{weidman_quantum_2024}, \cite{mottaEmergingQuantumComputing2022}). 

\vspace{0.5em}

Das Quantensystem wird so weiterentwickelt, dass die Energie als Phase im Hilfsregister gespeichert wird. Dazu sorgt jedes Hilfs-Qubit dafür, dass das System für eine bestimmte Zeit weiterläuft. Wenn das System eine bestimmte Energie hat, sammelt das Hilfsregister bei jeder Zeitentwicklung einen passenden Phasenwert. Um diese im Hilfsregister gespeicherten Phaseninformationen in eine konkrete Energieangabe zu überführen, wird eine inverse Quanten-Fourier-Transformation angewendet. Dieser Schritt sorgt dafür, dass sich die zuvor aufgebauten Phasen zu einem klaren Interferenzmuster überlagern. Dadurch entsteht ein charakteristisches Signal, aus dem die Energie des Systems als Bitfolge abgelesen werden kann (vgl. \cite{vonburgQuantumComputingEnhanced2021}, \cite{mottaEmergingQuantumComputing2022}). Im letzten Schritt wird das Hilfsregister gemessen. Wenn der Systemzustand zu Beginn bereits ein reiner Eigenzustand des Hamiltonoperators ist, liefert die Messung exakt die zugehörige Energie – ohne statistische Schwankungen. Dies ist die sogenannte Zero-Variance-Eigenschaft. Wenn der Startzustand jedoch eine Mischung verschiedener Eigenzustände war, projiziert die Messung das System zufällig auf einen dieser Zustände. Die Wahrscheinlichkeit, mit der ein bestimmter Energiewert dabei herauskommt, ist davon abhängig, wie stark der Anfangszustand mit dem jeweiligen Eigenzustand übereinstimmt. Wird dieser Vorgang mehrfach wiederholt oder der Startzustand gezielt gewählt, so kann die gewünschte Energie zuverlässig ermittelt werden. (vgl. \cite{vonburgQuantumComputingEnhanced2021},\cite{mottaEmergingQuantumComputing2022}).

\vspace{0.5em}

Ein Nachteil ist allerdings, dass QPE eine deutlich komplexere und längere Quantenberechnung erfordert. Deshalb ist sie auf zukünftige, fehlerkorrigierte Quantencomputer angewiesen. Doch im Gegensatz zu Methoden wie VQE, bei denen immer eine gewisse Unsicherheit bleibt, liefert QPE exakte und kontrollierbare Ergebnisse, sobald die Hardware leistungsfähig genug ist (vgl. \cite{vonburgQuantumComputingEnhanced2021},\cite{mottaEmergingQuantumComputing2022}).



\subsubsection*{Quantum Approximate Optimization Algorithm}
\label{Chemie_Algos_QAOA}

Neben diesen Verfahren zur präzisen oder variativen Bestimmung von Energiezuständen gibt es weitere hybride Algorithmen, die speziell für Optimierungsaufgaben konzipiert wurden. Ein besonders vielversprechender Vertreter ist der Quantum Approximate Optimization Algorithm (QAOA), der sich für kombinatorische Probleme eignet und zunehmend auch in der Chemie und Materialforschung Anwendung findet.

\vspace{0.5em}

QAOA gehört zur Familie dieser hybriden Variationsverfahren und wurde speziell für kombinatorische Optimierungsprobleme entwickelt, wodurch die optimalen Lösungen schneller ermittelt werden, als es durch brute force möglich wäre (vgl. \cite{guoHarnessingQuantumPower2024}). Das Problem wird durch einen sogenannten Kosten-Hamiltonian beschrieben. Dies ist ein Operator, der jeder möglichen Lösung einen Energiezustand zuweist, wobei die Lösung dem niedrigsten Energiezustand entspricht (vgl. \cite{mottaEmergingQuantumComputing2022}, \cite{guoHarnessingQuantumPower2024}). Ergänzt wird dieser durch einen sogenannten Mixer-Hamiltonian, der das System kontrolliert aus dem Startzustand in verschiedene Konfigurationen lenkt. Der Algorithmus führt anschließend abwechselnd beide Operatoren für bestimmte Zeitspannen aus, wobei die Parameter dieser Zeitspannen durch klassische Optimierung angepasst werden. Diese basieren auf wiederholten Messungen und der Bewertung der Ergebnisqualität (vgl. \cite{guoHarnessingQuantumPower2024}).
Diese Kombination von quantenmechanischer Zustandserzeugung und klassischer Rückkopplung ermöglicht es, gute Lösungen effizient zu finden, ohne den gesamten Lösungsraum durchsuchen zu müssen. Mit steigender Anzahl der Schichten (p-Level) im Raum wird die Lösungsqualität besser, allerdings steigt hiermit auch die Anforderung an die Hardware.

\vspace{0.5em}

In der Materialwissenschaft wurde QAOA bereits in Simulationen zur Gestaltung von Metamaterialien eingesetzt. Dabei wurde ein Quantenannealing-Modell genutzt, um transparente Kühlbeschichtungen mit optimierter Energieeffizienz zu entwerfen (vgl. \cite{guoHarnessingQuantumPower2024}). Währendessen werden Konzeptstudien zur molekularen Wirkstoffsuche erprobt, etwa durch die Optimierung molekularer Strukturen im Hinblick auf ihre Bindungsstärke. Guo et al. (2024) demonstrierten dies mithilfe eines quantengestützten Verfahrens, das verschiedene zweiatomige Moleküle hinsichtlich ihrer Bindungsstärke an eine Protein-Tasche analysierte. Die aktuell erhältliche Hardware ist allerdings begrenzt und schränkt die Anwendungsmöglichkeiten ein. Herausforderungen wie Rauschunempfindlichkeit und das Barren-Plateau-Problem müssen jedoch noch überwunden werden (vgl. \cite{weidman_quantum_2024}).


\section{Top 3 Zukunftsprojekte \& Forschungsinitiativen}
\label{Chemie_Projekte}

In diesem Kapitel werden drei Initiativen vorgestellt, die maßgeblich zur Weiterentwicklung des Quantencomputings beitragen. Sie stehen exemplarisch für unterschiedliche institutionelle und strategische Ansätze, mit denen Wissenschaft, Industrie und Politik den Weg zur praktischen Nutzung der Quanteninformationstechnologie gestalten. 

\subsection*{PASQuanS2 – EU-Flaggschiff-Initiative}
\label{Chemie_Projekte_PASQuanS2}

PASQuanS2 – EU-Flaggschiff-Initiative für Quantum Simulation
Das europäische „Quantum Flagship“ ist eine der drei großen Forschungsinitiativen der EU und verfolgt das Ziel, Europa als zentrale Kraft in der zweiten Quantenrevolution zu etablieren und eine wettbewerbsfähige Quantenindustrie aufzubauen. Mit einem Budget von rund einer Milliarde Euro über zehn Jahre konzentriert sich das Programm auf vier zentrale Technologiefelder: Quantenkommunikation, -computing, -simulation sowie -sensorik und -metrologie. Ein koordiniertes europäisches Vorgehen wurde notwendig, da es an standardisierten Schnittstellen, klaren Verwertungsstrategien und langfristigen industriellen Partnerschaften mangelte (vgl. \cite{rasanenPathEuropeanQuantum2021}). Einen entscheidenden politischen Impuls lieferte das 2016 veröffentlichte „Quantum Manifesto“, das die Quantenforschung als strategische Schlüsseltechnologie zur Sicherung digitaler Souveränität definierte (vgl. \cite{vandeventerEuropeanStandardsQuantum2022}). Um die Interoperabilität der Systeme und deren Skalierbarkeit zu gewährleisten, gilt die Entwicklung gemeinsamer technischer Standards als entscheidender Schritt. Auch wenn laut \cite{vandeventerEuropeanStandardsQuantum2022} bislang noch keine übergreifende Standardisierungsstrategie existiert, arbeiten Initiativen wie das Quantum Industry Consortium (QuIC) sowie mehrere Arbeitsgruppen des Quantum Flagship aktiv daran, regulatorische und technische Rahmenbedingungen europaweit zu harmonisieren. Auf dieser Grundlage entstand eine strategische Forschungsagenda, die technologische Meilensteine definiert und sowohl Grundlagenforschung als auch industriellen Transfer miteinander verzahnt. Im Folgenden werden zentrale technische Umsetzungsansätze und konkrete Forschungsprojekte innerhalb des Quantum Flagship näher beleuchtet.

\vspace{0.5em}

Ein herausragendes Beispiel dafür ist das Projekt PASQuanS2 (Programmable Atomic Large-Scale Quantum Simulation), das im Rahmen des Flagships finanziert wird. Ziel von PASQuanS2 ist die Entwicklung hochskalierbarer, programmierbarer Quantensimulatoren auf Basis ultrakalter Atome und Ionen, um realitätsnahe Vielteilchenprobleme effizient berechnen zu können (vgl. \cite{rasanenPathEuropeanQuantum2021}). Die aktuelle Projektphase PASQuanS2.1 (2023–2026) fokussiert auf Plattformen mit mindestens 2.000 steuerbaren Quantenteilchen und legt damit die Grundlage für spätere Systeme mit bis zu 10.000 Teilchen. 

\vspace{0.5em}

PASQuanS2 stützt sich technologisch auf drei sich ergänzende Plattformstrategien, die jeweils unterschiedliche physikalische Prinzipien nutzen. Die erste basiert auf Neutralatom-Arrays in optischen Gittern, bei denen ultrakalte Atome mithilfe von Lasern in regelmäßigen Strukturen angeordnet werden. Die zweite Plattform nutzt sogenannte optische Pinzetten (Tweezers-Systeme), in denen einzelne Atome präzise fixiert und über kontrollierte Rydberg-Wechselwirkungen miteinander gekoppelt werden. Die dritte Strategie setzt auf Ionenfallenarchitekturen, bei denen geladene Teilchen in elektromagnetischen Feldern eingeschlossen und durch kollektive Schwingungen miteinander verbunden werden. Jede dieser Technologien bringt unterschiedliche Stärken hinsichtlich Skalierbarkeit, Stabilität und Kopplungspräzision mit sich, verfolgt jedoch das gemeinsame Ziel, quantenmechanische Vielteilchendynamiken experimentell zugänglich zu machen (\cite{scherer_pasquans2_nodate}) Ein zentraler technischer Fokus liegt auf der Erhöhung der Zustandsfidelity auf unter \(0{,}1\,\%\), der Verbesserung von Lese- und Schreibprozessen sowie der Entwicklung verlässlicher Messprotokolle wie Randomized Benchmarking oder Hamiltonian Learning zur Validierung der Simulationsergebnisse. Gleichzeitig wird an modularen Softwareschnittstellen gearbeitet, um eine einfache Integration der Hardware in bestehende IT-Infrastrukturen und Cloud-Umgebungen zu ermöglichen. Dadurch sollen industrielle und wissenschaftliche Anwendergruppen ohne tiefergehende Hardwarekenntnisse Zugang zu den Quantensimulatoren erhalten – ein entscheidender Schritt in Richtung praktischer Nutzbarkeit (vgl. \cite{rasanenPathEuropeanQuantum2021}).

\vspace{0.5em}

Das Forschungsprojekt PASQuanS2 steht exemplarisch für den strategischen Versuch, wissenschaftliche Grundlagenforschung mit Fragen der Skalierbarkeit, Standardisierung und industriellen Anwendbarkeit zu verbinden. Das Programm verfolgt das Ziel, europäische Forschungsstrukturen im internationalen Vergleich zu stärken und langfristig eine technologisch wie wirtschaftlich konkurrenzfähige Quantenindustrie aufzubauen (vgl. \cite{vogiatzoglouEUsQuestDigital2025}).
 

\subsection*{IBM Quantum \& Qiskit Nature} 
\label{Chemie_Projekte_IBMQiskit}

IBM (International Business Machines Corporation) ist ein etabliertes Technologieunternehmen und hat die Entwicklung der Informatik über Jahrzehnte maßgeblich mitgestaltet, beispielsweise durch den Mainframe, relationale Datenbanken oder IBM Watson \cite{aruteQuantumSupremacyUsing2019a}. IBM positioniert sich strategisch im Quantencomputing. Mit der Initiative IBM Quantum verfolgt das Unternehmen das Ziel, skalierbare Quantenhardware zu entwickeln, über die Cloud zugänglich zu machen und durch das Open-Source-Framework Qiskit auch für Forschung und Industrie nutzbar zu gestalten (vgl. \cite{maksudul_shadat_akash_quantum_2025},  \cite{miceliQuantumComputationVisualization2018}). 

\vspace{0.5em}

IBM Quantum bezeichnet IBMs industrielle Quanten-Computing-Initiative, die sowohl modernste Quantenhardware als auch Cloud-Zugriff für Forschung und Industrie bereitstellt. Bereits 2016 stellte IBM den ersten frei zugänglichen Quantencomputer über die IBM Quantum Experience in der Cloud zur Verfügung. Zu Beginn des Jahres 2024 wurden bereits über drei Billionen Quantenoperationen über IBMs Quanten-Cloud durchgeführt und rund 2.800 wissenschaftliche Arbeiten unter Verwendung von IBM-Quantenprozessoren veröffentlicht. IBM gilt damit als einer der zentralen Akteure in der praktischen Umsetzung skalierbarer Quantencomputing-Anwendungen und adressiert ein breites Anwendungsspektrum von quantenchemischen Simulationen und Optimierungsproblemen bis hin zu Kryptographie und Machine Learning. Gleichzeitig entwickelt IBM die zugehörige Hardware. Dies zeigt sich 2023 mit der Vorstellung von „Condor“, ein supraleitender Quantenchip mit 1.121 Qubits, nachdem bereits 2021/22 die Prozessoren „Eagle“ (127 Qubits) und „Osprey“ (433 Qubits) technische Meilensteine gesetzt hatten. Die Kombination aus zugänglicher Hardware, Open-Source-Software und modularer Toolchain positioniert IBM Quantum als relevante Infrastruktur für anwendungsnahe Forschung im Bereich des Quantencomputings (vgl. \cite{abughanemIBMQuantumComputers2025}).

\vspace{0.5em}

Ein essenzieller Bestandteil von IBM Quantum ist das Open-Source-Software-Framework Qiskit, das 2017 von IBM veröffentlicht wurde. Qiskit stellt ein umfangreiches Software Development Kit (SDK) für Quantencomputing bereit und hat sich zum meistgenutzten Werkzeug dieser Art entwickelt. Es bietet eine benutzerfreundliche API zur Konstruktion von Quanten-Schaltkreisen und automatisiert die nötigen Schritte, um diese auf echter Quantenhardware auszuführen. Konkret definiert das Qiskit-„Patterns“-Modell einen strukturierten Workflow in vier Schritten. Zuallererst wird das ursprüngliche Problem in einen Quanten-Schaltkreis übersetzt. Anschließend wird diese Schaltung mittels Transpilation optimiert und an die Ziel-Hardware angepasst. Folgend wird der Schaltkreis auf einem ausgewählten Quantenprozessor (oder Simulator) ausgeführt und abschließend werden die Resultate ausgelesen und klassisch nachverarbeitet (vgl. \cite{abughanemIBMQuantumComputers2025}). Dieses Cloud-basierte Toolchain-Konzept ermöglicht es Nutzern, Quantenprogramme nahtlos vom Laptop bis auf die supraleitenden IBM-Quanten-Prozessoren zu bringen. Über Qiskit können Berechnungen entweder auf hochleistungsfähigen Simulatoren oder direkt auf den realen IBM-Quantencomputern in der Cloud durchgeführt werden. IBM stellt dafür mehrere Quantenprozessoren mit über IBM Quantum Experience frei zur Verfügung, zugänglich über ein webbasiertes Interface, welches insbesondere den Einstieg für eine breite Nutzerschaft erleichtert (vgl. \cite{wangAdvantagesTwoQuantum2025}). Die Hardware in Verbindung mit einer nahtlos integrierten Software-Umgebung hat dazu beigetragen, dass bereits Millionen von Experimenten mithilfe von Qiskit auf IBM-Systemen durchgeführt wurden. Damit zählt IBM zu den zentralen Akteuren in der praktischen Umsetzung und Skalierung von Quantencomputing-Anwendungen weltweit.

\vspace{0.5em}

Der hohe Praxisbezug von IBM Quantum zeigt sich insbesondere in der gezielten Anwendung wissenschaftlicher Algorithmen auf reale Problemstellungen. Ein zentrales Anwendungsfeld ist dabei die Quantenchemie, in der klassische Rechner schnell an ihre Grenzen stoßen. Hier kommen hybride Quantenalgorithmen wie der Variational Quantum Eigensolver zum Einsatz, der bereits im vorherigen Unterkapitel ausführlich behandelt wurde (vgl.  \cite{miceliQuantumComputationVisualization2018}). IBM’s Software-“Stack” unterstützt solche hybriden Ansätze mit spezialisierten Bibliotheken, wie Qiskit Nature (früher als Qiskit Chemistry bekannt). Dies ist ein Open-Source-Framework, das quantenmechanische Probleme aus den Naturwissenschaften, insbesondere in der Chemie, im Computer abbildet. Qiskit Nature stellt Funktionen bereit, um elektronische Strukturen von Molekülen in Form der Zweiten Quantisierung zu formulieren und an Quantenalgorithmen zu koppeln. So können klassische Quantenchemie-Programme für Vorabrechnungen integriert werden. Beispielsweise lässt sich ein Hartree-Fock-Grundzustand mit etablierten Programmpaketen berechnen und an Qiskit übergeben. Darauf aufbauend kommen quantum-classical Algorithmen wie VQE zum Einsatz, um die Korrektur zur Hartree-Fock-Lösung zu bestimmen. Typischerweise wird dabei ein Unitary Coupled Cluster-Ansatz mit Einfach- und Doppelanregungen (UCCSD) verwendet, der vom Hartree-Fock-Zustand ausgeht (vgl. \cite{avramidisGroundStateProperty2024}).

\vspace{0.5em}

Die Leistungsfähigkeit und Herausforderungen dieses Ansatzes werden in aktuellen Forschungsarbeiten deutlich. So wurden 2024 erstmals die Grundzustands-Eigenschaften mehrerer Lithiumhydrid-Komplexe (LiH, LiH$_2$, LiH$_3$ samt ihren Ionen) vollständig mit Qiskit und dem VQE-Algorithmus simuliert. Dabei diente ein UCCSD-Ansatz als quantenchemische Wavefunction, und es wurden neben den Grundzustandsenergien auch Elektronenaffinitäten, Ionisierungsenergien und Dipolmomente der Moleküle berechnet. Die Ergebnisse zeigen, dass VQE in der Lage ist, die elektronischen Grundzustandsgrößen dieser einfachen Moleküle mit hoher Genauigkeit zu reproduzieren (vgl. \cite{avramidisGroundStateProperty2024}).

\vspace{0.5em}

Zusammenfassend bildet IBM Quantum mit seinem ganzheitlichen Ansatz, von der leistungsfähigen Hardware über das umfangreiche Qiskit-Softwarepaket bis hin zu spezialisierten Modulen wie Qiskit Nature, eine zukunftsweisende Forschungsplattform. Als eines der “Top Zukunftsprojekte” treibt IBM Quantum damit sowohl die Weiterentwicklung der Quantencomputing-Technologie als auch deren Anwendung in wissenschaftlichen und industriellen Domänen voran.

\subsection*{Google Quantum AI \& Sycamore} 
\label{Chemie_Projekte_Google_Quantum_AI}

Googles Quantenforschungseinheit, Quantum AI, ist eine privatwirtschaftliche Initiative im Bereich der Quanteninformationsverarbeitung.  Eingebettet in die Alphabet-Struktur verfolgt Google einen Full-Stack-Ansatz. Dabei werden alle Ebenen der Quantencomputer-Architektur intern entwickelt. Dies umfasst die Hardware, darunter supraleitende Qubits und Quantenchips. Auch die Steuer- und Ausleseelektronik wird intern entwickelt. Hinzu kommen Softwarekomponenten wie Compiler, Fehlerkorrekturverfahren und das Open-Source-Framework Cirq, mit dem sich Quantenalgorithmen programmieren lassen. Im Mittelpunkt steht der 2019 vorgestellte Sycamore-Prozessor, ein supraleitender Quantenchip mit 53 funktionsfähigen Transmon-Qubits, der auf einem zweidimensionalen Gitter basiert (vgl. \cite{aruteQuantumSupremacyUsing2019a}).

\vspace{0.5em}

Der zentrale Durchbruch gelingt Google mit der Demonstration der Quantenüberlegenheit (quantum supremacy). Es ist definiert als die Fähigkeit eines Quantencomputers, eine spezifische Rechenaufgabe schneller zu lösen als jeder bekannte klassische Computer. Konkret ließ Google seinen Sycamore-Prozessor eine Zufalls-Sampling-Aufgabe ausführen, bei der bitstring-Ausgaben aus tiefen, pseudozufälligen Quanten-Schaltkreisen generiert wurden. Diese Aufgabe wurde in 200 Sekunden bewältigt. Im Vergleich dazu hätte ein klassischer Supercomputer wie Summit für dieselbe Aufgabe laut Google rund 10.000 Jahre benötigt (vgl. \cite{aruteQuantumSupremacyUsing2019a}).
Dieses Ergebnis wurde über die sogenannte Cross-Entropy Benchmarking (XEB)-Fidelity mit einem Wert von ca. 0.002 validiert (vgl. \cite{maksudul_shadat_akash_quantum_2025}). Dies weist darauf hin, dass die gemessenen Ausgaben signifikant von gleichverteilten Zufallswerten abweichen und somit mit den theoretisch erwarteten Quantenverteilungen übereinstimmen.

\vspace{0.5em}

IBM, als direkter Wettbewerber im Bereich supraleitender Qubits, widersprach dieser Interpretation und argumentierte, dass sich dieselbe Sampling-Aufgabe durch Optimierungen klassischer Algorithmen und Einsatz von Massenspeicher innerhalb von 2,5 Tagen auf klassischen Maschinen lösen ließe (vgl. \cite{pednault_quantum_2019}). Diese Diskussion verursachte nicht nur technische, sondern auch begriffliche Spannungen, wobei IBM den Begriff „Quantenüberlegenheit“ als irreführend kritisiert und stattdessen quantum advantage als neutraleren Ausdruck vorschlug (IBM Quantum Blog, 2019). Ungeachtet der Debatte bleibt Sycamore ein technologischer Meilenstein. Die auf dem Chip eingesetzten sogenannten Sycamore-Gates sind eine spezielle Form von fSim-Gates. Das ist eine Kombination aus rotationssymmetrischen iSWAP-Operationen und kontrollierten Phasenverschiebungen, die hohe Verschränkung bei gleichzeitig reduzierter Fehlerrate ermöglichen (vgl. \cite{abughanemPhotonicQuantumComputers2024}). Die durchschnittliche Die Fehlerquote beträgt bei simultaner Ausführung \(0{,}93\,\%\) für Zwei-Qubit-Gates und \(3{,}8\,\%\) bei der Auslesung (vgl. \cite{arute_quantum_2019}).

\vspace{0.5em}

Im Anschluss an das Supremacy-Experiment rückten Forschungsfragen zur Fehlertoleranz, Skalierbarkeit und plattformübergreifenden Vergleichbarkeit in den Vordergrund. Einen bemerkenswerten Beitrag dazu leistet die Studie von AbuGhanem \& Eleuch (2024), die eine vollständige Quantenprozess- und Zustandstomographie des Sycamore-Gates auf IBMs Quantencomputern durchführten (vgl. \cite{abughanemPhotonicQuantumComputers2024}). Die Autoren untersuchten dabei nicht nur Simulationen unter idealen und verrauschten Bedingungen, sondern führten die Messungen auch auf echter IBM-Quantenhardware durch. Die Ergebnisse auf IBMs Plattform zeigten eine Prozessfidelity von \(81{,}02\,\%\). Dieser Wert liegt nahe an dem, was unter realen Bedingungen auf aktueller Hardware technisch erreichbar ist, und zeigt, dass Sycamore-Gates auch auf einer konkurrierenden Architektur wie der von IBM mit hoher Qualität implementiert werden können (vgl. \cite{abughanem_full_2025}).
Solche herstellerübergreifenden Analysen sind wegweisend: Sie ermöglichen nicht nur Benchmarks für Gate-Qualitäten unter realistischen Bedingungen, sondern fördern auch eine Standardisierung der Metriken zur Bewertung von Quantencomputern (vgl. \cite{abughanemPhotonicQuantumComputers2024}).

\vspace{0.5em}

Mit einer eigens entwickelten Hardware-Architektur und begleitender Open-Source-Software trägt Google Quantum AI zur Weiterentwicklung skalierbarer Quantencomputing-Technologien bei. Langfristig fokussiert sich Google auf die Umsetzung eines fehlertoleranten, skalierbaren Quantencomputers mit Anwendungen in Chemie, Optimierung und KI (vgl. \cite{abughanem_full_2024}).
Damit steht Google Quantum AI mit Sycamore exemplarisch für den Übergang von wissenschaftlichen Demonstrationen hin zu strategisch nutzbaren Quantentechnologien. Die experimentellen Fortschritte auf Basis des Sycamore-Chips gelten als wichtige Beiträge zur Weiterentwicklung quantenbasierter Anwendungen, etwa in der Materialforschung, der Kryptographie oder der Modellierung komplexer Systeme.


\section{Bewertung anhand der Kriterien}
\label{Chemie_Bewertung}

\subsection{Vergleichende Bewertung der Anwendungsfelder}

Quantencomputer eröffnen neue Perspektiven für zentrale Fragestellungen in der Chemie und Materialwissenschaft. Die nachfolgende vergleichende Bewertung konzentriert sich auf drei Schlüsselanwendungsfelder: (1) die \textit{Simulation von Molekülen}, (2) die \textit{Materialforschung am Beispiel Batterien} und (3) die \textit{Modellierung von Elektronenkorrelationen und Defekten in Festkörpern}.
\newline\\
Diese Gegenüberstellung erfolgt anhand von fünf übergeordneten Bewertungskriterien: Technologischer Reifegrad, Marktrelevanz, gesellschaftlicher Nutzen, Forschungspotenzial sowie Risiken und ethische Implikationen. Diese wurden im Kapitel 5.7 Bewertung anhand der Kriterien bereits näher erläutert.
\newline\\
Ziel ist es, die unterschiedlichen Entwicklungsstände, Potenziale und Herausforderungen dieser Anwendungsbereiche systematisch einzuordnen – in direkter Anlehnung an die Methodik des vorangegangenen Kapitels zum Finanzwesen. Die folgende Tabelle fasst die Bewertung der drei Felder übersichtlich zusammen.

\begin{table}[h]
\centering
\begin{tabular}{|p{0.25\linewidth}|p{0.23\linewidth}|p{0.23\linewidth}|p{0.23\linewidth}|}
\hline
\textbf{Kriterium} & \textbf{Simulation von Molekülen} & \textbf{Materialforschung (Batterien)} & \textbf{Festkörper: Korrelation \& Defekte} \\
\hline
\textbf{Techno\-logischer Reife\-grad} & mittel & niedrig & niedrig \\
\hline
\textbf{Markt\-relevanz} & hoch & sehr hoch & hoch \\
\hline
\textbf{Gesell\-schaft\-licher Nutzen} & mittel & hoch & hoch \\
\hline
\textbf{Forschungs\-potenzial} & sehr hoch & sehr hoch & sehr hoch \\
\hline
\textbf{Risiken und Ethik} & gering & mittel & mittel--hoch \\
\hline
\end{tabular}
\caption{Bewertung der drei Hauptanwendungsfelder von Quantencomputing in der Chemie und Materialwissenschaft}
\label{tab:qc-chemie-bewertung}
\end{table}



\subsubsection{Technologischer Reifegrad}

Die \textit{Simulation von Molekülen}[\ref{Chemie_Simulation_Moleküle}] weist derzeit den höchsten technologischen Reifegrad unter den betrachteten Feldern auf. Erste experimentelle Umsetzungen des VQE wurden bereits auf realer Quantenhardware wie den IBM Q-Quantenprozessoren durchgeführt (vgl. \cite{kandala_hardware-efficient_2017}). Auch spezifische Tools wie \textit{Qiskit Nature} und \textit{OpenFermion} ermöglichen heute modulare quantenchemische Simulationen kleiner Moleküle wie H$_2$, LiH oder BeH$_2$, was den Anwendungsbezug weiter stärkt (vgl. \cite{the_qiskit_nature_development_team_qiskit_2023}, \cite{mcardle_quantum_2020}). Die theoretischen Grundlagen sind gut etabliert und wurden umfassend dargestellt, unter anderem von McArdle et al (vgl. \cite{mcardle_quantum_2020}). Dennoch bestehen Einschränkungen bei der Skalierung auf größere Systeme, insbesondere aufgrund der notwendigen Fehlerkorrektur und Qubitanzahl. Insgesamt ergibt sich daher ein \textbf{mittlerer Reifegrad}.

\vspace{0.5em}

Die \textit{Materialforschung im Bereich Batterien}[\ref{Chemie_Materialforschung_Batterien}] befindet sich derzeit überwiegend im konzeptionellen Stadium. Quantenmechanisch inspirierte Ansätze zur Berechnung von Gleichgewichtszellspannungen, zur Modellierung von Ionenmobilität sowie zur Simulation dynamischer Prozesse wie der SEI-Bildung werden aktuell vorwiegend auf klassischen Systemen oder Simulatoren erprobt (vgl. \cite{urban_computational_2016}, \cite{hanaor_computational_2024}, \cite{weidman_quantum_2024}). Zwar existieren theoretisch fundierte Konzepte und erste Pilotprojekte (vgl. \cite{demirApplicationQuantumComputing2024}, \cite{kaysser-pyzalla_dlr_nodate}), jedoch wurden diese bislang nicht auf realer Quantenhardware demonstriert. Entsprechend ist das Anwendungsfeld im Vergleich zu den anderen betrachteten Bereichen technologisch noch als  \textbf{niedrig entwickelt} einzustufen.

\vspace{0.5em}

Auch die \textit{Modellierung von Defekten und Korrelationen in Festkörpern}[\ref{Chemie_Elektronenkorrelation}] weist derzeit einen \textbf{niedrigen Reifegrad} auf. Zwar wurde in Pilotstudien der Einsatz von Quantencomputern zur Simulation von Farbzentren in Festkörpern wie Diamant oder Siliziumkarbid untersucht (vgl. \cite{cao_ab_2023}), jedoch bleibt die Modellierung stark korrelierter Systeme auf praktische Anwendungen mit wenigen Qubits beschränkt. Daley et al. zeigen zwar experimentelle Fortschritte im Bereich analoger Quantensimulationen, doch digitale Anwendungen in realistischen Materialszenarien sind bislang limitiert (vgl. \cite{daley_practical_2022}).


\subsubsection{Marktrelevanz}

Die \textit{Simulation von Molekülen}[\ref{Chemie_Simulation_Moleküle}] besitzt ebenfalls eine \textbf{hohe wirtschaftliche Bedeutung}, vor allem in der chemischen und pharmazeutischen Industrie. Eine präzisere Modellierung von Reaktionsmechanismen kann Entwicklungszeiten senken und gezieltere Designs neuer Wirkstoffe oder Katalysatoren ermöglichen (vgl. \cite{mcardle_quantum_2020}). Auch Anwendungen in der organischen Halbleiterentwicklung oder der Polymerchemie sind relevant.

\vspace{0.5em}

Die \textit{Materialforschung im Bereich Batterien}[\ref{Chemie_Materialforschung_Batterien}] besitzt die \textbf{höchste Marktrelevanz} der drei Felder. Fortschritte in der Simulation von Batteriekomponenten haben unmittelbare ökonomische Auswirkungen auf Elektromobilität, erneuerbare Energiesysteme und stationäre Energiespeicher. Der globale Markt für Lithium-Ionen- und Nachfolgetechnologien wächst rasant, und Simulationsmethoden gelten als strategisches Werkzeug zur Materialentwicklung (vgl. \cite{demirApplicationQuantumComputing2024}) Urban et al. betonen die Rolle rechnergestützter Vorhersagen zur Optimierung der Energie- und Leistungsdichte künftiger Batteriesysteme (vgl. \cite{urban_computational_2016}).

\vspace{0.5em}

Die \textit{Modellierung von Defekten und Korrelationen in Festkörpern}[\ref{Chemie_Elektronenkorrelation}] weist eine eher \textbf{spezialisierte Marktrelevanz} auf. Sie betrifft insbesondere die Halbleiterindustrie, optoelektronische Bauteile und Quanteninformationssysteme (vgl. \cite{bassett_quantum_2019}). Zwar sind Defekte entscheidend für die Leistungsfähigkeit moderner Bauelemente, doch beschränkt sich der direkte Markteinfluss auf Nischenanwendungen mit hohem technologischem Anspruch. Damit ist die wirtschaftliche Hebelwirkung insgesamt geringer als bei den beiden anderen Feldern.

\subsubsection{Gesellschaftlicher Nutzen}

Die \textit{Simulation von Molekülen}[\ref{Chemie_Simulation_Moleküle}] bringt einen \textbf{mittleren gesellschaftlichen Nutzen}. Eine präzisere Vorhersage von Reaktionen kann zur Entwicklung umweltfreundlicherer chemischer Verfahren und neuer Medikamente beitragen. Der Beitrag ist real, aber schwer quantifizierbar und indirekt in seiner gesellschaftlichen Wirkung (vgl. \cite{mcardle_quantum_2020}).

\vspace{0.5em}

Die \textit{Materialforschung im Bereich Batterien}[\ref{Chemie_Materialforschung_Batterien}] hat den \textbf{größten gesellschaftlichen Nutzen}. Fortschritte bei der Simulation ermöglichen die Entwicklung sicherer, langlebiger und nachhaltiger Speichertechnologien. Dies ist essenziell für die Elektrifizierung des Verkehrs, die Integration erneuerbarer Energien und damit für das Erreichen internationaler Klimaziele (vgl. \cite{demirApplicationQuantumComputing2024}). Der Nutzen ist somit direkt, breit wirksam und politisch hoch priorisiert.

\vspace{0.5em}

Auch die \textit{Modellierung von Defekten und Korrelationen in Festkörpern}[\ref{Chemie_Elektronenkorrelation}] leistet einen \textbf{hohen gesellschaftlichen Beitrag}. Anwendungen reichen von verbesserter Effizienz in Solarzellen bis hin zu stabileren und sichereren Mikroprozessoren. Zudem betrifft das Feld Quantenkommunikation und Quantenkryptographie, die langfristig zur Datensicherheit und digitalen Souveränität beitragen können (vgl. \cite{cao_ab_2023,bassett_quantum_2019}).


\subsubsection{Forschungspotenzial}

Alle drei Anwendungsfelder weisen ein \textbf{sehr hohes Forschungspotenzial} auf.

\vspace{0.5em}

Die \textit{Simulation von Molekülen} [\ref{Chemie_Simulation_Moleküle}] ist eines der historisch ersten und methodisch vielfältigsten Einsatzfelder des Quantencomputings. McArdle et al. zeigen eine breite Forschungslandschaft rund um VQE, QPE und andere hybride Verfahren (vgl. \cite{mcardle_quantum_2020}). Auch die Anwendung generativer Netze (qGANs) zur Modellierung komplexer Zustände wird aktiv untersucht (vgl. \cite{zoufal_quantum_2019}).

\vspace{0.5em}

Die \textit{Materialforschung im Bereich Batterien}[\ref{Chemie_Materialforschung_Batterien}] weist ebenso ein \textbf{enormes Innovationspotenzial} auf. Die Integration von quantenchemischer Simulation, Materialdesign und datenbankgetriebener Suche ist methodisch komplex und hochgradig interdisziplinär (vgl. \cite{urban_computational_2016}). Die Forschung steht noch am Anfang, bietet jedoch viele offene Fragestellungen zu Struktur–Eigenschafts-Beziehungen, Ionenmigration und Grenzflächendynamik und weist dadurch ein hohes Forschungspotenzial auf.

\vspace{0.5em}

Auch das \textit{Modellierung von Defekten und Korrelationen in Festkörpern}[\ref{Chemie_Elektronenkorrelation}] ist methodisch extrem fordernd. Daley et al. sprechen explizit von einem ``practical quantum advantage'' (vgl. \cite{daley_practical_2022}) in der Simulation solcher Systeme. Die Verbindung von Spinphysik, Vielteilchentheorie und \textit{ab-initio}-Materialsimulation ist komplex und birgt langfristig neue Konzepte für Materialdesign und Quantentechnologien (vgl. \cite{bassett_quantum_2019}).


\subsubsection{Risiken und ethische Implikationen}

Die \textit{Simulation von Molekülen}[\ref{Chemie_Simulation_Moleküle}] birgt vergleichsweise \textbf{geringe Risiken}. Fehlerhafte Simulationen können zu ineffizienten Designs oder längeren Entwicklungszeiten führen, betreffen jedoch keine kritischen Infrastrukturen. Zudem sind die meisten Simulationen mit klassischen Methoden überprüfbar (vgl. \cite{mcardle_quantum_2020}).

\vspace{0.5em}

Die \textit{Materialforschung im Bereich Batterien}[\ref{Chemie_Materialforschung_Batterien}] ist mit \textbf{mittleren Risiken} verbunden. Falsch modellierte Stabilitäts- oder Reaktivitätseigenschaften könnten zu sicherheitskritischen Fehlfunktionen führen (z.\,B.\ thermischer Durchgang, Elektrolytdegradation). Auch Fragen zur Ressourcenverteilung und Rohstoffabhängigkeit können ethisch relevant sein (vgl. \cite{demirApplicationQuantumComputing2024}).

\vspace{0.5em}

Die \textit{Modellierung von Defekten und Korrelationen in Festkörpern}[\ref{Chemie_Elektronenkorrelation}] ist am ehesten mit \textbf{mittelhohen Risiken} behaftet. Defekte in Halbleitern oder Quantenmaterialien wirken sich direkt auf die Funktion von Speicherchips, Sensoren und Quantenkommunikation aus. Fehlerhafte Simulationen in sicherheitskritischen Systemen könnten zu Fehlfunktionen führen, zudem ist die Nachvollziehbarkeit quantenmechanischer Multizustandsmodelle begrenzt (vgl. \cite{orus_quantum_2019,freysoldt_first-principles_2014}).

\subsubsection{Abschließende Gesamtbewertung}
Alle drei betrachteten Anwendungsfelder - die Simulation von Molekülen, die Materialforschung im Bereich Batterien und die Modellierung von Defekten in Festkörpern - weisen ein sehr hohes Forschungspotenzial auf, unterscheiden sich jedoch deutlich in Reifegrad, Marktrelevanz und gesellschaftlicher Wirkung. Die Molekülsimulation verfügt über den höchsten technologischen Entwicklungsstand und erste Anwendungen auf realer Quantenhardware, ist jedoch in ihrer gesellschaftlichen Wirkung begrenzter als andere Felder. Die Batteriematerialforschung ist methodisch noch wenig erprobt, hat jedoch die größte Relevanz für Wirtschaft und Gesellschaft, insbesondere im Kontext der Energiewende. Die Modellierung elektronischer Korrelationen und Defekte adressiert besonders komplexe physikalische Phänomene und bietet langfristig neue Impulse für Quantentechnologien, bleibt jedoch derzeit auf spezialisierte Anwendungen beschränkt. Auch im Hinblick auf Risiken und ethische Fragestellungen unterscheiden sich die Felder: Während die Molekülsimulation vergleichsweise geringe Risiken birgt, können fehlerhafte Modellierungen in der Batterieforschung sicherheitsrelevante Folgen haben. Die Defektmodellierung betrifft Anwendungen in sicherheitskritischen Technologien und erfordert besonders sorgfältige Validierung. Insgesamt zeigt sich, dass Quantencomputing in der Chemie und Materialforschung vielfältige Chancen eröffnet, deren Realisierung jedoch stark vom weiteren technologischen Fortschritt und einer verantwortungsvollen Einbettung in bestehende Forschungs- und Anwendungspraktiken abhängt.

\printbibliography
