%\motto{Use the template \emph{chapter.tex} to style the various elements of your chapter content.}
\chapter{Anwendung in der Chemie \& Materialforschung}
\label{trends} % Always give a unique label
% use \chaptermark{}
% to alter or adjust the chapter heading in the running head

\chapterauthor{Hüma Yilmaz, Sabine Weigand}

\abstract{some abstract}

\section{Relevanz \& Problemstellung}
{Forschung in der Chemie und Materialwissenschaften sind essenziell zur Lösung aktueller gesellschaftlicher Herausforderungen, etwa in der Entwicklung nachhaltiger Batterietechnologien \cite{hanaor_computational_2024}, effizienter Katalysatoren oder neuartiger funktionaler Materialien \cite{daley_practical_2022}. Ein tiefes Verständnis der zugrunde liegenden quantenmechanischen Prozesse ist dafür unerlässlich \cite{bauer_quantum_2020}. Klassische Simulationsmethoden wie die Dichtefunktionaltheorie (DFT), Hartree-Fock oder kraftfeldbasierte Molekulardynamik haben zwar große Fortschritte ermöglicht, stoßen jedoch bei komplexen elektronischen Systemen an ihre methodischen und rechnerischen Grenzen \cite{daley_practical_2022}\cite{vermaStatusChallengesDensity2020}\cite{cao_quantum_2019}. In genau diesem Spannungsfeld eröffnet das Quantencomputing neue Perspektiven. Es verspricht, viele dieser Einschränkungen zu überwinden und die Modellierung chemischer und materialwissenschaftlicher Systeme auf ein neues Niveau zu heben.

Klassische Methoden vereinfachen das Verhalten von Elektronen stark, um es überhaupt rechnerisch abbilden zu können \cite{vermaStatusChallengesDensity2020}. Die Dichtefunktionaltheorie (DFT) hat zwar zahlreiche Eigenschaften von Atomen, Molekülen und Festkörpern erfolgreich beschrieben und kann auch sehr große Systeme zu einem erschwinglichen Rechenaufwand behandeln. Sie ist jedoch insbesondere bei stark korrelierten Elektronenzuständen problematisch und ungenau \cite{vermaStatusChallengesDensity2020}\cite{hanaor_computational_2024}. In Molekülen mit Übergangsmetallen oder offenen d- und f-Schalen versagen viele DFT-Funktionale, da sie wichtige Effekte wie statische Korrelation oder Verschränkung nicht korrekt erfassen. Typische Fehlerquellen sind die Überdelokalisierung von Elektronendichten, Selbstwechselwirkungsfehler und eine unzureichende Beschreibung von Van-der-Waals-Kräften. Auch chemische Reaktionen, die Bindungsbrüche oder Elektronentransfer beinhalten, lassen sich mit klassischen Kraftfeldern kaum simulieren. Hinzu kommt, dass der Rechenaufwand bei exakten quantenmechanischen Methoden exponentiell mit der Systemgröße wächst, was ihre Anwendung auf größere Moleküle oder Materialien praktisch unmöglich macht\cite{vermaStatusChallengesDensity2020}\cite{hanaor_computational_2024}\cite{cao_quantum_2019}.

Quantencomputer arbeiten hingegen selbst nach den Gesetzen der Quantenmechanik und können damit die elektronischen Eigenschaften von Molekülen oder Festkörpern realistischer abbilden \cite{akromDevelopmentQuantumMachine2024}\cite{bauer_quantum_2020}. Sie sind in der Lage, Multireferenz-Zustände darzustellen, bei denen nicht nur eine einzige Elektronenkonfiguration zur Gesamtwellenfunktion beiträgt, sondern mehrere gleichzeitig. Das ist entscheidend für die präzise Beschreibung von Spin-Kopplungen, Ladungstransfer, elektronischer Verschränkung und Reaktionsdynamik \cite{akromDevelopmentQuantumMachine2024}. Quantenalgorithmen wie der Variational Quantum Eigensolver (VQE) oder die Quantum Phase Estimation (QPE) erlauben es, elektronische Grundzustände oder Anregungszustände mit hoher Genauigkeit zu berechnen \cite{weidman_quantum_2024}\cite{aspuru-guzik_simulated_2005}\cite{cao_quantum_2019}.

Auch dynamische Prozesse können auf Quantencomputern präzise beschrieben werden. Während klassische Verfahren oft auf Näherungsverfahren zurückgreifen müssen, kann ein Quantencomputer die zeitliche Entwicklung eines Moleküls auf Basis der quantenmechanischen Zeitentwicklung direkt simulieren \cite{cao_quantum_2019}. Das ist besonders nützlich für photochemische Reaktionen, ultraschnelle spektroskopische Prozesse oder nicht-adiabatische Übergänge, bei denen klassische Ansätze schnell an ihre Grenzen kommen \cite{weidman_quantum_2024}.

Trotz dieses Potenzials ist das Quantencomputing heute noch von technischen Einschränkungen geprägt. Aktuelle Quantencomputer befinden sich in der sogenannten ``Noisy Intermediate-Scale Quantum''-Phase, kurz NISQ. Sie verfügen nur über eine begrenzte Anzahl fehleranfälliger Qubits mit kurzen Kohärenzzeiten \cite{daley_practical_2022}. Doch bereits jetzt zeigen hybride Quanten-Klassische Algorithmen wie VQE vielversprechende Ergebnisse \cite{bauer_quantum_2020}\cite{cao_quantum_2019}. Fortschritte in der Quantenfehlerkorrektur und in der Entwicklung kompakterer Algorithmen ermöglichen bereits heute erste Simulationen chemischer Systeme, etwa über Plattformen wie Qiskit oder das IBM Quantum Lab \cite{liu_quantum_2020}.

Quantencomputing eröffnet den Zugang zu bisher unerreichbaren Fragestellungen. Systeme, deren Verhalten sich klassisch nicht mehr qualitativ bestimmen lässt, wie etwa bei mehr als 50 stark wechselwirkenden Quantenspins, lassen sich mit Quantenrechnern realistisch simulieren \cite{bauer_quantum_2020}. Die Verbindung von Quanteninformationstheorie und moderner Quantenchemie führt zu neuartigen algorithmischen Ansätzen, die nicht nur bestehende Methoden verbessern, sondern grundsätzlich neue Wege zur Berechnung physikalischer Realität eröffnen \cite{liu_quantum_2020}\cite{weidman_quantum_2024}. Quantencomputer erschließen auch neue Forschungsräume. Sie machen Prozesse sichtbar, die mit klassischen Methoden nicht abbildbar sind, und könnten zukünftig als eine Art Quantenmikroskop fungieren.
}

\section{Top 3 Anwendungsfelder (Praxis \& Theorie)}
{Quantencomputer eröffnen neue Perspektiven für zentrale Fragestellungen in der Chemie und Materialwissenschaft. In dieser Arbeit stehen drei Anwendungsfelder im Fokus: die Simulation von Molekülen, Materialforschung am Beispiel Batterien und die Modellierung elektronischer Korrelationen und Defekte in Festkörpern.
Diese Auswahl basiert auf drei Kriterien. Erstens besitzen alle drei Bereiche hohe praktische Bedeutung, etwa in der Energiewende, der Entwicklung neuer Katalysatoren und der Halbleitertechnologie. Zweitens stoßen klassische Simulationsmethoden in diesen Bereichen an Grenzen, da viele Prozesse durch stark korrelierte Elektronen, Multireferenzzustände, Defekte oder nichtlineare Dynamiken geprägt sind, die mit herkömmlichen Verfahren nur unzureichend erfasst werden können. Drittens gibt es in allen drei Feldern eine aktive Forschungsgemeinschaft und konkrete Projekte, die Quantenalgorithmen erproben und weiterentwickeln.
}
\subsection{Simulation von Molekülen}
{Die Simulation von Molekülen und chemischen Reaktionen ist ein zentrales Anwendungsfeld des Quantencomputings in der Chemie. Klassische Computer stoßen hierbei schnell an ihre Grenzen, da der Rechenaufwand zur Lösung der Schrödinger-Gleichung mit der Anzahl der Teilchen exponentiell ansteigt. Besonders bei größeren Molekülen oder Elektronensystemen mit starker Wechselwirkung werden etablierte Verfahren wie Hartree-Fock oder die Dichtefunktionaltheorie (DFT) entweder zu ungenau oder zu rechenintensiv (vgl. \cite{bauer_quantum_2020}).

\subsubsection{Historische Motivation und Grundidee}
Richard Feynman und Yuri Manin erkannten bereits in den 1980er Jahren, dass Quantencomputer die natürliche Komplexität quantenmechanischer Systeme effizient modellieren können. Quantencomputer werden dafür genutzt, quantenmechanische Phänomene wie Superposition und Verschränkung abzubilden. So lassen sich die komplexen Zustände von Molekülen und chemischen Reaktionen realistischer und effizienter darstellen (vgl. \cite{feynmanSimulatingPhysicsComputers1982}). 


\subsubsection{Verständnis und Vorhersage von Reaktionsmechanismen}

Ein zentrales Ziel in der chemischen Forschung ist das detaillierte Verständnis von Reaktionsmechanismen. Übergangszustände, Energiebarrieren und Reaktionspfade bestimmen, wie schnell und selektiv eine Reaktion abläuft. Quantencomputer bieten die Möglichkeit, mehrdimensionale Potenzialflächen vollständig zu simulieren, ohne auf vereinfachte Näherungen beschränkt zu sein. Dadurch können präzisere Aussagen über Aktivierungsenergien und Effekte wie kinetische Isotopieverschiebungen getroffen werden, was beispielsweise die Entwicklung effizienterer Katalysatoren und neuer Synthesewege unterstützt.

\subsubsection{Untersuchung elektronischer Strukturen und ihrer Dynamik}
Moleküle, besonders solche mit Übergangsmetallen oder ungepaarten Elektronen, besitzen komplexe Elektronenstrukturen. Klassische Methoden wie die Dichtefunktionaltheorie (DFT) stoßen hier an ihre Grenzen, da sie nur einen stark vereinfachten elektronischen Zustand berücksichtigen und viele quantenmechanische Effekte wie Elektronenkorrelationen oder Verschränkung nicht korrekt erfassen. Quantencomputer hingegen können diese Multireferenz-Zustände direkt darstellen und ermöglichen so eine detailliertere Modellierung von Spin-Kopplungen, Ladungstransfer-Prozessen oder Verschränkungsphänomenen im Elektronengas.

\subsubsection{Simulation von Moleküldynamik (Time Evolution) und Reaktionsverläufen}

Chemische Reaktionen verlaufen auf eine Weise, bei der während des Ablaufs sowohl die Elektronenverteilung als auch die Bewegung der Atomkerne gleichzeitig und aufeinander abgestimmt verändert werden. Das ist besonders in der Photochemie der Fall, wenn Licht eine Reaktion auslöst und dabei elektronische und strukturelle Änderungen eng miteinander verflochten sind. Herkömmliche Ansätze verwenden oft Surfacehopping-Approximationen (vgl. \cite{barbattiNonadiabaticDynamicsTrajectory2011}), um zwischen potenziellen Energieflächen zu wechseln. Quantenalgorithmen hingegen ermöglichen die direkte Simulation der zeitlichen Entwicklung (time evolution) des molekularen Quantenzustands (vgl. \cite{bauer_quantum_2020}). Dadurch lassen sich Reaktionskinetiken, kurzlebige Zwischenzustände und elektronische Übergänge vorhersagen, beispielsweise zur Interpretation ultrakurzer spektroskopischer Signale.
In der Praxis wird diese Methode eingesetzt, um die Mechanismen der Singulettspaltung in organischen Halbleitermaterialien zu entschlüsseln, was einen wichtigen Beitrag zur Effizienzsteigerung moderner Solarzellen leistet (vgl. \cite{motlagh_quantum_2025}, \cite{baldacchino_singlet_2022}). Ein weiteres Anwendungsfeld ist die Analyse lichtinduzierter Prozesse in der Biochemie, wie dem Sehvorgang oder der Photosynthese, bei denen quantencomputergestützte Simulationen dazu beitragen, elektronische Anregungen und molekulare Strukturänderungen in Echtzeit aufzuklären (vgl. \cite{macdonell_predicting_2023}).
}
\subsection{Materialforschung anhand von Batterien}
{Die Entwicklung und Optimierung von Batteriematerialien stellt eine zentrale Herausforderung der modernen Materialwissenschaft dar. Batterien sind hochkomplexe Vielteilchensysteme, in denen quantenmechanische Effekte auf atomaren Längen- und Femtosekunden-Zeitskalen überlagert zusammenspielen und unmittelbaren Einfluss auf Abschlussprozesse wie Ionentransport, Lade-/Entlade-Kinetiken und Grenzflächenbildung nehmen (vgl. \cite{bauer_quantum_2020}). Klassische Ansätze stoßen rasch an ihre Grenzen, weil sie die feingliedrigen elektronischen Wechselwirkungen nur näherungsweise abbilden können – dies limitiert unser Verständnis, wie Atome und Moleküle in Elektroden und Elektrolyten „miteinander kommunizieren“ und damit die Batterie-Performance langfristig einschränken (vgl. \cite{demirApplicationQuantumComputing2024}).

Aktuell dominieren Lithium-basierte Materialien den Markt, doch ihre Stabilität und Kapazität nehmen mit zunehmender Zyklenzahl ab, sodass Alternativen (z. B. Natrium-, Magnesium- oder Calcium-Batterien) dringend erforscht werden müssen (vgl. \cite{demirApplicationQuantumComputing2024}). Hinzu kommen externe Faktoren wie hohe Betriebstemperaturen oder hohe Lade-/Entladeströme, die Materialdegradation und unerwünschte Nebenreaktionen beschleunigen. Gerade im Kontext der Energiewende sind leistungsfähige, langlebige und umweltverträgliche Batteriesysteme ein Schlüssel für den flächendeckenden Einsatz erneuerbarer Energien.

Quantum Computing eröffnet hier neue Perspektiven: Durch die nativ quantenmechanische Beschreibung von Elektronen- und Ionenbewegungen lassen sich komplexe Wechselwirkungen und Grenzflächenprozesse auf atomarer Ebene präzise simulieren. Dies ermöglicht tiefere Einsichten in fundamentale Mechanismen wie Ladungstransfer, Redoxreaktionen und die Bildung stabiler Solid-Electrolyte-Interphasen, die mit klassischen Methoden nur unzureichend abgebildet werden können (vgl. \cite{bauer_quantum_2020}, \cite{demirApplicationQuantumComputing2024}).

\subsubsection{Berechnung von Gleichgewichtszellspannungen und Redoxpotentialen}
Die Gleichgewichtszellspannung $E^\circ$ einer elektrochemischen Zelle beschreibt die im Ruhezustand messbare Spannung zwischen Anode und Kathode. Sie ergibt sich aus der Differenz der Redoxpotentiale, die angeben, wie leicht eine Substanz Elektronen aufnimmt oder abgibt. Mit quantenmechanischen Methoden lassen sich Zellspannungen und Redoxpotentiale vorab berechnen, um neue Elektrodenmaterialien gezielt zu bewerten. Ein praktisches Beispiel ist die Berechnung der Zellspannung in Lithium-Ionen-Batterien, etwa für verschiedene Kathodenmaterialien wie LiCoO$_2$, um die Energiedichte und Leistungsfähigkeit neuer Batterien schon vor der Synthese vorherzusagen (vgl. \cite{urban_computational_2016}, \cite{hanaor_computational_2024}).


\subsubsection{Simulation von Ionenmobilität und Diffusionskoeffizienten}
Die Beweglichkeit von Ionen ist ein wesentlicher Faktor für die Ladegeschwindigkeit und Effizienz moderner Batterien. Damit Ionen wie Lithium oder Natrium während des Lade- und Entladevorgangs schnell und verlustarm durch das Elektrodenmaterial wandern können, müssen sie Energiebarrieren überwinden. Mithilfe quantenmechanischer Simulationen lassen sich diese Barrieren sowie die Diffusionskoeffizienten, also die Geschwindigkeit der Ionenbewegung im Festkörper, präzise auf atomarer Ebene berechnen (vgl. \cite{hanaor_computational_2024}, \cite{urban_computational_2016}). Quantenalgorithmen ermöglichen hierbei eine realistische Abbildung der elektronischen Struktur und der Wechselwirkungen im Material, wodurch der Einfluss verschiedener Kristallstrukturen quantitativ erfasst werden kann (vgl. \cite{aspuru-guzik_simulated_2005}, \cite{baker_simulating_2024}).
Diese Simulationen sind in der Materialentwicklung unverzichtbar, um elektrodenspezifische Ionentransportwege zu identifizieren und zu optimieren. Beispielsweise werden Kathodenmaterialien wie LiCoO$_2$ oder neuartige Festelektrolyte hinsichtlich ihrer Ionenmobilität systematisch untersucht. Solche Erkenntnisse unterstützen die Entwicklung von Batterien, die sowohl schnelle Ladeprozesse als auch eine lange Lebensdauer gewährleisten (vgl. \cite{hanaor_computational_2024}, \cite{urban_computational_2016}).

\subsubsection{Dynamische Prozesse: SEI-Bildung und Elektrolyt-Zersetzung (Time Evolution)}
Quantencomputer können die zeitliche Entwicklung chemischer Reaktionen (time evolution) direkt simulieren. So lassen sich beispielsweise die Ausbildung der SEI-Schicht an der Anode – ein Prozess, der innerhalb von Sekunden bis Minuten erfolgt – oder die schrittweise Zersetzung von Elektrolyten detailliert nachvollziehen. Der Einfluss des Hamilton-Operators auf solche Reaktionen ist mit klassischen Methoden nur schwer zu erfassen, während Quantenalgorithmen diese Prozesse effizient und atomgenau beschreiben können. Besonders leistungsfähig ist hier die auf quantenmechanischen Prinzipien basierende atomistische Molekulardynamik (AIMD), die Bindungsbildung und -bruch realistisch abbildet. Dadurch lassen sich wertvolle Einblicke in Mechanismen der Batteriealterung, Ionenmobilität und chemischen Stabilität gewinnen.
}
\subsection{Elektronenkorrelation und Defektmodellierung in der Festkörperchemie}

{Die Modellierung von Defekten in Festkörpern, durch Leerstellen, Zwischengitteratomen oder Dotierungen, gehört zu einem der wichtigsten und zugleich schwierigsten Themen der modernen Materialforschung. Ideale Kristalle selten und Abweichungen von der perfekten Gitterstruktur beeinflussen zentrale Eigenschaften wie Leitfähigkeit, Magnetismus, optische Absorption und mechanische Stabilität. Besonders in Halbleitern, Supraleitern oder Quantenmaterialien wie topologischen Isolatoren sind Defekte entscheidend für die Funktion des gesamten Systems (vgl. \cite{bassett_quantum_2019}).
Klassische Simulationen sind bei solchen Aufgaben oft unzureichend, da stark lokalisierte Elektronenzustände oder Spin-Zustände rund um Defektstellen nur ungenau beschrieben werden können (vgl. \cite{bauer_quantum_2020}). Gerade wenn mehrere Elektronenkonfigurationen gleichzeitig berücksichtigt werden müssen (Multireferenzcharakter), stoßen klassische Verfahren an ihre Grenzen (vgl. \cite{bassett_quantum_2019}).
Quantencomputer bieten hier neue Möglichkeiten, da sie stark korrelierte Vielteilchensysteme direkt auf Basis quantenmechanischer Prinzipien modellieren können (vgl. \cite{daley_practical_2022}).

\subsubsection{Stark korrelierte Materialien und Hochtemperatur-Supraleitung}

Stark korrelierte Materialien und Hochtemperatur-Supraleiter sind zentrale Forschungsgebiete, die ohne die Einbeziehung von Elektronenkorrelationen kaum verstanden werden könnten. Theoretisch beschäftigt man sich mit der Frage, wie kollektive Phänomene wie der Mott-Übergang, Spin-Ladungs-Entkopplung oder unkonventionelle Supraleitung aus der Wechselwirkung zahlreicher Elektronen hervorgehen (vgl. \cite{daley_practical_2022}). Traditionelle Methoden wie Hartree-Fock oder die Standard-DFT sind hier oft nicht ausreichend, weshalb weiterentwickelte Ansätze wie die Dynamische Mean-Field-Theorie (DMFT) oder Multi-Referenz-Methoden angewendet werden. Inzwischen ermöglichen Quantencomputer und Quantensimulatoren die realitätsnahe Modellierung dieser Systeme, indem sie die komplexen Vielteilchenzustände ab-initio behandeln können (vgl. \cite{baker_simulating_2024}).
In der Praxis haben Experimentalphysiker mit Hilfe analoger Quantensimulatoren, etwa auf Basis ultrakalter Atome, erstmals den Mott-Übergang im Fermi-Hubbard-Modell nachgestellt (vgl. \cite{daley_practical_2022}). Auch die Quantensimulation von antiferromagnetischen Phasenübergängen in Festkörpern wurde erfolgreich umgesetzt. Ebenso werden heute auf Quantenprozessoren mit optimierten Algorithmen wie dem Variational Quantum Eigensolver (VQE) neue Einsichten in die elektronischen Strukturen korrelierter Systeme gewonnen, was konkrete Impulse für das Materialdesign etwa von Hochtemperatur-Supraleitern liefert (vgl. \cite{weidman_quantum_2024}).

\subsubsection{Punktdefekte und Dotierungen in Halbleitern}

In Halbleitern sind Punktdefekte wie Leerstellen, Zwischengitteratome oder gezielte Dotierungen essenziell für grundlegende Eigenschaften wie Leitfähigkeit, Ladungsträgerdichte und optische Aktivitäten(vgl. \cite{bassett_quantum_2019}). Theoretisch wird dazu etwa die energetische Lage von Defektzuständen im Bandabstand oder die Wechselwirkung mit freien Ladungsträgern untersucht (vgl. \cite{freysoldt_first-principles_2014}). Klassische DFT-Ansätze unterschätzen meist die Bandlücke und beschreiben lokal gebundene Elektronenzustände bei Defektstellen oft nur ungenau. Quantencomputing eröffnet hier neue Wege, indem Defektbildungsenergien, Ladungsniveaus und sogar Diffusionsbarrieren auf ab-initio-Niveau bestimmt werden können (vgl. \cite{bassett_quantum_2019}).
Praktisch wurden mit hybriden Quantenalgorithmen bereits Simulationen von Farbzentren in Diamant oder Stickstoff-Leerstellen in Siliziumcarbid durchgeführt. Dies sind beides Systeme, die als Quantensensoren oder als Bauelemente für Quantencomputer zum Einsatz kommen könnten  (vgl. \cite{baker_simulating_2024}, \cite{cao_ab_2023}). Durch Quantensimulation lässt sich beispielsweise der Aufbau von Spin-Zuständen oder die Dynamik von Elektron-Loch-Paaren experimentell überprüfen und gezielt optimieren, was direkt in neue optoelektronische oder quantentechnologische Anwendungen einfließt (vgl. \cite{cao_ab_2023}).

\subsubsection{Defektmechanismen in Batteriematerialien}

Die Funktion von Batteriematerialien basiert auf der Bewegung von Ionen, die stets mit Defektbildungen wie Leerstellen oder Interkalationszentren verknüpft ist. In der Theorie werden dabei die elektronischen Zustände von Defekten und ihre Wechselwirkung mit mobilen Ionen untersucht, insbesondere wenn stark korrelierte Übergangsmetallionen beteiligt sind (vgl. \cite{hanaor_computational_2024}, \cite{freysoldt_first-principles_2014}). Klassische Berechnungsmethoden kommen dabei oft an ihre Grenzen, weil sie gemischte Ladungszustände oder komplexe chemische Reaktionen nicht genau erfassen können. Quantencomputer können dagegen die komplizierten elektronischen Strukturen und auch dynamische Vorgänge wie die Bewegung der Ionen, die Bildung von Schutzschichten (SEI) oder Phasenübergänge besser simulieren (vgl. \cite{urban_computational_2016}).
In der Praxis entwickeln Forschungsgruppen Quantenalgorithmen zur präzisen Berechnung der Redoxzustände und defektassoziierten Mechanismen. Erste Anwendungen zeigen, dass so wichtige Kenngrößen wie Zellspannung, Stabilität und Lebensdauer von Energiematerialien sehr viel realitätsnäher vorhergesagt werden können (vgl.\cite{baker_simulating_2024}). Dies eröffnet neue Perspektiven für das gezielte Design von Kathodenmaterialien und die Erhöhung der Energiedichte und Lebensdauer moderner Batterien (vgl. \cite{hanaor_computational_2024}).
}

\section{Top Technologien \& Algorithmen}
\section{Top 3 Zukunftsprojekte \& Forschungsinitiativen}
\section{Bewertung anhand der Kriterien}
\section{Teilfazit}


\printbibliography
