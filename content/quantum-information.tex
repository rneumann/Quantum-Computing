%\motto{Use the template \emph{chapter.tex} to style the various elements of your chapter content.}
\chapter{Quanteninformationen}
\label{qbits} % Always give a unique label
% use \chaptermark{}
% to alter or adjust the chapter heading in the running head

\abstract*{some abstract}

\abstract{some abstract}

\section{Section Heading}

\section{Praxisbeispiel: Bell-Zustand und Quantenkorrelation}
Bell-Zustände sind spezielle Zustände in der Quantenmechanik, in denen zwei Teilchen miteinander verschränkt sind. Das bedeutet: Ihre Zustände hängen so stark zusammen, dass man sie nicht unabhängig voneinander beschreiben kann – selbst, wenn die Teilchen weit voneinander entfernt sind. Diese Quantenkorrelationen lassen sich nicht mit klassischer Physik erklären. Benannt ist der Bell-Zustand nach dem Physiker John S. Bell. Dieser zeigte im Jahre 1964 auf, dass die Vorhersagen der Quantenmechanik im Widerspruch zu den Prinzipien des lokalen Realismus stehen – also der Vorstellung, dass Informationen nicht schneller als Licht übertragen werden können und dass physikalische Größen vor der Messung bereits festgelegt sind. (\cite[S.195]{bell_einstein_1964})
\\


Mit der von John S. Bell formulierten Bell-Ungleichung entwickelte Bell zudem ein mathematisches Kriterium, mit dem sich klassische und quantenmechanische Theorien experimentell unterscheiden lassen. Die Quantenmechanik sagt unter bestimmten Bedingungen eine Verletzung dieser Ungleichung voraus – was in zahlreichen Experimenten, den sogenannten Bell-Test, ab 1972 bestätigt wurde. Seitdem wurde die Verletzung der Bell-Ungleichung in zahlreichen Experimenten mit verschränkten Teilchenpaaren eindeutig nachgewiesen. In allen Fällen bestätigten die Ergebnisse die Vorhersagen der Quantenmechanik. 
(\cite[S.53-59]{homeister_quantum_2022})
\\


Insgesamt existieren vier verschiedene Bell-Zustände, die eine Situation maximaler Verschränkung zwischen zwei Qubits beschreiben. Das heißt: Wenn man eines misst, ist das Ergebnis des anderen automatisch bestimmt – selbst wenn sie weit voneinander entfernt sind. (\cite[S.53-55]{homeister_quantum_2022}) Die vier Bell-Zustände sind nachfolgend dargestellt, anschließend wird erläutert, wie ein Bell-Zustand erzeugt werden kann
\[
\begin{aligned}
\ket{\Phi^+} &= \frac{1}{\sqrt{2}} (\ket{00} + \ket{11}), \\
\ket{\Phi^-} &= \frac{1}{\sqrt{2}} (\ket{00} - \ket{11}), \\
\ket{\Psi^+} &= \frac{1}{\sqrt{2}} (\ket{01} + \ket{10}), \\
\ket{\Psi^-} &= \frac{1}{\sqrt{2}} (\ket{01} - \ket{10}).
\end{aligned}
\]
\\


\textbf{Schritt 1 – Superposition erzeugen:} \\
Zuerst wird auf das erste Qubit ein Hadamard-Gatter angewendet. Dadurch wird dieses Qubit in eine Superposition überführt:

\[
\frac{1}{\sqrt{2}} (\ket{0} + \ket{1}) \ket{0} = \frac{1}{\sqrt{2}} (\ket{00} + \ket{10})
\]
\\


\textbf{Schritt 2 – Verschränkung herstellen:} \\
Anschließend folgt ein CNOT-Gatter, bei dem das erste Qubit als Kontroll- und das zweite als Zielqubit fungiert. Dieses Gatter invertiert das Zielqubit nur dann, wenn das Kontrollqubit den Zustand \(\ket{1}\) hat. Dadurch entsteht der Zustand:

\[
\frac{1}{\sqrt{2}} (\ket{00} + \ket{11}) = \ket{\Phi^+}
\]
\\


Es resultiert die vollständige Quantenschaltung zur Erzeugung des Bell-Zustands. Diese ermöglicht es, jeden einfachen Zwei-Qubit-Eingangszustand (\(\ket{00}, \ket{01}, \ket{10}, \ket{11}\)) in einen Bell-Zustand zu überführen.

\[
\Qcircuit @C=1em @R=1em {
\lstick{\ket{0}} & \gate{H} & \ctrl{1} & \qw & \rstick{\frac{1}{\sqrt{2}}(\ket{00} + \ket{11})} \qw \\
\lstick{\ket{0}} & \qw      & \targ    & \qw & \qw
}
\]
\\


Dieser Zustand ist einer der zuvor vorgestellten \textbf{Bell-Zustände} – ein maximal verschränkter Zustand, in dem die Messungen der beiden Qubits perfekt korreliert sind. Wird in einem verschränkten Bell-Zustand das erste Qubit gemessen, so ergibt sich mit gleicher Wahrscheinlichkeit entweder der Zustand \( \ket{0} \) oder \( \ket{1} \). In beiden Fällen legt diese erste Messung sofort auch den Zustand des zweiten Qubits fest: Beobachtet man \( \ket{0} \) am ersten Qubit, ergibt sich insgesamt der Zustand \( \ket{00} \); misst man \( \ket{1} \), resultiert der Zustand \( \ket{11} \). Eine anschließende Messung des zweiten Qubits führt daher zwangsläufig zum gleichen Ergebnis wie beim ersten – entweder beide Qubits liefern 0 oder beide liefern 1. Je nach Eingangs-Zustand und der Reihenfolge der Gatter ergibt sich ein anderer Bell-Zustand. (\cite[S.53-54]{homeister_quantum_2022})
\\


Die Eigenschaften von Bell-Zuständen lassen sich mithilfe von Quantencomputern, wie denen von IBM, direkt simulieren und beobachten. In diesem Beispiel wurde ein einfacher Quantenschaltkreis aufgebaut, um einen Bell-Zustand zu erzeugen. Ausgangszustand waren zwei Qubits, die beide den Wert 0 hatten. Zunächst wurde auf das erste Qubit ein Hadamard-Gatter angewendet. Dadurch ging es in eine Überlagerung aus „0“ und „1“ über. Anschließend folgte ein CNOT-Gatter, das die beiden Qubits miteinander verschränkte: Wenn das erste Qubit auf „1“ übergeht, kippt das zweite automatisch ebenfalls auf „1“. Das Ergebnis ist ein Zustand, in dem die Qubits perfekt miteinander verbunden sind – man spricht von einem verschränkten Zustand. 

\begin{figure}[h]
    \centering
    \includegraphics[width=0.6\textwidth]{images/Schaltung_IBM.png}
    \caption{Quantenschaltung - IBM Quantum Learning Platform.}
    \label{fig:meinbild}
\end{figure}

Nach dem Aufbau der Schaltung wurden auf dem IBM-Quantencomputer Messungen durchgeführt. Das bedeutet, dass beide Qubits im sogenannten Z-Basiszustand ausgelesen werden um zu überprüfen, ob sie im Zustand „0“ oder „1“ sind. Die Auswertung zeigte, dass in nahezu allen Fällen entweder beide Qubits mit dem Ergebnis „0“ oder beide mit „1“ gemessen wurden – und zwar mit jeweils jeweils rund 50\% Wahrscheinlichkeit.

\begin{figure}[h]
    \centering
    \includegraphics[width=0.7\textwidth]{images/results_ibm.png}
    \caption{Wahrscheinlichkeitsverteilung - IBM Quantum Learning Platform.}
    \label{fig:meinbild}
\end{figure}

Wie das Histogramm zeigt, wurden ausschließlich die Zustände „00“ und „11“ beobachtet. Die Zustände „01“ oder „10“, bei denen sich die beiden Qubits unterscheiden würden, traten gar nicht auf. Dies ist ein typisches Verhalten eines Bell-Zustands und belegt, dass die Qubits verschränkt sind. Bemerkenswert ist dabei: Diese Quantenkorrelation bleibt bestehen, selbst wenn die Qubits räumlich voneinander getrennt wären. Die Messergebnisse des einen beeinflussen scheinbar sofort das andere – ein Verhalten, das mit klassischer Physik nicht erklärbar ist. (\cite{Bell State ZZ-Measurement}) 
\\


Genau darin zeigt sich der Informationsgehalt eines verschränkten Zustands: Die Information liegt nicht in einem einzelnen Qubit, sondern nur in ihrer gemeinsamen Beziehung. Diese Simulation macht das abstrakte Konzept der Quantenverschränkung greifbar und bieten einen intuitiven Zugang zu den Grundlagen der Quanteninformation. Sie zeigen anschaulich, wie Quantencomputer fundamentale Prinzipien der Quantenmechanik sichtbar und messbar machen. 
\printbibliography
