%\motto{Use the template \emph{chapter.tex} to style the various elements of your chapter content.}
\chapter{Quanteninformationen}
\label{qbits} % Always give a unique label
% use \chaptermark{}
% to alter or adjust the chapter heading in the running head

\chapterauthor{Karin Mustermann, Dominik Neumaier, Thilo Prünte, Burak Demirtas}

\abstract{some abstract}

\section{Der Qubit als Informationsträger}
Die fundamentale Funktionsweise klassischer Computer basiert auf einzelnen Bits, welche jeweils den Zustand $0$ (kein Strom) oder $1$ (Strom) annehmen können. Für Berechnungen werden dann Schaltkreise mit Logikgattern verwendet, die zu einer Ausgabe führen. 
\\

Im Kontext von Quantencomputing ist die kleinste Informationseinheit ein Quantenbit (Qubit). Ein Qubit ist in einer Superposition und kann somit mehr Zustände Annehmen als ein klassisches Bit. 

\section{Quantenverschränkung und Teleportation}

\subsection{Quantenverschränkung}
Die Quantenverschränkung ist ein Phänomen, bei dem zwei oder mehr Quantenobjekte sich in einem Zustand befinden, in dem sie, egal wie weit voneinander entfernt sie sind, gleich auf externe Reize reagieren. Solche Objekte, bei denen die Quantenverschränkung nachgewiesen werden konnte, sind Atome, Elementarteilchen wie Elektronen und Photonen, bis hin zu Kristallen.\\
\\
An einem Beispiel mit Elektronen lässt sich die Quantenverschränkung wie folgt erklären. Elektronen können sich in einem Zustand befinden, in dem sie keine eindeutige, exakte Position haben, sondern eine Menge an potentiellen Positionen. Diese potentiellen Positionen befinden sich alle im Umfeld der durchschnittlichen Position, dem Massenmittelpunkt des Elektrons, wie eine Wolke. Wenn von diesem Elektron seine Position gemessen wird, Antwortet dieses Elektron auf die Messung mit einem zufälligen Wert. Bei den nächsten Messungen wird wiederum mit einem anderen zufälligen Wert geantwortet. Es besteht ein Indeterminismus. In diesem Zustand ist es sogar möglich, dass sich das Elektron, wegen dem Indeterminismus auch an zwei oder mehr Positionen gleichzeitig befindet. Und dieses Elektron, an zwei oder mehr Positionen gleichzeitig, reagiert auf Reize auf die gleiche Weise. Dies wurde im Doppelspaltexperiment von Thomas Young nachgewiesen, bei dem ein Elektron auf einen Reiz an der ersten Position an einem Young‘schen Spalt reagiert, und auf einen Reiz an der zweiten Position an einem anderen Young’schen Spalt reagiert.\\
\\
Obwohl es möglich ist, dass ein Quantenobjekt keine exakte Position hat, und ein weiteres mit dem ersten Quantenobjekt verbundenes, „verschränktes“ Quantenobjekt ebenfalls keine eindeutige Position hat, ist die Distanz zwischen den beiden Quantenobjekten klar bestimmt. Wenn man also die Position der beiden Quantenobjekte misst, erhält man für beide immer einen zufälligen Wert. Aber die Differenz der zufälligen Werte voneinander ist immer genau gleich. Das gilt immer, auch wenn die beiden Quantenobjekte sehr weit voneinander entfernt sind. Die Position eines Quantenobjekts selbst ist nicht wohlbestimmt, die Position im Bezug auf ein verbundenes, verschränktes Quantenobjekt hingegen schon. Dass sich ein Quantenobjekt „hier“ und „einer festen Distanz von hier“ gleichzeitig befindet, wird „Superposition“ genannt. Und dieser superpositionierte Zustand ist ein verschränkter Zustand.\\
\\
Die Verschränkung fixiert nicht nur die Distanz von zwei oder mehr Quantenobjekten voneinander, sondern auch weitere Variablen, z.B. die Geschwindigkeit. Sie haben die gleiche Geschwindigkeit, welche aber selbst nicht fest ist, sondern eine von einer Menge potentieller Geschwindigkeiten.\\
\\
(Vgl. \cite[S.83-88]{gisin_unbegreifliche_2014}) 

\subsection{Quantenteleportation: Protokoll, um einen unbekannten Quantenzustand mithilfe eines verschränkten Paares und klassischer Kommunikation zu übertragen}
Ein Objekt besteht aus Materie und physikalischem Zustand. Bei Quantenobjekten ist die Materie die Masse und permanente Attribute wie z.B. die elektrische Ladung bei Elektronen, bzw. die Energie bei massenlosen Quantenobjekten wie z.B. Photonen. Der physikalische Zustand ist gebildet aus potentiellen Attributen, wie z.B. die potentiellen Positionen (denke Wolke um Massen-/ Energiemittelpunkt), die potentiellen Geschwindigkeiten bei Elektronen bzw. die potentiellen Schwingungsfrequenzen bei Photonen.\\
\\
In der Quantenteleportation wird der Zustand eines Quantenobjekts, der sog. Quantenzustand, ohne das Durchlaufen einer Zwischenstrecke, von einer Position auf eine andere Position, vorausgesetzt, dass sich in dieser anderen Position ein Quantenobjekt derselben Art befindet, direkt versetzt. Die Masse bzw. Energie kann nicht teleportiert werden, weil dies das Prinzip der Unmöglichkeit von Kommunikation ohne Signalübertragung verletzen würde. Dahingegen hat der Zustand weder eine Masse, noch eine Energie, denn sie ist potentiell – eine Wahrscheinlichkeit.\\
\\
Nach einer solchen Quantenteleportation verliert das Quantenobjekt seinen Zustand. Am Beispiel eines Photons kann man dies wie folgt erklären. Es gäbe ein Photon mit gut strukturierter Polarisation. Bei solch einem Photon schwingt das elektrische Feld regelmäßig in eine bestimmte Richtung. Nach der Quantenteleportation verliert dieses Photon seine Struktur. Übrig bleibt ein depolarisiertes Photon, ein Photon mit strukturloser Polarisation, dessen elektrisches Feld unregelmäßig in alle Richtungen schwingt. Der Quantenzustand des ersten Quantenobjekts nach der Quantenteleportation entspricht dem Quantenzustand des zweiten Quantenobjekts vor der Quantenteleportation.\\
\\
Der Quantenzustand, der teleportiert wird, ist ein Qubit. Voraussetzung einer Quantenteleportation ist das Dasein einer Menge verschränkter Quantenobjekte. In der nächsten Subsektion wird hierüber genauer erläutert.\\
\\
(Vgl. \cite[S.124-128]{gisin_unbegreifliche_2014}) 


\subsection{Bennett und Brassard (1993): wie drei Qubits (Senderzustand + 2 verschränkte) genutzt werden, um den Zustand über Distanz zu “teleportieren”}
Es gebe ein Quantenobjekt eines Senders mit unbekanntem Quantenzustand bzw. Qubit \(\ket{\Phi}\). Dieser Qubit soll zu einem dritten Quantenobjekt eines Empfängers teleportiert werden. Der Sender ist im Besitz eines zweiten Quantenobjekts mit dem ursprünglichen Zustand \(\ket{\alpha_0}\), „Ancilla“ bzw. „Ancilla-Qubit“ genannt. Im ersten Schritt der Quantenteleportation bringt der Sender das Quantenobjekt mit dem Qubit \(\ket{\Phi}\) und das Quantenobjekt mit dem Ancilla-Qubit dazu, so miteinander zu interagieren, dass das erste Quantenobjekt in einen Standard Zustand \(\ket{\Phi_0}\), und das zweite Quantenobjekt in einen unbekannten Quantenzustand \(\ket{\alpha}\), welches die vollständige Information über \(\ket{\Phi}\) enthält, versetzt wird. Diese Interaktion ist eine gemeinsame Messung der Quantenzustände beider Quantenobjekte des Senders. Allerdings muss erwähnt werden, dass eine solche Messung nur ein positives Ergebnis liefert, wenn der Qubit \(\ket{\Phi}\) einem orthonormalen Set angehört. Im zweiten Schritt der Quantenteleportation wird dann die neue Ancilla, bzw. in anderen Worten das (positive) Ergebnis der Messung im ersten Schritt, an das Quantenobjekt des Empfängers gesendet, wonach der Empfänger die Qubit-Veränderung auslösende Interaktion vom Sender rückgängig machen kann, und somit eine Replikation des originalen Quantenobjekts mit dem Qubit \(\ket{\Phi}\) herstellen kann. Auf dieser Weise können Informationen von Quantenobjekten, also Quanteninformationen, ausgetauscht werden, ohne dass ein Quantenobjekt geklont wird. Diese Methode der Quantenteleportation wird „spin-exchange“ genannt.\\
\\
(Vgl. \cite[S.1-2]{bennett_teleporting_1993})\\
\\
Am anfänglichen Zustand der „spin-exchange“ Methode sind das Quantenobjekt 2 auf der Senderseite und das Quantenobjekt 3 auf der Empfängerseite miteinander verschränkt. Durch die Interaktion auf Senderseite, wobei die Qubits der Quantenobjekte 1 und 2 verändert werden, wird die Verschränkung der Quantenobjekte 2 und 3 aufgehoben, und stattdessen die Quantenobjekte 1 und 2 verschränkt.\\
\\
(Vgl. \cite[S.2-3]{bennett_teleporting_1993})\\
\\
Mathematisch lässt sich der Prozess, wobei die Quantenobjekte in diesem Beispiel spin-\(\frac{1}{2}\) Partikel sind, mit folgenden Formeln darstellen:

\[ \ket{\Psi_{23}^{(-)}} = \sqrt{\frac{1}{2}} (\ket{\uparrow_2}\ket{\downarrow_3} + \ket{{\downarrow_2}\ket{\uparrow_3}}) \]
\\
Das zweite Quantenobjekt vom Sender (2) und das Quantenobjekt vom Empfänger (3) befinden sich in einem verschränkten Zustand. Dies ist die Ausgangssituation.

\[ \ket{\Psi_{12}^{(+)}} = \sqrt{\frac{1}{2}} (\ket{\uparrow_1}\ket{\downarrow_2} + \ket{{\downarrow_1}\ket{\uparrow_2}}) \]
\[ \ket{\Phi_{12}^{(\pm)}} = \sqrt{\frac{1}{2}} (\ket{\uparrow_1}\ket{\uparrow_2} \pm \ket{{\downarrow_1}\ket{\downarrow_2}}) \]
\\
Nun findet die gemeinsame Messung der beiden Quantenobjekte des Senders statt. Die vier sich in den Formeln befindenden Zustände bilden eine orthonormale Basis für die Quantenobjekte 1 und 2. Hiermit ist die Verschränkung der Quantenobjekte 2 und 3 aufgehoben, und stattdessen sind 1 und 2 verschränkt.

\[ \ket{\Phi} = \ket{\Phi_1} = a{\uparrow_1} + b{\downarrow_1} \]
\\
Für einfache Verständlichkeit wird der unbekannte Originalzustand von Quantenobjekt 1, \(\ket{\Phi}\), so geschrieben. Dabei ist \(|a|^2 + |b|^2 = 1\).\\
\\
(Vgl. \cite[S.2]{bennett_teleporting_1993})

\subsection{Bedeutung für Quantenkommunikation und als Demonstration von Quanteninformationstransfer}
Wie Anhand des wissenschaftlichen Artikels von Bennett, Brassard et. al. gezeigt wurde, sind Quantenverschränkung und Quantenteleportation die Prinzipien, durch die es zu einem Quanteninformationstransfer kommen kann. Daraus kann man schließen, dass beide Prinzipien eine sehr wichtige Bedeutung für Quantenkommunikation haben.

\section{Praxisbeispiel: Bell-Zustand und Quantenkorrelation}
Bell-Zustände sind spezielle Zustände in der Quantenmechanik, in denen zwei Teilchen maximal miteinander verschränkt sind. Das bedeutet: Ihre Zustände hängen so stark zusammen, dass man sie nicht unabhängig voneinander beschreiben kann – selbst, wenn die Teilchen weit voneinander entfernt sind. Benannt ist der Bell-Zustand nach dem Physiker John S. Bell. Dieser zeigte im Jahre 1964 auf, dass die Vorhersagen der Quantenmechanik im Widerspruch zu den Prinzipien des lokalen Realismus stehen – also der Vorstellung, dass Informationen nicht schneller als Licht übertragen werden können und dass physikalische Größen vor der Messung bereits festgelegt sind. (Vgl. \cite[S.195]{bell_einstein_1964})
\\


Mit der von John S. Bell formulierten Bell-Ungleichung entwickelte Bell ein mathematisches Kriterium, mit dem sich klassische und quantenmechanische Theorien experimentell unterscheiden lassen. Die Quantenmechanik sagt unter bestimmten Bedingungen eine Verletzung dieser Ungleichung voraus. Belegt wurde dies in zahlreichen Experimenten ab 1972, den sogenannten Bell-Test. Seitdem wurde die Verletzung der Bell-Ungleichung in zahlreichen Experimenten mit verschränkten Teilchenpaaren eindeutig nachgewiesen. In allen Fällen bestätigten die Ergebnisse die Vorhersagen der Quantenmechanik. 
(Vgl. \cite[S.53-59]{homeister_quantum_2022})
\\


Insgesamt existieren vier verschiedene Bell-Zustände, die eine Situation maximaler Verschränkung zwischen zwei Qubits beschreiben. Das heißt: Wird der Zustand eines Qubits gemessen, ist das Ergebnis des anderen automatisch bestimmt. Unabhängig von der Entfernung der Qubits voneinander. (Vgl. \cite[S.53-55]{homeister_quantum_2022}) 
\\


Die vier Bell-Zustände sind nachfolgend dargestellt: 
\[
\begin{aligned}
\ket{\Phi^+} &= \frac{1}{\sqrt{2}} (\ket{00} + \ket{11}), \\
\ket{\Phi^-} &= \frac{1}{\sqrt{2}} (\ket{00} - \ket{11}), \\
\ket{\Psi^+} &= \frac{1}{\sqrt{2}} (\ket{01} + \ket{10}), \\
\ket{\Psi^-} &= \frac{1}{\sqrt{2}} (\ket{01} - \ket{10}).
\end{aligned}
\]
\\


Die Erzeugung eines Bell-Zustands basiert auf zwei Schritten. Zu Beginn wird eine Superposition erzeugt, anschließend muss die Verschränkung zwischen den Qubits hergestellt werden. Diese Schritte werden nachfolgend erläutert.
\\


\textbf{Schritt 1 – Superposition erzeugen:} \\
Zuerst wird auf das erste Qubit ein Hadamard-Gatter angewendet. Dadurch wird dieses Qubit in eine Superposition überführt:

\[
\frac{1}{\sqrt{2}} (\ket{0} + \ket{1}) \ket{0} = \frac{1}{\sqrt{2}} (\ket{00} + \ket{10})
\]
\\


\textbf{Schritt 2 – Verschränkung herstellen:} \\
Anschließend folgt ein CNOT-Gatter, bei dem das erste Qubit als Kontroll- und das zweite als Zielqubit fungiert. Dieses Gatter invertiert das Zielqubit nur dann, wenn das Kontrollqubit den Zustand \(\ket{1}\) hat. Dadurch entsteht der Zustand:

\[
\frac{1}{\sqrt{2}} (\ket{00} + \ket{11}) = \ket{\Phi^+}
\]
\\


Es resultiert die vollständige Quantenschaltung zur Erzeugung des Bell-Zustands. Diese ermöglicht es, jeden einfachen Zwei-Qubit-Eingangszustand (\(\ket{00}, \ket{01}, \ket{10}, \ket{11}\)) in einen Bell-Zustand zu überführen.

\[
\Qcircuit @C=1em @R=1em {
\lstick{\ket{0}} & \gate{H} & \ctrl{1} & \qw & \rstick{\frac{1}{\sqrt{2}}(\ket{00} + \ket{11})} \qw \\
\lstick{\ket{0}} & \qw      & \targ    & \qw & \qw
}
\]
\\


Dieser Zustand ist einer der zuvor vorgestellten \textbf{Bell-Zustände} – ein maximal verschränkter Zustand, in dem die Messungen der beiden Qubits perfekt korreliert sind. Wird in einem verschränkten Bell-Zustand das erste Qubit gemessen, so ergibt sich mit gleicher Wahrscheinlichkeit entweder der Zustand \( \ket{0} \) oder \( \ket{1} \). In beiden Fällen legt diese erste Messung sofort auch den Zustand des zweiten Qubits fest: Beobachtet man \( \ket{0} \) am ersten Qubit, ergibt sich insgesamt der Zustand \( \ket{00} \); misst man \( \ket{1} \), resultiert der Zustand \( \ket{11} \). Eine anschließende Messung des zweiten Qubits führt daher zwangsläufig zum gleichen Ergebnis wie beim ersten – entweder beide Qubits liefern 0 oder beide liefern 1. Je nach Eingangs-Zustand und der Reihenfolge der Gatter ergibt sich ein anderer Bell-Zustand. (Vgl. \cite[S.53-54]{homeister_quantum_2022})
\\


Die Eigenschaften von Bell-Zuständen lassen sich mithilfe von Quantencomputern, wie denen von IBM, direkt simulieren und beobachten. In diesem Beispiel wurde ein einfacher Quantenschaltkreis aufgebaut, um einen Bell-Zustand zu erzeugen. Ausgangszustand waren zwei Qubits, die beide den Wert 0 hatten. Zunächst wurde auf das erste Qubit ein Hadamard-Gatter (rotes Gatter) angewendet. Dadurch ging es in eine Überlagerung aus „0“ und „1“ über. Anschließend folgte ein CNOT-Gatter (blaues Gatter), das die beiden Qubits miteinander verschränkte: Wenn das erste Qubit auf „1“ übergeht, kippt das zweite automatisch ebenfalls auf „1“. Das Ergebnis ist ein Zustand, in dem die Qubits perfekt miteinander verbunden sind – man spricht von einem verschränkten Zustand. 

\begin{figure}[h]
    \centering
    \includegraphics[width=0.4\textwidth]{images/Schaltung_IBM.png}
    \caption{Quantenschaltung - IBM Quantum Learning Platform.}
    \label{fig:meinbild}
\end{figure}

Nach dem Aufbau der Schaltung wurden auf dem IBM-Quantencomputer Messungen durchgeführt. Das bedeutet, dass die Qubits im Z-Basiszustand ausgelesen werden um zu überprüfen, ob sie im Zustand „0“ oder „1“ sind. 

\begin{figure}[h]
    \centering
    \includegraphics[width=0.6\textwidth]{images/results_ibm.png}
    \caption{Wahrscheinlichkeitsverteilung - IBM Quantum Learning Platform.}
    \label{fig:meinbild}
\end{figure}


Wie das Histogramm zeigt, wurden ausschließlich die Zustände „00“ und „11“ beobachtet - und zwar mit jeweils rund 50\% Wahrscheinlichkeit. Die Zustände „01“ oder „10“, bei denen sich die beiden Qubits unterscheiden würden, traten gar nicht auf. Dies ist ein typisches Verhalten eines Bell-Zustands und belegt, dass die Qubits verschränkt sind. Bemerkenswert ist dabei: Diese Quantenkorrelation bleibt bestehen, selbst wenn die Qubits räumlich voneinander getrennt wären. Die Messergebnisse des einen beeinflussen scheinbar sofort das andere – ein Verhalten, das mit klassischer Physik nicht erklärbar ist. (Vgl. \cite{Bell State ZZ-Measurement}) 
\\


Ein anschauliches Beispiel für die Eigenschaften verschränkter Zustände liefert auch ein Gedankenexperiment mit zwei beispielhaften Personen, Alice und Bob. Angenommen die beiden erzeugen gemeinsam entsprechend der vorhergehenden Ausführungen folgenden Bell-Zustand:
\[
\frac{1}{\sqrt{2}} (|00\rangle + |11\rangle).
\]
Alice erhält nun das erste Qubit, Bob das zweite Qubit. Angenommen die beiden befinden sich ursprünglich in einem Haus im gleichen Raum. Bob nimmt nun sein Qubit mit in ein anderes Zimmer. Solange keine Messung erfolgt und die Qubits vor äußeren Einflüssen geschützt sind, bleibt die Verschränkung erhalten – unabhängig von der räumlichen Trennung. Führt nun einer der beiden eine Messung durch, so ist das Ergebnis zufällig: Mit einer Wahrscheinlichkeit von 50\,\% wird $|0\rangle$ gemessen, mit 50\,\% $|1\rangle$. Erst wenn Alice und Bob ihre Ergebnisse miteinander vergleichen, zeigt sich die Besonderheit: Ihre Messergebnisse stimmen stets überein. Es resultiert eine perfekte Korrelation, unabhängig von Raum und Zeit. (Vgl. \cite[S.54]{homeister_quantum_2022})
\\


Genau darin zeigt sich der Informationsgehalt eines verschränkten Zustands: Die Information liegt nicht in einem einzelnen Qubit, sondern nur in ihrer gemeinsamen Beziehung. Diese Simulation macht das abstrakte Konzept der Quantenverschränkung greifbar und bietet einen intuitiven Zugang zu den Grundlagen der Quanteninformation. Es zeigt anschaulich, wie Quantencomputer fundamentale Prinzipien der Quantenmechanik sichtbar und messbar machen. 
\printbibliography
