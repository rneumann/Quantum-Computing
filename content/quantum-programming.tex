%\motto{Use the template \emph{chapter.tex} to style the various elements of your chapter content.}
\chapter{Quantensoftware und Programmierung}
\label{programming} % Always give a unique label
% use \chaptermark{}
% to alter or adjust the chapter heading in the running head

\chapterauthor{Konrad Maywald, Tom Williard, Daniel Purtov, Dennis Schweigert}

\abstract{some abstract}

\section{Programmiermodelle}

\section{Programmiersprachen und -Frameworks}
\subsection{Architektur von Quantenprogrammiersprachen}
\subsection{Übersicht gängiger Quantenprogrammiersprachen und -frameworks}
\subsection{Herausforderungen und Zukunftsperspektiven}

\section{Entwicklung von Quantenalgorithmen}

\subsection{Übertragung mathematischer Ideen in Quantenalgorithmen}

- Mathematische Problemstellung als Ausgangspunkt (z.B. Suche -> Verweis)
- Abbildung auf Prinzipien der Quantenmechanik
- Auswahl eines passenden Algorithmus (z.B. Grover -> Verweis)

\subsection{Fallstudie: Grover-Algorithmus}

- Quadratische Beschleunigung unstrukturierter Suche
- z.B. Orakel + Amplitudenverstärkung -> Verweise/Überschneidung
- Darstellung als Pseudocode -> Details in 7.4?

\subsection{Quantum Software Stack}

- Überblick über Schichten von Algorithmus zur Hardware
- Tooling
- Rolle von:
    - Compiler / Transpiler: z.B. Umwandlung von Circuit in hardware-spezifische Gates
    - Backends: Simulator vs. Quantenchip
    - Middleware: Warteschlange, Hardware-Auswahl, Jobverwaltung

\subsection{Mitigation von Fehlerquellen}

- Verweis auf Kapitel Fehlerkorrektur, Recap der typischen Fehlerarten (z.B. Dekohärenz, Gatter-Ungenauigkeit, Messfehler)
- Beschreibung von softwareseitigen Maßnahmen:
    - Zero-noise extrapolation
    - Randomized benchmarking
    - Dynamisches Decoupling, probabilistische Fehlerkorrektur
- Bedeutung für algorithmische Genauigkeit und Wiederholbarkeit

\subsection{Nutzung von Cloud-Plattformen}

- Überblick über Cloud Plattformen (IBM, Amazon, Google)
\\

Einfache Quantenprogramme können mit Frameworks wie Qiskit, Cirq oder Q\# lokal entwickelt und auf Simulatoren ausgeführt werden. Der Zugriff auf echte Quantenhardware ist bislang jedoch lediglich über Cloud-Plattformen möglich.
\\

Neben dem Zugang zu Quantenhardware stellen Cloud-Plattformen Werkzeuge für die Entwicklung, Visualisierung und Ausführung von Quantenalgorithmen zur Verfügung. Darüber hinaus wird der Entwicklungsprozess durch die Abstraktion der Hardwaresteuerung, automatisierte Fehlerabschätzungen und die Integration von Entwicklungsumgebungen erleichtert.
\\

Die führenden Anbieter von Quantenhardware IBM, Amazon und Google stellen jeweils eigene Cloud-Plattformen bereit, die sich vor allem in den unterstützten Programmiersprachen und Hardware unterscheiden. Im Folgenden werden die Plattformen vorgestellt. und ihre wichtigsten Features erläutert.

\section{Praxisbeispiel: Quantum Programm mit Qiskit}


\printbibliography
