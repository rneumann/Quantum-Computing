\preface

Die Quantenmechanik hat seit Beginn des 20. Jahrhunderts unser Verständnis der physikalischen Realität tiefgreifend verändert. Durch die Arbeiten von Wissenschaftlern wie Max Planck, Albert Einstein, Erwin Schrödinger und Werner Heisenberg wurde ein neues Fundament der Physik gelegt – eines, das klassische Konzepte von Kausalität und Determinismus in Frage stellte und den Weg für völlig neue Technologien ebnete.

Aus diesen Grundlagen entwickelte sich ab den 1970er Jahren ein neues Forschungsfeld: die Quanteninformationsverarbeitung. Erste theoretische Überlegungen zeigten, dass sich quantenmechanische Effekte nutzen lassen, um Informationen auf radikal neue Weise zu speichern, zu übertragen und zu verarbeiten. Diese Idee wurde in den 1990er Jahren durch die Entwicklung bedeutender Quantenalgorithmen konkretisiert – mit dem Versprechen, bestimmte Probleme deutlich effizienter zu lösen als klassische Computer es je könnten.

Heute stehen wir an einem technologischen Wendepunkt. Quantencomputer sind nicht mehr reine Theorie, sondern Realität: Mit sogenannten NISQ-Geräten (Noisy Intermediate-Scale Quantum) lassen sich bereits erste Anwendungen demonstrieren. Systeme auf Basis supraleitender Qubits, Ionenfallen oder photonischer Architekturen werden kontinuierlich weiterentwickelt. Neben großen Technologiekonzernen wie IBM, Google und Amazon sind auch Start-ups und universitäre Forschungsgruppen – auch in Deutschland – aktiv beteiligt. Parallel dazu entstehen staatlich geförderte Programme und eine globale Quantenökonomie.

Mit diesen Fortschritten jedoch treten neue gesellschaftliche Fragen in den Vordergrund. Die Aussicht auf massiv gesteigerte Rechenleistungen wirft sicherheitspolitische und ethische Herausforderungen auf. Wie können sensible Daten in Zukunft geschützt werden, wenn etablierte Verschlüsselungsverfahren durch Quantencomputer angreifbar werden? Sind Blockchain-basierte Systeme wie Kryptowährungen vor Manipulation sicher? Und wie lassen sich demokratische Infrastrukturen gegen hybride Bedrohungen und Cyberangriffe schützen?

Neben diesen praktischen Aspekten betreten wir auch philosophisches Neuland. Die wachsende Leistungsfähigkeit von Quantencomputern und künstlicher Intelligenz rückt die Frage in den Fokus, was menschliches Denken eigentlich ausmacht. Ist unser Bewusstsein bloß ein komplexer, deterministischer Prozess – simulierbar auf Maschinen? Oder liegt in der quantenmechanischen Unbestimmtheit möglicherweise ein Ursprung des freien Willens? Falls dem so ist, könnten Quantencomputer sogar eine Brücke zu neuartigen Formen maschinellen Lebens darstellen.

Dieses Buch widmet sich der faszinierenden Entwicklung von Quantencomputern und Quantentechnologien – von den physikalischen Grundlagen über den Stand der Technik bis hin zu zukünftigen Perspektiven. Es beleuchtet nicht nur die wissenschaftlich-technische Seite, sondern nimmt auch gesellschaftliche, ethische und sicherheitspolitische Implikationen in den Blick.

Entstanden ist dieses Buch im Rahmen eines Masterkurses mit dem Titel „What’s Next?“ an der Hochschule Karlsruhe. Es ist das Ergebnis eines didaktischen Experiments, bei dem sich eine Gruppe Studierender gemeinsam einem hochkomplexen Zukunftsthema nähert – mit dem Ziel, nicht nur Wissen zu erlangen und weiterzugeben, sondern auch zum kritischen Denken und Mitgestalten der technologischen Zukunft anzuregen.


\vspace{\baselineskip}
\begin{flushright}\noindent
Karlsruhe, Juli 2025\hfill {\it Rainer Neumann}\\
\end{flushright}
