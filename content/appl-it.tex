%\motto{Use the template \emph{chapter.tex} to style the various elements of your chapter content.}
\chapter{Grundlegende Anwendungsgebiete}
\label{trends} % Always give a unique label
% use \chaptermark{}
% to alter or adjust the chapter heading in the running head

\chapterauthor{Isabel Fritz, Felix Goos, Martin Maier, David Richard, Sabine Weigand,
Lola Bankai, Mareike Rennebaum, Siri Wandel, Hüma Yilmaz}

\abstract{some abstract}

\section{Quantum Machine Learning: Anwendungen und Perspektiven}
\subsection{Einordnung, Motivation und Relevanz}
\subsection{Technologischer Überblick}
\subsubsection{Technologische Grundlagen}
\subsubsection{Stand der Technik und Forschungsprojekte}
\subsubsection{Chancen, Risiken und technologische Herausforderungen}
\subsection{Fazit}

\vspace{2em}

\section{Kryptographie}
\subsection{Einleitung}
Kryptographische Verfahren sind integraler Bestandteil moderner digitaler Infrastrukturen. Insbesondere asymmetrische Kryptosysteme wie RSA, Diffie-Hellman und elliptische Kurvenkryptographie basieren auf der angenommenen Schwierigkeit mathematischer Probleme wie der ganzzahligen Faktorisierung oder der Berechnung diskreter Logarithmen. Diese Probleme gelten im klassischen Rechenmodell als nur mit exponentiellem Aufwand lösbar, wodurch die genannten Verfahren als sicher eingestuft werden. Symmetrische Kryptographie, beispielsweise AES, bietet hingegen Sicherheitsgarantien auf Basis der Schlüsselraumgröße und ist gegenwärtig gegenüber klassischen Angriffen hinreichend resistent.

Diese Sicherheitsannahmen werden jedoch durch die Entwicklung leistungsfähiger Quantencomputer substanziell infrage gestellt. Mit Shors Algorithmus (Shor 1994) wurde erstmals ein quantenmechanisches Verfahren vorgestellt, das die Faktorisierung großer Zahlen sowie die Berechnung diskreter Logarithmen in polynomialer Zeit ermöglicht und somit die Grundlage nahezu aller etablierten asymmetrischen Verfahren kompromittiert. Ergänzend dazu führt Grovers Algorithmus zu einer quadratischen Beschleunigung von Suchproblemen, was insbesondere symmetrische Verfahren betrifft, bei denen dadurch eine effektive Halbierung der Schlüssellänge notwendig wird, um das ursprüngliche Sicherheitsniveau zu erhalten (Grover 1996; vgl. auch Shor u Grover, Überblick in [63]).

Diese Algorithmen stellen keine abstrakten theoretischen Bedrohungen dar, sondern begründen bereits heute reale Risiken, insbesondere im Kontext sogenannter Harvest-now-decrypt-later-Angriffe. In diesem Szenario werden verschlüsselte Kommunikationsinhalte langfristig gespeichert, um sie bei Verfügbarkeit eines hinreichend leistungsfähigen Quantencomputers retrospektiv zu entschlüsseln (Mosca 2018). Die praktische Relevanz ergibt sich dabei aus der Latenz zwischen dem heutigen Einsatz kryptographischer Verfahren und der potentiellen Verfügbarkeit skalierbarer Quantencomputer. Michele Mosca beschreibt diese Problematik entlang dreier Größen: der angestrebten Geheimhaltungsdauer (x), der Migrationszeit zu quantenresistenten Verfahren (y) und der geschätzten Zeit bis zum Bruch aktueller Verfahren (z). Ist x + y > z, besteht akuter Handlungsbedarf, da zukünftige Angreifer auf heute übertragene Daten zugreifen können, ohne dass dies zum Zeitpunkt der Kommunikation verhindert werden kann (Mosca S. 38f.).

Diese Bedrohungslage hat eine breite wissenschaftliche, industrielle und regulatorische Reaktion ausgelöst. Insbesondere die vom U.S. National Institute of Standards and Technology (NIST) koordinierte Standardisierung postquantenkryptographischer Verfahren zielt darauf ab, praktikable Alternativen zu heute verbreiteten Algorithmen zu identifizieren, zu evaluieren und langfristig zu etablieren. Im Rahmen der vierten Wettbewerbsrunde wurden Verfahren wie CRYSTALS-Kyber, CRYSTALS-Dilithium und SPHINCS+ als zukünftige Standards ausgewählt \cite{alagic_status_2025}.

Das vorliegende Kapitel analysiert diese Entwicklung aus kryptographischer Perspektive. Es beginnt mit der detaillierten Darstellung der durch Quantenalgorithmen verursachten Bedrohung für symmetrische und asymmetrische Kryptosysteme und ordnet das Harvest-now-decrypt-later-Szenario systematisch ein. Daran anschließend werden aktuelle technische, normative und strategische Reaktionen untersucht. Ziel ist es, die Implikationen für die Gestaltung zukunftsfähiger Sicherheitssysteme darzustellen und die Notwendigkeit einer rechtzeitigen und koordinierten Umstellung auf quantenresistente Verfahren herauszuarbeiten.

\subsection{Technologische Grundlagen}
\subsubsection{Post-Quantum-Cryptography}
\noindent\textbf{Einleitung und Motivation}

\noindent
Die Sicherheit klassischer kryptographischer Verfahren basiert auf Problemen wie der Faktorisierung großer Zahlen oder der Berechnung diskreter Logarithmen – Aufgaben, die für klassische Rechner schwer, aber für zukünftige Quantencomputer effizient lösbar sind. Insbesondere Shors Algorithmus (1994) kann sowohl RSA als auch elliptische Kurven in polynomialer Zeit brechen, während Grovers Algorithmus (1996) symmetrische Verfahren wie AES durch quadratische Beschleunigung schwächt. Diese Entwicklungen bedrohen die langfristige Vertraulichkeit digitaler Kommunikation – besonders in Bereichen, in denen Daten über Jahrzehnte hinweg sicher bleiben müssen, etwa im Gesundheits- oder Militärwesen.
 (Chen et al., 2016, Sec. 1–2; Mosca, 2018, S. 38–39)
Post-Quantum Cryptography (PQK) verfolgt daher das Ziel, neue kryptographische Verfahren zu entwickeln, deren Sicherheit auch gegenüber leistungsfähigen Quantencomputern erhalten bleibt. Dabei handelt es sich um rein klassische Algorithmen, die auf mathematischen Problemen beruhen, für die bislang keine effizienten quantenbasierten Angriffe bekannt sind. Die zentrale Motivation hinter PQK liegt also nicht in kurzfristigen Leistungsverbesserungen, sondern in der Absicherung digitaler Infrastrukturen gegen zukünftige Bedrohungen – ein Paradigmenwechsel von „komplexitätsbasierter“ zu „quantenresistenter“ Sicherheit.
 (Bernstein et al., 2009, Kap. 1; Chen et al., 2016, Sec. 3; Mosca, 2018, S. 39–40)


\vspace{1em}
\noindent\textbf{Abgrenzung und Einordnung}

\noindent
Post-Quantum Cryptography (PQK) darf nicht mit Quantenkryptographie verwechselt werden. Während Quantenkryptographie wie Quantum Key Distribution (QKD) physikalische Quanteneffekte und spezielle Hardware (z. B. Photonenkanäle) nutzt, ist PQK vollständig klassisch: Die Verfahren laufen auf herkömmlichen Computern und beruhen auf mathematischen Problemstellungen, die auch Quantencomputer voraussichtlich nicht effizient lösen können.
 (Chen et al., 2016, Sec. 2.2; Mosca, 2018, S. 39)
PQK verfolgt somit einen softwarebasierten Ansatz, der sich weitgehend in bestehende Infrastrukturen integrieren lässt – etwa durch Austausch von Schlüsselaustausch- oder Signaturalgorithmen in Protokollen wie TLS, SSH oder S/MIME. Im Gegensatz zu QKD, das lediglich den Schlüsselaustausch absichert, können PQK-Verfahren alle Aufgaben moderner Kryptographie abdecken: Vertraulichkeit, Authentizität und Integrität.
 (Bernstein et al., 2009, Kap. 1; Chen et al., 2016, Sec. 4; National Academies, 2019, S. 95–97)
Die beiden Ansätze – PQK und Quantenkryptographie – gelten daher nicht als Konkurrenz, sondern als komplementär: PQK bietet eine pragmatische Übergangslösung mit globaler Skalierbarkeit, während QKD auf langfristige physikalische Sicherheit zielt, aber hohe infrastrukturelle Anforderungen stellt.
 (Mosca, 2018, S. 40; National Academies, 2019, S. 97–98)


\vspace{1em}
\noindent\textbf{Grundlagen der PQK}

\noindent
Im Kern zielt Post-Quantum Cryptography darauf ab, kryptographische Verfahren zu entwickeln, die auch durch Quantencomputer nicht effizient gebrochen werden können. Dabei basieren PQK-Verfahren auf mathematischen Problemstellungen wie Gitterproblemen, Code-Fehlerkorrektur oder Hashfunktionen, für die es bislang weder klassische noch quantenbasierte effiziente Algorithmen gibt. Diese Probleme sind in vielen Fällen sogar NP-schwer oder lassen sich auf solche reduzieren.
 (Chen et al., 2016, Sec. 3; Bernstein et al., 2009, Kap. 1)
Ein großer Vorteil von PQK liegt darin, dass keine neue Hardware erforderlich ist. Die Algorithmen können als Software auf bestehenden Systemen laufen und sind dadurch schnell skalierbar und kosteneffizient einsetzbar. Das macht PQK besonders attraktiv für Szenarien, in denen langfristige Datensicherheit gewährleistet sein muss – etwa im Kontext von Cloud-Infrastrukturen, öffentlicher Verwaltung oder medizinischer Archivierung.
 (Mosca, 2018, S. 39–40; National Academies, 2019, S. 92–93)
Insgesamt ermöglicht PQK einen schrittweisen Übergang von heute verbreiteten Verfahren zu quantensicheren Alternativen – ohne Brüche in Protokollen oder Softwarearchitekturen. Dies ist entscheidend, um Sicherheitssysteme rechtzeitig vor dem breiten Einsatz von Quantencomputern zu schützen.
 (Chen et al., 2016, Sec. 4; Bernstein et al., 2009, Kap. 1)

\vspace{1em}
\noindent\textbf{Problemklassen und Verfahren}
\subparagraph{Gitterbasierte Verfahren}
Gitterbasierte Kryptographie basiert auf geometrischen Problemen im mehrdimensionalen Raum, insbesondere auf dem Learning-With-Errors-Problem (LWE) und dem Shortest Vector Problem (SVP). Diese gelten sowohl für klassische als auch für Quantencomputer als hart. Die Verfahren zeichnen sich durch Effizienz, breite Anwendbarkeit (z. B. Signaturen, Schlüsselaustausch) und gute Sicherheitsnachweise aus. Prominente Vertreter sind Kyber (Key Encapsulation Mechanism) und Dilithium (Signaturverfahren), die von NIST standardisiert wurden.
 (Regev, 2005; Alagic et al., 2023; NIST, 2024 – FIPS 203)
 
\subparagraph{Codebasierte Verfahren}
Diese Ansätze nutzen die Schwierigkeit, zufällige lineare Fehlerkorrekturcodes zu dekodieren. Das klassische McEliece-System, das bereits 1978 vorgestellt wurde, ist bis heute gegen bekannte Angriffe – einschließlich quantenbasierter – resistent. Neuere Varianten wie MDPC-McEliece verringern den Speicherbedarf und ermöglichen effizientere Implementierungen. Hauptvorteil: extrem lange kryptographische Erfahrung; Nachteil: große Schlüssellängen.
 (Misoczki et al., 2013; Bernstein et al., 2009, Kap. 10)
 
\subparagraph{Hashbasierte Verfahren}
Diese Verfahren stützen sich allein auf die Sicherheit kryptographischer Hashfunktionen. Bekannteste Vertreter sind Lamport-Signaturen, Merkle-Bäume und darauf basierende Standards wie SPHINCS+. Da sie keine algebraische Struktur ausnutzen, gelten sie als sehr robust gegen strukturelle Angriffe – allerdings sind die Signaturen oft lang und benötigen komplexes Management für Schlüsselzustände.
 (Lamport, 1979; Bernstein et al., 2009, Kap. 5; Chen et al., 2016, Sec. 3.2)


\vspace{1em}
\noindent\textbf{Standardisierung und Status quo}

\noindent
Im Jahr 2016 startete das National Institute of Standards and Technology (NIST) einen mehrstufigen Auswahlprozess zur Standardisierung quantenresistenter Kryptographie. Ziel war es, aus Hunderten international eingereichter Vorschläge robuste, breit einsetzbare und quantensichere Verfahren für digitale Signaturen und Schlüsselaustausch zu identifizieren.
 (Alagic et al., 2023, S. 1–2)
Nach drei Wettbewerbsrunden wurden 2022 drei Verfahren für die Standardisierung ausgewählt: Kyber (Key Encapsulation), Dilithium (digitale Signaturen) und SPHINCS+ (hashbasiertes Signatursystem). Im März 2024 veröffentlichte NIST den ersten offiziellen PQC-Standard – FIPS 203, der Kyber spezifiziert. Die Standards zu Dilithium und SPHINCS+ folgen zeitnah.
 (NIST, 2024 – FIPS 203)
Die Relevanz für Wirtschaft und Staat ist hoch: Durch die Softwarekompatibilität dieser Verfahren ist ein schrittweiser Migrationspfad möglich – etwa in TLS, VPNs, Firmware-Updates oder digitaler Kommunikation in Verwaltung und Industrie. Unternehmen wie IBM, Cloudflare oder Amazon Web Services haben erste PQK-Pilotprojekte in Betrieb genommen.
 (Alagic et al., 2023, S. 3–5)
 

\subsubsection{Quantum-Key-Distribution}
\noindent\textbf{Einleitung und Motivation}

\noindent
Die Verteilung kryptographischer Schlüssel zählt zu den zentralen Herausforderungen moderner Informationssicherheit. Klassische Verfahren stützen sich auf die angenommene Schwierigkeit mathematischer Probleme wie der Faktorisierung großer Zahlen oder diskreter Logarithmen. Diese gelten zwar aktuell als sicher, könnten jedoch durch leistungsfähige Quantencomputer kompromittiert werden.
Quantum Key Distribution (QKD) verfolgt einen grundlegend anderen Ansatz: Sie nutzt nicht mathematische Komplexität, sondern quantenmechanische Prinzipien wie die Unschärferelation, Verschränkung und das No-Cloning-Theorem, um Sicherheit zu garantieren (Scarani et al., 2009).
Ein QKD-Protokoll erlaubt es zwei Parteien – klassisch Alice und Bob –, über einen Quantenkanal einen gemeinsamen geheimen Schlüssel zu erzeugen. Dieser beruht darauf, dass jede Messung durch einen Angreifer (Eve) zwangsläufig den Quantenzustand verändert und damit erkennbar stört (Bennett & Brassard, 1984).
Der Schlüssel wird anschließend über einen klassischen Kanal authentifiziert und genutzt. Da jeder Abhörversuch Spuren hinterlässt, gilt QKD theoretisch als informationstheoretisch sicher – unabhängig von technischer oder rechnerischer Kapazität (Ekert, 1991).


\vspace{1em}
\noindent\textbf{Abgrenzung und konzeptionelle Einordnung}

\noindent
Quantum Key Distribution (QKD) dient ausschließlich der sicheren Verteilung kryptographischer Schlüssel – die eigentliche Datenübertragung erfolgt weiterhin klassisch und muss gesondert verschlüsselt werden. QKD allein stellt also kein vollständiges Sicherheitssystem dar (Scarani et al., 2009).
Zudem erfordert QKD stets eine Authentifikation der Kommunikationspartner, um Angriffe wie „Man-in-the-Middle“ zu verhindern. Diese Authentifikation kann durch klassische Verfahren oder durch Post-Quantum-Kryptographie (PQK) erfolgen, die auf mathematisch schwer lösbaren Problemen basiert (Mosca, 2018).
QKD und PQK verfolgen dabei unterschiedliche Ansätze: Während PQK softwarebasiert ist und klassische Systeme absichert, nutzt QKD physikalische Gesetze zur Sicherheit. PQK ist leichter zu integrieren, QKD bietet im Idealfall theoretisch unbedingte Sicherheit, erfordert aber quantenphysikalische Infrastruktur (National Academies, 2019).
QKD ist damit ein spezialisierter Baustein, kein Ersatz für klassische oder PQK-gestützte Systeme, sondern ein ergänzender Bestandteil quantensicherer Kommunikation.


\vspace{1em}
\noindent\textbf{Grundlagen der Protokolle}

\subparagraph{BB84-Protokoll}
Das BB84-Protokoll wurde 1984 von Bennett und Brassard entwickelt und gilt als das erste konkrete Verfahren zur Quantum Key Distribution. Die zentrale Idee besteht darin, einzelne Photonen in zufälligen Polarisationszuständen zu senden – etwa horizontal/vertikal (|, ⎯) und diagonal (⧬, ×).
Der Sender (Alice) wählt dabei zufällig eine Basis (z. B. rectilinear oder diagonal) und sendet ein entsprechend polarisiertes Photon. Der Empfänger (Bob) misst das Photon ebenfalls in einer zufällig gewählten Basis. Nur wenn Sender und Empfänger dieselbe Basis verwendet haben, stimmen die Messergebnisse überein und können als Teil des gemeinsamen Schlüssels übernommen werden.
Ein Abhörversuch durch einen Dritten (Eve) würde das Photon stören und damit statistisch erkennbare Fehler erzeugen – was Grundlage der theoretischen Sicherheit ist.
 (Bennett & Brassard, 1984)
 
\subparagraph{E91-Protokoll}
Das E91-Protokoll, vorgeschlagen von Ekert (1991), basiert nicht auf Polarisation und Basiswahl, sondern auf quantum-entanglement (Verschränkung). Dabei wird ein Paar verschränkter Photonen erzeugt, von denen eines zu Alice und eines zu Bob geschickt wird.
Die beiden messen ihre Photonen in jeweils unterschiedlichen, zufällig gewählten Messrichtungen. Aufgrund der Quantenverschränkung sind die Messergebnisse korreliert – stärker, als es nach klassischer Physik möglich wäre. Diese Korrelationen können mit Bell-Tests überprüft werden und dienen als Sicherheitsnachweis: Wird die Bell-Ungleichung verletzt, ist ein Abhörversuch ausgeschlossen.
(Ekert, 1991)

\subparagraph{Vergleich BB84 vs. E91}
Beide Protokolle ermöglichen die sichere Generierung gemeinsamer Schlüssel, unterscheiden sich aber im Sicherheitsmodell:
BB84 stützt sich auf die Unmöglichkeit einer störungsfreien Messung und erfordert Vertrauen in die Quelle und die Geräte.
E91 nutzt dagegen verschränkte Zustände und erlaubt die Überprüfung der Sicherheit durch beobachtbare Bell-Korrelationen – potenziell sogar device-independent, also ohne vollständiges Vertrauen in die Hardware.

\vspace{1em}
\noindent\textbf{Erweiterte QKD-Konzepte}

\noindent
Um theoretisch sichere QKD-Protokolle an reale Bedingungen anzupassen, wurden mehrere Erweiterungen entwickelt, die bekannte Schwachstellen adressieren.
Decoy-State QKD
In der Praxis ist es oft technisch schwierig, wirklich einzelne Photonen zu senden. Viele Systeme verwenden daher abgeschwächte Laserpulse, bei denen gelegentlich mehrere Photonen gleichzeitig ausgesendet werden. Ein Angreifer könnte diese ausnutzen, indem er einen Photonenteil abzweigt und das restliche Signal weitersendet – ein sogenannter Photon-Splitting-Angriff.
Decoy-State QKD löst dieses Problem, indem zusätzlich zum eigentlichen Signal auch sogenannte Täuschungspulse („decoy states“) mit variabler Intensität gesendet werden. Diese erlauben es, den Kanal zu testen und solche Angriffe zu erkennen, ohne die Schlüsselerzeugung zu stören.
 (Scarani et al., 2009, Abschn. III.C)
Device-Independent QKD (DI-QKD)
Bei herkömmlichen Protokollen wird vorausgesetzt, dass die verwendeten Geräte korrekt und vertrauenswürdig arbeiten. In der Realität kann das aber durch Manipulation oder Fertigungsfehler nicht immer garantiert werden.
DI-QKD umgeht dieses Problem, indem es die Sicherheit direkt aus den beobachteten Messdaten ableitet – unabhängig von der internen Funktionsweise der Geräte. Möglich wird das durch quantenmechanische Bell-Tests, mit denen gezeigt werden kann, dass echte Nichtlokalität vorliegt und somit ein Abhören ausgeschlossen ist. Damit ist DI-QKD theoretisch besonders robust – aber experimentell sehr anspruchsvoll.
 (Scarani et al., 2009, Abschn. V.C)

\vspace{1em}
\noindent\textbf{Sicherheitsprinzipien und Angriffsmodelle}

\noindent
Die Sicherheit von QKD beruht auf fundamentalen Prinzipien der Quantenmechanik. Ein zentrales Element ist die Messstörung: Wird ein Quantenobjekt – etwa ein Photon im BB84-Protokoll – vom Angreifer gemessen, verändert sich sein Zustand. Dadurch entstehen Fehler in der Bitfolge, die von den Kommunikationspartnern erkannt werden können. Zudem verhindert das No-Cloning-Theorem, dass ein unbekannter Quantenzustand exakt kopiert werden kann. Damit ist es einem Angreifer prinzipiell unmöglich, eine perfekte Kopie der übertragenen Information zu erstellen.
 (Scarani et al., 2009)
Trotz dieser theoretischen Sicherheit existieren Angriffsmodelle, die Schwächen im Protokoll oder im Ablauf ausnutzen – etwa durch Ausnutzung falsch gewählter Basen oder durch die Manipulation von Messprozessen, z. B. durch das sogenannte Basis-Guessing. Auch Detektor-Manipulationen werden diskutiert, sofern Annahmen über das ideale Verhalten der Geräte nicht erfüllt sind.
 (Scarani et al., 2009)
Insgesamt gilt QKD als informationstheoretisch sicher, sofern alle theoretischen Annahmen erfüllt sind. In der Praxis jedoch hängt die tatsächliche Sicherheit stark von der korrekten Implementierung und der Kontrolle externer Fehlerquellen ab. Die Diskrepanz zwischen Theorie und Realität ist daher ein aktives Forschungsfeld, insbesondere im Hinblick auf Standards, Geräteverifikation und Modellierung von Angriffen.
 (Mosca, 2018)


\subsubsection{Quantum-Random-Number-Generator}
\noindent\textbf{Einleitung und Motivation}

\noindent
Ein Quantum Random Number Generator (QRNG) erzeugt echte Zufallszahlen auf Basis quantenmechanischer Prozesse. Im Gegensatz zu klassischen Zufallszahlengeneratoren, die deterministisch aus einem Startwert („Seed“) arbeiten, ist das Ergebnis bei QRNGs fundamental unvorhersagbar, da es nicht kausal vorherbestimmt ist.
 (Herrero-Collantes & Garcia-Escartin, 2017, Kap. I)
Pseudo-Random Number Generators (PRNGs) erzeugen zwar scheinbar zufällige Folgen, doch sind diese vollständig durch ihren Startwert bestimmt. Wird dieser bekannt oder erraten, lässt sich die gesamte Folge rekonstruieren – ein kritisches Risiko etwa bei Schlüsselgenerierung oder Challenge-Response-Verfahren.
 (NIST SP 800-90B, 2018, Sec. 3.1)
QRNGs bieten hier einen klaren Vorteil: Da sie auf quantenphysikalischer Messung beruhen – etwa bei der Entscheidung, ob ein Photon reflektiert oder transmittiert wird – entsteht das Ergebnis erst im Moment der Messung. Selbst der Hersteller kann die Bitfolge nicht vorhersagen, was sie einzigartig und nicht rekonstruierbar macht.
 (Ma et al., 2016, S. 2)
In der Kryptographie ist echter Zufall essenziell. Wird ein Schlüssel durch einen schwachen Zufallsprozess erzeugt, kann er unter Umständen rekonstruiert werden. Ein bekanntes Beispiel ist der Debian OpenSSL-Bug (2006–2008), bei dem fehlerhafte Zufallszahlen zu wiederverwendbaren Schlüsseln führten. QRNGs vermeiden solche Risiken, da sie eine verlässliche, nicht-reproduzierbare Entropiequelle liefern.
 (Herrero-Collantes & Garcia-Escartin, 2017, Kap. III)


\vspace{1em}
\noindent\textbf{Einsatzmöglichkeiten in der Kryptographie}

\noindent
QRNGs finden vor allem Anwendung in zwei Bereichen:
\smallskip
In Quantum Key Distribution (QKD): Hier dienen sie zur Erzeugung der zufälligen Bitfolgen, die später verschlüsselt oder verglichen werden. Da die Sicherheit von QKD-Protokollen wie BB84 maßgeblich von der Qualität des Zufalls abhängt, sind QRNGs hier ein kritischer Bestandteil. Ohne echten Zufall wäre z. B. ein Angriff durch Wiederverwendung von Schlüsseln möglich.
\smallskip
In klassischen kryptographischen Protokollen: Auch in herkömmlichen Systemen wie TLS, S/MIME oder SSH kann ein QRNG eine sichere Hardware-Entropiequelle für einen Pseudozufallszahlengenerator (PRNG) sein. Typischerweise wird das QRNG genutzt, um einen PRNG zu „seeden“, also mit einer Startentropie zu versorgen. So wird das Beste aus beiden Welten kombiniert: Echter Zufall + hohe Geschwindigkeit durch PRNGs.
\smallskip
(NIST SP 800-90B, 2018, Sec. 3.1)
Auch moderne Hardware-Sicherheitsmodule (HSMs) und Smartcards integrieren zunehmend QRNGs – sowohl zur Erhöhung der kryptographischen Sicherheit als auch zur Einhaltung regulatorischer Standards. In sensiblen Sektoren wie der Finanzbranche oder beim Militär ist eine verifizierbare Quelle für echten Zufall mittlerweile eine Anforderung.
(Sanguinetti et al., 2014)


\vspace{1em}
\noindent\textbf{Abgrenzung und Rolle in der Kryptographie}

\noindent
Quantum Random Number Generators (QRNGs) sind zentrale Bausteine sicherer Systeme, jedoch keine vollständigen Kryptosysteme. Ihre Aufgabe besteht ausschließlich in der Erzeugung echter, nicht vorhersagbarer Zufallszahlen – sie übernehmen weder Verschlüsselung noch Authentifikation. Als Grundlagentechnologie ersetzen oder ergänzen sie klassische Zufallsquellen wie Pseudozufallszahlengeneratoren (PRNGs).
 (Herrero-Collantes & Garcia-Escartin, 2017, Kap. IV)
QRNGs werden meist mit anderen Verfahren kombiniert. In der Quantenkryptographie, etwa bei Quantum Key Distribution (QKD), liefern sie die nötigen Zufallsbits. Auch in klassischen Anwendungen wie TLS oder VPNs dienen sie als sichere Entropiequelle zur Initialisierung von PRNGs. Moderne Hardware wie Security Modules (HSMs) oder Smartcards nutzt QRNGs zunehmend zur Stärkung kryptographischer Prozesse und zur Einhaltung regulatorischer Standards.
 (Ma et al., 2016, S. 4; Sanguinetti et al., 2014)
Trotz dieser Vorteile sind QRNGs nicht automatisch sicher. Bei kompromittierter oder fehlerhafter Hardware kann ein Angreifer Einfluss nehmen. Ma et al. (2016) unterscheiden daher drei Vertrauensmodelle: praktische QRNGs, die voll auf ihre Hardware angewiesen sind; semi-self-testing QRNGs mit interner Überprüfung; und fully self-testing QRNGs, bei denen quantenmechanische Tests (z. B. Bell-Tests) die Korrektheit unabhängig von der Gerätevertrauenswürdigkeit belegen. Letztere, auch als Device-Independent QRNGs (DI-QRNGs) bekannt, gelten als langfristiges Ziel kryptographischer Forschung.
 (Ma et al., 2016, S. 5–6)
QRNGs bieten somit das Potenzial für neue, robuste Sicherheitssysteme – ihre Wirksamkeit entfaltet sich jedoch nur im Zusammenspiel mit weiteren kryptographischen Komponenten. Ihre korrekte Einbindung ist entscheidend für die Integrität moderner Sicherheitsarchitekturen.


\vspace{1em}
\noindent\textbf{Funktionsprinzipien von QRNG}

\noindent
Die Funktionsweise von Quantum Random Number Generators (QRNGs) beruht auf einem fundamentalen Prinzip der Quantenmechanik: Das Ergebnis einer Messung ist nicht im Voraus bestimmbar – es entsteht erst im Moment der Messung. Genau dieses Verhalten macht sich die QRNG-Technologie zunutze, um echte, nicht reproduzierbare Zufallszahlen zu erzeugen.
 (Herrero-Collantes & Garcia-Escartin, 2017, Kap. II)
Ein klassisches Beispiel für dieses Prinzip ist der Strahlteiler-Ansatz (Beam Splitter), bei dem einzelne Photonen auf einen halbtransparenten Spiegel treffen. Das Photon wird dabei entweder reflektiert oder transmittiert – mit einer Wahrscheinlichkeit von jeweils 50 Prozent. Welchen der beiden Wege es nimmt, ist nicht kausal festgelegt, sondern ergibt sich erst durch die Messung an einem der beiden Detektoren. Jeder Detektor steht für eine Bitentscheidung („0“ oder „1“) – und damit entsteht ein echter Zufallswert.
 (Jennewein et al., 2000)
Ein weiteres praktisches Prinzip ist die Nutzung von Laserrauschen – also der zufälligen Phasenschwankungen von Licht, die bei der Emission in einem Laser auftreten. Diese Schwankungen können mit optischen Detektoren erfasst und in digitale Bitfolgen umgewandelt werden. Auch hier ist das Messergebnis nicht vorhersagbar und basiert auf quantenmechanischen Effekten.
(Ma et al., 2016, Abschn. „Practical architectures“)
Neben diesen beiden Grundtypen existieren noch weitere physikalische Realisierungen, die auf unterschiedlichen Quellen quantenmechanischer Unbestimmtheit beruhen: etwa das Quantentunnelungsrauschen in Halbleiterbauelementen, Phasenrauschen in Oszillatoren, oder Spontanemission in optischen Verstärkern. Die konkrete Umsetzung hängt dabei von der jeweiligen Anwendung ab – ob hohe Geschwindigkeit, Miniaturisierung oder maximale Vertrauenswürdigkeit im Vordergrund steht.
(Herrero-Collantes & Garcia-Escartin, 2017, Kap. II & III)
Allen Ansätzen gemeinsam ist, dass sie im Unterschied zu klassischen Zufallsgeneratoren nicht deterministisch sind. Es ist physikalisch unmöglich, das Ergebnis vorherzusagen – selbst mit vollständigem Wissen über den Aufbau des Systems. Genau das macht QRNGs zu einer so wertvollen Komponente in sicherheitskritischen Bereichen der Kryptographie.


\vspace{1em}
\noindent\textbf{Anforderungen an sichere QRNGs}

\noindent
Ein Quantum Random Number Generator (QRNG) muss für den Einsatz in kryptographischen Systemen spezifische Anforderungen erfüllen. Diese betreffen nicht nur die physikalische Qualität der Zufallszahlen, sondern auch ihre Testbarkeit und Vertrauenswürdigkeit. Vier zentrale Kriterien sind entscheidend.
 (Herrero-Collantes & Garcia-Escartin, 2017, Kap. II; Ma et al., 2016, S. 2–3)
Erstens: Nicht-Vorhersagbarkeit. Anders als bei PRNGs lässt sich bei QRNGs das Ergebnis nicht rekonstruieren – selbst mit vollständigem Wissen über Aufbau und Zustand. Die Ursache liegt in der quantenmechanischen Superposition: Ein Ergebnis entsteht erst im Moment der Messung, z. B. wenn ein Photon zufällig reflektiert oder transmittiert wird.
 (Herrero-Collantes & Garcia-Escartin, 2017, Kap. II; Ma et al., 2016, S. 2–3)
Zweitens: Zufälligkeitstestbarkeit. Um die Gleichverteilung und Unabhängigkeit der erzeugten Bitfolgen zu prüfen, werden standardisierte statistische Tests eingesetzt – etwa die NIST Statistical Test Suite (STS) oder Dieharder. Ein QRNG muss diese regelmäßig bestehen, um weiterbetrieben werden zu dürfen.
 (NIST SP 800-90B, 2018, Sec. 4)
Drittens: Entropiequellenanalyse. Die Entropie – also der tatsächlich enthaltene Zufall – muss ausreichend hoch und gegen technische Störungen abgesichert sein. Systeme müssen in der Lage sein, die effektive Entropie zu schätzen und im Zweifel zu verwerfen. Diese Prüfung ist Teil vieler Zertifizierungen, etwa FIPS 140-3.
 (Ma et al., 2016; NIST SP 800-90B, Abschn. 5.2)
Viertens: Device Independence. Ideal ist ein QRNG, das auch dann sichere Zufallszahlen liefert, wenn die Hardware nicht vertrauenswürdig ist. Dies wird durch quantenmechanische Tests wie Bell-Tests erreicht, die Unabhängigkeit von internen Annahmen garantieren. Solche Geräte sind technisch anspruchsvoll, aber ein zentrales Ziel künftiger Entwicklungen.
 (Ma et al., 2016, Abschnitt „Classification of QRNGs“)
 
\subsection{Aktuelle Anwendungsprojekte}
\subsubsection{Post-Quantum-Cryptography}
\cite{}
Im Rahmen der Post-Quantum-Kryptographie-Standardisierung verfolgt das NIST das Ziel, kryptographische Algorithmen zu identifizieren, die auch gegenüber zukünftigen Quantenangriffen resistent sind. Die vierte Runde des PQC-Prozesses hat zur Auswahl von Verfahren für Public-Key-Verschlüsselung sowie digitale Signaturen geführt. Im Fokus stehen insbesondere die Verfahren CRYSTALS-Kyber, CRYSTALS-Dilithium, and SPHINCS+, welche bereits breite Beachtung in industriellen Pilotprojekten und Implementierungsstudien gefunden haben.

CRYSTALS-Kyber ist ein auf dem Module-Learning-with-Errors (Module-LWE)-Problem basierendes Verfahren zur sicheren Schlüsselaushandlung. Das Design legt besonderen Wert auf eine effiziente Implementierbarkeit über verschiedene Plattformen hinweg, darunter auch ressourcenbeschränkte Systeme wie eingebettete Geräte. Ein wesentliches Merkmal ist die Resistenz gegen Seitenkanalangriffe, insbesondere durch eine auf konstante Laufzeit ausgelegte Referenzimplementierung (vgl. NIST IR 8545, S. 4; kyber blog). Kyber wurde frühzeitig in praktische Anwendungen integriert. So testete Google gemeinsam mit Cloudflare den Algorithmus im Rahmen einer hybriden TLS-1.3-Implementierung, um dessen Praxistauglichkeit im Internetverkehr zu evaluieren. Auch Amazon Web Services (AWS) integrierte Kyber im Key Management Service als experimentellen Post-Quantum-Kandidaten \cite{sullivan_securing_2020, weibel_round_2020}.

Für digitale Signaturen wurde mit CRYSTALS-Dilithium ein weiteres Verfahren aus der CRYSTALS-Familie standardisiert. Dilithium basiert auf der kombinatorischen Härte des Short Integer Solution (SIS)- und des Module-LWE-Problems und verwendet das Fiat-Shamir with Aborts-Paradigma. Ein wesentliches Designziel ist die einfache und sichere Implementierbarkeit, insbesondere durch Verzicht auf zustandsbehaftete Operationen und durch Nutzung gleichmäßig verteilter Zufallswerte anstelle diskreter Gauss-Verteilungen, die als anfällig gegenüber Seitenkanalangriffen gelten (dilithium paper S. 3–5). Die Signaturen werden entweder deterministisch oder randomisiert erzeugt; Letzteres erhöht die Sicherheit in adversen Umgebungen, in denen Seitenkanalangriffe drohen. Zur Performanceoptimierung sind Implementierungen mit AVX2- und AES-Unterstützung verfügbar, die deutliche Geschwindigkeitsvorteile gegenüber Referenzimplementierungen zeigen (dilithium paper S. 8–9). Die Sicherheitsnachweise erfolgen sowohl im klassischen Random-Oracle-Modell (ROM) als auch im Quantum-ROM (QROM) unter Annahme der Härte der Module-LWE- und SIS-Probleme \cite[S. 6-7]{schwabe_dilithium_nodate}.

Das Verfahren SPHINCS+ verfolgt im Gegensatz zu Kyber und Dilithium einen hashbasierten Ansatz und setzt ausschließlich auf die Sicherheit kryptographischer Hashfunktionen. Dies ermöglicht ein besonders konservatives Sicherheitsmodell, das selbst unter Annahme fortgeschrittener algebraischer Quantenalgorithmen tragfähig bleibt \cite[s. 1-2]{schwabe_sphincs_2025}. Ein Alleinstellungsmerkmal von SPHINCS+ ist das stateless Design, welches im Gegensatz zu klassischen, zustandsbehafteten Hash-basierten Verfahren die Gefahr von Schlüsselverlusten durch fehlerhafte Zustandsverwaltung eliminiert. Die Signaturerzeugung basiert auf der Kombination mehrerer kryptographischer Bausteine, darunter Winternitz-One-Time-Signatures, Merkle-Bäume und HORST (eine Few-Time-Signature-Variante). Die daraus resultierenden Signaturen sind mit etwa 41 KB vergleichsweise groß, jedoch bieten sie hohe Sicherheit und Resistenz gegen eine breite Klasse von Angriffen, inklusive solcher unter Einbeziehung von Quantenrechnern \cite[S. 4-5]{schwabe_sphincs_2025}. Aufgrund dieser Eigenschaften empfiehlt sich SPHINCS+ insbesondere für Anwendungen mit langfristiger Vertrauenswürdigkeit, wie etwa Firmware-Signaturen oder die Verifikation kritischer Softwarekomponenten.

Die Auswahl und Standardisierung dieser Verfahren wurde von NIST sowohl unter Berücksichtigung mathematischer Sicherheitsannahmen als auch praktischer Implementierbarkeit und Performance getroffen. Insbesondere wurde ein Augenmerk auf Diversität der zugrundeliegenden Problemklassen gelegt, um im Sinne der Kryptoagilität eine robuste Sicherheitsarchitektur gegenüber zukünftigen algorithmischen Durchbrüchen zu gewährleisten \cite[S. 4, 9]{alagic_status_2025}. CRYSTALS-Kyber und CRYSTALS-Dilithium basieren auf Gitterproblemen, während SPHINCS+ auf die Sicherheit von Hashfunktionen setzt.

Die ausgewählten Algorithmen sind mittlerweile Bestandteil offizieller NIST-Standards: CRYSTALS-Kyber als ML-KEM in FIPS 203, CRYSTALS-Dilithium als ML-DSA in FIPS 204 und SPHINCS+ als SLH-DSA in FIPS 205. Erste industrielle Anwendungen und Pilotprojekte zeigen eine zunehmende Integration in sicherheitskritische Kommunikationsprotokolle und Cloud-Infrastrukturen \cite{alagic_status_2025, sullivan_securing_2020, weibel_round_2020}. Insgesamt stellt die NIST-Standardisierung eines der zentralen Anwendungsprojekte im Bereich PQK dar und liefert eine belastbare Grundlage für die künftige Migration kryptographischer Systeme in das Post-Quantum-Zeitalter.

\subsubsection{Quantum-Key-Distribution}

Die Quantum Key Distribution (QKD) stellt eine der derzeit technologisch am weitesten entwickelten Anwendungen der Quantenkommunikation dar. Ihr Ziel ist die abhörsichere Verteilung kryptographischer Schlüssel basierend auf quantenmechanischen Prinzipien. Aufgrund fundamentaler physikalischer Eigenschaften erlaubt QKD die Erkennung von Abhörversuchen und bietet damit in bestimmten Szenarien informationstheoretische Sicherheit. Dennoch bestehen Herausforderungen hinsichtlich Reichweite, Infrastrukturkosten und Interoperabilität mit bestehenden Kommunikationssystemen. Zwei bedeutende internationale Anwendungsprojekte verdeutlichen die verschiedenen strategischen Ansätze im Umgang mit diesen Herausforderungen: das chinesische Satellitenprogramm um den Micius-Satelliten sowie die europäische EuroQCI-Initiative.

Im Jahr 2016 startete China mit dem Satelliten \cite{courtland_chinas_2016} das erste operative Weltraumexperiment zur quantenmechanischen Schlüsselverteilung \cite{courtland_chinas_2016}. Der in einer Umlaufbahn von ca. 500~km operierende LEO-Satellit demonstrierte die Realisierbarkeit satellitengestützter QKD zwischen weit entfernten Bodenstationen und legte damit den Grundstein für eine globale Quantenkommunikationsarchitektur \cite{wang_modeling_2021}. Im Vergleich zu optischen Glasfaserverbindungen, deren Reichweite durch Dämpfung und Fehlerraten physikalisch limitiert ist, ermöglicht die freie Ausbreitung im Vakuum des Alls verlustärmere Verbindungen über mehrere tausend Kilometer.

Parallel zur Satellitenentwicklung wurde ein nationales QKD-Netz realisiert, das sich von Beijing bis Shanghai über ca. 2000~km erstreckt und 32 terrestrische Knoten umfasst, die als trusted nodes operieren \cite{wang_modeling_2021}. Diese Architektur wirft jedoch grundlegende Fragen hinsichtlich Vertrauen und Skalierbarkeit auf, da kompromittierte Knoten die Gesamtsicherheit untergraben können. Obwohl sie aus technologischer Sicht einfacher zu realisieren sind als quantenmechanische Repeater, bleibt ihre Eignung für hochsichere Anwendungen begrenzt. Aus diesem Grund rückt in China zunehmend die Entwicklung von Satellitenkonstellationen mit Inter-Satellite-Links (ISLs) in den Fokus, die eine redundante und dynamisch steuerbare Verteilung von Schlüsseln ermöglichen sollen \cite{wang_modeling_2021}.

Die Europäische Union verfolgt mit der EuroQCI den Aufbau einer sicheren paneuropäischen Quantenkommunikationsarchitektur. Ziel ist ein hybrides Netz, das terrestrische QKD-Übertragungen über Glasfaser mit satellitengestützten Komponenten kombiniert \cite{noauthor_european_2020}. Die Initiative umfasst dabei sowohl nationale Quanten-Backbones als auch grenzüberschreitende Knoten und wird in enger Kooperation mit der European Space Agency (ESA) umgesetzt.

Ein zentrales Teilprojekt ist der Satellit \cite{noauthor_eagle-1_2025}, der ab 2024 getestet werden soll. Das Vorhaben zielt auf die Demonstration von QKD via Satellit im Kontext europäischer Sicherheitsinfrastrukturen und wurde speziell auf regulatorische und industrielle Bedarfe innerhalb der EU abgestimmt (eagle1). Im Gegensatz zum chinesischen Micius-Satelliten handelt es sich bei Eagle-1 explizit um ein technologievalidierendes Pilotprojekt, das vor allem Schnittstellen und Interoperabilität mit terrestrischen Netzen erprobt.

Erste Systemdesignstudien wurden unter anderem vom Institut de Ciències Fotòniques (ICFO) in Spanien koordiniert und adressieren die technischen Herausforderungen von Synchronisation, Detektion und Netzwerkmanagement im Kontext von Low Earth Orbit Systemen \cite{van_deventer_towards_2022}. Ein interessantes Detail ist die geplante Modularität des EuroQCI-Systems, das eine sukzessive Erweiterung über Mitgliedsstaaten hinweg vorsieht, statt eine zentralistische Netzarchitektur zu forcieren.

Im direkten Vergleich zeigt sich, dass China auf Demonstrationen mit operationellem Charakter und nationaler Souveränität fokussiert, während die europäischen Aktivitäten stärker kooperativ, regulativ eingebettet und netzarchitektonisch offen angelegt sind. Beide Strategien erscheinen unter ihren jeweiligen geopolitischen und technologischen Rahmenbedingungen plausibel. Eine kritische Reflexion muss jedoch auch die langfristige Wartbarkeit, Standardisierung und Kostenstruktur berücksichtigen. Gerade hier wird sich zeigen, ob der europäische Ansatz mit seiner Betonung auf Interoperabilität und systemischer Integration über den Prototypenstatus hinaus skaliert werden kann.

\subsubsection{Quantum-Random-Number-Generator}

Während großskalige Quantenkommunikationssysteme wie QKD-Netzwerke in der Regel erhebliche infrastrukturelle Investitionen und spezifische physikalische Rahmenbedingungen erfordern, lassen sich Komponenten wie Quantum Random Number Generators (QRNG) bereits heute in einer Vielzahl bestehender Systeme integrieren. QRNG fungiert dabei als grundlegende Bausteintechnologie in kryptographischen Anwendungen, indem es echte, nicht-deterministische Zufallszahlen auf Basis quantenmechanischer Prozesse erzeugt. Die Kommerzialisierung dieser Technologie erfolgt primär in Form eigenständiger Hardware-Module oder integrierbarer Chips, die sich durch geprüfte Entropiequellen, regulatorische Konformität und standardisierte Schnittstellen auszeichnen. Im Folgenden werden vier aktuelle Anwendungsprojekte vorgestellt, die unterschiedliche Implementierungsstrategien, Zielmärkte und technologischen Reifegrade illustrieren.

ID Quantique entwickelt seit mehreren Jahren miniaturisierte QRNG-Chips, die sich insbesondere für eingebettete Systeme, mobile Endgeräte und IoT-Infrastrukturen eignen. Die Quantis-QRNG-Chips basieren auf der quantenmechanischen Messung von Schussrauschen eines lichtempfindlichen Elements. Konkret wird dabei ein Verstärkerschaltkreis zur Erfassung der Quantenfluktuationen eines Photodetektors verwendet, aus denen durch nachgeschaltete Entropieextraktion bitweise Zufallsdaten generiert werden \cite{noauthor_quantis_2025}. Die verwendeten Verfahren stellen sicher, dass die erzeugten Bits nicht durch klassische Prozesse vorherbestimmt oder rekonstruierbar sind. Ziel ist eine kontinuierliche, rückführbare und stromsparende Generierung echter Zufallszahlen in Anwendungen, bei denen weder hohe Datenraten noch externe optische Komponenten verfügbar sind – etwa in Secure Elements oder als Entropiequelle in Embedded Devices. Die Chips sind für industrielle Zertifizierungsprozesse vorbereitet und werden mittlerweile in verschiedenen Mobilplattformen evaluiert, beispielsweise in vernetzten Fahrzeugkomponenten und sicherheitskritischen Sensornetzwerken.

Ein zentraler Treiber für die Akzeptanz und Skalierbarkeit von QRNG ist die Standardisierung der zugrundeliegenden Technologien. Im Rahmen des ETSI Industry Specification Group QKD arbeitet ein Konsortium unter Beteiligung von ID Quantique an der Formulierung technischer Spezifikationen für QRNG-Systeme. Die aktuelle Spezifikation GS QKD 014 legt Kriterien zur Charakterisierung von QRNG-Komponenten fest, insbesondere hinsichtlich Modellierung der quantenmechanischen Entropiequelle, Entropieextraktion, Testbarkeit und Systemarchitektur \cite{curran_idq_2023}. Zusätzlich wird zwischen „strong quantum“, „quantum-origin“, und „pseudo-random“ unterschieden, je nachdem, ob die Zufallsquelle rein quantenbasiert, hybrid oder deterministisch ist \cite{van_deventer_towards_2022}. Diese Kategorisierung dient nicht nur der Interoperabilität, sondern adressiert auch die Zertifizierbarkeit und Sicherheitsklassifikation in kritischen Infrastrukturen. Die ETSI-Initiative schlägt darüber hinaus vor, QRNGs künftig als definierte Komponenten in Public Key Infrastructures (PKI) zu integrieren, etwa zur Schlüsselinitialisierung oder als Seed für deterministische Algorithmen in hybriden Kryptosystemen.

Die von Toshiba entwickelte USB-QRNG-Serie richtet sich an Anwendungen in professionellen Desktop- und Rechenzentrumsumgebungen. Die Geräte nutzen eine optische Methode, bei der einzelne Photonen auf einen Strahlteiler gelenkt und anschließend detektiert werden. Die Entscheidung, ob ein Photon auf Detektor A oder B trifft, basiert auf fundamentaler Quantenunsicherheit, wodurch Bitfolgen mit maximaler Entropie erzeugt werden \cite{toshiba_europe_cambridge_research_laboratory_quantum_2025}. Der Vorteil dieser Architektur liegt in der unmittelbaren physikalischen Rückführbarkeit der Zufälligkeit sowie in der einfachen Integration über eine standardisierte USB-Schnittstelle. Die Bitraten liegen im Bereich einiger Mbit/s und eignen sich damit sowohl zur Initialisierung kryptographischer Prozesse (z.B. RSA-Schlüsselerzeugung, TLS-Handshakes) als auch als Entropiequelle für Betriebssysteme oder Sicherheitsmodule. Toshiba hebt besonders die Resistenz gegenüber externen Einflussgrößen hervor, da die Detektion in einem abgeschirmten optischen System erfolgt. Die Geräte wurden unter anderem in der britischen Post-Quantum-Initiative evaluiert und werden derzeit auch in Kombination mit PQC-Algorithmen getestet.

Ein umfassender Überblick über aktuelle und potenzielle QRNG-Anwendungen findet sich im Bericht „QRNG Report 2021“ von evolutionQ. Dort wird deutlich, dass sich die Einsatzfelder weit über klassische Kryptographie hinaus erstrecken. QRNG wird beispielsweise in der Tokenisierung von Finanztransaktionen, bei der Erzeugung von Seeds für Blockchain-Schlüsselpaare oder in der Protokollinitialisierung sicherer Multi-Party-Computing-Systeme eingesetzt \cite{??????} <- Piani et al. Besonders hervorgehoben wird die Rolle von QRNG in virtualisierten und containerisierten Cloud-Infrastrukturen, bei denen klassische Entropiequellen wie /dev/random nicht hinreichend isoliert oder manipulationssicher sind. QRNG kann hier als dedizierter Hardware-Entropieprovider über PCIe, USB oder Netzwerkprotokolle eingebunden werden. Ein weiteres Anwendungsfeld ist die Verwendung in sicherheitskritischer Hardware wie Hardware Security Modules (HSMs), wo QRNG zur Generierung nicht reproduzierbarer Einmalschlüssel beiträgt. Laut evolutionQ liegt der entscheidende Vorteil in der messbaren Entropieherkunft, die sowohl regulatorischen Anforderungen (z.B. eIDAS, FIPS) als auch zukünftigen Anforderungen an Post-Quantum-Resilienz gerecht wird.

Die vorgestellten Anwendungsprojekte zeigen, dass QRNG nicht mehr als isolierte Labortechnologie zu betrachten ist, sondern sich zunehmend als integraler Bestandteil moderner Sicherheitsarchitekturen etabliert. Während sich einzelne Implementierungen – etwa in Form von USB-Geräten oder Chips – technisch stark unterscheiden, eint sie das Ziel, vertrauenswürdige und physikalisch nachvollziehbare Zufallsquellen für kryptographische Kernfunktionen bereitzustellen. In der Gesamtschau ergibt sich ein technologieoffenes, aber klar sicherheitsgetriebenes Innovationsfeld: QRNG schließt eine zentrale Lücke in der Kette quantensicherer Systeme, indem es nicht nur den algorithmischen Teil, sondern auch die zugrunde liegende Entropiequelle absichert. Damit kommt dieser Technologie eine Schlüsselrolle im Übergang von klassischen zu postquantenresilienten Sicherheitsinfrastrukturen zu – und zwar nicht nur auf konzeptioneller, sondern bereits auf operativer Ebene.

\subsection{Fazit}
%Zusammenfassung:

%Bewertung des Standes:

%Zukunftsausblick:

%Fazit/Abschluss:

\printbibliography
