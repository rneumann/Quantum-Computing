%\motto{Use the template \emph{chapter.tex} to style the various elements of your chapter content.}
\chapter{Grundlegende Quantenalgorithmen}
\label{basic_algorithms} % Always give a unique label
% use \chaptermark{}
% to alter or adjust the chapter heading in the running head

\chapterauthor{Vladimir Alyoshin, Gian-Luca Eberling}

\abstract*{some abstract}

\abstract{some abstract}

\section{Shor-Algorithmus: Faktorisierung und diskreter Logarithmus}
\label{first:shor-algorithm}
Die klassische Kryptografie basiert auf der Annahme, dass bestimmte mathematische Probleme – wie die Zerlegung großer Zahlen in Primfaktoren – selbst für moderne Supercomputer unlösbar bleiben. Genau hier setzt Shor’s Algorithmus an: 1994 entwickelte der Mathematiker Peter Shor einen Quantenalgorithmus, der dieses Problem effizient lösen kann. Damit stellte er nicht nur eine der zentralen Säulen klassischer Verschlüsselungsverfahren in Frage, sondern zeigte auch das enorme Potenzial von Quantencomputern für konkrete Anwendungen.\\

\noindent In diesem Kapitel erklären wir die Grundidee von Shor’s Algorithmus, zeigen, wie er die periodische Struktur von Funktionen mithilfe der Quanten-Fourier-Transformation nutzt, und machen deutlich, warum seine Effizienz klassisch nicht erreichbar ist. Ziel ist es, die mathematischen Grundlagen verständlich zu machen und Schritt für Schritt zur Quantenlösung des Faktorisierungsproblems zu führen.\\

\subsection{Mathematische Grundlagen}

\subsubsection*{Euklidischer Algorithmus}
Der Euklidische Algorithmus berechnet den größten gemeinsamen Teiler (ggT) zweier Zahlen effizient:
\[
\gcd(a, b) = \gcd(b, a \bmod b)
\]
Dies ist notwendig zur Validierung in Shor's Algorithmus und zur Auswahl der Basis \( a \).

\subsubsection*{Modulare Arithmetik}
In Shor's Algorithmus wird exponentielle modularer Arithmetik verwendet:
\[
f(x) = a^x \bmod N
\]
Die Funktion ist periodisch, was die Grundlage für die Periodenfindung darstellt.

\subsubsection*{Periodizität und Ordnung}
Die Ordnung \( r \) von \( a \bmod N \) ist die kleinste positive ganze Zahl, für die gilt:
\[
a^r \equiv 1 \mod N
\]
Diese Ordnung entspricht der Periode der Funktion \( f(x) \).

\subsubsection*{Kettenbruchentwicklung}
Um aus einer gemessenen Näherung \( j/Q \) eine rationale Zahl \( s/r \) zu finden, verwendet man die Kettenbruchentwicklung:
\[
\frac{j}{Q} \approx \frac{s}{r}
\]
Dabei wird ein Näherungsbruch mit möglichst kleinem Nenner gesucht.

\subsection{Idee: Primfaktorzerlegung}
Ziel ist es, eine zusammengesetzte Zahl \( N = pq \) effizient zu faktorisieren. Klassische Algorithmen benötigen dazu exponentielle Laufzeit. Shor's Algorithmus nutzt die Periodenstruktur der Funktion
\[
f(x) = a^x \bmod N
\]
um über einen quantenmechanischen Mechanismus – insbesondere die Quanten-Phasenschätzung (QPE) – die Periode \( r \) dieser Funktion effizient zu bestimmen.

\noindent Dabei kommt **Quanteninterferenz** als zentrales Werkzeug zum Einsatz: Im Quantenalgorithmus wird ein Superpositionszustand über viele mögliche Werte \( x \) erzeugt und durch die Funktion \( f(x) \) moduliert. Dies geschieht durch Anwendung einer Einheitär-Operation \( U_f \), die \( |x\rangle \mapsto |f(x)\rangle \) abbildet. Anschließend wird durch die Quanten-Phasenschätzung in Kombination mit einer inversen Quanten-Fourier-Transformation (iQFT) eine Interferenz zwischen den Zuständen herbeigeführt, die bestimmte periodische Anteile konstruktiv verstärkt und andere destruktiv auslöscht.\\

\noindent Diese Interferenz sorgt dafür, dass bei der anschließenden Messung mit hoher Wahrscheinlichkeit ein Ergebnis nahe einer rationalen Näherung \( s/r \) (mit \( r \) = Periode) beobachtet wird. Der Bruch \( s/r \) wird schließlich mithilfe der Kettenbruchentwicklung in den exakten Wert von \( r \) umgewandelt. Sobald \( r \) bekannt ist und gewisse Bedingungen erfüllt (insbesondere: \( r \) gerade und \( a^{r/2} \not\equiv -1 \bmod N \)), kann man die Teiler von \( N \) effizient mit dem Euklidischen Algorithmus bestimmen:\\
\[
\gcd(a^{r/2} \pm 1, N)
\]

\noindent So ermöglicht die **Quanteninterferenz**, zusammen mit der Fourier-Analyse, die entscheidende Extraktion der verborgenen Periodenstruktur, was die Grundlage der Faktorisierung bildet.


\subsection{Diskreter Logarithmus}
Shor’s Algorithmus kann auch auf das Problem des diskreten Logarithmus angewendet werden. Dieses besteht darin, für eine gegebene Gruppe \( G \), ein Element \( g \in G \) und ein weiteres Element \( h \in G \), die Zahl \( x \) zu finden, sodass:
\[
g^x = h
\]
Auch dieses Problem lässt sich durch Reduktion auf eine Periodensuche effizient mit einem Quantenalgorithmus lösen.

\subsection{Quanten Fourier Transformation}

\subsubsection*{Definition}
Die Quanten-Fourier-Transformation (QFT) auf einem \( n \)-Qubit-System transformiert einen Basiszustand \( \ket{x} \) wie folgt:
\[
\text{QFT}(\ket{x}) = \frac{1}{\sqrt{2^n}} \sum_{k=0}^{2^n-1} e^{2\pi i xk / 2^n} \ket{k}
\]
mit
\begin{itemize}
    \item \( n \): Anzahl der Qubits im Quantenregister, also die Größe des Systems.
    \item \( \ket{x} \): Ein Basiszustand des \( n \)-Qubit-Systems, wobei \( x \in \{0, 1, \ldots, 2^n - 1\} \) eine ganze Zahl ist, die den Basiszustand beschreibt.
    \item \( 2^n \): Die Dimension des Hilbertraums, also die Gesamtanzahl aller Basiszustände für das \( n \)-Qubit-System.
    \item \( k \): Laufindex in der Summe, der ebenfalls von 0 bis \( 2^n - 1 \) reicht und die möglichen Basiszustände nach der Transformation angibt.
    \item \( e^{2 \pi i x k / 2^n} \): Die komplexe Phase, die im Überlagerungszustand die Interferenzmuster erzeugt und so Frequenzinformationen kodiert.
\end{itemize}

\noindent Sie ist zentral für die Phasenerkennung bei der Periodensuche.

\subsubsection*{Komplexität}
Die QFT kann auf einem Quantencomputer effizient in \( \mathcal{O}(n^2) \) durchgeführt werden, im Gegensatz zur klassischen Fourier-Transformation mit \( \mathcal{O}(n 2^n) \).

\subsection{Quanten-Phasenabschätzung}

\noindent Die Quanten-Phasenabschätzung ermöglicht die Bestimmung der Phase \( \phi \) eines Eigenwerts \( e^{2\pi i \phi} \) einer unitären Matrix \( U \), wobei:
\[
U \ket{u} = e^{2\pi i \phi} \ket{u}
\]
Sie ist ein zentrales Werkzeug zur Ermittlung der Periode \( r \) im Shor-Algorithmus, da man damit die Periode indirekt über eine Phase ablesen kann.

\subsection{Ablauf des Shor-Algorithmus zur Faktorisierung}

\noindent
Ziel ist es, eine zusammengesetzte Zahl \( N = pq \) in ihre Primfaktoren \( p \) und \( q \) zu zerlegen. Dazu nutzt Shor's Algorithmus einen Quantencomputer zur Berechnung der Periode einer bestimmten Funktion. Der Algorithmus gliedert sich in einen klassischen und einen quantenmechanischen Teil.

\subsubsection*{Schritt 1: Auswahl einer zufälligen Basis}
Wähle eine ganze Zahl \( a \) mit:
\[
1 < a < N \quad \text{und} \quad \gcd(a, N) = 1
\]
Falls \( \gcd(a, N) > 1 \), wurde bereits ein echter Teiler von \( N \) gefunden.

\subsubsection*{Schritt 2: Reduktion auf das Periodenproblem}
Definiere die Funktion
\[
f(x) = a^x \bmod N
\]
Die Funktion \( f(x) \) ist periodisch mit einer Periode \( r \), sodass:
\[
a^r \equiv 1 \mod N
\]
Die Suche nach der kleinsten solchen Zahl \( r \) ist das zentrale Problem, das der Quantencomputer löst.

\subsubsection*{Schritt 3: Periodenfindung mit einem Quantencomputer}

\begin{enumerate}
  \item[3.1] \textbf{Initialisierung:}  
  Zwei Register werden initialisiert mit dem Zustand:
  \[
  \ket{0}^{\otimes n} \otimes \ket{1}
  \]

  \item[3.2] \textbf{Hadamard-Transformation:}  
  Auf das erste Register wird die Hadamard-Transformation angewendet, um eine Superposition zu erzeugen:
  \[
  \frac{1}{\sqrt{Q}} \sum_{x=0}^{Q-1} \ket{x} \otimes \ket{1}
  \]
  
  Dabei ist:
  \begin{itemize}
    \item \( n \): Anzahl der Qubits im ersten Register.
    \item \( Q = 2^n \): Anzahl der möglichen Zustände im ersten Register
    \item \( x \): Laufvariable über alle Basiszustände von \( \ket{0} \) bis \( \ket{Q-1} \).
    \item \( \ket{1} \): Anfangszustand des zweiten Registers, auf das später die modulare Exponentiation angewendet wird.
  \end{itemize}

  Durch die Hadamard-Gates wird das Register in eine Superposition gebracht, in der alle \( Q \) Basiszustände gleichwahrscheinlich sind. Diese Superposition ist notwendig, um später durch die Phaseninterferenz Information über die Periode der Funktion zu extrahieren.\\

  \item[3.3] \textbf{Modulare Exponentiation:}  
  Eine kontrollierte Operation berechnet:
  \[
  \ket{x} \otimes \ket{1} \rightarrow \ket{x} \otimes \ket{a^x \bmod N}
  \]

  \item[3.4] \textbf{Inverse Quanten-Fourier-Transformation (iQFT):}  
  Auf das erste Register wird die iQFT angewendet, um Periodeninformationen in der Amplitudenverteilung zu kodieren.\\

  \item[3.5] \textbf{Messung:}  
  Das erste Register wird gemessen und ergibt einen Wert \( j \), der näherungsweise gilt:
  \[
  \frac{j}{Q} \approx \frac{s}{r}
  \]

  \item[3.6] \textbf{Klassische Kettenbruchentwicklung:}  
  Die rationale Näherung \( \frac{j}{Q} \approx \frac{s}{r} \) wird mithilfe einer Kettenbruchentwicklung bestimmt.  
  Dabei sucht man nach einem Bruch \( \frac{s}{r} \), der \( \frac{j}{Q} \) mit kleinem Nenner \( r \) gut approximiert.\\

  \item[3.7] \textbf{Validierung des Kandidaten:}  
  Überprüfe, ob:
  \[
  a^r \equiv 1 \mod N
  \]
  Falls nicht, versuche eine andere Näherung oder starte mit anderem \( a \) neu.
\end{enumerate}

\subsubsection*{Schritt 4: Klassischer Abschluss}
Wenn eine gültige Periode \( r \) gefunden wurde (und \( r \) gerade ist), berechne:
\[
\gcd\left(a^{r/2} - 1,\; N\right) \quad \text{und} \quad \gcd\left(a^{r/2} + 1,\; N\right)
\]
Diese Werte liefern mit hoher Wahrscheinlichkeit echte Teiler von \( N \). Ist das nicht der Fall, beginnt der Algorithmus erneut mit einer anderen Basis \( a \).

\subsubsection*{Zusammenfassung des Verfahrens}
Der Shor-Algorithmus zeigt, wie Quantencomputer periodische Strukturen in Funktionen aufdecken können, die klassisch nur mit exponentiellem Aufwand zu finden wären. Die Kombination aus Quanten-Phasenabschätzung, Fourier-Transformation und Kettenbruchentwicklung ermöglicht eine effiziente Faktorisierung großer Zahlen – ein fundamentaler Angriff auf die Sicherheit klassischer Verschlüsselung.\\

\noindent Die Operationen des Shor's Algorithmus zur Faktorisierung lassen sich wie folgt darstellen:

\begin{algorithm}
\label{algorithm:shor}
\caption{Shor's Algorithmus zur Faktorisierung einer Zahl \( N \)}
\begin{algorithmic}[1]
\Require Eine zusammengesetzte Zahl \( N \) aus zwei unbekannten Primzahlen \( p \) und \( q \)
\Ensure Ein nicht-trivialer Teiler von \( N \)
\State Wähle zufällig eine ganze Zahl \( a \) mit \( 1 < a < N \)
\If{\( \gcd(a, N) \neq 1 \)}
    \State \Return \( \gcd(a, N) \) \Comment{Zufällig gewähltes \( a \) war schon ein Teiler}
\EndIf
\State Bestimme die Periode \( r \) der Funktion \( f(x) = a^x \bmod N \) mittels Quanten-Phasenschätzung:
    \begin{enumerate}
        \item Initialisiere zwei Register: Eins mit Superposition aller \( x \), eins mit Zustand \( |1\rangle \)
        \item Wende die Einheitäre Operation \( U_f \colon |x\rangle|1\rangle \mapsto |x\rangle|a^x \bmod N\rangle \) an
        \item Wende die inverse Quanten-Fourier-Transformation (iQFT) auf das erste Register an
        \item Miss das erste Register → erhalte Näherung \( s/r \)
        \item Berechne \( r \) durch Kettenbruchentwicklung
    \end{enumerate}
\If{\( r \) ungerade oder \( a^{r/2} \equiv -1 \bmod N \)}
    \State \Return \textbf{Fehlschlag, wähle anderes \( a \)} und wiederhole
\EndIf
\State Berechne \( \gcd(a^{r/2} \pm 1, N) \)
\State \Return Einer der berechneten Teiler ist eine Primzahl \( p \) oder \( q \)
\end{algorithmic}
\end{algorithm}

\subsection{Anwendung auf den RSA-Algorithmus}

\noindent
Der RSA-Algorithmus basiert auf der Schwierigkeit, eine große Zahl \( N = pq \) zu faktorisieren. Shor’s Algorithmus bricht diese Sicherheit, indem er in polynomialer Zeit die Faktoren \( p \) und \( q \) bestimmt. Hat ein Angreifer Zugriff auf einen leistungsfähigen Quantencomputer, kann er aus dem öffentlichen Schlüssel die privaten Schlüsselparameter rekonstruieren:
\begin{itemize}
  \item Berechne \( p \) und \( q \) mit Shor.
  \item Berechne \( \phi(N) = (p-1)(q-1) \).
  \item Berechne den privaten Schlüssel \( d \) durch:
  \[
  d \equiv e^{-1} \mod \phi(N)
  \]
\end{itemize}
Dies zeigt, dass RSA bei hinreichend großen Quantencomputern als unsicher gelten muss.

\subsection{Anwendung auf das diskrete Logarithmusproblem}

Shor’s Algorithmus kann nicht nur die Primfaktorzerlegung beschleunigen, sondern auch das diskrete Logarithmusproblem effizient lösen. Dieses Problem ist die Grundlage vieler kryptografischer Verfahren wie Diffie-Hellman oder ElGamal. \\

\noindent Das Ziel ist, für gegebenes \( a, b \) in einer Gruppe \( \mathbb{Z}_p^* \) den Exponenten \( x \) zu finden, sodass:
\[
a^x \equiv b \mod p
\]

\noindent Shor’s Algorithmus reduziert das Problem auf die Periodensuche einer geeigneten Funktion durch Quanten-Phasenabschätzung, ähnlich wie bei der Faktorisierung. Dabei wird aus der Überlagerung von Zuständen und anschließender Quanten-Fourier-Transformation die gesuchte Periode extrahiert, die Rückschlüsse auf den Exponenten \( x \) erlaubt.\\

\begin{algorithm}
\caption{Quantenalgorithmus zur Bestimmung des diskreten Logarithmus \( x \) aus \( g^x \equiv h \bmod p \)}
\begin{algorithmic}[1]
\Require Eine Primzahl \( p \), eine erzeugende Basis \( g \in \mathbb{Z}_p^* \) und ein Element \( h \in \mathbb{Z}_p^* \), sodass \( h = g^x \bmod p \)
\Ensure Der Exponent \( x \)
\State Setze eine ausreichend große Zahl \( Q = 2^n \) mit \( Q > p^2 \)
\State Bestimme die Periode \( r \) der Funktion \( f(x) = g^x \bmod p \) mittels Quanten-Phasenschätzung:
    \begin{enumerate}
        \item Initialisiere zwei Register: Erstes Register in Superposition über \( x = 0, \ldots, Q-1 \), zweites Register in Zustand \( |1\rangle \)
        \item Wende die unitäre Operation \( U_g \colon |x\rangle|1\rangle \mapsto |x\rangle|g^x \bmod p\rangle \) an
        \item Wende die inverse Quanten-Fourier-Transformation (iQFT) auf das erste Register an
        \item Miss das erste Register → erhalte eine Näherung des Bruchs \( s/r \)
        \item Berechne \( r \) mittels Kettenbruchentwicklung
    \end{enumerate}
\State Nutze \( r \), \( s \) und \( h \) mit klassischer Nachbearbeitung, um den diskreten Logarithmus \( x \) zu bestimmen
\State \Return \( x \)
\end{algorithmic}
\end{algorithm}


\section{Grover-Algorithmus: Quantum-Suche}
\label{sec:grover-algorithm}
In der Informatik zählt die Datenverarbeitung zu den gefragtesten Aufgaben. In der heutigen Zeit ist die Möglichkeit, große Datenmengen zu durchsuchen — insbesondere unstrukturierte Daten — ein wesentlicher Aspekt der Datenanalyse. Um dies zu veranschaulichen, betrachten wir folgendes Beispiel:\\ 

\noindent Angenommen, wir möchten einen bestimmten Kunden in der Datenbank eines Online-Shops finden (unter der Annahme, dass die Datenbank nicht indexiert ist). Bei Verwendung eines klassischen Computers benötigen wir im schlechtesten Fall $N$ Iterationen, wobei $N$ die Anzahl der Zeilen in der Tabelle darstellt. Im durchschnittlichen Fall sind etwa $N/2$ Operationen erforderlich. Das bedeutet, dass alle Elemente der Tabelle durchlaufen werden müssen, sofern uns die genaue Position des gesuchten Kunden nicht bekannt ist — bis dieser schließlich gefunden wird.\\

\noindent Wenn jedoch unsere Datenbank auf Quantenberechnungen basiert, können wir Quanten-Suchalgorithmen wie den Grover-Algorithmus einsetzen. In diesem Fall lässt sich das gesuchte Element mit etwa $\sqrt{N}$ Operationen finden. Zwar führt dieser Algorithmus nicht zu einer exponentiellen Beschleunigung der Suche und muss häufig mehrfach iterativ aufgerufen werden, dennoch stellt die quadratische Laufzeit bereits eine bedeutende Optimierung dar.

\subsection{Grundlagen: Unstrukturiertes Suchproblem}
Bevor wir uns dem Grover-Algorithmus zuwenden, betrachten wir folgendes Beispiel: Stellen wir uns vor, auf dem Tisch liegen vier Spielkarten verdeckt. Eine dieser vier Karten ist die Pik-Dame, und unsere Aufgabe besteht darin, sie zu finden. Wie viele Karten müssen wir im Durchschnitt aufdecken, um die Pik-Dame zu identifizieren? Im besten Fall finden wir die gesuchte Karte bereits beim ersten Versuch. Im schlechtesten Fall müssen wir drei Karten umdrehen — wenn keine von ihnen die Pik-Dame ist, dann bleibt nur noch die vierte, nicht aufgedeckte Karte als einzig mögliche Lösung. Im Durchschnitt müssen wir 2{,}25 Karten aufdecken, um die Pik-Dame zu finden.\\

\noindent Formulieren wir nun das Beispiel um: Anstelle von Spielkarten betrachten wir eine binäre Reihenfolge natürlicher Zahlen — nämlich $00$, $01$, $10$ und $11$. Nehmen wir an, die Pik-Dame entspricht dabei der Zahl $11$. Wir definieren eine Funktion $f$, die den Wert $1$ zurückgibt, wenn das Eingabemuster $11$ ist, und $0$ in allen anderen Fällen. Somit lässt sich das Problem wie folgt formal ausdrücken:

\begin{definition}[Unstrukturiertes Suchproblem]
Sei $f\colon \{0, 1\}^n \rightarrow \{0, 1\}$ eine abstrakte Funktion mit $f(x) \in \{0, 1\}$. Gesucht ist ein Wert $x$, sodass $f(x) = 1$, falls ein solcher Wert existiert; andernfalls soll das Ergebnis $0$ sein.
\end{definition}

\noindent Die Komplexität des unstrukturierten Suchproblems ergibt sich daraus, wie viele Iterationen der Funktion erforderlich sind, um das gesuchte Element zu finden. Wir benötigen im schlechtesten Fall $N - 1$ Funktionsauswertungen, wobei $N = 2^n$ gilt. Auf diese Weise prüfen wir alle möglichen Eingaben. Ähnlich wie im Beispiel mit den Karten entspricht $N$ dem letzten Element, vorausgesetzt, die vorherigen wurden bereits geprüft. Der Grover-Algorithmus kann dieses Problem jedoch deutlich schneller lösen und erreicht dabei eine quadratische Laufzeitverbesserung.\\

\noindent Jedoch repliziert der Grover-Algorithmus noch ein Suchproblem. Wie wir im Beispiel mit den Spielkarten gesehen haben, müssen wir Karte für Karte umdrehen, um zu prüfen, ob es die gesuchte Karte ist oder nicht. Dieser Prozess ist iterativ und wird durch das \textit{heuristische Suchproblem} beschrieben:

\begin{definition}[Heuristisches Suchproblem]
Gegeben sei die Möglichkeit, einen probabilistischen „Rate“-Algorithmus \( A \) auszuführen, und eine „Prüf“-Funktion \( f \), sodass

\[
\Pr\left[ A \text{ gibt } w \text{ aus, sodass } f(w) = 1 \right] = \varepsilon,
\]
\end{definition}

\noindent Eine Möglichkeit, das heuristische Suchproblem klassisch zu lösen, besteht darin, den Algorithmus \( A \) wiederholt auszuführen und jedes Ergebnis mit der Funktion \( f \) zu überprüfen. Dies führt zu durchschnittlich \( O\left(\frac{1}{\varepsilon}\right) \) Auswertungen von \( f \). In der nächsten Sektion werden wir sehen, wie der Grover-Algorithmus dieses Verfahren implementiert.

\subsection{Aufbau von dem Grover Algorithmus}

\noindent Um den Grover-Algorithmus vollständig zu verstehen, ist es wichtig, zunächst die grundlegenden Quantenoperationen zu begreifen. Lov Grover führte die folgende Definition für eine bestimmte Klasse von Operationen ein.\cite[1-2]{grover-placeholder}

\begin{definition}[Unitäre Operationen]
Quantenmechanische Operationen, die in kontrollierter Weise ausgeführt werden können, sind \emph{unitäre Operationen}, welche in jedem Schritt auf eine kleine Anzahl von Qubits wirken.
\end{definition}

\noindent Der Grover-Algorithmus basiert auf einer Folge solcher unitären Operationen, die auf einen reinen Anfangszustand angewendet werden. Diese Operationen dienen dazu, die Wahrscheinlichkeit des gesuchten Ergebnisses zu verstärken. Am Ende des Algorithmus erfolgt eine Messung des resultierenden Zustands. Diese Messung projiziert das überlagerte Quantensystem auf einen klassischen Zustand, der mit hoher Wahrscheinlichkeit die Lösung des zugrunde liegenden Suchproblems enthält.\\

\noindent Lov Grover definierte die folgenden unitären Operationen als zentrale Bestandteile seines Algorithmus:

\begin{enumerate}
    \item \textbf{Initialisierung in einen Superpositionszustand:} 
    Die erste Operation besteht darin, ein Quantenregister mit $n$ Qubits (entsprechend $N = 2^{n}$ möglichen Zuständen) in einen Zustand zu überführen, in dem alle Basiszustände mit der gleichen Wahrscheinlichkeit auftreten – also eine \textit{Gleichverteilung}. Die \textit{Gleichverteilung} wird dadurch erreicht, dass die Amplituden aller Zustände gleich sind. Wenn das System anfangs im Zustand $\ket{0}^{\otimes n}$ ist (alle Qubits auf 0), dann erzeugen wir durch Anwendung der Hadamard-Transformation auf jedes Qubit die Superposition:

$$
\ket{\psi_0} = \frac{1}{\sqrt{N}} \sum_{x=0}^{N-1} \ket{x}
$$

Dies ist die gleichmäßige Superposition über alle möglichen Zustände $\ket{x}$ mit $x \in {0, \ldots, N-1}$.\\
    \item \textbf{Walsch-Hadamard Transformation}: Hadamard Transformation ist eine wichtige Quantummechanische Operation, die durch eine Operation $H$ definiert ist. Diese Operation wird auf ein einzelnes Qubit angewendet und wird durch die folgende Matrix dargestellt:
    $$
H = \frac{1}{\sqrt{2}} \begin{pmatrix}
1 & 1 \\
1 & -1 \\
\end{pmatrix}
$$
Ein Qubit im Zustand \( \lvert 0 \rangle \) wird in eine Superposition der beiden Zustände überführt: \( \left( \frac{1}{\sqrt{2}}, \frac{1}{\sqrt{2}} \right) \). Die Wirkung auf die Basiszustände ist:

$$
H\ket{0} = \frac{1}{\sqrt{2}}(\ket{0} + \ket{1}), \quad H\ket{1} = \frac{1}{\sqrt{2}}(\ket{0} - \ket{1})
$$
Wendet man $H$ auf jedes der $n$ Qubits an, so entsteht eine Transformation $H^{\otimes n}$, die eine $2^n \times 2^n$-Matrix darstellt.
Besonders interessant ist, dass das Vorzeichen (also die Phase) des Amplitudenwertes nach der Hadamard-Transformation vom Skalarprodukt der Eingabe- und Ausgabebits abhängt:

$$
\text{Vorzeichen} = (-1)^{x \cdot y}
$$

Dabei ist $x \cdot y$ das bitweise Skalarprodukt der $n$-Bit-Binärzahlen $x$ und $y$.\\

    \item \textbf{Selektive Phasenrotation:} Die dritte elementare Operation ist die selektive Phasenrotation der Amplitude in bestimmten Zuständen. Die Transformation, die dies für ein Zwei-Zustands-System beschreibt, hat die Form:

    $$
R = \begin{pmatrix}
e^{i\phi_1} & 0 \\
0 & e^{i\phi_2}
\end{pmatrix}
$$

wobei \( \phi \) eine beliebige reelle Zahl ist. Es ist zu beachten, dass – im Gegensatz zur Walsh-Hadamard-Transformation und anderen Zustandsübergangsmatrizen – die Wahrscheinlichkeit in jedem Zustand gleich bleibt, da das Quadrat des Betrags der Amplitude in jedem Zustand unverändert bleibt.
Im Grover-Algorithmus wird die Phasenrotation speziell dafür genutzt, um den gesuchten Zustand $\ket{x\_{\text{target}}}$ zu markieren, indem dessen Amplitude mit $-1$ multipliziert wird, während alle anderen Zustände unverändert bleiben:

$$
\ket{x} \mapsto 
\begin{cases}
- \ket{x}, & \text{wenn } x = x_{\text{target}} \\
\ket{x}, & \text{sonst}
\end{cases}
$$

\end{enumerate}

\noindent Zusammenfassend lässt sich einfach sagen: Der Grover-Algorithmus verwendet diese Operationen, um pro Iteration zwei Schritte auszuführen:
\begin{enumerate}
  \item Vorzeichenumkehr der Wahrscheinlichkeitsamplitude des gesuchten Elements.
  \item Verstärkung der Wahrscheinlichkeitsamplitude.
\end{enumerate}

\noindent Mit dem Verständnis der oben genannten unitären Operationen lässt sich der Grover-Algorithmus durch die folgenden Schritte beschreiben:

\begin{algorithm}[H]
\caption{Grover-Suchalgorithmus}
\begin{algorithmic}[1]
\State Erzeuge ein Register aus \( n \) Qubits im Zustand \( \ket{0}^{\otimes n} \).
\State Wende die Hadamard-Operation \( H \) auf jedes Qubit an, um eine Superposition zu erzeugen.
\Repeat
  \State\textbf{Oracle:}
  \Statex \hspace{1em} Sei das System im Zustand \( \ket{x} \).
  \If{ \( f(x) = 1 \) }
    \State Rotiere die Phase von \( \ket{x} \) um \( \pi \) Radiant.
  \Else
    \State Belasse \( \ket{x} \) unverändert.
  \EndIf
  \State Wende die Grover-Diffusionstransformation $D$ an.
  \State Messe das Register zur Bestimmung des gesuchten Index.
\If{das Ergebnis ist keine gültige Lösung}
  \State Gehe zu Schritt 3 zurück.
\EndIf
\Until{eine gültige Lösung mit hoher Wahrscheinlichkeit gefunden wurde}
\end{algorithmic}
\end{algorithm}

\noindent Ein entscheidender Schritt in diesem Algorithmus ist die \textit{Grover-Diffusionstransformation}, die wesentliche Änderungen an den Amplituden bewirkt. Nämlich, sie verstärkt gezielt die Amplitude des gesuchten Zustands. Um Diffusionstransformation zu verstehen, erwähnt Grover in seiner Arbeit das Konzept der \textit{lokalen Übergangsmatrizen} ein.

\begin{definition}[Lokale Übergangsmatrizen]
Lokale Übergangsmatrizen sind Matrizen, in denen nur eine konstante Anzahl von Elementen in jeder Spalte ungleich null ist.
\end{definition}

\noindent Anschließend führt Grover den Begriff der Diffusionstransformation ein:

\begin{definition}[Grover-Diffusionstransformation]
\( D \) ist keine lokale Übergangsmatrix, da es Übergänge von jedem Zustand zu allen \( N \) Zuständen gibt:
$$
D_{ij} = 
\begin{cases}
\frac{2}{N}, & \text{wenn } i \ne j \\
1 - \frac{2}{N}, & \text{wenn } i = j
\end{cases}
$$
Unter Verwendung der Walsh-Hadamard-Transformationsmatrix kann \( D \) als Produkt von drei lokalen unitären Transformationen implementiert werden:
$$
D = H \cdot R \cdot H
$$
Hierbei sind:
\begin{itemize}
\item $H$: Die \textbf{Hadamard-Transformation} auf allen $n$ Qubits.
\item $R$: Eine \textbf{Phaseninversionsmatrix} (Phasenrotationsmatrix), die nur den Zustand $\ket{0}^{\otimes n}$ (alle Qubits sind 0) mit $-1$ multipliziert, alle anderen Zustände bleiben unverändert.
\end{itemize}
\end{definition}

\noindent Alle für den Grover-Algorithmus nötigen Operationen bestehen aus nur zwei fundamentalen Bausteinen - \textbf{Hadamard-Gates} und \textbf{Phaseninversionen}. Dadurch ist der Grover-Algorithmus im Vergleich zu anderen Quantenalgorithmen – z.,B. solchen wie Shor's Algorithmus \ref{algorithm:shor}, die eine Quantum Fourier-Transformation(QFT) benötigen – relativ einfach zu implementieren. In der nächsten Sektion betrachten wir den Grover-Algorithmus Implementierung Schritt für Schritt anhand eines konkreten Beispiels.

\subsection{Beispielanwendung von Grover-Algorithmus}

\subsection{Geometrische Darstellung}
\subsection{Komplexitätsanalyse}
Der Grover-Algorithmus (1996) führt eine quantengestützte Suche durch, indem er sukzessive die Amplitude des gesuchten Zustands verstärkt. Nach etwa $\sqrt{N}$ Iterationen erhält man das Ziel mit hoher Wahrscheinlichkeit.\\
Dieser Algorithmus ist informationstheoretisch optimal für die unstrukturierte Suche. Er wird typischerweise durch eine Schleife aus Orakel-Abfrage (markiert das gesuchte Element) und Inversion der Amplituden um den Mittelwert realisiert.\\
Grover findet Anwendungen als Subroutine in vielen Quantenalgorithmen (z.B. für Optimierung oder Simulation), etwa zur Stichprobensuche oder in hybriden Ansätzen.

\subsection{Quantenlauf-Suchalgorithmen}

Neben dem Grover-Ansatz gibt es Suchalgorithmen auf Basis von Quantenläufen (Quantum Walks). Hier wird die klassische zufällige Suche durch einen quantenmechanischen Spaziergang auf einem Graphen ersetzt.\\
Solche Algorithmen können z.B. die Suche in speziellen Strukturen beschleunigen: Etwa wird auf einem zweidimensionalen Gitter in $O(\sqrt{N\log N})$ statt $O(N)$ Schritten gesucht. Weitere Anwendungen sind die Elementar-Unterscheidung oder die Suche in sozialen Netzwerken.\\
Einführende Übersichten (z.B. Ambainis 2004) fassen die wichtigsten Suchalgorithmen anhand von Quantenläufen und ihre Komplexitäten zusammen.

\subsection{Praxisanwendung und Stand der Technik}

Grover’s Algorithmus wurde bereits auf ersten Quantenrechnern experimentell getestet. So implementierte IBM mit einem 127-Qubit-Supersystem eine 3-Qubit-Grover-Schaltung und untersuchte deren Leistung.\\
Der praktische Einsatz ist derzeit wegen begrenzter Qubit-Zahlen, niedriger Kohärenzzeiten und Fehlerraten noch eingeschränkt. Meist dienen kleine Grover-Experimente als Benchmark für Hardware.\\
Zukünftige Anwendungen könnten in Bereichen wie Datenbankrecherche, Kryptanalyse (Brute-Force-Angriffe) oder NISQ-nahe Optimierungsaufgaben liegen; der Fortschritt in Hardware-Fehlerkorrektur wird entscheidend sein.

\section{Quantumoptimierung}
Quantum-Optimierungsalgorithmen zielen darauf ab, kombinatorische und numerische Optimierungsprobleme schneller zu lösen oder mindestens gute Approximationslösungen zu finden. Typische Ansätze sind der Quanten-Aiabatischer Algorithmus (Quantenannealing) für NP-Probleme, hybride Algorithmen wie VQE und QAOA für Optimierung mit Quanten-Schaltkreisen, sowie Algorithmen zur Lösung linearer Gleichungssysteme (etwa HHL), die in Optimierungs- und Datenanalyseanwendungen münden.

\subsection{Adiabatische Optimierung}

Der Quanten-adiabatische Algorithmus (Farhi et al., 2000) übersetzt kombinatorische Probleme wie 3-SAT in die Suche nach dem Grundzustand eines zeitabhängigen Hamiltonians.\\
Man beginnt im Grundzustand eines einfachen Anfangs-Hamiltonians und ändert langsam zum Zielfall, dessen Grundzustand die Lösung enthält. Nach dem adiabatischen Theorem landet man bei entsprechender Langsamkeit im gewünschten Zustand.\\
Die benötigte Laufzeit hängt vom Energieabstand (Gap) zwischen dem Grund- und dem ersten angeregten Zustand ab. In allgemeinen Fällen ist dieser Abstand schwer abzuschätzen, weshalb eine polynomielle Laufzeit nur für spezielle Problemklassen nachgewiesen wurde.\\

\subsection{Variationaler Quanten-Eigenlöser (VQE)}

Der VQE-Algorithmus (Peruzzo et al., 2014) nutzt das Variationsprinzip auf Quantencomputern: Ein parametrisiertes Quantenschaltkreis $\ket{\psi(\vec\theta)}$ wird auf einen Hamiltonian angewandt, dessen Erwartungswert energetisch bewertet wird. Ein klassischer Optimierer passt die Parameter $\vec\theta$ an, um die Energie zu minimieren.\\
VQE eignet sich insbesondere zur Berechnung von Grundzustandsenergien in Quantenchemie und Festkörperphysik. Dieser hybride Ansatz kann komplexe Vielteilchensysteme in polynomialer Zeit simulieren und ist wegen geringer Quantenressourcen auch für Noisy Intermediate-Scale Quantum (NISQ) Geräte vielversprechend.\\
Trotz der theoretisch guten Skalierung müssen in der Praxis Laufzeit und Güte (sogenannte Pre-Faktoren) sorgfältig abgeschätzt werden; Rauschresistenz ist ein weiterer Vorteil von VQE.

\subsection{Quanten-Algorithmus zur näherungsweisen Optimierung (QAOA)}

QAOA (Farhi et al., 2014) ist ein hybrider Ansatz zur kombinatorischen Optimierung: Er wechselt zwischen „Kosten“- und „Mixer“-Hamiltonians mit tunbaren Parametern in $p$ Schritten.\\
Durch Abstimmung dieser Parameter mithilfe eines klassischen Optimierers können gute Näherungslösungen für Probleme wie MaxCut oder andere NP-harte Optimierungsaufgaben gefunden werden. Die Qualität der Lösung verbessert sich mit zunehmendem $p$.\\
Die Schaltkreustiefe wächst linear mit $p$ und der Problemgröße; wenn $p$ fest ist, kann man den Algorithmus effizient klassisch vorprozessieren. Die Frage, ob QAOA für bestimmte Probleme gegenüber klassischen Algorithmen Vorteile bietet, wird intensiv erforscht.\\

\subsection{Quantum Data Fitting: Lineare Gleichungssysteme lösen}

Quantum Data Fitting bezieht sich darauf, lineare Regressions- oder Ausgleichsaufgaben mithilfe von Quantenalgorithmen durchzuführen. Kern ist meist der HHL-Algorithmus (Harrow, Hassidim, Lloyd 2009), der lineare Gleichungen $Ax=b$ in Logarithmus-Zeit löst.\\
Beispielsweise kann man Parameter für ein lineares Modell auf einem großen Datensatz bestimmen oder die Qualität einer Kleinste-Quadrate-Anpassung untersuchen.\\
Ein neuer Quantenalgorithmus berechnet effizient die Güte einer linearen Regression über ein exponentiell großes Datenset, indem er das lineare Gleichungssystem quantenmechanisch löst.
 

\section{Quantum-safe Algorithmen in Kryptographie}
\subsection{Aktuelle Stand der Kryptographie}
\subsection{Vektoren in der asymmetrischen Verschlüsselung}

\section{Grover-Algorithmus für Fine-tuning von LLM}
\subsection{Idee: Transformer Architektur}
Transformer-Architekturen (Vaswani et al., 2017) bilden die Grundlage moderner großer Sprachmodelle (LLMs). Sie verwenden Self-Attention-Mechanismen, um in einem Eingabesequenz-Kontext relevante Zusammenhänge zu gewichten.\\
Ein Nachteil ist die quadratische Rechenkomplexität in der Sequenzlänge: Standard-Self-Attention erfordert $O(n^2)$ Operationen für eine Sequenz der Länge $n$. Das limitiert die Effizienz bei langen Eingaben.\\
Dieses Kapitel bietet einen Überblick über Transformer und Attention-Blöcke und ihre Rolle in LLMs, um die Ausgangsbasis für quantenbeschleunigte Ansätze zu erklären.

\subsection{Berechnung der Attention}
Gao et al. (2023) zeigten kürzlich, wie man Grover’s Suche nutzen kann, um die Attention-Matrix in Transformers effizienter zu berechnen. Sie betrachten insbesondere spärliche Attention-Matrizen und erreichen einen polynomialen Quantenvorteil gegenüber klassischen Methoden.\\
Ihr Algorithmus berechnet die relativ kleinen (sparse) Einträge der Attention-Matrix schneller, indem er Suche statt vollständiger Matrixmultiplikation verwendet.\\
Solche Quanten-Transformer sind ein aktives Forschungsthema: Sie könnten zukünftig helfen, die Rechenzeit von LLMs zu reduzieren, indem sie die Aufmerksamkeitsschichten quantenbeschleunigt umsetzen.
\printbibliography
