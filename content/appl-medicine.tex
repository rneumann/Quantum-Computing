%\motto{Use the template \emph{chapter.tex} to style the various elements of your chapter content.}
\chapter{Medizinische Anwendungsgebiete}
\label{trends} % Always give a unique label
% use \chaptermark{}
% to alter or adjust the chapter heading in the running head

\chapterauthor{Martin Maier, Siri Wandel}

\abstract{some abstract}

\section{Relevanz \& Problemstellung}


\section{Top 3 Anwendungsfelder (Praxis \& Theorie)}
Die potenziellen Einsatzmöglichkeiten von Quantencomputern im Gesundheitswesen sind vielfältig, wie Abbildung \ref{fig:use-cases-medicine} verdeutlicht. Sie reichen von molekularer Simulation über Krebsbehandlung bis hin zu Bildanalyseverfahren. In diesem Abschnitt liegt der Fokus auf den drei Anwendungsfeldern, die aktuell sowohl in der Forschung als auch in der industriellen Entwicklung eine herausragende Rolle spielen: Wirkstoffentwicklung , Proteinstrukturvorhersage und Personalisierte Medizin.\\

\begin{figure}[ht]
    \centering
    \includegraphics[width=.8\textwidth]{images/medicine/AnwendungsfelderMedizin.png}
    \caption{Übersicht möglicher Einsatzfelder von Quantencomputern in der Medizin. (Eigene Darstellung nach \cite{dhande_quantum_2023}.)}
    \label{fig:use-cases-medicine}
\end{figure}

Diese Auswahl basiert auf drei qualitativen Kriterien: Erstens adressieren diese Bereiche relevante medizinische Herausforderungen. Zweitens zeichnen sie sich durch eine besonders hohe rechnerische Komplexität aus, die den Einsatz von Quantencomputern sinnvoll erscheinen lässt. Drittens bestehen bereits erste praxisnahe Anwendungen und Pilotprojekte, die eine Brücke zwischen theoretischem Potenzial und konkreter Nutzung bilden. Die Auswahl orientiert sich somit nicht nur an theoretischem Potenzial, sondern auch an praktischer Umsetzbarkeit und gesellschaftlicher Relevanz.\\

\subsection{Wirkstoffentwicklung}
Die Entwicklung neuer Wirkstoffe ist grundlegend mit hohen Kosten verbunden. Das liegt besonders an langen Entwicklungszyklen und einer geringen Erfolgsquote. Der Einsatz von Computern, besonders im Rahmen des Computer"=Aided Drug Design (CADD), ist in der pharmazeutischen Forschung etabliert. Dies stößt jedoch bei der Modellierung komplexer molekularer Systeme an seine Grenzen (Vgl. \cite{bertl_quantum_2025}).\\

\subsubsection*{Compound Screening \& Lead"=Optimierung}
Quantencomputer versprechen, diese Prozesse durch genauere und schnellere Simulationen von Molekülen und deren Wechselwirkungen erheblich zu verbessern. Dabei konzentrieren sich aktuelle Quantencomputing"=Anwendungen in der pharmazeutischen Forschung besonders auf zwei frühe Phasen der Wirkstoffentwicklung: das \textit{Compound Screening} und die \textit{Lead"=Optimierung}. Beim Compound Screening werden sehr viele verschiedene chemische Verbindungen daraufhin untersucht, ob sie grundsätzlich an ein Zielmolekül (z.B. ein krankheitsrelevantes Protein) binden können. In der Lead"=Optimierung werden dann vielversprechende Kandidaten gezielt verbessert, um deren Wirksamkeit und Verträglichkeit zu steigern. Quantencomputer können in diesen Phasen helfen, indem sie Wechselwirkungen zwischen Molekülen genauer simulieren und energetisch günstige Strukturen besser vorhersagen (Vgl. \cite{zinner_quantum_2021}).\\

\subsubsection*{Quantenchemische Simulation}
Auf theoretischer Ebene gilt die quantenchemische Berechnung von Molekül"=Energien als eines der vielversprechendsten Anwendungsfelder des Quantencomputings. Sie ist essenziell für die Ermittlung von Bindungsaffinitäten, Reaktionspfaden und energetisch bevorzugten Konformationen, welche zentrale Größen in der computergestützten Wirkstoffentwicklung sind. Wie \cite{cao_quantum_2019} betonen: Das klassische Problem, dessen Lösung vom Quantencomputing erwartet wird, ist die Berechnung von Grundzustands- und angeregten Zustandsenergien kleiner Moleküle. Solche Berechnungen dienen als Ausgangspunkt für die Bestimmung vieler nützlicher Größen, wie etwa Reaktionspfaden, Bindungsenergien und Reaktionsgeschwindigkeiten chemischer Prozesse. (eigene Übersetzung nach \cite{cao_quantum_2019}.) Zentral ist demnach die Berechnung von Molekülenergien, da sie viele wichtige chemische Eigenschaften bestimmt.\\

\subsubsection*{Virtuelles Screening und datengetriebene Modellierung}
Beim \textit{virtuellen Screening} handelt es sich um einen frühen Schritt im Prozess der Wirkstoffentwicklung. Dabei werden mit Hilfe von Computern große Datenbanken nach Molekülen durchsucht, die an ein bestimmtes krankheitsrelevantes Ziel binden könnten. So lassen sich vielversprechende Wirkstoffkandidaten schneller finden und unnötige Labortests vermeiden. Klassische Methoden stoßen dabei aber oft an Grenzen: Die Modelle sind manchmal ungenau, liefern viele falsche Treffer und brauchen viel Rechenleistung. Neue Ansätze wie Quantum Machine Learning (siehe Kapitel \ref{alg:qml}) können hier helfen. Mit Quantencomputern lassen sich Molekülwechselwirkungen genauer simulieren und komplexe Daten effizienter verarbeiten. Techniken wie Quanten"=Neuronale Netzwerke, Quanten"=Kernel oder spezielle Quanten"=Suchalgorithmen ermöglichen ein gezielteres und schnelleres Screening. (Vgl. \cite{kumar2024})


\subsection{Proteinstrukturvorhersage}
\label{med:protein}
Die dreidimensionale Struktur eines Proteins bestimmt maßgeblich seine Funktion und ist damit zentral für die Medikamentenentwicklung. Klassische Methoden oder KI"=gestützte Systeme wie AlphaFold stoßen bei hochkomplexen Proteinen an Rechen- und Genauigkeitsgrenzen (Vgl. \cite{dhande_quantum_2023}). Quantencomputer bieten hier eine Alternative: Sie können potenziell viele mögliche Faltungszustände gleichzeitig analysieren und so schneller zu strukturell relevanten Lösungen führen.\\

\citeauthor{doga_perspective_2024} zeigen, dass Quantencomputer schon heute in der Lage sind, erste biologische Faltungsprobleme zu bearbeiten. Mithilfe eines hybriden Ansatzes, der klassische Optimierung mit einem variationalen Quantenalgorithmus kombiniert, konnten sie die Struktur einer kurzen Proteinschleife vorhersagen. Dabei erreichten sie eine höhere Genauigkeit als klassische Methoden oder AlphaFold. Das zeigt, dass Quantencomputer auch mit heutiger Hardware bereits sinnvolle Ergebnisse liefern können. (Vgl. \cite{doga_perspective_2024})\\

Ergänzend zeigt \citeauthor{robert_resource-efficient_2021}, wie sich Gittermodelle für Polypeptidketten mit nur 22 Qubits effizient lösen lassen. Beide Studien nutzen hybride Quantenalgorithmen, die sich insbesondere für energieoptimierende Strukturvorhersagen eignen. (Vgl. \cite{robert_resource-efficient_2021})\\

\subsection{Personalisierte Medizin}
Die personalisierte Medizin verfolgt das Ziel, medizinische Entscheidungen stärker an den individuellen Eigenschaften eines Patienten auszurichten, etwa an genetischen Merkmalen, Krankheitsverläufen oder Therapieansprechen. Das bedeutet, dass aus komplexen, oft hochdimensionalen Daten wie Genomsequenzen, Laborwerten oder Bilddaten präzise Vorhersagen abzuleiten sind. Klassische Methoden stoßen hierbei zunehmend an ihre Grenzen, insbesondere bei kleinen Datensätzen, hoher Variabilität oder stark nichtlinearen Zusammenhängen (Vgl. \cite{gupta_systematic_2025}).\\

Des Weiteren wird die personalisierte Medizin zunehmend im Sinne der P5"=Medizin verstanden, die fünf Dimensionen umfassen: predictive, preventive, personalized, participatory und psychocognitive. Letztere erweitert das ursprüngliche P4-Modell um die psychologischen und kognitiven Bedürfnisse der Patienten, etwa Entscheidungsverhalten, Werte, Lebensqualität und Gesundheitskompetenz, die maßgeblich beeinflussen, wie Menschen mit Krankheit umgehen und Therapien annehmen (Vgl. \cite{gorini_p5_2011}).\\

Quantencomputing kann in diesem Kontext neue Wege eröffnen. Durch seine Fähigkeit, viele Datenmuster gleichzeitig zu analysieren, lassen sich komplexe Strukturen in Gesundheitsdaten schneller und in höherer Qualität erkennen. Besonders vielversprechend ist hier der Einsatz von Quantum Machine Learning (QML), also von Modellen, die klassische Lernverfahren mit quantenmechanischen Elementen kombinieren (Vgl. \cite{bertl_quantum_2025}.)\\

Eine aktuelle Übersichtsarbeit zeigt, dass sich erste klinisch relevante QML-Anwendungen zum Beispiel in der EKG"=Analyse, Genomik oder Bildgebung bereits auf echter Quantenhardware testen lassen, wenn auch bisher in begrenztem Umfang (Vgl. \cite{gupta_systematic_2025}). Besonders Genomik und Bildgebung ermöglichen eine präzisere Diagnostik und auf individuelle genetische Profile abgestimmte Therapieansätze.\\

Ein konkretes Beispiel liefert eine Studie zur Vorhersage von Arzneimittelwirkungen bei Krebspatienten. In dieser erzielt ein hybrides Quantenneuronales Netzwerk (QNN) eine bis zu 15 Prozent höhere Genauigkeit als klassische Modelle, bei signifikant kürzerer Trainingszeit (Vgl. \cite{sagingalieva_hybrid_2023}).\\

\begin{table}[ht]
\centering
\renewcommand{\arraystretch}{1.3}
\resizebox{\textwidth}{!}{
\begin{tabular}{|p{2.3cm}|p{4.5cm}|p{2.5cm}|p{4.5cm}|}
\hline
\textbf{Anwendungsfeld} & \textbf{Rechnerische Herausforderung} & \textbf{Quantenansatz / Algorithmus} & \textbf{Warum QC relevant?} \\
\hline
Drug Discovery & Molekül-Energieberechnung, Bindungsaffinitäten & VQE, QPE & Klassische Methoden zu ungenau oder zu langsam \\
\hline
Proteinstruktur & Kombinatorische Explosion möglicher Faltungen & QAOA, Hybridalgorithmen & QC kann Zustände simultan analysieren \\
\hline
Personalisierte Medizin & Mustererkennung in hochdimensionalen Daten & QML, QNN & Klassische Modelle überfordert bei komplexer Nichtlinearität \\
\hline
\end{tabular}
}
\caption{Anwendungsfelder des Quantencomputings in der Medizin und Pharmazie}
\label{tab:qc_medizin}
\end{table}



\section{Top Technologien \& Algorithmen}

\subsection{Technologische Frameworks}

\subsubsection*{Wirkstoffentwicklung}
Die praktische Umsetzung der Quantenalgorithmen für medizinische Anwendungen erfordert leistungsfähige Software"=Stacks, die Quanten-Hardware und klassische Algorithmen verbinden. Für die Wirkstoffentwicklung werden vor allem Frameworks genutzt, die komplexe \textit{Molekül"=Hamilton"=Operatoren} bearbeiten und Verfahren wie VQE oder QPE effizient implementieren. Dabei handelt es sich um mathematische Ausdrücke, die alle relevanten physikalischen Eigenschaften eines Moleküls quantenmechanisch beschreiben. Der Hamilton-Operator bildet somit die Grundlage für die Simulation von Molekülzuständen, da er die Energieverteilung eines Systems vollständig bestimmt (Vgl. \cite{mcardle}). So bietet IBM Qiskit mit dem Qiskit Nature"=Modul spezialisierte Werkzeuge zur Darstellung chemischer Systeme und zum Lösen von Grundzustandsproblemen in Molekülen. (Vgl. \cite{noauthor_qiskit_2022})\\

Microsofts Q\# Quantum Development Kit (QDK) enthält ebenfalls eine Chemie-Bibliothek mit qubitisierten Fermionen-Operatoren sowie Implementierungen von Hamilton-Simulatoren (Trotterisierung, Qubitierung). (Vgl. \cite{msazure}) \\

Zudem erlaubt Amazon Braket als Cloud-Service den Zugriff auf unterschiedliche Quantenprozessoren (z.B. Rigetti"=Supraleiter, IonQ-Ionenfallen) und Simulationsengines über ein einheitliches SDK. (Vgl. \cite{aws})\\

Dieser flexible Ansatz wird beispielsweise im AWS-Toolkit QCEDD genutzt, das eingebaute Beispiele für Aufgaben wie molekulares Docking und Faltungsprobleme in der Wirkstoffforschung enthält. (Vgl. \cite{amazonscience})\\ %evtl. git repo zitieren

%besonders beim virtuellen screening relevant 

\subsubsection*{Proteinstrukturvorhersage}
Die Vorhersage von Proteinstrukturen ist ein hochdimensionales Optimierungsproblem. Klassischerweise modelliert man Proteine auf Gitternetzen, um die Konformationsräume lösbar zu machen. Wie bereits in Kapitel \ref{med:protein} dargelegt, sind Quantenoptimierungsalgorithmen wie QAOA dafür gut geeignet. (Vgl. \cite{boulebnane})\\

Zur praktischen Umsetzung von Algorithmen wie QAOA in der Proteinstrukturvorhersage wird geeignete Software benötigt, die die Komplexität quantenmechanischer Optimierung handhabbar macht. Ein bekanntes Framework in diesem Bereich ist \textit{PennyLane}. Es ist eine plattformunabhängige Open"=Source"=Bibliothek für Quantencomputing und Quantum Machine Learning, die für wissenschaftliche Anwendungen entwickelt wurde. Durch PennyLane können Quantenalgorithmen mit klassischen Lernverfahren kombiniert werden. Das ist besonders bei der Proteinstrukturvorhersage von Vorteil, da hier sowohl quantenbasierte Optimierung, als auch klassische Auswertungsverfahren erforderlich sind. (Vgl. \cite{PennyLane})\\

Das Start"=up Menten AI nutzt PennyLane für die Entwicklung neuer proteinbasierter Wirkstoffe. Ziel ist es, Quantencomputing und klassische Verfahren so zu kombinieren, dass neue Wirkstoffkandidaten schneller und gezielter gefunden werden können. Dabei setzt Menten AI auf PennyLane, um Quantenalgorithmen effizient einzubinden und bioaktive Peptide für neue Therapien zu entwickeln. Das heißt, sie nutzen die Software, um schneller passende Wirkstoffe zu finden, die gut an bestimmte Krankheitsziele andocken können. (Vgl. \cite{xanadu2022})\\



\subsubsection*{Personalisierte Medizin}
In der personalisierten Medizin stehen große, hochdimensionale Datensätze (Genomik, Bildgebung, EKG, klinische Parameter) im Mittelpunkt. Hier kommen Frameworks zum Einsatz, die Quantenmodelle in klassische Machine"=Learning"=Pipelines einbetten. Ein Beispiel ist die Kombination aus \textit{Cirq} und \textit{TensorFlow Quantum}. Damit lassen sich Quantenalgorithmen direkt in klassische Lernmodelle integrieren. So entstehen hybride neuronale Netze, die klassische und quantenbasierte Rechenschritte verbinden. Das hilft dabei, komplexe medizinische Daten genauer auszuwerten und Muster zu erkennen, die für individuelle Behandlungsentscheidungen wichtig sind. (Vgl. \cite{tensorflowQuantum2020})

\section{Wichtige Unternehmen \& Akteure}


\section{Top 3 Zukunftsprojekte \& Forschungsinitiativen}


\section{Bewertung anhand der Kriterien}


\section{Teilfazit}


\printbibliography
