%\motto{Use the template \emph{chapter.tex} to style the various elements of your chapter content.}
\chapter{Fehlerkorrektur}
\label{error_correction} % Always give a unique label
% use \chaptermark{}
% to alter or adjust the chapter heading in the running head

\chapterauthor{Niklas Bodfeld, Tim Boschert, Manuel Meixner, Ann-Kathrin Wenzel}

\abstract{some abstract}

\section{Fehlertypen in Quantenrechnern}\label{chap:QEC1}

\subsection{Die Herausforderung von Fehlern in Quantencomputern}
Die praktische Realisierung der Quantencomputer steht vor einer grundlegenden Hürde. Die Fehler. Diese treten in Quantensystemen häufiger auf als in klassischen Rechnern. Zudem sind sie viel schwerer zu kontrollieren.

Während klassische Systeme mit extrem niedrigen Fehlerraten arbeiten (unter 10⁻¹⁷), liegen diese bei heutigen Quantenprozessoren um Größenordnungen zwischen 10⁻³ und 10⁻¹. Der Grund dafür liegt unteranderem in der hohen Empfindlichkeit von Qubits gegenüber Umwelteinflüssen wie elektromagnetischen Feldern, Temperaturfluktuationen oder Strahlungen. Hinzu kommen Fehler durch ungenaue Hardware, Crosstalk zwischen benachbarten Qubits und unzuverlässige Mess- oder Initialisierungsprozessen.\ref{chap:QEC3}.). \cite[Seite 48-49]{tutschku_quantencomputing_2023}; \ref{chap:QEC3}.)

Die Auswirkungen dieser Fehler sind tiefgreifend. Schon ein einzelner Fehler kann sich durch Verschränkung auf das gesamte System ausbreiten. Besonders kritisch sind Phasenfehler, da sie die Interferenzmuster zerstören können, auf denen viele Quantenalgorithmen beruhen.

Deshalb ist die Entwicklung effektiver Fehlerkorrekturmethoden ein zentrales Thema für die Zukunft des Quantencomputings. Nur wenn bestimmte Mechanismen greifen, kann der theoretische Nutzen in der Praxis ausgeschöpft werden.


\subsection{Physikalische Fehlerursachen in Quantenrechnern}
Quantencomputer arbeiten nicht mit Bits sondern mit Qubits, die sich gleichzeitig in mehreren Zuständen befinden können. Die quantenmechanischen Eigenschaften von Qubits, insbesondere die Superposition, ermöglichen das enorme Potenzial und die Leistungsstärke von Quantencomputern. Gleichzeitig machen sie Quantencomputer allerdings auch extrem empfindlich. Schon geringe äußere Einflüsse können die Zustände stören und zu Fehlern führen. Im Folgenden werden zentrale physikalische Ursachen näher betrachtet.


\textbf{Dekohärenz}

Da sich physikalische Qubits nicht vollständig von ihrer Umgebung isolieren lassen, kommt es durch unvermeidliche Wechselwirkungen zu Dekohärenz, einem zentralen Fehlermechanismus in der Quanteninformatik.
Dekohärenz beschreibt den Prozess, bei dem ein Qubit durch den Kontakt mit seiner Umgebung „gestört“ wird und dabei seine Fähigkeit verliert, sich in einem Überlagerungszustand zu befinden. Stattdessen verhält es sich zunehmend wie ein klassisches Bit. Das bedeutet, dass die Quanteninformation, die zuvor im Zustand des Qubits gespeichert war, verloren geht. Anders als klassische Fehler entsteht Dekohärenz nicht durch eine fehlerhafte Bedienung oder ungenaue Operationen, sondern ist ein grundlegendes physikalisches Phänomen, das automatisch auftritt, sobald ein Quantensystem mit seiner Umgebung interagiert.

Es wird dabei zwischen Phasendekohärenz und Amplitudendekohärenz unterschieden. Bei der Phasendekohärenz bleibt das Qubit formal im selben Zustand, also zum Beispiel in ∣0⟩ oder ∣1⟩, jedoch verändert sich die relative Phase zwischen diesen Zuständen. Diese Phase ist zentral für Quanteninterferenz, einem Grundprinzip, das es Quantencomputern ermöglicht, durch Überlagerung unterschiedliche Rechenwege effizient zu nutzen. Wird die Phase gestört, kann der Quantencomputer falsche oder keine sinnvollen Ergebnisse mehr liefern. 

Amplitudendekohärenz hingegen beschreibt einen echten Energieverlust des Qubits. Wenn es aus einem angeregten Zustand ∣1⟩ in den Grundzustand ∣0⟩ zurückfällt. Solche Prozesse treten unter anderem durch spontane Emission oder Wärmeaustausch mit der Umgebung auf und sind in physischen Systemen wie supraleitenden Qubits oder Ionenfallen besonders häufig zu beobachten.

Die Wirkung solcher Umweltkopplungen lässt sich mathematisch mithilfe der sogenannten Operator-Summen-Zerlegung (Kraus-Zerlegung) beschreiben. Die Zerlegung erlaubt es, die Störung eines Qubits durch die Umgebung als Wahrscheinlichkeitsmischung mehrerer Fehlerprozesse darzustellen. Dies ist entscheidend für die Entwicklung von Quantenfehlerkorrektur und bildet die theoretische Grundlage für den Umgang mit Umwelteinflüssen.

Ein häufig verwendetes Modell zur Beschreibung solcher Fehler ist das lokale und Markov’sche Fehlermodell. Es geht davon aus, dass jedes Qubit nur mit seiner eigenen lokalen Umgebung interagiert (lokal) und dass sich diese Umgebung bei jedem Zeitschritt erneuert bzw. unabhängig vom Zustand in vorherigen Zeitintervallen ist (Markov). Die Gesamtwirkung auf ein Mehr-Qubit-System lässt sich dann als Produkt von Einzeleffekten auf die jeweiligen Qubits darstellen. Dieses Modell ist realistisch und erlaubt es, die meisten heute verwendeten Quantenfehlerkorrekturverfahren zu formulieren. 

Insgesamt zeigt sich, dass Dekohärenz  ein unvermeidlicher, aber mathematisch gut beschreibbarer Effekt ist. Durch geeignete Fehlerkorrektur und die Annahme realistischer Fehlermodelle kann ihre Auswirkung auf Quanteninformationen deutlich reduziert werden. \ref{chap:QEC3}.). \cite[Seite 332-339]{rieffel_eleanor_g_and_wolfgang_h_polak_quantum_2011}


\textbf{Rauschen und Vibration}

Doch Dekohärenz ist nicht die einzige physikalische Fehlerursache in Quantencomputern. Weitere Ursachen liegen in verschiedenen Rauschprozessen. Das sind zufällige oder unkontrollierte Störungen aus der Umgebung. Dazu zählen unter anderem thermisches Rauschen, das durch Temperaturunterschiede entsteht, sowie elektromagnetische Fluktuationen, verursacht durch Stromleitungen, elektronische Geräte oder sogar Mobilfunkstrahlung. Auch mechanische Vibrationen stellen eine Störquelle dar. Diese können beispielsweise durch Pumpen oder Bewegungen im Kühlapparat ausgelöst werden und sich auf die empfindlichen Qubits übertragen. In Systemen wie Ionenfallen können solche Vibrationen die Bewegung der Ionen stören. Dies führt dazu, dass die Quantenobjekte ihre Überlagerungszustände verlieren und sich zunehmend klassisch verhalten. Gleichzeitig wirken zufällige elektrische Feldschwankungen auf die Teilchen, wodurch es zu unkontrollierten Energie- und Positionsänderungen kommen kann. Wird dabei das harmonische Potenzial gestört, verlassen die Ionen ihren quantenmechanischen Grundzustand, was zu Rechenfehlern führt. Besonders problematisch wird dies bei stark gekoppelten Qubits, bei denen viele Operationen wie CNOT-Gatter notwendig sind. Denn jede zusätzliche Operation erhöht die Anfälligkeit für Rauschen und überfordert in vielen Fällen die heutigen Fehlermitigationstechniken. Um diese Fehler zu minimieren, sind hochstabile Betriebsbedingungen sowie gezielte Kühl- und Isolationsmaßnahmen unerlässlich.\ref{chap:QEC3}.). \cite[Seite 39-43]{tutschku_quantencomputing_2023}; \ref{chap:QEC3}.). \cite[Seite 353-356]{nielsen_michael_a_and_isaac_l_chuang_quantum_2010}


\textbf{Unvollständige Isolation}

Ein besonderes Problem ist die unzureichende Isolation der Qubits. Idealerweise sollte ein Qubit vollständig von seiner Umgebung abgeschottet sein. In der Realität ist das fast nicht möglich. Selbst im Vakuum oder bei extrem tiefen Temperaturen gibt es noch mikroskopisch kleine Wechselwirkungen mit anderen Teilchen, Materialien oder Defekten. Manche Materialien enthalten zum Beispiel sogenannte Zwei-Niveau-Systeme, die mit den Qubits interagieren können und als zusätzliche Störquellen wirken. Auch kosmische Strahlung oder Magnetfelder aus der Umgebung können Störungen verursachen (Tutschku et al., 2023).

Aus all diesen Gründen ist es extrem wichtig, die Quantenhardware so gut wie möglich gegen äußere Einflüsse zu schützen. Das geschieht zum Beispiel durch ultratiefe Temperaturen im Millikelvin-Bereich, durch spezielle Abschirmungen gegen elektromagnetische Wellen, durch mechanische Entkopplung der Geräte und durch den Einsatz besonders reiner Materialien ohne Defekte. Trotzdem lässt sich nie ganz vermeiden, dass äußere Einflüsse auf das System wirken. Daher ist es bedeutend, zusätzlich auf softwareseitige Verfahren zur Fehlerkorrektur zu setzen.\ref{chap:QEC3}.). \cite[Seite 24-26]{tutschku_quantencomputing_2023}

Quantenfehler sind ein Zusammenspiel aus unvermeidbaren physikalischen Prozessen, technologischen Grenzen und Umwelteinflüssen. Sie zu verstehen und zu kontrollieren ist eine der größten Herausforderungen beim Bau und Betrieb eines funktionierenden Quantencomputers.

\subsection{Klassifizierung von Quantenfehlern}\label{chap:QEC1.3}
Die Klassifikation von Quantenfehlern stellt eine bedeutsame Grundlage für die Entwicklung stabiler Fehlerkorrekturverfahren im Quantencomputing dar. Aufgrund der Superposition können Fehler nicht nur den Zustand selbst, sondern auch die relative Phase, Amplitude oder die Kohärenzeigenschaften eines Qubits beeinflussen. Die Einteilung unterscheidet zwischen diskreten, kontinuierlichen, zufälligen und systematischen Fehlern.

Die Diskrete Fehlerarten sind die Pauli-Fehler. Die einfachste und zugleich mathematisch fundamentale Klasse von Quantenfehlern basiert auf den sogenannten Pauli-Fehlern, benannt nach den drei Pauli-Matrizen 
X, Y und Z. Diese Operatoren stellen Basisoperationen im zweidimensionalen Qubit-Zustandsraum dar und bilden eine vollständige Fehlerbasis für beliebige Ein-Qubit-Störungen.


\textbf{Bit-Flip-Fehler (X-Fehler)}

Ein Bit-Flip-Fehler ist eine der einfachsten und grundlegendsten Fehlerarten in der Quanteninformatik. Er beschreibt eine Umkehrung des Zustands eines Qubits, also einen Wechsel von ∣0⟩ nach ∣1⟩ oder von ∣1⟩ nach ∣0⟩. Formal wird dieser Fehler durch den X-Operator (auch Pauli-X-Gatter genannt) dargestellt, der wie ein klassisches NOT-Gatter wirkt: X∣0⟩=∣1⟩, X∣1⟩=∣0⟩. Ein solcher Fehler kann in der physikalischen Realität zum Beispiel durch thermische Anregung entstehen. Also wenn ein Qubit spontan aus dem Grundzustand ∣0⟩ in einen angeregten Zustand ∣1⟩ übergeht, weil es mit seiner Umgebung Energie austauscht.

Beim Bit-Flip-Fehler geht es also ausschließlich um die Veränderung der Besetzungszustände eines Qubits, nicht um seine Phase oder Superposition. Er ist die quantenmechanische Entsprechung eines klassischen Bitfehlers, bei dem eine „0“ zu einer „1“ wird oder umgekehrt. \ref{chap:QEC3}.). \cite[Seite 246-251]{rieffel_eleanor_g_and_wolfgang_h_polak_quantum_2011}(Nielsen & Chuang, 2010, S.449).



\textbf{Phase-Flip-Fehler (Z-Fehler)}

Ein Phase-Flip-Fehler (auch Z-Fehler) ist eine fundamentale Art von Fehler in der Quanteninformationstheorie, die nicht den Zustand ∣0⟩ oder ∣1⟩ selbst verändert, sondern die relative Phase zwischen ihnen. Formal wird dieser Fehler durch den Z-Operator (Pauli-Z-Gatter) dargestellt: Z∣0⟩=∣0⟩,Z∣1⟩=−∣1⟩. Obwohl sich der Zustand ∣1⟩ dabei nicht in ∣0⟩ umwandelt, bekommt er ein negatives Vorzeichen. Die sogenannte Phaseninversion. Für Zustände wie ∣0⟩ oder ∣1⟩ alleine ist dieser Effekt nicht direkt sichtbar. Doch sobald der Qubit-Zustand eine Superposition ist, wirkt sich der Fehler messbar aus. Besonders deutlich wird das in diesem Zustand: \[
|+\rangle = \frac{1}{\sqrt{2}}(|0\rangle + |1\rangle)
\]
Wird darauf der Z-Operator angewendet, entsteht:
\[
Z|+\rangle = \frac{1}{\sqrt{2}}(|0\rangle - |1\rangle) = |-\rangle
\]

Damit wird die Phase zwischen den beiden Zustandsanteilen vertauscht. Das hat gravierende Folgen für Quantenalgorithmen, die auf Interferenz und kohärenter Phasenevolution beruhen. Phase-Flip-Fehler stören also genau diese empfindlichen quantenmechanischen Effekte, die Quantencomputer leistungsfähig machen. \ref{chap:QEC3}.). \cite[Seite 251-252]{rieffel_eleanor_g_and_wolfgang_h_polak_quantum_2011}


\textbf{Kombinierte Bit- und Phase-Fehler (Y-Fehler)}

Der Y-Operator kombiniert Bit- und Phase-Flip in einem einzigen Fehler: Y=iXZ. Er wirkt gleichzeitig wie ein X- (Bit-Flip) und ein Z-Operator (Phase-Flip), und ist damit der vollständigste Einzelfehler im Pauli-Basisset. Solche Fehler treten typischerweise bei komplexeren physikalischen Störungen auf. Dazu zählt unteranderem Crosstalk zwischen benachbarten Qubits, Störfelder mit nichtlokaler Auswirkung oder gekoppelte Fluktuationen in Steuerleitungen. In der Fehlerkorrektur werden Y-Fehler daher als gleichzeitiges Auftreten von X- und Z-Fehlern behandelt und können entsprechend über Stabiliser-Messungen erkannt werden (Rieffel & Polak, 2011, S.375).

Diese drei Operatoren bilden eine vollständige Fehlerbasis für Ein-Qubit-Störungen. Jeder beliebige Fehler auf einem Qubit kann mathematisch als Linearkombination von I, X, Y und Z dargestellt werden. Diese Eigenschaft ist von Bedeutung für den Aufbau von Quantenfehlerkorrekturcodes wie dem Shor-Code oder dem Steane-Code, welche gezielt auf Pauli-Fehler reagieren. \ref{chap:QEC3}.). \cite[Seite 252-253]{rieffel_eleanor_g_and_wolfgang_h_polak_quantum_2011}


\textbf{Dämpfung}

Neben diskreten Fehlern gibt es die kontinuierliche Fehlerprozesse, die sich über sogenannte Quantenkanäle modelliert werden. Die zwei wichtigsten Beispiele sind Amplitudendämpfung und die Phasendämpfung. Sie hängen eng mit den physikalischen Prozessen der Phasen- und Amplitudendämpfung zusammen.

Die Amplitudendämpfung beschreibt den Verlust von Energie durch das Qubit, etwa infolge spontaner Emission oder thermischer Relaxation. Der Zustand ∣1⟩ relaxiert dabei mit einer gewissen Wahrscheinlichkeit γ in den Grundzustand ∣0⟩.
 \ref{chap:QEC3}.). \cite[Seite 380-383]{nielsen_michael_a_and_isaac_l_chuang_quantum_2010}
(Nielsen & Chuang, 2010, S.376–377)

Bei der Phasendämpfung (Dephasing) geht keine Energie verloren, aber Kohärenz. Die Off-Diagonal-Elemente der Dichtematrix, also die, die Superpositionen erzeugen, werden gedämpft. Dies geschieht durch Umwelteinflüsse wie Fluktuationen elektromagnetischer Felder. In der Praxis ist Phasendämpfung oft die dominante Dekohärenzquelle. \ref{chap:QEC3}.). \cite[Seite 383-386]{nielsen_michael_a_and_isaac_l_chuang_quantum_2010}(Rieffel & Polak, 2011, S.374–375).


\textbf{Zufällige vs. Systematische Fehler}

Quantenfehler lassen sich nicht nur nach ihrer physikalischen Form, sondern auch nach ihrer Entstehungsart klassifizieren. Hierzu zählen zufällige Fehler und systematische Fehler.

Zufällige Fehler (stochastisch) entstehen unvorhersehbar durch thermisches Rauschen, Photonen-Einfälle, kosmische Strahlung oder spontane Kopplungen an die Umgebung. Sie sind oft kurzzeitig, unkorreliert und lassen sich statistisch modellieren.
Systematische Fehler beruhen auf wiederholbaren, deterministischen Einflüssen, wie falscher Kalibrierung von Pulssequenzen, ungenauen Gatterzeiten oder falsch modellierter Kopplung. Solche Fehler summieren sich über Zeit auf und können ganze Rechenprozesse verfälschen. Sie sind schwerer zu korrigieren, da sie nicht durch Mittelwertbildung „herausrauschen“.\ref{chap:QEC3}.). \cite[Seite 3-5]{Quantum error correction below the surface code threshold_2024}
Die Unterscheidung ist für die Architekturplanung bedeutend, da systematische Fehler oft durch verbesserte Hardware, kontrollierte Steuerungen oder modellgestützte Korrektur vermieden werden können.

Ein wachsendes Problem in größeren Quantenprozessoren ist das Auftreten von korrelierten Fehlern. Solche Fehler beeinflussen mehrere Qubits gleichzeitig und können durch gemeinsame Störquellen oder durch ungewollte Kopplungen entstehen. Anders als unabhängige Fehler breiten sie sich nicht lokal aus, sondern erzeugen komplexe Fehlerbilder, die von klassischen Fehlerkorrekturmodellen oft nicht erfasst werden.


\subsection{Hardwarebedingte Fehler und Systematische Grenzen}
Neben den oben beschriebenen Fehlern spielen auch hardwarebedingte Fehler eine zentrale Rolle in der Fehlertheorie des Quantencomputings. Diese entstehen durch Unzulänglichkeiten in der technischen Umsetzung der Steuerung, der Auslese und der physikalischen Qubit-Architektur. In realen Quantenprozessoren, wie supraleitenden Schaltkreisen, Ionenfallen oder spinbasierten Systemen, sind diese Fehlerquellen ein wesentlicher begrenzender Faktor für die Skalierbarkeit und Zuverlässigkeit der Systeme.


\textbf{Gatteroperationen}

Ein wesentliches Problem sind fehlerhafte Gatteroperationen. Idealerweise sollen Quantenlogikgatter wie das CNOT- oder Hadamard-Gate eine definierte unitäre Transformation auf den Zustand der Qubits ausführen. In der Praxis weichen die implementierten Operationen von diesem Ideal ab. Ursachen dafür sind unter anderem folgende Fehler:

Timing-Fehler: Ungenauigkeiten in der Pulslänge oder -frequenz führen zu nicht vollständig abgeschlossenen Rotationen.
Crosstalk: Eine ungewollte Kopplung zwischen benachbarten Qubits oder Steuerleitungen kann zu Störungen führen, die nicht auf das Zielqubit beschränkt bleiben.
Leakage-Fehler: Besonders bei supraleitenden oder mehrstufigen Systemen kann ein Qubit durch ein Gatter in höhere Energiezustände außerhalb des Rechenraums (z.B. |2⟩) übergehen, was spätere Gates unbrauchbar macht.
Falsche Kalibrierung: Abweichungen in der Kalibrierung der Steuerpulse führen dazu, dass Operationen (z.B. ein X-Gate) nicht exakt das tun, was vorgesehen ist – etwa eine Rotation um 178° statt 180°.

Diese Gatterfehler häufen sich über tiefe Schaltkreise hinweg und führen dazu, dass die logische Fehlerrate mit zunehmender Schaltungstiefe exponentiell ansteigt. Selbst bei modernen Quantenprozessoren liegt die mittlere Gatterfidelität, also die Wahrscheinlichkeit, mit der ein Gatter korrekt ausgeführt wird, typischerweise nur bei 99–99,9%. \ref{chap:QEC3}.). \cite[Seite 2-5]{Quantum error correction below the surface code threshold_2024}; \ref{chap:QEC3}.). \cite[Seite 5-13]{Time–adaptive single–shot crosstalk detector on superconducting quantum computer_2025}


\textbf{Messfehler}

Ein weiterer kritischer Aspekt sind Messfehler. Die Auslese eines Qubits erfolgt in der Regel über die Verstärkung und Analyse eines quantenmechanischen Signals. Dabei können verschiedene Fehlerquellen auftreten. Zum Beispiel elektronisches Rauschen in der Verstärkerschaltung, Überlappung der Signalverteilungen für die Zustände ∣0⟩ und ∣1⟩ oder Verzögerungen und Sättigungseffekte in der Ausleseelektronik. Solche Störungen können dazu führen, dass der tatsächliche Zustand des Qubits falsch interpretiert wird. Dies beeinträchtigt nicht nur die Genauigkeit von Quantenalgorithmen, sondern besonders auch die Fehlerkorrektur, die auf das zuverlässige Auslesen von Syndromen angewiesen ist.
 \ref{chap:QEC3}.). \cite[Seite 49-51]{tutschku_quantencomputing_2023}; \cite[Seite 5-7]{Time–adaptive single–shot crosstalk detector on superconducting quantum computer_2025}


\textbf{Qubitinitalisierung}

Die Initialisierung von Qubits ist ein grundlegender Schritt zu Beginn jeder Quantenberechnung. Ziel ist es, sämtliche Qubits zuverlässig in den Grundzustand ∣0⟩ zu bringen. In der Praxis gelingt dies jedoch nicht immer mit perfekter Genauigkeit. Mögliche Ursachen sind zum Beispiel thermische Anregungen bei unzureichender Kühlung, unvollständige Relaxation aus angeregten Zuständen oder Restkopplungen an Resonatoren oder benachbarte Qubits. Fehlerhafte Initialisierungen beeinflussen unmittelbar die Ausführung der ersten Gatteroperationen und können sich über den gesamten Quantenschaltkreis hinweg auswirken. \ref{chap:QEC3}.). \cite[Seite 3-5]{Quantum error correction below the surface code threshold_2024}


\textbf{Konnektivität}

In skalierbaren Architekturen, vor allem in zweidimensionalen Gittern supraleitender Qubits, sind nicht alle Qubits direkt miteinander verbunden. Diese eingeschränkte Konnektivität führt dazu, dass logische Operationen, die Qubits an weit entfernten Positionen betreffen, durch zusätzliche SWAP-Gatter oder Relaisoperationen realisiert werden müssen. Diese zusätzlichen Schritte erhöhen nicht nur die Rechenzeit, sondern fügen dem System zusätzliche Fehlerquellen hinzu. Besonders kritisch ist dies in tiefen Schaltkreisen oder bei komplexen Algorithmen. \ref{chap:QEC3}.). \cite[Seite 49-51]{tutschku_quantencomputing_2023}


\textbf{Kalibrierfehler}

Alle oben genannten Fehlerarten werden zusätzlich durch Kalibrierfehler verstärkt. Diese entstehen, wenn die physikalischen Parameter des Systems, wie Frequenzen, Kopplungsstärken, Pulsamplituden oder Dämpfungsraten nicht exakt bekannt oder nicht stabil sind. Selbst kleine Abweichungen in der Kalibrierung können über viele Rechenzyklen hinweg zu signifikanten Abweichungen im Endergebnis führen. Studien zeigen, dass Kalibrierfehler insbesondere bei nicht regelmäßig durchgeführten Rekalibrierungszyklen zu systematischer Fehlentwicklung der Qubit-Zustände führen. \ref{chap:QEC3}.). \cite[Seite 5-7]{Quantum error correction below the surface code threshold_2024}

\subsection{Quantifizierung und Modellierung von Quantenfehlern}
Die präzise Modellierung und Quantifizierung von Fehlern in Quantencomputern ist wichtig, um deren Auswirkungen auf die Informationsverarbeitung zu verstehen und wirksame Fehlerkorrekturstrategien zu entwickeln. Aufgrund der Besonderheiten quantenmechanischer Systeme, insbesondere Superposition, Nichtklassikalität und Unitarität, ist eine Beschreibung von Fehlerprozessen nur durch formalisierte mathematische Modelle möglich, die über klassische Störmodelle weit hinausgehen. In der Theorie offener Quantensysteme werden solche Fehler durch Quantenkanäle beschrieben, deren Wirkung über sogenannte Krausoperatoren modelliert wird.


\textbf{Fehlerkanäle und Krausoperatoren}

Ein Quantenkanal beschreibt die Wirkung eines Umgebungseinflusses auf ein Quantensystem, das sich dadurch in einen neuen Zustand überführt. Formal wird ein Quantenkanal (E) durch eine completely positive, trace-preserving (CPTP) Abbildung auf dem Raum der Dichtematrizen definiert. Die Wirkung eines solchen Kanals auf einen Zustand ρ wird durch die Kraus-Zerlegung beschrieben:

E(\rho) = \sum_k E_k \rho E_k^\dagger, \quad \text{mit} \quad \sum_k E_k^\dagger E_k = I.

Hierbei sind  E_k die Krausoperatoren, die jeweils eine mögliche Fehlerwirkung modellieren. Diese Darstellung ist besonders mächtig, da sie sowohl diskrete als auch kontinuierliche Fehlermechanismen integriert. Je nach Art des Rauschprozesses ergeben sich unterschiedliche konkrete Kanalformen:
Amplitudendämpfungskanal: Modelliert Energieverluste wie spontane Emission. Dabei relaxiert das Qubit mit Wahrscheinlichkeit 
γ vom angeregten Zustand ∣1⟩ in den Grundzustand ∣0⟩. Die zugehörigen Krausoperatoren lauten:

E_0 = \begin{pmatrix} 1 & 0 \\ 0 & \sqrt{1-\gamma} \end{pmatrix},E_1 = \begin{pmatrix} 0 & 0 \\ \sqrt{\gamma} & 0 \end{pmatrix}


Depolarisationskanal: Beschreibt symmetrisches Rauschen, bei dem mit Wahrscheinlichkeit p ein zufälliger Pauli-Fehler (X, Y, Z) auftritt:

E(\rho) = (1-p)\rho + \frac{p}{3}(X\rho X + Y\rho Y + Z\rho Z)


Diese Modelle bilden die Grundlage für theoretische Fehleranalysen und sind integraler Bestandteil moderner Simulationen von Quantenalgorithmen.\ref{chap:QEC3}.). \cite[Seite 379-397]{nielsen_michael_a_and_isaac_l_chuang_quantum_2010}


\textbf{Fehlerraten}

Fehlerraten spielen eine zentrale Rolle bei der Quantifizierung und Modellierung von Quantenfehlern, insbesondere im Kontext der derzeit verfügbaren NISQ-Quantencomputer („Noisy Intermediate-Scale Quantum“). Diese Systeme sind durch verschiedene Störfaktoren stark limitiert. Dazu zählen kurze Kohärenzzeiten (T1 und T2), begrenzte Qubit-Konnektivität, Crosstalk und insbesondere die Fehlerraten bei Quantengattern und Messungen. Die Gate-Fidelity, also die Wahrscheinlichkeit, mit der ein Quantengatter korrekt ausgeführt wird, liegt derzeit bei etwa 0,1% Fehler für 1-Qubit-Gatter und etwa 1% Fehler für 2-Qubit-Gatter. Auch bei der Initialisierung und Auslesung treten mit etwa 1% signifikante Fehler auf. Diese Fehlerraten sind um Größenordnungen höher als bei klassischen Computern und schränken die maximale Schaltungstiefe sowie die Verlässlichkeit von Quantenalgorithmen massiv ein. Um diese Fehler systematisch zu erfassen und vergleichbar zu machen, wurden Metriken wie das „Quantenvolumen“ entwickelt. Es kombiniert Aspekte wie Qubit-Anzahl, Gate-Fidelity und Schaltungstiefe zu einer einzigen Kenngröße. Allerdings ist diese Metrik nicht unumstritten, da die Aussagekraft stark vom konkreten Anwendungsfall abhängt. In jedem Fall ist eine präzise Quantifizierung und Modellierung der Fehlerraten unerlässlich, um die Einsatzfähigkeit von Quantencomputern realistisch bewerten zu können und eine effektive Fehlerkorrektur zu ermöglichen. \ref{chap:QEC3}.). \cite[Seite 379-397]{https://www.quantencomputer-info.de/quantencomputer/welche-quantencomputer-gibt-es-jetzt-schon/}


\textbf{Fehlergrenzen und Fehlertolernaz}

Ein zentrales Konzept in der Quantenfehlertoleranz ist die sogenannte Fehlergrenze (Threshold). Diese beschreibt den maximal tolerierbaren physikalischen Fehler \( p \) der Qubits, unterhalb dessen ein Fehlerkorrekturcode in der Lage ist, die logische Fehlerwahrscheinlichkeit 
\( p_{L} \) 
durch Vergrößerung des Codes zu unterdrücken. Ist der physikalische Fehler kleiner als der Schwellenwert \( p_{L} \) , kann die logische Fehlerquote beliebig klein gemacht werden. Der Code wirkt also stabilisierend. Liegt der physikalische Fehler jedoch über diesem Schwellenwert, wird die Fehlerkorrektur kontraproduktiv und verschlechtert den Zustand. Für den in der Praxis besonders relevanten Surface Code liegt der theoretische Schwellenwert bei etwa 10,9•% (bei separater Behandlung von X- und Z-Fehlern). Mit realistischen Decodierungsalgorithmen liegt der praktische Schwellenwert bei etwa 10,3•%.

Ein weiterer kritischer Aspekt ist die Fehlertoleranz (Fault Tolerance) von Quantenfehlerkorrekturverfahren. Dabei geht es nicht nur darum, Fehler im Qubit-Zustand selbst zu erkennen und zu beheben, sondern auch Fehler, die während des Korrekturprozesses auftreten, beispielsweise bei Zwei-Qubit-Gates oder bei der Syndrommessung, korrekt zu behandeln. Eine fehlertolerante Fehlerkorrektur muss gewährleisten, dass sich solche Fehler nicht unkontrolliert ausbreiten. Dies wird beispielsweise durch spezielle Schaltungen, redundante Hilfsqubits (Ancillas) oder wiederholte Messungen über mehrere Zyklen erreicht. Diese zusätzlichen Anforderungen erhöhen jedoch den Ressourcenbedarf erheblich. In realistischen Szenarien, insbesondere bei Surface Codes mit fehlerhaften Syndrommessungen, reduziert sich der Schwellenwert dadurch auf etwa 1 %. Dennoch ist dieser Wert für moderne Quantenhardware erreichbar, da aktuelle Systeme bereits Fehlerquoten unterhalb dieser Grenze aufweisen.\ref{chap:QEC3}.). \cite[Seite 13-15]{roffe_Quantum_error_correction_2019} 

\subsection{Auswirkungen von Fehlern auf Quantenalgorithmen und Systemarchitektur}
Fehler haben tiefgreifende Auswirkungen auf die Funktionsweise von Quantenalgorithmen und die Gestaltung von Quantencomputer-Architekturen. Bereits kleinste Abweichungen durch Umwelteinflüsse oder ungenaue Operationen können den empfindlichen Quantenzustand stören. Solche Störungen zeichnen sich in Form von Dekohärenz, Gate-Fehlern und Messfehlern ab. Besonders kritisch ist die Dekohärenz, also der Verlust von Quantenkohärenz, die essenziell für Superpositions- und Verschränkungszustände ist. Algorithmen wie Shor’s oder Grover’s, die stark auf Interferenzeffekten beruhen, werden dadurch extrem fehleranfällig. Ein einzelner Fehler kann das gesamte Interferenzmuster zerstören und damit das korrekte Ergebnis verhindern. \ref{chap:QEC3}.). \cite[Seite 381-385]{nielsen_michael_a_and_isaac_l_chuang_quantum_2010} 

Diese hohe Fehleranfälligkeit wirkt sich direkt auf die algorithmische Struktur aus. Je größer die Schaltungstiefe, also die Anzahl nacheinander geschalteter Operationen, desto wahrscheinlicher treten Fehler auf. Aktuelle Hardware beschränkt die maximale Tiefe daher erheblich. Entwickler müssen daher bereits beim Entwurf von Quantenalgorithmen Strategien zur Fehlertoleranz berücksichtigen. Ein zentrales Konzept in diesem Zusammenhang ist der sogenannte Fehlerschwellenwert (Threshold), der angibt, bis zu welchem Maß an physikalischen Fehlern eine zuverlässige Fehlerkorrektur überhaupt möglich ist \ref{chap:QEC3}.). \cite[Seite 293-308]{rieffel_eleanor_g_and_wolfgang_h_polak_quantum_2011}.

Auch auf der Systemarchitektur wirkt sich die Fehleranfälligkeit massiv aus. Um fehlerhafte physikalische Qubits nutzbar zu machen, werden sie in logische Qubits zusammengefasst, die durch Fehlerkorrekturcodes geschützt sind, wie durch den Steane- oder Shor-Code . Dieser Schritt ist jedoch teuer. Die Anzahl der tatsächlich benötigten physikalischen Qubits steigt exponentiell mit dem Anspruch an Fehlertoleranz. Beispielsweise kann ein logisches Qubit je nach Code und Fehlerrate mehrere Hundert oder Tausend physikalische Qubits benötigen \ref{chap:QEC3}.). \cite[Seite 435–437]{nielsen_michael_a_and_isaac_l_chuang_quantum_2010};\ref{chap:QEC3}.). \cite[Seite 246-253]{rieffel_eleanor_g_and_wolfgang_h_polak_quantum_2011}.

Liegt die effektive Fehlerrate pro Gatteroperation oder Qubit unterhalb dieser Fehlerschwelle, so kann durch wiederholte Korrekturzyklen ein logisch fehlerfreier Zustand aufrechterhalten werden. Wird sie überschritten, kann auch ein noch so ausgeklügelter Fehlerkorrekturcode das System nicht mehr retten.

Die Gestaltung der Quantenhardware und -architektur muss daher darauf ausgerichtet sein, mit einer begrenzten Anzahl physikalischer Qubits möglichst effektive Fehlerresistenz zu gewährleisten. Dazu gehören unteanderm Fehlertolerante Layouts, z.B. zweidimensionale Qubit-Gitter mit lokaler Nachbarschaftsstruktur, die einfache Syndrome-Messung und Korrektur ermöglichen.
Die konkrete Umsetzung solcher Codes, wie etwa des 9-Qubit Shor-Codes, zeigt, dass verschiedene Arten von Fehlern, wie der Bit-Flip, Phase-Flip und kombinierte Fehler, durch geeignete Verschränkung und Syndrommessung identifiziert und korrigiert werden können. Dies setzt jedoch eine Systemarchitektur voraus, die gezielte Paarung und Interaktion zwischen Qubits erlaubt, was hohe Anforderungen an die Qubit-Konnektivität und Kohärenz stellt.

Fehler wirken sich aber nicht nur auf die Hardware, sondern auch auf die Softwareebene aus. Quantenalgorithmen müssen so entworfen sein, dass sie entweder robust gegenüber Fehlern sind oder mit reduzierten Ressourcen für Fehlerkorrektur arbeiten. Dafür werden auch softwarebasierte Fehlermitigationstechniken verwendet, etwa die statistische Nachbearbeitung von Messergebnissen oder hybride Strategien, bei denen klassische Optimierung zur Steuerung kurzer Quantenzyklen genutzt wird \ref{chap:QEC3}.). \cite[Seite 305-306]{rieffel_eleanor_g_and_wolfgang_h_polak_quantum_2011}.

Zusammenfassend lässt sich sagen, dass die Fehleranfälligkeit in Quantencomputern kein nachträgliches Problem darstellt, sondern ein grundlegendes Designkriterium auf allen Ebenen. Von der Wahl der Algorithmen über die Anzahl und Anordnung der Qubits bis hin zur Steuerungselektronik. Nur wenn es gelingt, Fehler systematisch zu modellieren, zu korrigieren oder zu umgehen, können Quantencomputer ihr volles Potenzial entfalten.


\section{Grundprinzipien in Quantenfehlerkorrektur}\label{chap:QEC2}
\subsection{Fehlererkennung ohne Zustandsmessung}
Eines der zentralen Konzepte der Quanten-Fehlerkorrektur ist, dass man Fehler erkennen kann, ohne die kodierte Quanteninformation zu messen und somit zu zerstören. In klassischen Systemen könnten wir etwa redundante Bitmuster verwenden, um Fehler aufgrund direkter Messung zu erkennen und zu korrigieren. In einem quantenmechanischen System ist es jedoch nicht so einfach möglich, eine Messung durchzuführen, da ein solcher Schritt den Superpositionszustand des Qubits kollabieren ließe und damit die gespeicherte Information unter Umständen irreversibel zerstören könnte. \cite{nielsen_michael_a_and_isaac_l_chuang_quantum_2010}

Um das zu umgehen, führt man im Bereich der Quanten-Fehlerkorrektur sogenannte Syndrommessungen durch. Dabei handelt es sich um Messungen, bei denen nicht der Zustand des Qubits direkt gemessen wird, sondern nur Information über das Eintreten von Fehlern. Man misst dabei den sogenannten Stabilisator-Operator, der zu einem speziellen Quanten-Fehlerkorrekturcode gehört. Diese Operatoren definieren eine Menge erlaubter Zustände – sogenannte +1-Eigenzustände der Stabilizer-Gruppe. Tritt ein Fehler auf, so verändert sich die Eigenwertstruktur des Stabilisators. Ein Wechsel zwischen \(+1\) und \(-1\) ist zum Beispiel ein  Zeichen dafür, dass ein bestimmter Fehler passiert. \cite[Seite 444-446]{nielsen_michael_a_and_isaac_l_chuang_quantum_2010}

Die Syndrommessung erfolgt in der Praxis meist so, dass man Hilfsqubits (Ancillae) einführt, auf denen man den Stabilisator wirken lässt und so Informationen darüber erhält, was der Stabilizer auf den gekodeten Zustand angewendet hätte, ohne hierbei den gekodeten Zustand zu verändern. So lassen sich ernsthaft Fehler zum Beispiel durch die sogenannten kontrollierten Operatoren auch zu dem Hilfs-Qubit "weiterleiten" und damit dort auslesen. Dabei bleibt die gekodete Quanteninformation erhalten. Man nutzt dies dazu, das sogenannte Fehlersyndrom zu bestimmen um dann gezielt unitäre Korrektur-Operatoren durchzuführen (siehe Kapitel \ref{chap:QEC3}.). \cite[Seite 444-446]{nielsen_michael_a_and_isaac_l_chuang_quantum_2010}

Ein Schlüsselelement des fehlertoleranten Quantencomputings ist die Möglichkeit, einen Quantenzustand messen zu können, ohne dass er in seinen Ursprungszustand zu zwei Drittelwahrscheinlichkeit kollabiert. Dank dieser Fähigkeit können Codes konstruiert werden, welche die Integrität von Quanteninformation auch dann noch lange Zeit sicherstellen, wenn wiederkehrende Fehler laufend erkannt und ausgebessert werden.

\subsection{Redundanz durch Kodierung}
Quanten-Fehlerkorrektur beruht auf zentraler Idee, Quantenzustände vor Fehlern zu schützen, indem Redundanz eingefügt wird. Das bedeutet, ein einzelnes logisches Qubit wird nicht durch ein einziges physikalisches Qubit dargestellt, sondern auf mehrere physikalische Qubits verteilt. Genau durch diese zusätzliche Kodierungskapazität können Fehler eintreten, ohne dass dabei die logische Quanteninformation verloren geht.

Im Unterschied zur klassischen Fehlerkorrektur, bei der Redundanz verwendet wird (z.B. dreifache Paritätsbits), muss in der Quantenwelt solch ein Ansatz jedoch mit Bedacht gewählt werden. Denn das No-Cloning-Theorem verhindert, dass man auch einfach unbekannte Quantenzustände duplizieren kann. \cite{nielsen_michael_a_and_isaac_l_chuang_quantum_2010} Stattdessen wird die Kodierung mithilfe einer kohärenten Verschränkung mehrerer Qubits erreicht, sodass die Quanteninformation nicht in einem einzelnen Qubit, sondern in einem höherdimensionalen Unterraum des Gesamtsystems gespeichert wird.

Ein bekanntes Beispiel ist der Shor-Code, bei dem ein logisches Qubit auf neun physikalische Qubits abgebildet wird. Dieser Code schützt sowohl vor Bit-Flip- als auch vor Phase-Flip-Fehlern, indem er erst eine dreifache Kopie erstellt, um Bit-Flips zu entdecken, dann jede dieser Kopien nochmals mit einer Phase-Flip-Korrektur kodiert. \cite{shor_scheme_1995}
Ebenso gut ist der Steane-Code, der sieben Qubits zum Darstellen eines logischen Qubits verwendet und den klassischen (7, 4, 3)-Hamming-Code verwendet. \cite{Steane}

Mit Hilfe einer solchen Kodierung kann Quanteninformation in einem sogenannten Code-Raum abgelegt werden, der durch eine Gruppe von Stabilisator-Operatoren definiert ist. Tritt ein Fehler auf, so wird der Zustand in einen orthogonalen Unterraum verschoben, der außerhalb des Code-Raums liegt. Dieses Umschichten lässt sich über die Syndrommessung identifizieren – und zwar ohne, dass der ursprüngliche Zustand vermessen wird. Wesentlich ist, dass die Redundanz dabei hilft, Fehler zu lokalisieren und Korrekturoperationen durchzuführen, ohne die Quantenkohärenz zu stören.

Die Redundanz durch Kodierung ist somit die Voraussetzung für alle Quanten-Fehlerkorrekturverfahren: Denn nur indem man Information auf viele Qubits verteilt, kann man dem System Fehler \(\)verzeihen, ohne dass es aus funktionalem Zerfallen lässt.

\subsection{Fehlerzustände müssen unterscheidbar sein}
Ein entscheidendes Konzept in der Quantenfehlerkorrektur ist, dass Fehler auf einfache Weise voneinander unterscheidbar sein müssen. Nur wenn Fehler Zustände auf unterschiedliche Weise beschädigen, hat man eine Chance, festzustellen welcher der Fehler aufgetreten ist, mit dem Ziel diesen gezielt zu korrigieren ohne die Quanteninformation zu verlieren.

In der Sprache der Quanteninformatik bedeutet dies, dass die durch Fehler erstellten Zustände orthogonal zueinander sein müssen, beziehungsweise so aufgebaut, dass man durch Messung eines fehlerhaften Zustands auf den Code-Raum schließen kann. Andernfalls besteht die Gefahr, dass Fehler, die zu gleichem Messausgang führen, nicht unterschieden werden können und die Messung zu einer falschen Korrektur und dem Verlust von Information führt. \cite[Seite 449–451]{nielsen_michael_a_and_isaac_l_chuang_quantum_2010}

Dieses Konzept wird durch das sogenannte Fehlerkorrekturkriterium mathematisch präzisiert. Für eine Menge von Fehlern \(\left\{E_{i}\right\}\) , die von einem Quantencode korrigiert werden sollen, gilt:
\begin{equation}
    \left\langle\psi_{a}\right| E_{i}^{\dagger} E_{j}\left|\psi_{b}\right\rangle=C_{i j} \delta_{a b}
\end{equation}
für alle Zustände \(
    \left|\psi_{a}\right\rangle,\left|\psi_{b}\right\rangle
\)  im Code-Raum. Hierbei hängt die Konstante \(
    C_{i j}
\) vom Fehler,  nicht jedoch vom kodierten Zustand ab. Dieses Kriterium ist notwendig um sicherzustellen, dass die Wirkung eines Fehlers unabhängig von der gespeicherten Information eindeutig bestimmbar ist.

Aber aufgepasst: Wenn zwei unterschiedliche Fehler denselben Effekt auf den Code haben – also nicht unterscheidbar sind –, kann keine gezielte Korrektur erfolgen. Die Fähigkeit zur Unterscheidung solcher Fehlerzustände ist also die Voraussetzung für \textit{irgendeine} syndrombasierte Fehlerkorrektur. Nur dadurch können beispielsweise Look-Up-Tabellen oder automatische Fehlerkorrekturmechanismen sicher die \textit{einzig mögliche} Korrektur wählen.

Das Prinzip hat übrigens eine Parallele zur Konstruktion von Stabilizer-Codes, bei denen unterschiedliche Fehler unterschiedliche Syndrommuster hinterlassen. Durch geschickte Wahl der Stabilisatorgruppe kann man erreichen, dass in einem Stabilizer-Code alle Fehler innerhalb der Toleranzgrenze eindeutige Syndrome haben. \cite{gottesmann Stabilizer Codes}

\subsection{Reversibilität}
Ein entscheidendes Prinzip der Quanten-Fehlerkorrektur ist es, dass sämtliche eingesetzten Operationen im Fehlerkorrekturprozess umkehrbar sein müssen. In der Quantenmechanik entsprechen sämtliche möglichen zeitlichen Entwicklungen von isolierten Systemen unitären Transformationen – also Operationen, die völlig umkehrbar sind und keine Information vernichten. Diese Forderung gilt es auch für die Fehlerkorrektur durchzusetzen, da eine unwiderrufliche Fehlerkorrektur notwendigerweise den kodierten Quantenzustand zerstören und die Kohärenz der Quanteninformation verlieren würde. \cite[Seite 450-451]{nielsen_michael_a_and_isaac_l_chuang_quantum_2010}

Anders ausgedrückt: Wenn bei einer Kette von Ereignissen ein Fehler durch das Syndrom erkannt wurde, erfolgt ihre Korrektur durch eine bestimmte, unitäre Operation, welche uns den Fehlerzustand wieder ins Urteil des Codes zurückführt. Ist dieser Zustand erreicht, ist unser Fehler nicht nur korrigiert, sondern auch so, dass wir ihn\textbf{ kohärent rekonstruieren} können – und dies ist für weiteres Quantenrechnen und für fehlerresistente Quantenrechner notwendig.

Dieses Erfordernis der Rückkehr zur Ausgangslage besitzt auch eine sehr klare und bequeme theoretische Beschreibung: Das sogenannte Nicht-Trivialitätskriterium für Fehlerkorrektur sagt uns, dass für jede nichttriviale Fehlerwirkung \(E_i\) eine zugehörige Korrekturoperation \(R_i\) existiert, so dass 
\begin{equation}
    R_{i} E_{i}|\psi\rangle=|\psi\rangle \quad \text { für alle }|\psi\rangle \in \mathcal{C}
\end{equation}
Die Fehlererkennung und -korrektur insgesamt bildet somit einen Prozess, bei dem der Quanteninformationsträger lediglich entlang eines Umwegs durch Code-Hilfsraum entlanggeführt wird und die Logikqubit-Effektivität erhalten bleibt, um in der Lage zu sein, auch weitere Quantengatter auf diese anzuwenden.

Darum unterscheidet sich die Quanten-Fehlerkorrektur deutlich z.B. von klassischen Korrekturverfahren, welche nicht unbedingt reversibel sein müssen. In der Quanteninformatik wiederum ist Reversibilität jedoch ein \textbf{fundamentales physikalisches Prinzip}, das sowohl theoretisch als auch bei der Praxis der Implementierung beachtet werden muss.

\subsection{Fehlerlokalisierung}
Ein weiteres wichtiges Konzept, das hinter der Quanten-Fehlerkorrektur steckt, ist das der Fehlerlokalisierung. Damit ist gemeint, dass ein aufgetretener Fehler am besten so gefunden und behandelt werden kann, dass er sowohl räumlich als auch logisch eng eingegrenzt ist, und dass seine Folgen sich nicht unkontrolliert über das gesamte System ausbreiten. Nur wenn ein Fehler örtlich identifiziert und korrigiert werden kann, ist eine skalierbare und robuste Fehlerkorrektur möglich. \cite[Seite 451-452]{nielsen_michael_a_and_isaac_l_chuang_quantum_2010}

Fehler können in einem Quantencomputer verschiedene Ursprünge haben – z. B. Dekohärenz, Wechselwirkung mit der Umwelt oder Fehler bei Gates. Nach dem Prinzip der Fehlerlokalisierung sollte jeder dieser Fehler anfänglich lediglich eine begrenzte Anzahl physikalischer Qubits beeinflussen. Diese Voraussetzung soll es ermöglichen, mittels sogenannter Syndrommessungen zu erkennen, wo ein Fehler aufgetreten ist, ohne dabei den globalen Quantenzustand zu zerstören.

Dieses Konzept ist eng mit der Verwendung sogenannter lokaler Fehlerkorrektur-Codes wie z. B. topologischen Codes wie dem Surface Code verbunden. Bei diesen Codes sind die Qubits in einer regelmäßigen, zweidimensionalen Gitteranordnung angeordnet, bei der jeder Stabilizer-Operator nur Qubits betrachtet, die benachbart sind. Ein auftretender Fehler produziert ein lokales Syndrom, das sich aus den Nachbarn des fehlerhaften Qubits ergibt. Dadurch kann der Fehler nicht nur gefunden, sondern auch eingegrenzt und so gezielt korrigiert werden. \cite{Fowler et al}

Insbesondere für große, skalierbare Quantenrechner ist die Fähigkeit zur Fehlerlokalisierung essentiell: Denn ohne diese müssten sämtliche Qubits miteinander verschränkt und abgefragt werden, was einen exponentiellen Kontroll- und Ressourcenaufwand bedeuten würde. Bei lokalisierbaren Fehlern hingegen können lineare oder sogar konstante Skalierungsstrategien verfolgt werden – wodurch fehlertolerante Quantencomputer in greifbare Nähe rücken.

Zusammengefasst: Fehlerlokalisierung sorgt dafür, dass weder einzelne Fehler den gesamten Code-Raum beeinträchtigen, noch langfristige Korrekturprozesse destabilisieren. Sie ist damit eine notwendige Voraussetzung für praktische, modulare und skalierbare Quanten-Architekturen.

\subsection{Fehlerschwelle (Threshhold Theorem)}
Ein fundamentaler Mechanismus, um fehlertolerante Quantencomputer auch technisch umsetzen zu können, ist das Fehlerschwellenprinzip. Dieses besagt, dass sich Quanten-Fehlerkorrektur auf lange Sicht nur dann lohnt, wenn bei den einzelnen Qubits und Operationen die Wahrscheinlichkeit eines physikalischen Fehlers unter einem bestimmten Schwellenwert liegt, der als Fehlerschwelle bezeichnet wird. Wird die Schwelle unterschritten, so können beliebig exakte Rechnungen dadurch durchgeführt werden, dass man die Fehlerkorrektur immer wieder einsetzt – selbst wenn fortwährend physikalische Fehler passieren. \cite[Seite 454-456]{nielsen_michael_a_and_isaac_l_chuang_quantum_2010}

Dieses Prinzip wurde im Rahmen des Threshold Theorems auch rigoros bewiesen \cite{Aharonov und Ben-Or}. Das Theorem besagt, dass bei einer hinreichend kleinen Fehlerrate jedes Quanten-Computing mit beliebiger Genauigkeit durchgeführt werden kann – vorausgesetzt, es gibt genügend Redundanz und Korrekturschritte. Oder anders ausgedrückt: wenn Fehler selten genug und die Möglichkeit zur Fehlerkorrektur groß genug sind, dann können Fehler „überlebt“ werden.

Der genaue Wert der Fehlerschwelle hängt vom gewählten Code, der Architektur und der Art der Fehler ab. Er beträgt typischerweise  Wert zwischen \(
    10^{-4}
\)  und \(
    10^{-2}
\) und zählt derzeit Surface Codes zu den robustesten Verfahren. Deren experimentelle Fehlerschwelle liegt bei etwa \(1 \%\) für 
ein Gatemodell. \cite{Fowler et al.}

Von zentraler Bedeutung ist das Fehlerschwellenprinzip für die Entwicklung von Quantenhardware und -architektur, da es ermöglicht Fehler unterhalb einer beherrschbaren Schwelle und nicht als absolutes Hindernis aufzufassen. Es ist damit im Grunde eine theoretische Grundlage für das fehlertolerante Quantencomputing: Man kann Rechner auch mit unzuverlässigen Bauteilen aufbauen, solange die Wahrscheinlichkeit von Fehlern unterhalb der Fehlerschwelle liegt.

\section{Praktische Realisierung der Fehlerkorrektur}\label{chap:QEC3}

Dieses Kapitel konzentriert sich auf die praktische Umsetzung der Fehlerkorrektur.
Zuerst betrachten wir einige grundlegende Code-Beispiele, die auf den in Kapitel \ref{chap:QEC2} erläuterten Prinzipien aufbauen (Abschnitt \ref{chap:QEC3.1}). Anschließend wird der Fehlerkorrekturzyklus detailliert am Beispiel der vielversprechenden Oberflächencodes dargestellt (Abschnitt \ref{chap:QEC3.2}). Abschließend werden aktuelle Schwierigkeiten auf dem Weg zu fehlertoleranten Quantencomputern diskutiert und ein Ausblick auf zukünftige Entwicklungen gegeben (Abschnitt \ref{chap:QEC3.3}). Der Fokus liegt dabei auf Aspekten der praktischen Umsetzung sowie den damit verbundenen Problemen und Lösungen.


\subsection{Anwendung von QEC-Prinzipien: Erste Code-Beispiele}\label{chap:QEC3.1}

Wie in Abschnitt \ref{chap:QEC1} ausführlich dargestellt wurde, sind Qubits anfällig für diverse
Fehler, und wie in Abschnitt \ref{chap:QEC2} erläutert, erfordern die grundlegenden Prinzipien der
Quantenmechanik (wie das No-Cloning-Theorem und das Messproblem) spezielle
Strategien zur Fehlerkorrektur. Anstatt diese Prinzipien hier erneut zu vertiefen,
konzentrieren wir uns darauf, wie sie in ersten, einfachen Fehlerkorrekturcodes
praktisch angewendet werden. Diese Codes verdeutlichen die grundlegende Idee,
logische Information durch Kodierung über mehrere physikalische Qubits und durch
Syndrommessungen zu schützen.
Erste historische Codes demonstrieren diese Prinzipien exemplarisch. Insbesondere betrachten wir den Shor-Code (9-Qubit-Code) und den Steane-Code (7-Qubit-Code) als frühe Beispiele, sowie den einfachen 3-Qubit-Code, um die Funktionsweise und mathematischen Grundlagen der QEC zu verdeutlichen.
\cite[Seite 427-430]{nielsen_quantum_2010}\\

\paragraph{3-Qubit-Wiederholungscode (Bit-Flip Code):}
Als einfachstes Beispiel dient der 3-Qubit-Wiederholungscode, der gegen Bit-Flips schützt. Hier wird ein logisches Qubit auf drei physische Qubits verteilt, z.\,B.
\[
\lvert 0_L \rangle = \lvert 000 \rangle, \quad \lvert 1_L \rangle = \lvert 111 \rangle.
\]
Durch diese Redundanz kann ein einzelner Bitfehler erkannt werden, ohne den Logikzustand direkt zu messen. Man misst dazu sogenannte Paritätsoperatoren (Stabilisatoren), die nur verraten, ob unter den drei Qubits ein Ungleichgewicht vorliegt, aber nicht, was das logische Bit ist. Konkret prüft man, ob alle drei Qubits gleich sind (\(\lvert 000 \rangle\) oder \(\lvert 111 \rangle\)) oder ob eines abweicht. Diese Syndrommessungen liefern einen Fehlerindikator (z.\,B. welches Qubit anders ist), ohne die Superposition zu zerstören. Anschließend kann durch eine entsprechende Pauli-\(X\)-Operation der Fehler rückgängig gemacht werden.

Der 3-Qubit-Code illustriert somit die Prinzipien der Fehlererkennung und -korrektur: Die Fehler sind durch orthogonale Syndrome unterscheidbar, und die Korrekturoperation (Bit-Flip auf dem identifizierten fehlerhaften Qubit) stellt den ursprünglichen Zustand wieder her (Reversibilität). Allerdings kann dieser einfache Code nur einen Bit-Flip-Fehler korrigieren und versagt, falls zwei oder mehr Qubits gleichzeitig flippen.
\cite[Seite 430-431]{nielsen_quantum_2010}\\

\paragraph{3-Qubit-Wiederholungscode (Phasenflip-Code):}\label{chap:QEC1.4}

Wie in Abschnitt~\ref{chap:QEC1.3} beschrieben, wirkt ein Phasenfehler durch Anwendung des Pauli-\(Z\)-Operators auf einen Qubit-Zustand. Dabei bleibt \(|0\rangle\) unverändert, während \(|1\rangle\) in \(-|1\rangle\) übergeht:
\[
Z(a|0\rangle + b|1\rangle) = a|0\rangle - b|1\rangle
\]
Ein solcher Vorzeichenwechsel ist in der Standardbasis \(\{|0\rangle, |1\rangle\}\) nicht direkt messbar. Betrachtet man jedoch den Zustand in der sogenannten \emph{Hadamard-Basis}, also in der Basis \(\{|+\rangle, |-\rangle\}\) mit
\[
|+\rangle = \frac{1}{\sqrt{2}}(|0\rangle + |1\rangle), \quad |-\rangle = \frac{1}{\sqrt{2}}(|0\rangle - |1\rangle),
\]
so wirkt der Phasenfehler wie ein klassischer Bitflip:
\[
Z|+\rangle = |-\rangle, \quad Z|-\rangle = |+\rangle
\]
Diesen Zusammenhang macht man sich zunutze, indem man einen Bitflip-Code in der Hadamard-Basis anwendet (vgl.\ Nielsen und Chuang, 2010, S.~430–431; Devitt et al., 2013, S.~4). Die logischen Zustände des Phasenflip-Codes lauten:
\[
|0_L\rangle = |+++\rangle, \quad |1_L\rangle = |---\rangle
\]
Ein einzelner Phasenfehler auf einem der drei Qubits kehrt den lokalen Zustand von \(|+\rangle\) zu \(|-\rangle\) um (oder umgekehrt), wodurch mithilfe von Paritätsmessungen (Syndromen) analog zum klassischen Wiederholungscode erkannt werden kann, welches Qubit betroffen ist. Eine anschließende \(Z\)-Operation korrigiert den Fehler, ohne die Superposition zu zerstören.\\

\paragraph{Shor-Code (9-Qubit-Code):}

Peter Shor entwickelte 1995 den ersten Quantenfehlerkorrekturcode, der beliebige Ein-Qubit-Fehler – Bit-Flip (\(X\)), Phasen-Flip (\(Z\)) oder eine Kombination daraus (\(Y = iXZ\)) – korrigieren kann. Der Code kombiniert zwei einfache Schutzmechanismen: einen gegen Bitflips und einen gegen Phasenflips.

Ein logisches Qubit wird dabei auf neun physikalische Qubits verteilt, indem zwei Schutzmechanismen nacheinander kombiniert werden:  
Zunächst wird der Zustand mithilfe eines Bit-Flip-Codes dreifach wiederholt – das schützt gegen Bitfehler. Anschließend wird diese Dreifachstruktur nochmals kodiert, sodass auch Phasenfehler erkennbar werden. Dadurch ergibt sich ein kodierter Zustand, der sich als verschränkte Superposition folgender 9-Qubit-Zustände schreiben lässt:
\[
\begin{aligned}
|0_L\rangle &= \frac{1}{2\sqrt{2}} \left( |000\rangle + |111\rangle \right) \otimes \left( |000\rangle + |111\rangle \right) \otimes \left( |000\rangle + |111\rangle \right) \\
|1_L\rangle &= \frac{1}{2\sqrt{2}} \left( |000\rangle - |111\rangle \right) \otimes \left( |000\rangle - |111\rangle \right) \otimes \left( |000\rangle - |111\rangle \right)
\end{aligned}
\]
(vgl.\ Devitt et al., 2013, S.~9)\\\\
Die logischen Zustände \(|0_L\rangle\) und \(|1_L\rangle\) bestehen somit jeweils aus drei Blöcken, in denen sich das Muster \(|000\rangle \pm |111\rangle\) wiederholt. Die Addition bzw. Subtraktion erzeugt eine Superposition, die sowohl Bitflips als auch Phasenflips innerhalb eines Blocks sichtbar macht.

Um Fehler zu erkennen, misst man sogenannte Stabilizer – spezielle Operatoren, die anzeigen, ob und wo ein Fehler aufgetreten ist, ohne den gespeicherten Zustand direkt zu zerstören. So kann z.\,B. ein einzelner Phasenfehler (siehe Abschnitt \ref{chap:QEC1.3}) durch einen Wechsel des Vorzeichens in einem Block wie \(|000\rangle + |111\rangle \mapsto |000\rangle - |111\rangle\) erkannt werden. Durch geeignete Syndrommessungen kann man anschließend den betroffenen Qubit identifizieren und den Fehler durch Anwendung eines Pauli-Operators (\(X\) oder \(Z\)) rückgängig machen.

Da ein \(Y\)-Fehler einer Kombination von \(X\) und \(Z\) entspricht, genügt es, beide Einzelfehlerarten korrigieren zu können. Der Shor-Code erfüllt dies und hat die Parameter \([[9,1,3]]\): Er speichert ein logisches Qubit in neun physischen und erkennt bis zu zwei Fehler, wobei ein einzelner immer korrekt korrigiert werden kann.

Auch wenn der Code in der Praxis aufwendig ist, war er ein Meilenstein: Er zeigte erstmals, dass sich Quanteninformation durch geschickte Kodierung und Messung zuverlässig schützen lässt. \cite[Seite 432–434]{nielsen_quantum_2010}\\

\paragraph{Steane-Code (7-Qubit-Code):}

Der 1996 von Andrew Steane entwickelte 7-Qubit-Code ist einer der bekanntesten Quantenfehlerkorrekturcodes. Er basiert auf einem klassischen Hamming-Code und gehört zur Familie der sogenannten CSS-Codes (Calderbank-Shor-Steane). Ziel ist es, Ein-Qubit-Fehler aller Art (Bitflips \(X\), Phasenflips \(Z\) und Kombinationen daraus) mit möglichst wenigen Qubits zu erkennen und zu korrigieren.

Der Steane-Code nutzt eine clevere Strategie: Er wählt bestimmte Bitmuster aus dem klassischen \([7,4,3]\)-Hamming-Code, die sich so stark voneinander unterscheiden, dass ein einzelner Fehler sie nicht ineinander überführen kann. Diese klassischen Wörter werden anschließend in eine Superposition überführt – so entsteht echte Quantenredundanz.

Die beiden logischen Zustände \(|0_L\rangle\) und \(|1_L\rangle\) bestehen jeweils aus einer Superposition von acht solchen 7-Bit-Wörtern. Alle ausgewählten Bitmuster erfüllen spezielle Paritätsbedingungen, die es erlauben, auftretende Fehler mit sogenannten Stabilizer-Messungen zu identifizieren und zu korrigieren:
\[
\begin{aligned}
|0_L\rangle &= \frac{1}{\sqrt{8}} \big( 
|0000000\rangle + |1010101\rangle + |0110011\rangle + |1100110\rangle + {} \\
&\quad |0001111\rangle + |1011010\rangle + |0111100\rangle + |1101001\rangle \big) \\
|1_L\rangle &= \frac{1}{\sqrt{8}} \big( 
|1111111\rangle + |0101010\rangle + |1001100\rangle + |0011001\rangle + {} \\
&\quad |1110000\rangle + |0100101\rangle + |1000011\rangle + |0010110\rangle \big)
\end{aligned}
\]

Durch diese Struktur lässt sich mit nur sieben Qubits ein vollständiger Schutz gegen Ein-Qubit-Fehler erreichen – genau wie beim Shor-Code, aber effizienter.\\

Die Zustände \(|0_L\rangle\) und \(|1_L\rangle\) bestehen aus Superpositionen scheinbar willkürlicher Bitmuster wie \(|1010101\rangle\) oder \(|1100110\rangle\). Tatsächlich sind diese Muster jedoch gezielt gewählt:

\begin{itemize}
  \item Sie stammen aus dem klassischen \([7,4,3]\)-Hamming-Code, der bekannte Fehlerkorrektureigenschaften hat.
  \item Die Bitfolgen erfüllen bestimmte Paritätsbedingungen und sind so konstruiert, dass man bei einem einzelnen Bitflip noch erkennen kann, welches Bit betroffen ist.
  \item Durch die Überlagerung mehrerer solcher gültiger Codewörter (Superposition) entsteht ein Quantenfehlerkorrekturcode, der sowohl Bit- als auch Phasenfehler erkennen kann.
\end{itemize}

\noindent
Diese scheinbar komplizierten Muster sind also notwendig, um mit nur sieben Qubits vollständige Ein-Qubit-Fehlerkorrektur zu ermöglichen.
\cite[Seite 12]{devitt_qec_2013}\\

\paragraph{Ausblick auf fortgeschrittene Codes:}

Die bisher vorgestellten Codes – wie der Drei-Qubit-Code, der Shor-Code und der Steane-Code – zeigen, wie sich einzelne Bit- oder Phasenfehler gezielt erkennen und korrigieren lassen. Während sie bereits vollständige Quantenfehlerkorrektur auf einem logischen Qubit ermöglichen, stoßen sie bei wachsender Qubit-Anzahl und praktischer Realisierung an ihre Grenzen.

Für den Aufbau skalierbarer Quantencomputer werden daher weiterentwickelte Codefamilien benötigt, die neben der Korrektur beliebiger Fehler auch architektonische Vorteile bieten – etwa durch lokale Wechselwirkungen und effiziente Layouts. Ein prominentes Beispiel dafür sind die Oberflächencodes, die im folgenden Abschnitt behandelt werden. Sie beruhen auf denselben Grundprinzipien, setzen jedoch auf topologische Strukturen und fehlertolerante Operationen im Gitter.


\subsection{Der Fehlerkorrekturzyklus am Beispiel von Oberflächencodes}\label{chap:QEC3.2}

Der Oberflächencode basiert auf einem zweidimensionalen Gitter aus physikalischen Qubits, typischerweise in Form eines rechteckigen Flächenstücks (\emph{Patch}). Dabei unterscheidet man zwei Arten von Qubits: \emph{Daten-Qubits}, welche die eigentliche logische Quanteninformation tragen, und \emph{Mess-Qubits} (auch \emph{Ancilla-Qubits} genannt), die zur Fehlererkennung verwendet werden.

Man kann sich die Struktur wie ein Schachbrett vorstellen: Die Daten-Qubits liegen auf Kanten des Gitters, während die Mess-Qubits zwischen ihnen positioniert sind. Diese Mess-Qubits überwachen jeweils vier benachbarte Daten-Qubits. Dabei gibt es zwei Typen von Stabilizer-Operatoren:
\begin{itemize}
  \item \textbf{\(Z\)-Stabilizer} (auch \emph{Plaquette-Operatoren}\footnote{Plaquettes bezeichnen kleine quadratische Flächen im Gitter. Ein \(Z\)-Stabilizer wirkt auf die vier Daten-Qubits an den Ecken einer solchen Plaquette und erkennt Bitfehler.}) überprüfen die Parität in Bezug auf Bit-Flip-Fehler.
  \item \textbf{\(X\)-Stabilizer} (auch \emph{Stern-Operatoren}\footnote{Stern-Operatoren wirken auf die vier Daten-Qubits, die an einem Gitterpunkt zusammenlaufen, und erkennen Phasenfehler.}) erkennen Phasenfehler.
\end{itemize}

\paragraph{Syndrommessung und Fehlererkennung:}

Die Fehlerkorrektur erfolgt zyklisch. In jedem Zyklus werden alle \(X\)- und \(Z\)-Stabilizer gemessen. Dazu wird jedes Mess-Qubit mit den vier zugehörigen Daten-Qubits durch eine Reihe kontrollierter Gatter verschränkt, z.\,B. mit CNOT-Operationen. Nach dieser Verschaltung enthält das Mess-Qubit die Paritätsinformation: Bei \(Z\)-Stabilizern etwa, ob eine gerade oder ungerade Anzahl von \( |1\rangle \)-Zuständen in der Gruppe vorliegt. Anschließend wird das Mess-Qubit gemessen – in der \(Z\)-Basis für \(Z\)-Stabilizer und in der \(X\)-Basis für \(X\)-Stabilizer.

Das Ergebnis der Messung – 0 (Eigenwert \(+1\)) oder 1 (Eigenwert \(-1\)) – wird als \emph{Syndrom-Bit} bezeichnet. Es zeigt an, ob sich seit der letzten Messung ein Fehler ereignet hat. Wichtig ist: Diese Messungen beeinflussen den logischen Zustand nicht. Sie liefern nur Informationen über potenzielle Fehler, ohne die kodierte Quanteninformation direkt zu zerstören.

\paragraph{Decodierung und Fehlerlokalisierung:}

Die Menge aller Syndrom-Bits ergibt ein \emph{Syndrommuster}, aus dem auf die wahrscheinlichsten Fehler geschlossen werden kann. Ein einzelner Bitfehler an einem Daten-Qubit führt typischerweise dazu, dass zwei benachbarte \(Z\)-Stabilizer von 0 auf 1 wechseln. Analog dazu führt ein Phasenfehler zu Abweichungen bei zwei benachbarten \(X\)-Stabilizern.

Diese lokalen Muster bilden die Grundlage für die sogenannte \emph{Decodierung}: Ein Algorithmus analysiert die Verteilung der Fehlerhinweise und rekonstruiert den wahrscheinlichsten Fehlerpfad. Häufig wird dazu ein Graph aufgebaut, in dem Syndromabweichungen als Knoten erscheinen. Der Decoder sucht dann eine möglichst kurze Verbindung dieser Knoten – z.\,B. mithilfe eines Minimum-Weight-Perfect-Matching-Algorithmus. Auch modernere Verfahren wie Union-Find oder maschinelles Lernen kommen zum Einsatz.

Da auch Messungen fehlerhaft sein können, betrachtet man oft mehrere Zyklen hintereinander und baut einen 3D-Graphen mit der Zeitachse als dritter Dimension. Inkonstistente Syndrome über die Zeit hinweg deuten dann auf Messfehler hin.

\paragraph{Korrekturoperation:}

Nach der Decodierung steht eine Hypothese über die Art und Position der Fehler zur Verfügung. Man könnte nun direkt eine Pauli-Korrekturoperation auf das betroffene Qubit anwenden – etwa ein weiteres \(X\), um einen Bitflip rückgängig zu machen.

In der Praxis verwendet man häufig stattdessen ein sogenanntes \emph{Pauli-Frame}\footnote{Ein Pauli-Frame ist ein rein klassisches Register, in dem alle erkannten Fehler notiert werden. Die tatsächliche physische Korrektur wird dadurch vermieden und stattdessen bei der Interpretation der Messdaten berücksichtigt. So spart man potenziell fehleranfällige Operationen.}. Dabei wird die Fehlerinformation nicht physisch korrigiert, sondern nur logisch „mitgeführt“ und am Ende beim Auslesen berücksichtigt. Das spart Ressourcen und reduziert zusätzliche Fehlerquellen.

Nach Abschluss eines Fehlerkorrekturzyklus beginnt sofort der nächste – denn in einem realen Quantencomputer können ständig neue Fehler auftreten. Die kontinuierliche Wiederholung des Zyklus ermöglicht daher langfristig stabile und fehlertolerante Quanteninformation.\\

\paragraph{Einfacher Überblick: So funktioniert ein Fehlerkorrekturzyklus im Oberflächencode}Damit ein Quantencomputer trotz fehleranfälliger Qubits zuverlässig funktioniert, wird in kurzen Abständen automatisch überprüft, ob Fehler aufgetreten sind. Der Ablauf sieht dabei so aus:

\begin{enumerate}
  \item \textbf{Start:}  
  Alle Qubits werden in bekannte Anfangszustände gebracht. Die sogenannten Mess-Qubits sind bereit, nach Fehlern zu suchen.

  \item \textbf{Fehlersuche:}  
  Die Mess-Qubits „fragen“ ihre Nachbarn (die Daten-Qubits), ob alles in Ordnung ist. Aus ihren Antworten entsteht ein Muster aus Nullen und Einsen – das sogenannte \emph{Syndrom}.

  \item \textbf{Auswertung:}  
  Ein klassischer Computer schaut sich das Syndrom an und erkennt daraus, wo wahrscheinlich ein Fehler passiert ist.

  \item \textbf{Reaktion:}  
  Der Computer entscheidet, ob und wie ein Fehler korrigiert werden muss. Oft merkt man sich den Fehler einfach und berücksichtigt ihn später automatisch.

  \item \textbf{Neustart:}  
  Der nächste Zyklus beginnt sofort. So werden ständig neue Fehler früh erkannt – auch während der Quantencomputer rechnet.
\end{enumerate}

\subsection{Aktuelle H\"urden und was die Zukunft bringt}\label{chap:QEC3.3}

Trotz erster erfolgreicher Implementierungen steht die Quantenfehlerkorrektur (QEC) noch am Anfang ihrer praktischen Nutzbarkeit. Der Weg zu einem fehlertoleranten Quantencomputer ist gepflastert mit technischen und konzeptionellen Herausforderungen. In diesem Abschnitt beleuchten wir die zentralen H\"urden und geben einen Ausblick auf vielversprechende Entwicklungen.\\

\paragraph{Qubit-Qualit\"at und Fehlerschwellen} 
Ein zentrales Problem ist die Qualit\"at der physikalischen Qubits selbst. Quantenfehlerkorrektur wirkt nur, wenn die zugrunde liegenden Fehlerraten hinreichend niedrig sind. Theoretisch existiert eine Fehlerschwelle, unterhalb derer Redundanz die Zuverl\"assigkeit erh\"oht\footnote{Siehe das Threshold-Theorem in Abschnitt~\ref{chap:QEC2}}. Praktisch liegt diese Schwelle typischerweise bei Fehlerraten $\lesssim 10^{-3}$. Viele Architekturen, wie supraleitende Qubits oder Ionenfallen, erreichen in Labordemonstrationen bereits zweiqubit-Gates mit Fehlerraten im niedrigen Promillebereich, was vielversprechend ist. Doch um lange Quantenalgorithmen (z.B. Faktorisierung mit Shor bei $n=2048$) fehlerfrei auszuf\"uhren, m\"ussen logische Qubits \emph{stundenlang} koh\"arent bleiben -- ein Ziel, das nur mit sehr stabilen physikalischen Qubits erreichbar ist. Korrelierte Fehler, Leakage und seltene St\"orereignisse (z.B. kosmische Teilchen) stellen dabei zus\"atzliche Risiken dar.

\paragraph{Massiver Qubit-Overhead}
Fehlerkorrektur ist teuer: Selbst einfache Codes wie der Oberfl\"achencode ben\"otigen $O(d^2)$ physikalische Qubits f\"ur ein logisches Qubit mit Distanz $d$. Realistische Anwendungen erfordern daher \emph{Millionen} physikalischer Qubits. Derzeitige Systeme bewegen sich im Bereich von wenigen Hundert bis Tausend Qubits -- die Skalenl\"ucke ist also noch immens. Jeder weitere Qubit bringt nicht nur Nutzen, sondern auch neue Herausforderungen: Verkabelung, Kalibrierung und thermische Isolation m\"ussen f\"ur jedes Qubit mitgedacht werden.

\paragraph{Technische Limitierungen bei Kryogenik und Auslese}
Besonders supraleitende Systeme m\"ussen bei 10--20\,mK betrieben werden. Hier stoßen Forscher schnell an Grenzen: Die K\"uhlung gro\ss er Quantenprozessoren erfordert aufw\"andige Verd\"unnungsk\"uhler, und die Zahl der nach innen f\"uhrenden Kabel wird schnell zum Flaschenhals. Auch die Kontrolle \"uber klassische Elektronik ist herausfordernd: Steuereinheiten befinden sich oft au\ss erhalb des K\"uhlsystems und m\"ussen mit tausenden Kan\"alen kommunizieren -- ein thermisches und technisches Problem. Ans\"atze wie Cryo-CMOS (Steuerelektronik bei tiefen Temperaturen) werden intensiv erforscht.

\paragraph{Dekodierung: Komplex und latenzkritisch}
Selbst wenn alle Syndromdaten vorliegen, ist die Decodierung (Fehlerdiagnose und Korrekturvorschlag) ein nichttriviales Problem. Bei QEC-Zyklen im Mikrosekundenbereich und Millionen von Qubits entstehen Datenraten im Bereich hunderter Terabyte pro Sekunde. Diese m\"ussen in Echtzeit verarbeitet werden, idealerweise mit Latenzen $<1\,\mu$s. Daher ben\"otigt man spezialisierte Hardware-Decoder\footnote{Siehe z.B. Riverlane's Decodierchips oder FPGA-basierte Systeme.} und ausgekl\"ugelte Algorithmen wie Minimum Weight Matching oder Machine-Learning-gest\"utzte Ans\"atze.

\paragraph{Nicht-ideale Fehler und Modellabweichungen}
Die meisten QEC-Codes beruhen auf idealisierten Fehlermodellen (z.B. unkorrelierte, Pauli-artige Fehler). In realen Systemen treten jedoch korrelierte Fehler, Kaskadeneffekte und Leakage auf, die nicht leicht mit bestehenden Codes zu behandeln sind. Solche Abweichungen senken die effektive Fehlerschwelle und machen eine robuste Fehlerkorrektur schwieriger.\\

\paragraph{Ausblick und neue Ans\"atze}
Die Forschung reagiert mit vielf\"altigen Strategien:
\begin{itemize}
    \item \textbf{Hardwareverbesserung:} Längere Kohärenzzeiten, stabilere Qubits und besseres Materialdesign sind zentrale Entwicklungsziele.
    \item \textbf{Topologische Qubits:} Systeme wie Majorana-Moden versprechen inhärente Fehlerresistenz, sind aber noch experimentell.
    \item \textbf{Modulare Architekturen:} Mehrere kleine Qubit-Module könnten vernetzt werden, um skalierbare Systeme aufzubauen.
    \item \textbf{Neue Codes:} Quantum-LDPC-Codes, Floquet-Codes oder Color Codes bieten bessere Raten und Flexibilität, sind aber schwerer umzusetzen.
    \item \textbf{Ko-Design:} Eine enge Verzahnung von Hardware, Software und Fehlermodellen ist notwendig, um das Gesamtsystem zu optimieren.
\end{itemize}

\paragraph{Fazit} 
Die Quantenfehlerkorrektur hat sich in den letzten Jahren von einem theoretischen Konzept zu einem praktischen Forschungsfeld mit greifbaren Fortschritten entwickelt. Dennoch sind viele zentrale Herausforderungen ungel\"ost. Die kommenden Jahre werden entscheidend sein, um aus den heutigen Demonstratoren robuste, fehlertolerante Quantencomputer zu formen. Dabei ist klar: Ohne QEC ist skalierbares Quantencomputing unm\"oglich -- mit ihr aber wird es eines Tages m\"oglich sein, Probleme zu l\"osen, die klassisch unzug\"anglich bleiben.


\section{Praxisbeispiel: Ein einfacher Quantenfehlerkorrekturcode}
\begin{itemize}
    \item Mit Qiskit Library und qiskit.providers.aer Rauschen simulieren und die Notwendigkeit von Fehlerkorrektur anhand eines Vergleichs aufzeigen
    \item Verschiedene Algorithmen zur Fehlerminimierung verwenden
    \item Verschiedene Rauschen und Ansätze vergleichen und kontrastisieren
    \item Qiskit Visualization Library erlaubt Darstellung von Zusammenhängen
\end{itemize}



\printbibliography
