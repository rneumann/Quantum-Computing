%\motto{Use the template \emph{chapter.tex} to style the various elements of your chapter content.}
\chapter{Fehlerkorrektur}
\label{error_correction} % Always give a unique label
% use \chaptermark{}
% to alter or adjust the chapter heading in the running head

\chapterauthor{Niklas Bodfeld, Tim Boschert, Manuel Meixner, Ann-Kathrin Wenzel}

\abstract{some abstract}

\section{Fehlertypen in Quantenrechnern}\label{chap:QEC1}

\subsection{Relevanz von Fehlern in Quantenrechnern}
\begin{itemize}
\item Überblick
\item Woher kommen Fehler
\item Wie wirken diese sich aus
\item Wo passieren sie
\end{itemize}

\subsection{Physikalische Fehlerursachen in Quantenrechnern}
\begin{itemize}
\item Dekohärenz
\item Phasendekohärenz
\item Amplitudendekohärenz 
\item Rauschen
\item Unvollständige Isolation 
\item Vibration-
\item Umwelteinflüsse
\end{itemize}

\subsection{Klassifizierung von Quantenfehlern}
\begin{itemize}
\item Bit-Flip
\item Phase-Flip
\item Bit und Phasenflip
\item Amplitudendämpfung
\item Phasendämpfung
\end{itemize}



\subsection{Hardwarebedingte Fehler und Systematische Grenzen}
\begin{itemize}
\item Gatteroperationen
\item Messung falsch
\item Qubitinitalisierung
\item Konnektivität
\item Kalibrierfehler
\end{itemize}

\subsection{Quantifizierung und Modellierung von Quantenfehlern}
\begin{itemize}
\item Fehlerkanäle
\item Krausoperation
\item Fehlerrate
\item wie werden Fehler gemessen
\end{itemize}

\subsection{Auswirkungen von Fehlern auf Quantenalgorithmen und Systemarchitektur}
\begin{itemize}
\item Rechenabbrüche
\item Bedarf an Fehlertoleranz
\item Fehlerbudget
\end{itemize}


\section{Grundprinzipien in Quantenfehlerkorrektur}\label{chap:QEC2}
\begin{itemize}
\item Fehlererkennung ohne Zustandsmessung
\item Redundanz durch Kodierung
\item Fehlerzustände müssen unterscheidbar sein
\item Reversibilität
\item Fehlerlokalisierung
\item Fehlerschwelle (Threshhold Theorem)
\end{itemize}


\section{Praktische Realisierung der Fehlerkorrektur}\label{chap:QEC3}

Dieses Kapitel konzentriert sich auf die praktische Umsetzung der Fehlerkorrektur (Quantum Error Correction, QEC).\\
Zuerst betrachten wir einige grundlegende Code-Beispiele, die auf den in Kapitel 5.2 erläuterten Prinzipien aufbauen (Abschnitt 5.3.1). Anschließend wird der Fehlerkorrekturzyklus detailliert am Beispiel der vielversprechenden Oberflächencodes dargestellt (Abschnitt 5.3.2). Abschließend werden aktuelle Schwierigkeiten auf dem Weg zu fehlertoleranten Quantencomputern diskutiert und ein Ausblick auf zukünftige Entwicklungen gegeben (Abschnitt 5.3.3). Der Fokus liegt dabei auf Aspekten der praktischen Umsetzung sowie den damit verbundenen Problemen und Lösungen.

\subsection{Anwendung von QEC-Prinzipien: Erste Code-Beispiele}

Wie in Abschnitt 5.1 ausführlich dargestellt wurde, sind Qubits anfällig für diverse Fehler, und wie in Abschnitt 5.2 erläutert, erfordern die grundlegenden Prinzipien der Quantenmechanik (wie das No-Cloning-Theorem und das Messproblem) spezielle Strategien zur Fehlerkorrektur. Anstatt diese Prinzipien hier erneut zu vertiefen, konzentrieren wir uns darauf, wie sie in ersten, einfachen Fehlerkorrekturcodes praktisch angewendet werden. Diese Codes verdeutlichen die grundlegende Idee, logische Information durch Kodierung über mehrere physikalische Qubits und durch Syndrommessungen zu schützen.

Um zu zeigen, wie die in Abschnitt 5.2 besprochenen Grundideen von Redundanz und Syndrommessung in der Praxis funktionieren, können wir uns einfache Fehlerkorrekturcodes ansehen. Der Drei-Qubit-Bitflip-Code ist so ein Beispiel. Hier wird ein logisches Qubit, z.B. $\alpha|0\rangle_L + \beta|1\rangle_L$, als $\alpha|000\rangle+\beta|111\rangle$ auf drei physikalischen Qubits gespeichert. Tritt beispielsweise ein Bitflip-Fehler auf einem der drei physikalischen Qubits auf (z.B. wird aus $|000\rangle$ der Zustand $|100\rangle$), kann dieser mittels spezifischer Prüfoperationen – der Messung von Stabilisatoroperatoren wie $Z_1Z_2$ und $Z_2Z_3$ – erkannt werden. Diese Messungen, deren Grundprinzip in Abschnitt 5.2 erläutert wurde, verraten, ob benachbarte Qubits gleich sind oder nicht, ohne den gespeicherten logischen Zustand selbst preiszugeben. Das Ergebnis dieser Syndrommessung identifiziert das fehlerhafte Qubit, woraufhin ein entsprechender X-Operator zur Fehlerkorrektur angewendet werden kann.\cite[Seite 427-430]{nielsen_quantum_2010}\\

Analog dazu existiert der Drei-Qubit-Phasenflip-Code, der Phasenfehler korrigieren kann. Hierbei wird der Zustand beispielsweise als $\alpha\ket{+++}+\beta\ket{---}$ kodiert, wobei $\ket{+}=(\ket{0}+\ket{1})/\sqrt{2}$ und $\ket{-}=(\ket{0}-\ket{1})/\sqrt{2}$. Durch Anwendung von Hadamard-Transformationen vor und nach einem Kanal, der Phasenfehler verursacht, kann dieser Code Phasenfehler effektiv in Bitfehler umwandeln, die dann analog zum Bitflip-Code korrigiert werden können.\cite[Seite 430-431]{nielsen_quantum_2010}\\

Diese einfachen Codes können jedoch jeweils nur einen spezifischen Fehlertyp (entweder Bitflip oder Phasenflip) auf einem einzelnen Qubit korrigieren. Um allgemeine Fehler, d.h. beliebige Kombinationen von Bit- und Phasenflips, auf mehreren Qubits zu korrigieren, sind komplexere Codes notwendig. Beispiele hierfür sind der von Peter Shor entwickelte Shor-Code, der ein logisches Qubit in neun physikalischen Qubits kodiert und einen beliebigen einzelnen Fehler korrigieren kann, oder die bereits erwähnten Oberflächencodes, die aufgrund ihrer praktischen Vorteile im nächsten Abschnitt detaillierter behandelt werden. Diese komplexeren Codes bauen auf den gleichen Grundprinzipien auf, nutzen aber weiter entwickelte Kodierungs- und Syndrommessverfahren.\cite[Seite 432ff]{nielsen_quantum_2010}\\
Die Notwendigkeit spezifischer Quantenfehlerkorrekturverfahren und die Komplexität ihrer praktischen Umsetzung ergeben sich somit direkt aus den in Kapiteln ~\ref{chap:QEC1} und ~\ref{chap:QEC2} diskutierten quantenmechanischen Gegebenheiten. Die folgenden Abschnitte vertiefen die Realisierung anhand leistungsfähigerer Code-Familien


\subsection{Der Fehlerkorrekturzyklus mit am Beispiel von Oberflächencodes}

Oberflächencodes (Surface Codes) gelten aktuell als einer der besten Ansätze für die praktische Quantenfehlerkorrektur. Sie sind deshalb so interessant, weil sie relativ hohe Fehlerraten erlauben (wie durch das in \ref{chap:QEC2} erläuterte Threshold-Theorem vorhergesagt) und gut zu vielen gängigen Qubit-Bauweisen passen.\\

\textbf{Einführung in Oberflächencodes (Surface Codes)}



\begin{itemize}
    \item \textbf{Der QEC-Zyklus -- ein Überblick:} Kodieren $\rightarrow$ Fehler passieren $\rightarrow$ Fehler finden $\rightarrow$ Fehler deute $\rightarrow$ Fehler beheben.
    \item \textbf{1. Logische Qubits kodieren:}
    \item ...
    \item \textbf{2. Fehler aufspüren (Syndrommessung):}
    \item ...
    \item \textbf{3. Fehler interpretieren (Dekodierung):}
    \item ...
    \item \textbf{4. Fehler beheben (Korrekturoperation):}
    \item ...
\end{itemize}

\subsection{Aktuelle Hürden und was die Zukunft bringt}
\begin{itemize}
    \item \textbf{Herausforderungen \& Ausblick -- ein Überblick:} Großer Qubit-Bedarf, Zeitaufwand, interne Fehlertoleranz des QEC-Prozesses, Dekoder-Performance, reale Fehlermodelle und die Suche nach besseren Lösungen.
    \item \textbf{Hürde: Massiver Qubit-Overhead (z.B. $O(d^2)$)}
    \item ...
    \item \textbf{Hürde: Zeitlicher Overhead und Latenz der Zyklen}
    \item ...
    \item \textbf{Hürde: Fehlertoleranz des Fehlerkorrekturprozesses selbst}
    \item ...
    \item \textbf{Hürde: Komplexität und Geschwindigkeit der Dekoder}
    \item ...
    \item \textbf{Hürde: Korrelierte Fehler und Abweichungen von idealen Fehlermodellen}
    \item ...
    \item \textbf{Ausblick: Entwicklung effizienterer Codes, Hardware-Verbesserungen, Co-Design, Fortschritte bei fehlertoleranten Gattern}
    \item ...
\end{itemize}

\section{Praxisbeispiel: Ein einfacher Quantenfehlerkorrekturcode}
\begin{itemize}
    \item Mit Qiskit Library und qiskit.providers.aer Rauschen simulieren und die Notwendigkeit von Fehlerkorrektur anhand eines Vergleichs aufzeigen
    \item Verschiedene Algorithmen zur Fehlerminimierung verwenden
    \item Verschiedene Rauschen und Ansätze vergleichen und kontrastisieren
    \item Qiskit Visualization Library erlaubt Darstellung von Zusammenhängen
\end{itemize}



\printbibliography
