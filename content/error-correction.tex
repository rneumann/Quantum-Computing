%\motto{Use the template \emph{chapter.tex} to style the various elements of your chapter content.}
\chapter{Fehlerkorrektur}
\label{error_correction} % Always give a unique label
% use \chaptermark{}
% to alter or adjust the chapter heading in the running head

\chapterauthor{Niklas Bodfeld, Tim Boschert, Manuel Meixner, Ann-Kathrin Wenzel}

\abstract{some abstract}

\section{Fehlertypen in Quantenrechnern}

\subsection{Relevanz von Fehlern in Quantenrechnern}
\begin{itemize}
\item Überblick
\item Woher kommen Fehler
\item Wie wirken diese sich aus
\item Wo passieren sie
\end{itemize}

\subsection{Physikalische Fehlerursachen in Quantenrechnern}
\begin{itemize}
\item Dekohärenz
\item Phasendekohärenz
\item Amplitudendekohärenz 
\item Rauschen
\item Unvollständige Isolation 
\item Vibration-
\item Umwelteinflüsse
\end{itemize}

\subsection{Klassifizierung von Quantenfehlern}
\begin{itemize}
\item Bit-Flip
\item Phase-Flip
\item Bit und Phasenflip
\item Amplitudendämpfung
\item Phasendämpfung
\end{itemize}



\subsection{Hardwarebedingte Fehler und Systematische Grenzen}
\begin{itemize}
\item Gatteroperationen
\item Messung falsch
\item Qubitinitalisierung
\item Konnektivität
\item Kalibrierfehler
\end{itemize}

\subsection{Quantifizierung und Modellierung von Quantenfehlern}
\begin{itemize}
\item Fehlerkanäle
\item Krausoperation
\item Fehlerrate
\item wie werden Fehler gemessen
\end{itemize}

\subsection{Auswirkungen von Fehlern auf Quantenalgorithmen und Systemarchitektur}
\begin{itemize}
\item Rechenabbrüche
\item Bedarf an Fehlertoleranz
\item Fehlerbudget
\end{itemize}


\section{Grundprinzipien in Quantenfehlerkorrektur}
\begin{itemize}
\item Fehlererkennung ohne Zustandsmessung
\item Redundanz durch Kodierung
\item Fehlerzustände müssen unterscheidbar sein
\item Reversibilität
\item Fehlerlokalisierung
\item Fehlerschwelle (Threshhold Theorem)
\end{itemize}


\section{Praktische Realisierung der Fehlerkorrektur}

\subsection{Warum Fehlerkorrektur? Grundlagen und erste Ideen}
\begin{itemize}
    \item \textbf{Grundlagen der Fehlerkorrektur -- ein Überblick:} Notwendigkeit durch Fehlerquellen, Lösungsansatz durch Fehlerschwellen-Theorem, Kernideen von QECCs und das Beispiel der Oberflächencodes.
    \item \textbf{Fehlerquellen und deren Limitierung (Kap. 6.1)}
    \item Kurzer Recap zu 6.1
    \item ...
    \item \textbf{Das Fehlerschwellen-Theorem (Kernidee \& Bedeutung)}
    \item Was ist das
    \item Warum wichtig
    \item ...
    \item \textbf{Grundprinzipien von QECCs (z.B. Redundanz, No-Cloning)}
    \item ...
    \item \textbf{Stabilisator-Formalismus und Logische Operatoren}
    \item ...
    \item \textbf{Oberflächencodes als führender Kandidat (2D-Gitter, Distanz $d$, hohe Schwelle)}
    \item ...
\end{itemize}

\subsection{Der Fehlerkorrekturzyklus mit am Beispiel von Oberflächencodes}
\begin{itemize}
    \item \textbf{Der QEC-Zyklus -- ein Überblick:} Kodieren $\rightarrow$ Fehler passieren $\rightarrow$ Fehler finden $\rightarrow$ Fehler deute $\rightarrow$ Fehler beheben.
    \item \textbf{1. Logische Qubits kodieren:}
    \item ...
    \item \textbf{2. Fehler aufspüren (Syndrommessung):}
    \item ...
    \item \textbf{3. Fehler interpretieren (Dekodierung):}
    \item ...
    \item \textbf{4. Fehler beheben (Korrekturoperation):}
    \item ...
\end{itemize}

\subsection{Aktuelle Hürden und was die Zukunft bringt}
\begin{itemize}
    \item \textbf{Herausforderungen \& Ausblick -- ein Überblick:} Großer Qubit-Bedarf, Zeitaufwand, interne Fehlertoleranz des QEC-Prozesses, Dekoder-Performance, reale Fehlermodelle und die Suche nach besseren Lösungen.
    \item \textbf{Hürde: Massiver Qubit-Overhead (z.B. $O(d^2)$)}
    \item ...
    \item \textbf{Hürde: Zeitlicher Overhead und Latenz der Zyklen}
    \item ...
    \item \textbf{Hürde: Fehlertoleranz des Fehlerkorrekturprozesses selbst}
    \item ...
    \item \textbf{Hürde: Komplexität und Geschwindigkeit der Dekoder}
    \item ...
    \item \textbf{Hürde: Korrelierte Fehler und Abweichungen von idealen Fehlermodellen}
    \item ...
    \item \textbf{Ausblick: Entwicklung effizienterer Codes, Hardware-Verbesserungen, Co-Design, Fortschritte bei fehlertoleranten Gattern}
    \item ...
\end{itemize}

\section{Praxisbeispiel: Ein einfacher Quantenfehlerkorrekturcode}
\begin{itemize}
    \item Mit Qiskit Library und qiskit.providers.aer Rauschen simulieren und die Notwendigkeit von Fehlerkorrektur anhand eines Vergleichs aufzeigen
    \item Verschiedene Algorithmen zur Fehlerminimierung verwenden
    \item Verschiedene Rauschen und Ansätze vergleichen und kontrastisieren
    \item Qiskit Visualization Library erlaubt Darstellung von Zusammenhängen
\end{itemize}



\printbibliography
