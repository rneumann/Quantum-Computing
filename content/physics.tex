%\motto{Use the template \emph{chapter.tex} to style the various elements of your chapter content.}
\chapter{Physikalische Grundlagen}
\label{physics} % Always give a unique label

\chapterauthor{Amanda Hagan, Lucie Hartmann, Leon Lukacin, Ole Pross}

\abstract{some abstract}

\section{(Einführung in die Quantenmechanik für das Quantencomputing)}
\subsection{Motivation und Abgrenzung zur klassischen Physik }
\subsection{Wichtige Konzepte der Quantenmechanik }
\subsubsection{\textit{\textit{Zustände und Wellenfunktion} }}
\subsubsection{\textit{Observable und Messung}} 
\subsubsection{\textit{(Zeitentwicklung und Schrödingergleichung??)} }

\section{Zentrale Quantenphänomene }
\subsection{Superposition }

\begin{itemize}
\item Licht verhält sich wie Welle (Doppelspaltexperiment) und Teilchen (Photon)
\item Welle Teilchen Dualismus
\item Beispiel mit Schrödingers Katze
\item In beiden Zuständen bis gemessen wird
\item Teilchen z.B. an zwei Orten gleichzeitig, aber auch andere (sich ausschließende) Eigenschaften parallel -> Superposition -> Teilchen in allen möglichen Zuständen gleichzeitig
\item Einwirkung von außen oder Messung zerstört Superposition
\item Heisenbergsche Unschärferelation
\item Beide Zustände der Superposition haben jeweils einen Anteil, der genaue Wert eines Anteils ist nicht bekannt
\item Mit einer Wahrscheinlichkeit abhängig vom Anteil nimmt ein Teilchen einen der beiden Zustände an

\item Schrödingergleichung/Wellenfunktion
\item Linearkombination (widersprüchlicher) Zustände z.B. 0 und 1/ tot und lebendig
Bei Messung erfolgt Kollaps der Wellenfunktion -> am Ende ein Zustand
\item Wellenfunktion beschreibt ein quantenmechanisches System und enthält alle Informationen über dieses System
\item Eigenvalues und Eigenstates
\end{itemize}

\subsection{Quanteninterferenz}

\begin{itemize}
\item Interferenzmuster aus Doppelspaltexperiment -> Wahrscheinlichkeitsamplitude
Wahrscheinlichkeitsamplituden können sich verstärken oder auslöschen (konstruktiv/destruktiv)
\item Interferometer
\item Einzelnes Photon löst allein Interferenz aus -> entgegen Intuition -> Photon muss in Superposition sein
\item Messung zerstört Verhalten als Welle -> nur noch als Teilchen
Nutzbar in Gattern
\item Tunneleffekt
\end{itemize}

\subsection{Verschränkung }

Korrelation mehrerer Qubits
\begin{itemize}
    \item Beispiel zweier Münzen bei einem verschränkten Coin Flip
    \item lokale Unabhängigkeit (non-classical information)
\end{itemize}

Hidden Variable Theory (disproved)
\begin{itemize}
    \item keine Erklärung von Verschränkung in der klassischen Physik (reines Quantenphänomen)
\end{itemize}

Wie entsteht Verschränkung (kurz Prozess der Erzeugung von Entanglement)

Quantenteleportation und Spukhafte Fernwirkung

Mathematische Erschließung vs. menschliches Bild der Realität
\begin{itemize}
    \item Problem der Kausalität und Lokalität in der klassischen Physik
    \item Debatte über Interpretation der experimentellen Ergebnisse (lokal-realistische Theorien vs. Kopplung in Quantenmechanik)
\end{itemize}

((Evtl. Bell-Zustand, Bell'sche Ungleichung)

\section{Quantenmessung }
\subsection{Grundlagen der Quantenmessung (projektive Messungen, Messoperatoren)}
Vergleich der Messung in klassischer Physik (Beobachtung des System beeinflusst in Quantenmechanischer Welt das Verhalten des Systems 
- zweites Postulat der Quantenmechanik)
- Zweites Messpostulat (projektive Messung): Quantenmessung entspricht einer Projektion des Zustandes auf eine Basis

Beispiel: Photon und Polarisationsfilter 
- einzelnes Photon nicht teilbar, daher Wahrscheinlichkeit der horizontalen / Vertikalen Verteilung beim Passieren des Polarisationsfilters (Zustand ändert sich)
- im Kleinen (Einzelphoton) Zufall, jedoch mit Vielzahl relative Unsicherheit sehr klein (Verhalten wirkt klassisch und stabil) - Quantenfluktuationen verschwinden

Genauer Zustand kann nicht ermittelt werden - zur Ermittlung des Zustandes, muss man Messen

'Observable Operators'
- Zuordnung eines Observable: Zuordnung von Messzuständen und entsprechenden Werten (Eigenwerte)
- Möglichkeiten der statististischen Berechnung (Erwartungswert, Varianz)
- Gleichzeitige Messung von zwei Observables (Uncertainty Principle)

\subsection{Nichtdeterminismus}
Beobachtung nur durch Betrachten des Outputs probabilistisch nach Durchlaufen eines Quantum Circuits möglich
- keine direkte Auslese des Zustandes (s. Kollaps  der Wellenfunktion)
- statistische Verteilung als Aussage über den Zustand bei wiederholtem Durchgang desselben Prozesses / Algorithmus

\subsection{Messprozess (Kollaps der Wellenfunktion, Einfluss auf das Quantensystem, Messung als irreversible Operation) }
Kollaps der Wellenfunktion: Zustand springt auf Basiszustand bei Messung
- Wahrscheinlichkeitsregel: Bornsche Regel
- Kontroversität durch deterministisches Weltbild (in Quantenmechanik ist der Zufall grundlegend) 
- Superposition zerstört und klassischer Zustand angenommen 
- Quanten-Folgezustand 
- No-Cloning Theorem (Quanteninformation kann nicht erneut gemessen werden, da Messung System zerstört) 


(((Technische Realisierung von Messungen, z.B. Fluoreszenz, Tunnelstrom, Photodetektion – eventuell aufgreifen in physikalische Realisierung (s. Supraleiter, Ionenfallen, etc.))) 


\section{Mathematische Beschreibung von Qubits}

\subsection{Hilbertraum und Zustandsvektoren}
\textbf{Zustand eines Qubits} durch einen normierten Vektor in einem zwei-dimensionalen komplexen Hilbertraum.
\begin{itemize}
    \item Beschreibung des Hilbertraums:
    \begin{itemize}
        \item Komplexer Vektorraum mit Skalarprodukt.
    \end{itemize}
    \item Zustände als Basisvektoren.
    \item Allgemeine Qubit-Zustände als Linearkombination.
    \item Bedeutung der Normierung (Wahrscheinlichkeitsinterpretation und komplexe Koeffizienten).
\end{itemize}

\subsection{Bra-Ket-Notation}
\begin{itemize}
    \item Erklärung von \textit{Ket-Vektoren} $\ket{\psi}$ und \textit{Bra-Vektoren} $\bra{\phi}$.
    \item Inneres Produkt: $\braket{\phi | \psi}$.
    \item Operatoren als Transformationen im Hilbertraum, z. B. $U \ket{\psi}$.
    \item Nutzen der Bra-Ket-Notation für spätere Abschnitte wie Quantenalgorithmen, Gates und Messungen.
\end{itemize}

\subsection{Bloch-Kugel-Darstellung}
\begin{itemize}
    \item Darstellung von Qubit-Zuständen geometrisch als Punkte auf der Oberfläche einer Kugel im 3D-Raum.
    \item Bedeutung von Nord-/Südpol und Äquator (Superpositionen).
    \item Veranschaulichung von Rotation durch Quanten-Gates:
    \begin{itemize}
        \item Pauli-X als Spiegelung.
        \item Hadamard-Gate als Rotation.
    \end{itemize}
    \item Bedeutung der Darstellung für intuitive Visualisierung von Quantenzuständen und -operationen.
\end{itemize}

\subsection{Zwei-Niveau-Systeme als Qubits}
\begin{itemize}
    \item Physikalische Systeme, die als Qubits geeignet sind:
    \begin{itemize}
        \item Spin-1/2-Systeme.
        \item Polarisationszustände von Photonen.
    \end{itemize}
    \item Erfüllung der mathematischen Anforderungen eines Qubit-Zustands durch physikalische Zwei-Niveau-Systeme.
    \item Beispiele:
    \begin{itemize}
        \item Elektronenspin ($\ket{\uparrow}$, $\ket{\downarrow}$).
        \item Lichtpolarisation (horizontal/vertikal).
    \end{itemize}
\end{itemize}


\section{Physikalische Realisierung von Qubits }
\subsection{Anforderungen an physikalische Systeme }
\subsubsection{Allgemeine Voraussetzungen}
(Kontrollierbarkeit, Isolation, Skalierbarkeit, Messbarkeit) 
\subsubsection{DiVincenzo Kriterien}
\textbf{1. Präzise definierter und skalierbarer Zustandsraum (Hilbertraum): Qubits als klar identifizierbare, kombinierbare Subsysteme} 

- Freiheitsgrade zur Speicherung/Verarbeitung von Infos müssen als Dimensionen im Hilbertraum eines Quantensystems realisiert sein 

- Isolation des Systems (wenig Wechselwirkung mit Umgebung) 

- Hilbertraum muss präzise bestimmbar und in ein direktes Produkt kleiner Teilsysteme zerlegbar sein (mehrere Qubits) 

- Zustandsraum soll exponentiell mit der Zahl der Qubits wachsen, sonst gibt es keinen Quantencomputing-Vorteil 

\textbf{{2. Möglichkeit eines Startzustands} }

Zb “alle Spins down” -> erreichbar durch Kühlung des Quantensystems auf den Grundzustand; Aufwand abhängig vom System zB bei Atomfallen: Kühlung auf Nano-Kelvin 

-> muss kontrolliert reproduzierbar sein 

\textbf{3. Isolation und Erhaltung der Kohärenz} 

- System muss von der Umgebung abgeschirmt sein um Dekohärenz zu vermeiden -> Fehler entstehen, wenn Quantensystem mit der Umgebung verschränkt wird -> führt zu einem gemischten Zustand 

\textbf{- Toleranz für Fehler sehr gering: }derzeit praktikabel nur bei ε < 10⁻⁶ 

\textbf{4.Man muss kontrollierte Transformationen am System durchführen können} 

(Quantengatter)Operationen müssen selektiv steuerbar sein und dürfen keine ungewollten Nebenwirkungen erzeugen -> meist über zeitabhängige Hamiltonian-Änderungen realisiert (Laser, Magnetfelder etc) 

5. \textbf{.Messung einzelner Qubits im Eigenbasiszustand} 

- “Starke” Messung notwendig -> eindeutige Projektion des Zustands auf den Basiszustand 

- “Schwache” Messung reicht nicht aus (liefert nur Wahrscheinlichkeiten) 

\cite{divincenzo_topics_nodate}
 
\subsubsection{Einführung in Dekohärenz / Fehlerquellen}

Def Quantencomputer (vs klassischer Computer): A quantum 

computer is one whose operation exploits certain very special transformations of its internal state 

Für diese Transformationen müssen “carefully controlled conditions” herrschen 

Physikalisches System darf keine physikalischen Interaktionen haben, die nicht unter der Kontrolle des Programms sind: \textit{All other interactions,} 

\textit{however irrelevant they might be in an ordinary computer – which} we shall call classical – introduce potentially catastrophic disruptions

\textit{into the operation of a quantum comput}er zB Interaktionen mit externer Welt (Luftmoleküle / minimale thermische Energie etc) 

-> führen zu Dekohärenz 
\cite{mermin_quantum_2012}
 

Nielsen 2010: Quantum Computation and Quantum information. S. 46 

2 obstructions : noise as a fundamental barrier \& failed quantum mechanics (incorrect) 

\textbf{Noise as a fundamental barrier -> Threshold Theorem}: noise (Umwelteinflüsse) kann unter einen Schwellenwert reduziert werden und mit error-correcting codes noch weiter reduziert werden 

\textbf{failed / incorrect quantum mechanics:} ein Grund warum das Gebiet des Quantencomputing so interessant ist, ist um die Validität von Quantenmechanik zu beweisen (besonders large scale quantum systems) 

\cite{nielsen_michael_a_and_isaac_l_chuang_quantum_2010}
Fraunhofer Quantum computing compact training program 2025, S. 58 

Noise sources:

• Thermal fluctuations 

• Electromagnetic interference 

• Imperfections in quantum gates 

• Interactions with the environment 

• Influence of neighboring qubits 

\textbf{Consequences in computations:} 

• Flip the state of the qubit 

• Errors can cause qubits to lose their 

(phase) superposition and entanglement 

properties 

• Limits the size and complexity of feasible 

quantum algorithms 

-----------------------------------------------
 
Dekohärenz vermeiden: To avoid decoherence individual bits cannot in general be encoded

in physical systems of macroscopic size,because such systems except} under very special circumstances) cannot be isolated from their own irrelevant internal properties.

Deswegen: bits werden kodiert in wenigen Zuständen von\textit{\textbf{ systems of atomic size }}(interne features sind egal, weil sie nicht existieren oder sie zu hohe Energie erfordern, um relevant zu werden 

Diese Systeme müssen decoupled from their surroundings sein; except for the completely controlled interactions that are associated with the computational process itself

S.2 warum die Situation nicht hoffnungslos ist

Abstände zwischen diskreten Energieniveaus deutlich größer als bei makroskopischen Systemen -> dynamische Isolation auf atomarer Ebene einfacher zu erreichen (man  braucht einen größeren Einfluss um ein Atom aus dem Grundzustand zu bewegen) 

\textbf{Fehlerkorrektur ist möglich}, wenn Einflüsse der Umwelt ausreichend niedrig sind (bei klassischen Computern ist das Routine, bei Quantencompuern muss Fehlerkorrektur stattfinden, ohne den Status der Bits (weder original noch “disrupted”) zu kennen 
 

 \cite{mermin_quantum_2012}
\subsection{Supraleitende Qubits }

\begin{itemize}
\item In Supraleitern fließt Strom verlustlos
\item Temperatur nahe absolutem Nullpunkt
\item In ringförmigen Supraleitern treten Quantenphänomene auf
\item Erzeugung eines zirkulierenden Dauerstroms
\item Bitzustände realisiert durch Flussrichtung des Stroms im Ring
\item Zwei Elektronen ein Cooper-Paar -> Verhalten sich wie ein Teilchen
\item Alle Cooper-Paare haben gemeinsame Wellenfunktion
\item Josephson-Kontakt -> Unterbrechung durch Isolator/Normalleiter
\item Tunneleffekt
\item Messung von Spannung
\end{itemize}

\subsection{Ionenfallen-Qubits }
\subsection{Photonenbasierte Qubits}

Single Photons as Qubits 

Single photons are largely free of the noise, or decoherence, that plagues other systems; can be easily manipulated to realize one-qubit logic gates; and enable encoding in any of several degrees of freedom, for example, polarization, time bin, or path. 
\cite{obrien_optical_2007}



Photons are chargeless particles, and do not interact very strongly with each other, or even 

with most matter. They can be guided along long distances with low loss in optical fibers, 

delayed efficiently using phase shifters, and combined easily using beamsplitters. Photons 

exhibit signature quantum phenomena, such as the interference produced in two-slit ex- 

periments. Furthermore, in principle, photons can be made to interact with each other, 

using nonlinear optical media which mediate interactions 

\textbf{In elektromagnetischem Resonator ist Energie nicht kontinuierlich sondern in units -> “Photonen” -> kleinste Einheiten von Licht / elektromagnetischer Strahlung. }

\textbf{Energiemenge eines Photons hängt von Frequenz der Schwingung ab ℏω (}das reduzierte Plancksche Wirkungsquantum, eine fundamentale Naturkonstant) 

-> in dem Resonator kann es null oder genau ein Photon geben (wie beim Qubit -> Superposition) 

 \cite{nielsen_quantum_2010}
 
\subsection{Spin-Qubits}
\subsection{Neutralatom-Qubits }



Zitat \cite{alhazmi_live_2024}

\cite{bergou_quantum_2021}

\printbibliography
