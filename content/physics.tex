%\motto{Use the template \emph{chapter.tex} to style the various elements of your chapter content.}
\chapter{Physikalische Grundlagen}
\label{physics} % Always give a unique label

\chapterauthor{Karin Mustermann, Max Mustermann}

\abstract{some abstract}

\section{(Einführung in die Quantenmechanik für das Quantencomputing)}
\subsection{Motivation und Abgrenzung zur klassischen Physik }
\subsection{Wichtige Konzepte der Quantenmechanik }
\subsubsection{\textit{\textit{Zustände und Wellenfunktion} }}
\subsubsection{\textit{Observable und Messung}} 
\subsubsection{\textit{(Zeitentwicklung und Schrödingergleichung??)} }
\section{Zentrale Quantenphänomene }
\subsection{Superposition }
\subsection{Quanteninterferenz}
\subsection{Verschränkung }
\section{Quantenmessung }
\subsection{Grundlagen der Quantenmessung }
(projektive Messungen, Messoperatoren)
\subsection{Nichtdeterminismus}
(keine Vorhersagbarkeit, Vergleich mit klassischer Statistik, Rolle in Quantenalgorithmen) 
\subsection{Messprozess }
(Kollaps der Wellenfunktion, Einfluss auf das Quantensystem, Messung als irreversible Operation) 
\subsection{(((Technische Realisierung von Messungen, z.B. Fluoreszenz, Tunnelstrom, Photodetektion – eventuell aufgreifen in physikalische Realisierung (s. Supraleiter, Ionenfallen, etc.))) }
\section{Mathematische Beschreibung von Qubits }
\subsection{Hilbertraum und Zustandsvektoren }
\subsection{Bra-Ket-Notation }
\subsection{Bloch-Kugel-Darstellung }
\subsection{Zwei-Niveau-Systeme als Qubits (?) }
\section{Physikalische Realisierung von Qubits }
\subsection{Anforderungen an physikalische Systeme }
\subsubsection{Allgemeine Voraussetzungen}
(Kontrollierbarkeit, Isolation, Skalierbarkeit, Messbarkeit) 
\subsubsection{DiVincenzo Kriterien}
\subsubsection{Einführung in Dekohärenz / Fehlerquellen}
\subsection{Supraleitende Qubits }
\subsection{Ionenfallen-Qubits }
\subsection{Photonenbasierte Qubits}
\subsection{Spin-Qubits}
\subsection{Neutralatom-Qubits }



Zitat \cite{alhazmi_live_2024}

\cite{bergou_quantum_2021}

\printbibliography
