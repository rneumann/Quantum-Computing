%\motto{Use the template \emph{chapter.tex} to style the various elements of your chapter content.}
\chapter{Quantenhardware}
\label{hardware} % Always give a unique label
% use \chaptermark{}
% to alter or adjust the chapter heading in the running head

\chapterauthor{Dennis Hülsken, Jakob Krumke, Marc Meyer, Daniel Roth, Tom Slatosch}

\abstract{some abstract}

\section{Die verschiedenen Quanten Hardwareplattformen}
\subsection{Einleitung und Unterschiede}
\subsection{Unterschied atomare und Festkörper Plattformen}

\section{Supraleitende Qubits}
\subsection{physikalisches Prinzip supraleitende qubits}
\subsection{Gatterimplementierung}
\subsection{Beispiele supraleitende Qubits}
\subsection{Herausforderungen supraleitende Qubits}

\section{Quantencomputer aus Ionenfallen-Qubits}
\subsection{Physikalisches Prinzip}
    - Zwei-Niveau-System
    - Photonen Manipulation (Elektromagnetisches Felder)
\subsection{Gatterimplementierung}
    - Wie funktioniert ein Logischer Gatter bei Ionenfallen-Quantencomputern?
    - Ein-Qubit-Gatter
    - Zwei-Qubit-Gatter
    - Cirac-Zoller-Gatter
    - Mølmer-Sørensen-Gatter
\subsection{Verwendungsbereiche und Merkmale der Ionenfallen-Qubits}
    - Stand der Technologie
    - Ionenfallen in Quantensimulation und Metrologie
    - Ionenfallen in Quantencomputer
    - Wie ist ein Ionenfallen Quantencomputer aufgebaut?
    - Wie und unter welchen Bedingungen wird ein Ionenfallen Quantencomputer benutzt/verwaltet?
\subsection{Herausforderungen und technische Limitationen}
    - Warum gibt es die aktuellen Probleme?
    - Skalierbarkeit
    - Gatterzeit
    - Potenzial
    - Ansätze zur Lösung aktueller Limitationen: z. B. Quantum Networking 

\section{Quantencomputer auf Basis diamantbasierter Qubits (NV-Zentren)}
\subsection{Physikalisches Prinzip}
    - Zwei-Niveau-System: Elektronenspins von Stickstoff-Fehlstellen (NV-Zentren) im Diamantgitter
    - Optische Kontrolle
    - Mikrowellensteuerung
    - Kernspins
\subsection{Gatterimplementierung}
    - Ein-Qubit-Gatter
    - Zwei-Qubit-Gatter
\subsection{Verwendungsbereiche und Merkmale der diamantbasierten Qubits}
    - Stand der Technologie
    - Anwendungen
    - Systemaufbau
    - Betriebsbedingungen
\subsection{Herausforderungen und technische Limitationen}
    - Warum gibt es die aktuellen Probleme?
    - Skalierbarkeit
    - Gatterzeiten:
    - Potenzial und Ansätze:



\section{Titel tbd}
\subsection{Photonische Quantencomputer}
    - Grundlegende Prinzipien und Qubit-Kodierung
    - Schlüsselkomponenten: Photonenerzeugung, -manipulation und -detektion
    - Vorteile und spezifische Herausforderungen
    - Aktueller Stand, jüngste Fortschritte (2024-2025) und Ausblick
    - Engineering- und Skalierungslösungen
    - Quantenfehlerkorrektur

\subsection{Halbleiterbasierte Qubits: Spin qubits}
    - Physikalische Realisierung in Halbleitern
    - Kontroll- und Auslesemechanismen
    - Vorteile und spezifische Herausforderungen
    - Aktueller Stand, jüngste Fortschritte (2024-2025) und Ausblick
    - Engineering- und Skalierungslösungen
    - Quantenfehlerkorrektur

\subsection{Neutralatom-Quantencomputer}
    - Prinzipien: Atomkühlung, Fallen und Qubit-Kodierung
    - Steuerung und Auslesung mittels Lasertechnologie
    - Vorteile und spezifische Herausforderungen
    - Aktueller Stand, jüngste Fortschritte (2024-2025) und Ausblick
    - Engineering- und Skalierungslösungen
    - Quantenfehlerkorrektur

\subsection{Topologische Qubits}
    - Theoretische Konzepte: Nichtabelsche Anyonen und Majorana-Fermionen
    - Ansätze zur Realisierung und Manipulation
    - Potenzial für inhärente Fehlertoleranz und aktuelle Herausforderungen
    - Aktueller Stand, jüngste Fortschritte (2024-2025) und Ausblick
    - Engineering- und Skalierungslösungen
    - Quantenfehlerkorrektur (über den inhärenten Schutz hinaus)



\section{Quantencomputer-Architekturen und Vernetzung}
\subsection{Aufbau eines Quantenprozessors: Qubit-Array, Kopplungsmechanismen}
\subsection{Skalierungsstrategien: Modulare Systeme}
\subsection{Unterstützende Infrastruktur: Kryo-Elektronik, Hochfrequenzelektronik, Steuerungseinheiten, Filter gegen thermisches Rauschen}
\subsection{Erste Netzwerke: Konzepte des Quanteninternets (Architekturmodell), Quantenrepeater, Quantenrouter, Verschränkungsverteilung}

\section{Praxisbeispiel(e): Im Inneren eines IBM-Quantencomputers}

\subsection{Aufbaus eines kommerziellen Quantencomputers - IBM Q System One}
Beschreibung des Aufbaus eines kommerziellen Quantencomputers - IBM Q System One
Dieser Abschnitt beschreibt die physikalische Struktur eines typischen supraleitenden Quantencomputers. Im Fokus steht der Qubit-Chip, der auf einer stark heruntergekühlten Plattform montiert ist – einer sogenannten Verdünnungskühlstufe mit Temperaturen im Millikelvin-Bereich. Verschachtelte Abschirmungen und Vakuumkammern sorgen für eine minimale Störung durch Wärme, Strahlung oder elektromagnetische Einflüsse von außen.

\subsection{Foto-Illustration}
Foto-Illustration: Kaltes Verdünnungskryostat mit hängender Chip-Ebene (Gold-Coax-Kabel zu Qubits)
Hier wird mithilfe eines Bildes gezeigt, wie ein realer Kryostat aufgebaut ist, in dem der Qubit-Chip „hängt“. Die goldfarbenen Koaxialkabel, die an den Chip führen, dienen der Steuerung und Auslesung der Qubits mit hochfrequenten Mikrowellensignalen. Das Bild veranschaulicht die aufwendige technische Infrastruktur, die notwendig ist, um Quantenoperationen durchzuführen.

\subsection{Erläuterung eines einzelnen supraleitenden Qubits (Transmon)}
Erläuterung eines einzelnen supraleitenden Qubits (Transmon) und wie ein Zwei-Qubit-Gatter durch kapazitive Kopplung realisiert wird
Ein Transmon-Qubit ist ein spezieller supraleitender Schaltkreis, der zur Stabilisierung gegen Ladungsrauschen designt wurde. In diesem Teil wird erklärt, wie durch gezielte Mikrowellenpulse Zustände manipuliert und gelesen werden können. Zusätzlich wird gezeigt, wie zwei Transmon-Qubits über kapazitive Kopplung ein kontrolliertes Quantenlogikgatter bilden – ein zentrales Element zur Realisierung von Quantenalgorithmen.

\printbibliography
