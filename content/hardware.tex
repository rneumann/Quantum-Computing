%\motto{Use the template \emph{chapter.tex} to style the various elements of your chapter content.}


\chapter{Quantenhardware}
\label{hardware} % Always give a unique label
% use \chaptermark{}
% to alter or adjust the chapter heading in the running head

\chapterauthor{Dennis Hülsken, Jakob Krumke, Marc Meyer, Daniel Roth, Tom Slatosch}

\abstract{some abstract}

\section{Die verschiedenen Quanten Hardwareplattformen}
Die Realisierung von Quantencomputern basiert wie beschrieben auf verschiedenen physikalischen Plattformen, die sich durch ihre spezifischen Qubit-Implementierungen, Steuerungsmechanismen und Anwendungsbereiche unterscheiden. Es wird zwischen zwei verschiedenen übergeordneten Hardwareplattformen unterschieden:


\textbf{    - Festkörperplattformen, welche meistens aus Supraleiter oder Diamant basierte NV-Zentren bestehen } 

\textbf{    - Atomare Plattformen, welche auf Ionen oder Photonen basieren} 

Diese Unterschiede in Plattformen wirken sich erheblich auf die Skalierbarkeit, die Kohärenzzeiten und die potenziellen Anwendungsfelder bei Quantencomputer aus. Für beide Hauptplattformen existieren weitere Plattformen, wie z. B. Neutralatom-fallen oder Quanten-Punkte, sind jedoch aktuell nicht so forschungsrelevant wie die vorgestellten Plattformen.

\subsection{Unterschied atomare und Festkörper Plattformen}

Wie im vorherigen Kapitel vorgestellt, existieren in der Benutzung der Hauptplattformen Unterschiede.
- Genaue Vorstellung Atomare Plattformen
- Genaue Vorstellung Festkörper Plattformen

Die folgende Tabelle gibt einen kompakten Überblick über die wichtigsten Eigenschaften beider Plattformarten:

\begin{tabular}{ |p{4cm}|p{5cm}|p{5cm}|  }
 \hline
 \textbf{Aspekt}& \textbf{Atomare Plattformen} & \textbf{Festkörperplattformen}\\
 \hline
 \textbf{Beispiele}   & Ionenfallen-Qubits, neutrale Atome    &Supraleitende Qubits, Halbleiter-Spin-Qubits, NV-Zentren\\
 \textbf{Qubit-Implementierung}&   Nutzung einzelner Atome oder Ionen als Qubits	  & Nutzung elektronischer oder magnetischer Zustände
\\
 \textbf{Kohärenzzeiten} &Sehr lang (Sekunden bis Minuten)	 & Kürzer (Mikrosekunden bis Millisekunden)
\\
 \textbf{Steuerung}    &Laser- und Mikrowellensteuerung	 & Mikrowellen- oder elektrischer Steuerung
\\
 \textbf{Herstellung}&   Aufwändige atomare Präparation und Kühlung	  & Integration in Halbleiterfertigung
\\
 \textbf{Skalierbarkeit}& Schwierig durch komplexe Fallen- oder Lasersysteme	  & Gut integrierbar, skalierbar
   \\
 \textbf{Anwendungen}& Präzise Quantenlogik, Metrologie, Quantensimulationen	  & Algorithmische Anwendungen, Quantenchemie, Optimierung
\\
 \hline
\end{tabular}

\section{Supraleitende Qubits}
Supraleitende Quantencomputer nutzen die Prinzipien der Quantenmechanik und die einzigartigen Eigenschaften von Supraleitern, um Quantenberechnungen durchzuführen. Diese Systeme sind darauf ausgelegt, komplexe Aufgaben in Bereichen wie Quantenchemie, Simulation, Kryptographie und Optimierung zu bewältigen, was ihnen potenzielle Vorteile gegenüber klassischen Systemen verschafft. Die Realisierung dieser Rechenvorteile hängt jedoch von der effizienten und skalierbaren Ausführung von Quantenprogrammen auf robuster Hardware ab, die für Quantenoperationen konzipiert ist.
\subsection{physikalisches Prinzip supraleitende qubits}
\subsection{Implementierung von Quantenlogik}
\subsubsection{Grundlegende Prinzipien und Typen}
Die fundamentalen Einheiten der Quanteninformation in supraleitenden Systemen sind supraleitende Qubits, die als künstliche Atome mit quantisierten Energieniveaus fungieren. Diese Qubits werden typischerweise aus supraleitenden Schaltkreisen gefertigt, die Materialien wie Aluminium oder Niob verwenden und bei extrem niedrigen Temperaturen, nahe dem absoluten Nullpunkt (ca. 15 Millikelvin), betrieben werden(The ultimate Guide to Superconducting Quantum Computers, 2025). Die Supraleitung, die bei diesen Temperaturen erreicht wird, gewährleistet einen widerstandslosen Stromfluss, was für die Aufrechterhaltung der Kohärenz der Qubits und ihrer Quantenzustände unerlässlich ist.
\\\\
Ein supraleitendes Qubit besteht häufig aus einem Induktor und einem Kondensator (LC-Oszillator), die durch einen Josephson-Kontakt verbunden sind. Der Josephson-Kontakt ist ein entscheidendes nichtlineares Element, das eine Anharmonizität in das Energiespektrum des Schaltkreises einführt. Diese Anharmonizität ist von größter Bedeutung, da sie sicherstellt, dass nur die beiden niedrigsten Energiezustände - der Grundzustand 0 and der erste angeregte Zustand 1 - als Qubit-Zustände dienen. Ohne diese Nichtlinearität würde der Quanten-LC-Schaltkreis als einfacher harmonischer Oszillator mit äquidistanten Energieniveaus fungieren, wodurch eine isolierte Zwei-Niveau-System-Nutzung als Qubit unmöglich wäre. Die präzise technische Gestaltung des Josephson-Kontakts und die Auswahl des supraleitenden Materials sind daher nicht nur Fertigungsdetails, sondern grundlegend für die Schaffung eines stabilen und adressierbaren Zwei-Niveau-Systems. Sie bestimmen die grundlegenden Eigenschaften des Qubits wie Anharmonizität, Kohärenzzeit und Rauschresistenz, die für zuverlässige Quantenoperationen unerlässlich sind. Die Betriebsumgebung von supraleitenden Quantencomputern erfordert extrem niedrige Temperaturen, typischerweise im Bereich von 10 bis 20 Millikelvin(mK). Um diese Temperaturen zu erreichen, werden komplexe Kühlsysteme, meist Dilution Refridgerators, benötigt. Um eine solch extreme Kühlung zu ermöglichen, wird die Phasenmischung von Helium-3 und Helium-4 verwendet. Eine genauere Beschreibung zum Aufbau der Kühlsysteme und deren Funktionsweise sowie aktuelle Modelle, sind im Kapitel 1.2.4 zu finden.
\\\\
Der am häufigsten verwendete Qubit-Typ, bei supraleitenden Quantencomputern, ist das Transmon-Qubit. Dieser Transmon-Qubit weißt einen spezifische Hardwareaufbau auf. Er ist eine Weiterentwicklung des Cooper-Pair-Box-Qubits und besteht im Wesentlichen aus einem Josephson-Kontakt, der parallel zu einem relativ großen Shunt-Kondensator geschaltet ist. Dieser Kondensator, oft in einer "Kreuz"-Form (bekannt als Xmon), ist entscheidend, da er die Empfindlichkeit des Qubits gegenüber Ladungsrauschen drastisch reduziert und somit längere Kohärenzzeiten ermöglicht. Die typischen Betriebsfrequenzen von Transmonen liegen um 5 GHz, können aber in-situ zwischen etwa 4 und 6 GHz abgestimmt werden. Für die Frequenzabstimmung werden häufig zwei Josephson-Kontakte parallel in einer SQUID-Schleife (Superconducting QUantum Interference Device) angeordnet. Das Anlegen eines externen Magnetflusses an diese Schleife ermöglicht die dynamische Anpassung der effektiven Josephson-Energie und damit der Resonanzfrequenz des Qubits. (Roth Thomas, 04.2023). Ein alternativer Qubit-Typ ist das Fluxonium-Qubit. Dieses Qubit gewinnt aktuell an Interesse, da es durch einen leicht abgeänderten Aufbau höhere Kohärenzzeiten und hohe Gatter-Fidelitäten erreichen kann. Der Herstellungsprozess dieser Qubits umfasst mehrere Schritte, darunter lithographische Strukturierung, Metallabscheidung, Nass- oder Trockenätzen und die kontrollierte Oxidation von Supraleiterfilmen zur Bildung der Josephson-Kontakte. Fortschritte in der industriellen Halbleiterfertigung, wie optische Lithographie und reaktives Ionenätzen auf großen 300-mm-Siliziumwafern, zeigen vielversprechende Wege zur Skalierung mit hohen Ausbeuten an funktionsfähigen Qubits auf. (Kady Bentley, 2025)

\begin{figure}[ht]
    \centering
    \includegraphics[width=1\textwidth]{images/quanten-hardware/Transmon Qubit.jpg}
    \caption{(a) vollständiges Qubit, (b) Zoom auf die SQUID-Schleife, (c) Josephson-Kontakt}
    \label{fig:transom_image}
    \end{figure}

\subsubsection{Ein-Qubit-Gatter}
Ein-Qubit-Gatter sind die grundlegendsten Operationen in einem Quantencomputer und ermöglichen die gezielte Manipulation des Zustands eines einzelnen Qubits.Diese Gatter, die Rotationen auf der Bloch-Kugel darstellen (z.B. X-, Y- und Z-Rotationen), werden in supraleitenden Quantencomputern durch die präzise Anwendung von Mikrowellenpulsen realisiert.
\\\\
Ein-Qubit-Gatter werden durch das Anlegen von Mikrowellenpulsen realisiert, die auf die Übergangsfrequenz des Qubits abgestimmt sind. Diese Pulse induzieren kohärente Oszillationen, sogenannte Rabi-Oszillationen, zwischen den \big \vert0⟩ und \big \vert1⟩ Zuständen des Qubits. Die Dauer und Phase dieser Pulse bestimmen die Art der Rotation auf der Bloch-Kugel. Beispielsweise können X- und Y-Rotationen durch entsprechend phasenverschobene Mikrowellenpulse erzeugt werden. Z-Rotationen können oft virtuell durch Software implementiert werden, indem die Phase der nachfolgenden Mikrowellenanregungen angepasst wird, was keine physische Pulsdauer erfordert (Characterizing Superconductiong Qubits).\\\\Der Signalpfad für die Qubit-Steuerung ist eine komplexe Kette, die von Raumtemperatur-Elektronik bis zum Qubit im Dilutionskühler reicht. Die Erzeugung der präzisen Mikrowellenpulse beginnt bei Raumtemperatur mit Arbitrary Waveform Generators (AWGs) und Digital-Analog-Wandlern (DACs), die die Basisband-Pulswellenformen erzeugen. Diese Basisband-Signale werden dann mittels Mischer mit einem lokalen Oszillator (LO) zu Mikrowellenfrequenzen (typischerweise 2-10 GHz) hochkonvertiert (Bao Zenghui, 2024). Die Mikrowellensignale werden über Koaxialkabel durch die verschiedenen Temperaturstufen des Kryostaten (z.B. 3K, 1K, 100mK, 10mK) zum Qubit geleitet. Entlang dieses Pfades sind Dämpfungsglieder und Filter angebracht, die an den jeweiligen Temperaturstufen thermisch verankert sind. Ihre Funktion ist es, das von Raumtemperatur eindringende Rauschen zu reduzieren und unerwünschte Frequenzen zu unterdrücken, die die Qubit-Kohärenz stören könnten. Am kältesten Punkt (mK-Stufe) erreichen die Signale den Qubit-Chip über On-Chip-Drive-Lines, die kapazitiv oder induktiv mit den Qubits gekoppelt sind. Die Skalierung dieser Verkabelung stellt eine erhebliche Herausforderung dar, da jede zusätzliche Leitung Wärme in den Kryostaten einbringt und die Kühlleistung begrenzt ist. Dies treibt die Forschung an monolithisch integrierter Steuerungselektronik direkt auf dem Qubit-Chip voran, um den Verkabelungsaufwand und die Wärmelast zu reduzieren (Bao Zenghui, 2024).

\subsubsection{Zwei-Qubit-Gatter}
Zwei-Qubit-Gatter sind für die Durchführung komplexer Quantenalgorithmen unerlässlich, da sie die Verschränkung von Qubits ermöglichen. Ihre Implementierung erfordert eine kontrollierte Wechselwirkung zwischen den Qubits, die auf verschiedene Weisen hardwareseitig realisiert werden kann. Die Entwicklung der Zwei-Qubit-Kopplung hat sich von direkten zu vermittelten, dynamisch steuerbaren Interaktionen entwickelt, um Skalierungs- und Fehlerreduktionsanforderungen zu erfüllen.\\\\
\textbf{Kapazitive Kopplung:}\\
Die kapazitive Kopplung ist eine grundlegende Methode zur Realisierung von Zwei-Qubit-Gattern. Sie wird physisch durch die Platzierung von leitenden Strukturen (Pads oder \grqq{}Coupler Arms\grqq{} ) in unmittelbarer Nähe der Qubits auf dem Chip erreicht, wodurch eine elektrische Kopplung entsteht. Häufig werden hierfür interdigitierte Kondensatoren oder direkte Pads verwendet, die eine effektive Kapazität zwischen den Qubits herstellen.\\
Eine Herausforderung bei der direkten kapazitiven Kopplung in größeren Systemen ist das unerwünschte Übersprechen (Crosstalk) zwischen nicht benachbarten Qubits, das die Gatter-Fidelität beeinträchtigen kann. Um dies zu mindern, wurden fortgeschrittene Designs entwickelt, die \grqq{}Waveguide Extenders\grqq{} oder \grqq{}Coupler Arms\grqq{} mit interdigitierten oder Gap-Kondensatoren nutzen, um die Kopplung über größere physische Distanzen zu vermitteln und gleichzeitig das Übersprechen zu reduzieren. Diese Extender ermöglichen eine flexiblere Platzierung der Qubits auf dem Chip und schaffen Raum für zusätzliche Komponenten wie Ausleseresonatoren und Purcell-Filter.
\\\\
\textbf{Resonatorbasierte Kopplung (Quantenbus):}\\
Eine weit verbreitete und skalierbare Technik zur Kopplung mehrerer Qubits ist die Verwendung eines Mikrowellenresonators als "Quantenbus". Dieser Resonator fungiert als Vermittler und ermöglicht es Qubits, miteinander zu interagieren, selbst wenn sie nicht direkt benachbart sind. Die Interaktion erfolgt über den Austausch virtueller Photonen mit dem Resonator. Dieses Konzept wird als "Circuit QED" (Quantenelektrodynamik in Schaltkreisen) bezeichnet, analog zur optischen Kavität-QED.\\
Durch gezieltes Abstimmen der Qubits in und aus der Resonanz mit dem Bus können verschränkende Zwei-Qubit-Gatter wie Controlled-NOT (CNOT) oder iSWAP implementiert werden. Der Quantenbus bietet eine effektive All-to-All-Konnektivität zwischen einer Gruppe von Qubits, was für bestimmte Algorithmen vorteilhaft ist. Die "Quantum Bus"-Architektur ist nicht nur ein Kopplungsmechanismus, sondern eine grundlegende architektonische Wahl, die flexible und höhergradige Konnektivität zwischen Qubits ermöglicht. Diese Modularität ist entscheidend für die Entwicklung von Quantenprozessoren, die an spezifische Algorithmen und Fehlerkorrekturcodes angepasst werden können, wodurch die Einschränkungen von festen Topologien und spärlicher Konnektivität in größeren Quantensystemen direkt angegangen werden.

\subsection{Beispiele supraleitende Qubits}
** geplant ist hier die Prozessoren Google Scaymore und IBM Eagle vorzustellen. Muss nochmal abgesprochen werden mit Kapitel 1.7, dass dort keine Überschneidungen entstehen**
\subsection{Herausforderungen supraleitende Qubits}
**Einleitungsabsatz zu den Herausforderungen von supraleitenden Quantencomputern**
\\\\
Der Betrieb supraleitender Qubits erfordert extrem niedrige Temperaturen, typischerweise unter 15 Millikelvin (mK), um die Kohärenz aufrechtzuerhalten und die Supraleitung zu ermöglichen.(The ultimate Guide to Superconducting Quantum Computers, 2025). Diese Bedingungen werden durch spezialisierte Kühlsysteme, hauptsächlich Verdünnungskryostate, erreicht. Innerhalb eines Verdünnungskryostat befinden sich die Qubits auf der kältesten Stufe (~10 mK), während die Mikrowellenelektronik auf höheren Temperaturstufen (1 K, 4 K) angeordnet ist. (What Is Cryogenic Quantum Computing and Why It Matters, 2025). Umfassende Abschirmung und Filterung sind entscheidend, um elektromagnetische Interferenzen zu verhindern und thermisches Rauschen zu unterdrücken, das die fragilen Quantenzustände stören und Dekohärenz verursachen kann. Die kryogene Umgebung ist somit nicht nur ein passives Kühlsystem, sondern eine aktive und komplexe Infrastruktur, die die empfindlichen Quantenzustände grundlegend ermöglicht und schützt. Ihre Rolle geht über die reine Kühlung hinaus, da sie aktiv die notwendige Quantenkohärenz aufrechterhält, indem sie verschiedene Formen von Umgebungsrauschen (thermisch, elektromagnetisch) mindert. Verdünnungskryostate sind spezialisierte kryogene Geräte, die die einzigartigen Eigenschaften eines Gemischs aus den Helium-Isotopen Helium-3 und Helium-4 nutzen, um kontinuierlich Temperaturen von bis zu 2 mK zu erreichen. Historisch gesehen wurden "nasse" Verdünnungskryostate verwendet, die eine kontinuierliche Versorgung mit flüssigem Helium zur Vorkühlung benötigten. Diese Systeme waren effektiv, aber mit hohen Betriebskosten und logistischem Aufwand verbunden. Seit etwa 2010 hat sich ein signifikanter Wandel hin zu kryogenfreien ("trockenen") Systemen vollzogen. Diese trockenen Systeme verwenden geschlossene Pulskühlröhren (Pulse Tube Refrigerators, PTRs) zur Vorkühlung auf etwa 4 K, wodurch die Notwendigkeit einer externen Flüssigheliumversorgung entfällt. Der Übergang zu kryokogenfreien Verdünnungskryostaten ist nicht nur im Trend sondern auch strategisch notwendig für die Skalierung von Quantencomputern. Trockene Systeme bieten kontinuierlichen Betrieb und eliminieren die Notwendigkeit des Nachfüllens von Flüssighelium, was längere und unterbrechungsfreie Quantenexperimente ermöglicht und die Betriebseffizienz erhöht. Allerdings geht dieser Vorteil oft mit erhöhten mechanischen Vibrationen von den Pulskühlern einher. Trotz dieser Vibrationsherausforderung ist der Wechsel zu trockenen Systemen für die Skalierbarkeit unerlässlich, da er längere, unterbrechungsfreie Experimente ermöglicht und die Betriebseffizienz für größere Quantensysteme verbessert. Die Hersteller investieren daher stark in Vibrationsdämpfungstechniken, um diesen Nachteil zu mindern(J.Rothe, 2025).
\\\\
Aktuelle Branchenführer verwenden unterschiedliche Kühlsysteme von verschiedenen Anbietern um die individuellen Anforderungen zu erfüllen. IBM hat sich für das im Dezember 2023 erste modulare Quantencomputer System für das Kühlsystem "KIDE Cryogenic Platform" von dem finnischen Unternehmen Bluefors entschieden. Dieses System ist ein kroykogenfreies System, dass auf die Verbindung mehrere Prozessoren ausgelegt ist. Hier wird der von IBM gesetzte Fokus auf skalierbare und modulare Systeme deutlich. Um die reißige Anzahl an Qubits kühlen zu können werden hier neun Pulskühlröhren verwendet, die dafür sorgen verschiedene Temperaturstufen versorgen zu können (Bluefors, 2023). Durch die Nutzung von Pulskühlröhren anstelle von flüssigem Helium können dabei Kühlzeiten von bis zu 3 Jahren realisiert werden. Um modular zu bleiben, wurde KIDE als hexagonale Kammer mit Zugangstüren für Wartung und Möglichkeiten mehrere Cryostat-Einheiten zu koppeln konzipiert. (IBM Quantum, 2024). Ergänzend arbeitet IBM aktuell an dem Goldeye-Projekt mit dem Ziel eine "Super-Kühlschrank" mit 1,7 Kubikmetern Volumen zu entwickeln um zukünftig größere Quantensysteme kühlen zu können. (Pat Gumann, Jerry Chow, 2022) 

Google und Intel setzen ebenfalls auf kryogene Plattformen von Bluefors um ihre supraleitenden zu betreiben \cite{noauthor_cryogenic_2025}(Cryogenic Quantum Hardware: The Cold Engine of Quantum Tech, 2025). **Weiteres Ausführen zu Google und Intel **
\\\\
** Hier muss noch auf Verkabelung und Skalierbarkeit eingegangen werden**
\subsection{Ausblick und Weiterentwicklung}

**Hier kommen inhaltlich Punkte zur Verbesserung von Kohärenzzeiten und Gatterfidelity sowie Miniaturisierung

\section{Quantencomputer aus Ionenfallen-Qubits}
\subsection{Logische Gatter in Ionenfallencomputer}

In einer Ionenfalle werden geladene Atome (meist \( ^{40}\mathrm{Ca}^+ \), \( ^{171}\mathrm{Yb}^+ \) oder \( ^{9}\mathrm{Be}^+ \)) durch elektrische Felder in einem räumlich fixierten Gitter gehalten. Die internen elektronischen Zustände dieser Ionen dienen als Basiszustände \( \lvert 0 \rangle \) und \( \lvert 1 \rangle \) eines Qubits. 

Dabei sind die Eigenschaften der Laserstrahlung – insbesondere Ausbreitungsrichtung, Wellenlänge, Frequenz, Polarisation sowie die Dauer der Bestrahlung – entscheidend, um gezielt Überlagerungszustände (Superpositionen) und Verschränkungen für das Quantencomputing zu erzeugen (Baumann, 2024). 

Es existieren verschiedene Typen von Quantengattern in Ionenfallen, die je nach Art eine unterschiedliche Anzahl an Qubits beeinflussen. Diese Gatter unterscheiden sich sowohl in ihrer Komplexität als auch in ihrem Einsatzbereich.

Das folgende Kapitel stellt exemplarisch mehrere dieser Quantengatter vor – beginnend mit Ein-Qubit-Gattern bis hin zu Zwei-Qubit-Gattern wie dem Mølmer-Sørensen-Gatter – um deren jeweilige Funktionsweise und Unterschiede vorzustellen.

\subsubsection{Ein-Qubit-Gatter}

Ein gängiges physikalisches Modell zur Beschreibung eines Ein-Qubit-Gatters in Ionenfallencomputern ist der sogenannte \glqq Starre Rotor\grqq. Dabei wird ein einzelnes Ion betrachtet, das aus einem positiv geladenen Atomkern und einem oder mehreren Elektronen besteht. In einem stark vereinfachten Zwei-Niveau-System kann das Elektron als Teilchen angesehen werden, das den Kern in einer festen Kreisbahn umkreist, wodurch die beiden Qubit-Zustände \( \lvert 0 \rangle \) und \( \lvert 1 \rangle \) realisiert werden.

\begin{figure}[ht]
    \centering
    \includegraphics[width=0.6\textwidth]{images/quanten-hardware/Rotor.png}
    \caption{Schematische Darstellung eines Rotors nach Baumann (2024)}
    \label{fig:meinbild}
\end{figure}

Ein Laserstrahl mit passender Wellenlänge wird genutzt, um Übergänge zwischen diesen Zuständen zu induzieren. Liegt die Frequenz des Lasers genau im Resonanzbereich der Energiedifferenz der beiden Zustände, so kann das Ion durch sogenannte Rabi-Oszillationen in eine beliebige Superposition gebracht werden. Die Dynamik des Übergangs lässt sich mathematisch durch die Rabi-Frequenz \( \Omega \) beschreiben, welche sowohl von der Intensität des Lasers als auch von der Stärke des Dipolmoments des Übergangs abhängt.

Die Kontrolle über Pulsdauer, Polarisation und Phase erlaubt die Realisierung verschiedener logischer Operationen wie beispielsweise das Hadamard-Gatter oder das Pauli-\( X \)-Gatter. Visualisiert wird der Zustand eines Qubits in der Praxis häufig durch die sogenannte Bloch-Kugel, in welcher jede mögliche Superposition von \( \lvert 0 \rangle \) und \( \lvert 1 \rangle \) einem Punkt auf der Kugeloberfläche entspricht.

Im Experiment werden zur Umsetzung meist extrem präzise, kurze Laserpulse verwendet, die eine hohe Adressiergenauigkeit einzelner Ionen gewährleistet (Baumann, 2024).

\subsubsection{Zwei-Qubit-Gatter}

Zwei-Qubit-Gatter ermöglichen es, Korrelationen zwischen Qubits herzustellen und bilden zusammen mit Ein-Qubit-Gattern eine universelle Menge an logischen Operationen. Die wohl bekannteste Zwei-Qubit-Operation ist das bereits vorgestellte CNOT-Gatter (Controlled-NOT-Gatter), das auf zwei Qubits wirkt: ein Kontroll- und ein Ziel-Qubit. Ist das Kontroll-Qubit im Zustand \( \lvert 1 \rangle \), wird ein NOT-Gatter auf das Ziel-Qubit angewendet; andernfalls bleibt es unverändert. Diese logische Verknüpfung bildet die Grundlage für viele quantenlogische Algorithmen und Verschlüsselungsverfahren.

In Ionenfallen wird ein solches Gatter durch die gemeinsame Kopplung der Ionen an die quantisierten Schwingungsmoden realisiert. Dabei dienen diese kollektiven Moden als ein „Phonon-Bus“, über den Zustände zwischen den Ionen ausgetauscht werden können (Baumann, 2025).

Zu den bedeutendsten realisierten Zwei-Qubit-Gattermodellen gehört das Cirac-Zoller- und das Mølmer-Sørensen-Gatter, welche im folgenden Abschnitt vorgestellt werden:

\newpage
\textbf{Cirac-Zoller-Gatter} 

Das von Cirac und Zoller im Jahr 1995 vorgeschlagene Gatter gilt als eines der ersten theoretisch und experimentell realisierten Modelle für ein entanglierendes Zwei-Qubit-Gatter in Ionenfallen. Hierbei wird ein drittes, sogenanntes Bus-Qubit über gezielte Laserpulse angeregt und zwischen zwei speichernden Ionen als Vermittler genutzt. Die Gatteroperation nutzt eine präzise definierte Sequenz aus Resonanzanregungen und kontrollierten Phasenverschiebungen, um über den Schwingungszustand gezielt einen Quantenbit-Zustand auf das andere zu übertragen.

Der Zustand \( \lvert 0 \rangle \lvert 1 \rangle \) kann beispielsweise durch das Gatter in eine Überlagerung mit \( \lvert 1 \rangle \lvert 0 \rangle \) überführt werden, wobei die Phononen nur temporär real angeregt werden. Die präzise Steuerung der Frequenzdifferenz, Laserstärke und Phasenlage ist essenziell. In der Praxis ist die Methode allerdings experimentell aufwendig, da sie eine sehr hohe Kontrolle über den Grundzustand der Ionen erfordert. Dennoch wurde sie erfolgreich in Experimenten mit Kalziumionen demonstriert(Cirac and Zoller, 1995; Baumann, 2025). 

\textbf{Mølmer-Sørensen-Gatter} 


Das Mølmer-Sørensen-Gatter (MS-Gatter) ist eine Weiterentwicklung, die sich in modernen Ionenfallen-Architekturen durchgesetzt hat. Anders als das Cirac-Zoller-Gatter benötigt es kein Bus-Qubit, sondern nutzt kollektive Schwingungsmoden zur gleichzeitigen Adressierung mehrerer Qubits.

Ein typisches MS-Gatter basiert auf einem bichromatischen Laserpuls, welcher symmetrisch zwei Seitenbänder ober- und unterhalb der Qubit-Resonanzfrequenz anspricht. Dabei wird die Schwingung nur virtuell angeregt, was die Robustheit gegenüber thermischen Zuständen erhöht. Durch diese Wechselwirkung entsteht eine effektive Spin-Spin-Kopplung, mit der sich verschränkte Zustände, wie z.\,B. Bell-Zustände, effizient erzeugen lassen. Das Gatter benötigt dabei nur minimale Kontrolle über die Bewegungszustände der Ionen, was es besonders fehlerresistent und skalierbar macht. Zudem kann es gleichzeitig mehrere Qubit-Paare verschränken (Mølmer and Sørensen, 2000; Baumann, 2025).

\begin{table}[ht]
    \centering
    \caption{Vergleich zwischen Cirac-Zoller- und Mølmer-Sørensen-Gatter WIP}
    \label{tab:vergleich_gatter}
    \begin{tabular}{p{4.5cm} p{5.5cm} p{5.5cm}}
        \textbf{Eigenschaft} & \textbf{Cirac-Zoller-Gatter} & \textbf{Mølmer-Sørensen-Gatter} \\
        Jahr & 1995 & ~2000 \\
        Vermittlung & Bus-Qubit notwendig & Keine zusätzlichen Qubits \\
        Schwingungsmoden & Echt angeregt & Virtuell angeregt \\
        Robustheit & Gering (gegenüber thermischen Zuständen) & Hoch \\
        Skalierbarkeit & Eingeschränkt & Gut skalierbar \\
        Experimentelle Anforderungen & Hohe Präzision & Weniger streng \\
        Anwendungsbereich & Proof-of-Concept & Industrielle Prototypen \\
    \end{tabular}
\end{table}


\subsection{Aufbau und Betriebsweise eines Ionenfallencomputers}

Während theoretische Betrachtungen von Ionenfallen-Qubits sich stark auf die physikalischen Prinzipien konzentrieren, wie in Abschnitt 1.6.3 beschrieben, unterscheidet sich der tatsächliche Aufbau eines Ionenfallen-Quantencomputers in der Praxis erheblich durch seine technische Komplexität. Moderne Systeme verbinden physikalisch hochpräzise Komponenten mit klassischer Computertechnik und Softwareintegration, welche im folgenden Kapitel genauer vorgestellt werden.
\subsubsection{Beispiele in Reale Systemen}

Ein anschauliches Beispiel für die Realisierung eines kommerziellen Ionenfallen-Quantencomputers liefert das österreichische Unternehmen Alpine Quantum Technologies (AQT). Der Quantenprozessor von AQT ist vollständig in ein standardisiertes 19-Zoll-Gehäuse integriert, das äußerlich kaum von einem klassischen Serverschrank unterscheidbar ist. Im Inneren befinden sich jedoch Hochvakuumkammern, mikrostrukturierte Paul-Fallen, Lasermodule, modulare Optiksysteme und eine komplexe Steuerungseinheit. Die Quantenhardware wird durch klassische Komponenten ergänzt, die für Temperaturkontrolle, Impulssteuerung, Datenausgabe und externe Kommunikation zuständig sind (Bischoff et al., 2024).

    \begin{figure}[ht]
    \centering
    \includegraphics[width=1\textwidth]{images/quanten-hardware/AQT.jpg}
    \caption{Bild eines AQT Ionenfallencomputers}
    \label{fig:meinbild}
    \end{figure}

AQT entwickelt seine Systeme auf der Grundlage langjähriger Forschung an der Universität Innsbruck, insbesondere in Kooperation mit dem Experimentalphysiker Rainer Blatt, Mitgründer von AQT, der als einer der Pioniere der Ionenfallenplattform gilt (Blatt and Wineland, 2008). Die Geräte verwenden typischerweise linear angeordnete Ionenketten aus Kalziumionen, die in einer mikrochip-basierten Paul-Falle fixiert werden. Diese Chips sind modular aufgebaut und lassen sich bei Bedarf austauschen oder erweitern. Die Laseransteuerung, die Ioneninitialisierung, das Auslesen der Zustände und die zeitlich exakte Koordination der Gatteroperationen erfolgen über eine integrierte Steuerelektronik, meist in Form von FPGA-gestützter Signalverarbeitung .

Auch andere Systeme wie Quantinuum (ehemals Honeywell Quantum) oder IonQ verfolgen vergleichbare Ansätze: Der eigentliche Quantenprozessor wird in eine vollständig abgeschirmte Umgebung eingebettet, kombiniert mit photonischer Signalverarbeitung und einem modularen optischen Kontrollsystem. So wurde etwa 2024 an der Universität Innsbruck erstmals eine Verschränkung von 56 Ionen auf einem einzigen Chip demonstriert, ein Rekord für die Ionenfallenplattform (Bischoff et al., 2024). 

Neben AQT existieren heute weltweit mehrere Firmen, die Ionenfallen-Quantencomputer entwickeln oder bereits kommerziell anbieten, darunter eQtron in Siegen, Universal Quantum in Sussex, IonQ in Maryland (USA) sowie Quantumium in Colorado. 
Trotz des unterschiedlichen Designs verfolgen alle diese Plattformen einen ähnlichen Architekturansatz: Die Trennung von quantenphysikalischem Kernsystem (Ionenchip, Lasermodule, Vakuum) und klassischer Steuerungseinheit, die die Verbindung zur Außenwelt und Benutzersteuerung herstellt. Diese einzelnen Komponenten werden in den nächsten Kapiteln genauer erläutert.

\subsubsection{Ionenchips}

Der Ionenchip bildet das zentrale Bauteil eines Ionenfallen-Quantencomputers. Auf ihm wird die tatsächliche Quanteninformation gespeichert und verarbeitet. Die Aufgabe des Chips besteht darin, einzelne Ionen präzise in einer räumlich fixierten Anordnung zu halten und gleichzeitig eine kontrollierte Wechselwirkung zwischen ihnen zu ermöglichen. Technisch wird dies durch eine sogenannte mikrostrukturierte Paul-Falle realisiert, die auf dem Prinzip oszillierender elektrischer Felder basiert – wie bereits in Abschnitt 1.6.3 beschrieben. Während klassische Paul-Fallen in Form von makroskopischen Elektroden in einer Vakuumkammer realisiert wurden, kommt in modernen Quantencomputern eine mikrochipbasierte Realisierung zum Einsatz.

Um den Betrieb bei hoher Stabilität zu ermöglichen, wird der Chip in eine Ultrahochvakuumkammer (UHV) integriert. Das Vakuum schützt vor Stößen mit Luftmolekülen und reduziert unerwünschte elektrische Entladungen. Zusätzlich wird die Umgebungstemperatur konstant gehalten, um thermische Drift im Chipmaterial zu minimieren. Da geladene Ionen empfindlich auf äußere Felder reagieren, ist die gesamte Kammer elektromagnetisch abgeschirmt.

\subsubsection{Optische Systeme}

Das optische System ist ein weiteres zentrales Element der Quantenhardware in Ionenfallen-Quantencomputern. Es übernimmt alle Aufgaben der Qubit-Manipulation, Kühlung und Zustandsmessung, indem es kontrollierte Laserimpulse gezielt auf einzelne Ionen in der Falle richtet. Hierzu werden hochpräzise Laserquellen verwendet, die über modulare optische Systeme mit den gespeicherten Ionen interagieren. 

Die Lichtquellen bestehen typischerweise aus frequenzstabilisierten Laserdioden, deren Wellenlänge exakt auf die atomaren Übergänge der verwendeten Ionen abgestimmt ist. Die Stabilität der Frequenz ist dabei essenziell, da kleine Abweichungen von wenigen Megahertz bereits zu fehlerhaften Zustandsänderungen führen können. Aus diesem Grund werden die Laserdioden häufig mithilfe von Referenzkavitäten frequenzstabilisiert oder über nichtlineare Kristalle frequenzverdoppelt, um schwer zugängliche UV-Wellenlängen zu erreichen.

Ein entscheidender Aspekt ist die Kontrolle der optischen Pulsparameter, da diese direkt die Quantengatteroperationen bestimmen. Für eine vollständige Kontrolle müssen vier Parameter in Echtzeit gesteuert werden:

\begin{itemize}
    \item die Pulsdauer (bestimmt die Rotation auf der Bloch-Kugel),
    \item die Intensität (beeinflusst die Rabi-Frequenz),
    \item die exakte Frequenzlage (z.\,B. für detuning-sensitives Targeting),
    \item und die Phasenlage der Welle (wichtig für kohärente Operationen und Verschränkungsgatter).
\end{itemize}

Diese Steuerung erfolgt über ein klassisches Kontrollsystem bestehend aus FPGAs, Taktgebern und Analogwandlern, welches die Laserpulse auf Nanosekundenebene formt und synchronisiert.

Das optische System stellt deshalb insgesamt einen der empfindlichsten und kritischsten Bausteine ionenbasierter Quantencomputer dar. Seine Qualität bestimmt direkt die erreichbare Gate-Fidelity, die Kohärenzzeit und die Skalierbarkeit des Gesamtsystems.

\subsubsection{Kühlung und Abschirmung}
WIP
\subsubsection{Elektronik}
WIP
\subsubsection{Modularität und Skalierbarkeit}
WIP



\subsection{Herausforderungen und technische Limitationen}
WIP
    - Warum gibt es die aktuellen Probleme?
    - Skalierbarkeit
    - Gatterzeit
    - Potenzial
    - Ansätze zur Lösung aktueller Limitationen: z. B. Quantum Networking 

\section{Quantencomputer auf Basis diamantbasierter Qubits (NV-Zentren)}
\subsection{Physikalisches Prinzip}
    - Zwei-Niveau-System: Elektronenspins von Stickstoff-Fehlstellen (NV-Zentren) im Diamantgitter
    - Optische Kontrolle
    - Mikrowellensteuerung
    - Kernspins
\subsection{Gatterimplementierung}
    - Ein-Qubit-Gatter
    - Zwei-Qubit-Gatter
\subsection{Verwendungsbereiche und Merkmale der diamantbasierten Qubits}
    - Stand der Technologie
    - Anwendungen
    - Systemaufbau
    - Betriebsbedingungen
\subsection{Herausforderungen und technische Limitationen}
    - Warum gibt es die aktuellen Probleme?
    - Skalierbarkeit
    - Gatterzeiten:
    - Potenzial und Ansätze:


% \addbibressource{content/jakob/citations.bib}

\section{Photonen und andere Ansätze}
Die Realisierung eines skalierbaren, fehlertoleranten Quantencomputers stellt eine der größten wissenschaftlichen und technischen Herausforderungen unserer Zeit dar (\cite{QuantumHardwareExplained}). Weltweit werden diverse physikalische Systeme intensiv erforscht, um die komplexen Anforderungen an Quantenhardware zu erfüllen (\cite{chengNoisyIntermediatescaleQuantum2023}). Diese Vielfalt an Ansätzen spiegelt die Tatsache wider, dass noch kein einzelnes System alle notwendigen Kriterien optimal erfüllt und der Weg zu universellen Quantencomputern nicht singulär erscheint. Vielmehr deutet die aktuelle Forschungslandschaft auf eine längere Phase paralleler Entwicklungen und potenzieller Hybridisierungen hin, in der verschiedene Technologien koexistieren und möglicherweise integriert werden, anstatt dass eine einzelne Technologie die anderen schnell verdrängt (\cite{QuantumHardwareExplained}).
\newline \newline
Ein etablierter Rahmen zur Bewertung der verschiedenen Hardwareplattformen sind die DiVincenzo-Kriterien (\cite{QuantumComputingArchitecture}). Diese umfassen typischerweise die Existenz gut charakterisierter Qubits, die Möglichkeit zur Initialisierung der Qubits in einen definierten Zustand, lange Kohärenzzeiten, die Fähigkeit zur Durchführung universeller Quantengatter sowie eine zuverlässige Auslesemethode für Qubits. Die Erfüllung dieser Kriterien unterliegt signifikanten ingenieurtechnischen Hürden, die Expertise aus Physik, Kryotechnik, Materialwissenschaften und Elektronik erfordern (\cite{QuantumHardwareExplained}).
\newline \newline
Dieses Kapitel fokussiert auf photonische Quantencomputer als einen prominenten alternativen Ansatz und beleuchtet darüber hinaus weitere vielversprechende Technologien wie Halbleiter-Spin-Qubits und topologische Qubits. Es wird anerkannt, dass etabliertere Plattformen wie supraleitende Qubits (\cite{QuantumComputingArchitecture}) und gefangene Ionen (\cite{sruthisomarouthuQuantumComputingDigital2025}) bereits einen höheren Reifegrad erreicht haben und vermutlich in anderen Kapiteln dieses Buches ausführlich behandelt werden. Die hier diskutierten Ansätze zeichnen sich durch spezifische Vorteile und innovative Lösungsstrategien für die Kernprobleme der Quanteninformationsverarbeitung aus.
\newline \newline
Die aktuelle Ära der Quantencomputerentwicklung wird häufig als „Noisy \linebreak Intermediate-Scale Quantum“ (NISQ) bezeichnet (\cite{chengNoisyIntermediatescaleQuantum2023}). In dieser Phase sind die verfügbaren Quantenprozessoren noch fehleranfällig und die Anzahl der Qubits begrenzt. Ein zentrales Forschungsziel ist daher die Entwicklung und Implementierung von Strategien zur Fehlerminderung und -korrektur sowie die Realisierung sogenannter logischer Qubits, die eine höhere Robustheit gegenüber physikalischen Fehlern aufweisen (\cite{QuantumHardwareExplained}). Dieser Trend zur Fehlertoleranz ist ein kritisches Moment in der Entwicklung und signalisiert eine Verschiebung von der reinen Demonstration quantenmechanischer Phänomene hin zur Bewältigung der ingenieurtechnischen Herausforderungen für praktische Quantenvorteile. Jede Bewertung der hier vorgestellten „anderen Ansätze“ muss daher nicht nur die rohe Qubit-Anzahl, sondern auch den Fortschritt und die Strategien auf dem Weg zur Fehlertoleranz berücksichtigen.

\subsection{Photonische Quantencomputer}
Photonische Quantencomputer nutzen Lichtquanten (Photonen) als Träger von Quanteninformation. Dieser Ansatz bietet eine Reihe einzigartiger Vorteile, steht aber auch vor spezifischen Herausforderungen, die intensive Forschungs- und Entwicklungsanstrengungen erfordern.

\subsubsection{Grundlegende Prinzipien und Qubit-Kodierung}
Photonen werden oft als "fliegende Qubits" bezeichnet, da sie sich naturgemäß mit Lichtgeschwindigkeit bewegen, eine geringe Wechselwirkung mit ihrer Umgebung aufweisen (was zu langer Kohärenz führt) und potenziell bei Raumtemperatur betrieben werden können \cite{abughanemPhotonicQuantumComputers2024}. Die Kodierung von Qubits in Photonen kann auf verschiedene Weisen erfolgen. Eine verbreitete Methode ist die Dual-Rail-Kodierung, bei der die Zustände $\ket{0}$ und $\ket{1}$ durch die Anwesenheit eines Photons in einem von zwei räumlichen Moden (z.B. zwei Wellenleitern) oder zwei orthogonalen Polarisationen repräsentiert werden \cite{slussarenkoPhotonicQuantumInformation2019}. Alternativ werden kontinuierliche Variablen (Continuous Variable, CV) Ansätze verfolgt, bei denen Qubits beispielsweise in gequetschten Lichtzuständen (Squeezed States) kodiert werden, wie es von Unternehmen wie Xanadu erforscht wird \cite{QuantumComputingArchitecture}. Auch die Zeitmultiplex-Kodierung (Time-Bin Encoding), bei der die Ankunftszeit eines Photons relativ zu einem Referenzpuls das Qubit definiert, ist eine gängige Methode \cite{LinearOpticsScalable}.

\subsubsection{Schlüsselkomponenten}
Die Realisierung photonischer Quantencomputer hängt von drei Schlüsselkomponenten ab:
\begin{itemize}
    \item Photonenquellen: Eine Herausforderung ist die Erzeugung einzelner Photonen bei Bedarf (deterministisch). Probabilistische Quellen, basierend auf spontaner parametrischer Fluoreszenz (Spontaneous Parametric Downconversion, SPDC), sind zwar etabliert, aber für skalierbare Systeme ineffizient \cite{slussarenkoPhotonicQuantumInformation2019}. Daher liegt ein starker Fokus auf der Entwicklung deterministischer oder angekündigter (heralded) Einzelphotonenquellen, beispielsweise basierend auf Quantenpunkten \cite{LinearOpticsScalable} oder Halbleiter-Quantenemittern, wie sie Quandela einsetzt \cite{QuandelaAnnounces100000fold2025}. Kürzlich wurde über Quantenpunktquellen mit Helligkeiten über 106 Photonenpaaren/s/mW und angekündigte Einzelphotonenquellen mit Reinheiten über 99\% berichtet \cite{LinearOpticsScalable}.
    \item Manipulation: Die Manipulation photonischer Qubits erfolgt durch lineare optische Elemente wie Strahlteiler und Phasenschieber \cite{slussarenkoPhotonicQuantumInformation2019}. Für erhöhte Stabilität, Miniaturisierung und Skalierbarkeit werden zunehmend integrierte photonische Schaltkreise (Photonic Integrated Circuits, PICs) eingesetzt \cite{abughanemPhotonicQuantumComputers2024}. Diese PICs werden auf verschiedenen Materialplattformen wie Silizium, Siliziumnitrid oder Lithiumniobat realisiert.
    \item Detektion: Hocheffiziente Einzelphotonendetektoren sind unerlässlich. Supraleitende Nanodraht-Einzelphotonendetektoren (Superconducting Nanowire Single-Photon Detectors, SNSPDs) erreichen Detektionseffizienzen von über 98\% \cite{LinearOpticsScalable}, stellen aber hohe Anforderungen an die Kühlung. Eine Herausforderung bleibt die photonenzahlauflösende Detektion \cite{slussarenkoPhotonicQuantumInformation2019}.
\end{itemize}
\subsubsection{Vorteile und spezifische Herausforderungen}

Photonische Quantencomputer bieten eine Reihe signifikanter Vorteile. Ein wesentlicher Pluspunkt ist ihr Potenzial für den Raumtemperaturbetrieb bei einigen Komponenten, wenngleich Quellen und Detektoren oft eine Kühlung erfordern \cite{abughanemPhotonicQuantumComputers2024}. Ferner zeichnen sich photonische Qubits durch eine geringe Dekohärenz aus, bedingt durch die schwache Wechselwirkung von Photonen mit ihrer Umgebung, was zu langen Kohärenzzeiten führt\cite{abughanemPhotonicQuantumComputers2024}. Darüber hinaus eignen sich Photonen hervorragend für die Quantenkommunikation und -netzwerke, da sie als natürliche Informationsträger über lange Distanzen fungieren \cite{abughanemPhotonicQuantumComputers2024}. Diese Eigenschaft positioniert die Photonik einzigartig für modulare Architekturen, bei denen spezialisierte photonische Module andere Arten von Quantenprozessoren miteinander verbinden und somit als "Quanten-Internet-Backbone" dienen könnten.
\newline
\newline
Dem gegenüber stehen die spezifischen Herausforderungen Photonenverlust, probabilistische Qubit-Gatter und die skalierbarkeit von Quellen, sowie Detektoren.
\paragraph{Photonenverlust}
Dies ist ein Hauptproblem, das alle Komponenten betrifft – von der Quelle über die Manipulation bis zur Detektion – und die Fidelität stark beeinträchtigt \cite{LinearOpticsScalable}. Strategien zur Minderung umfassen die Entwicklung von Materialien mit extrem geringen Verlusten (z.B. Siliziumnitrid) und die Verbesserung der Effizienz von Komponenten und Kopplungen \cite{salavrakosPhotonnativeQuantumAlgorithms2025}.

\paragraph{Probabilistische Zwei-Qubit-Gatter}
Die schwache native Wechselwirkung zwischen Photonen macht deterministische Zwei-Qubit-Gatter sehr schwierig \cite{slussarenkoPhotonicQuantumInformation2019}. Übliche Ansätze basieren auf messungsinduzierter Nichtlinearität unter Verwendung von Hilfsphotonen \textcolor{green}{Fachausdruck?} und Postselektion, was zu einem hohen Ressourcenaufwand führt. Alternative Ansätze wie Cluster-Zustands-basierted oder fusionsbasiertes Quantencomputing werden intensiv erforscht, um diese Hürde zu überwinden \cite{salavrakosPhotonnativeQuantumAlgorithms2025}.

\paragraph{Skalierbarkeit von Quellen und Detektoren}
Die zuverlässige Erzeugung und Detektion einer großen Anzahl von Photonen mit hoher Qualität bleibt eine technische Herausforderung \cite{LinearOpticsScalable}.

\subsubsection{Aktueller Stand der Technik}
Das Feld der photonischen Quantencomputer verzeichnet rasante Fortschritte, die maßgeblich von akademischer Forschung und industriellen Akteuren vorangetrieben werden. Führende Unternehmen verfolgen dabei unterschiedliche, aber gleichermaßen ambitionierte Strategien.
Das kanadische Unternehmen Xanadu stellte Anfang 2025 seinen Prototyp „Aurora“ vor, ein modulares System, das aus 35 photonischen Chips und 13 Kilometern Glasfaser besteht. Eine strategische Partnerschaft mit dem U.S. Air Force Research Laboratory (AFRL) unterstreicht das Ziel, fehlertolerante Prozessoren zu entwickeln und die Quellen für verschränkte Photonen sowie für gequetschtes Licht zu optimieren \cite{kareemXanaduUSAir2025}.
PsiQuantum verfolgt einen Ansatz, der auf eine Skalierung im Millionen-Qubit-Bereich abzielt und entwickelt dafür das „Omega“-Chipset. In Partnerschaft mit Regierungen plant das Unternehmen den Bau dedizierter Quantenrechenzentren in Brisbane, Australien, und Chicago, USA. Für die Fertigung der Chips nutzt PsiQuantum die Kapazitäten des etablierten Halbleiterherstellers GlobalFoundries \cite{DARPAEyesCompanies}. Diese „Fabless“-Strategie, bei der die Produktion ausgelagert wird, steht im Kontrast zum Vorgehen von Unternehmen wie Quantum Computing Inc. (QCi), die eine eigene Fabrik für Dünnschicht-Lithiumniobat-Chips (TFLN) errichten \cite{QuantumComputingInc}. Diese unterschiedlichen Geschäftsmodelle – „Fabless“ versus „integrierter Hersteller“ – spiegeln Entwicklungen wider, die bereits in der klassischen Halbleiterindustrie zu beobachten waren, und dürften langfristige Auswirkungen auf Kostenstrukturen, Innovationsgeschwindigkeit und die Verbreitung der Technologie haben. \newline
Das französische Unternehmen Quandela verfolgt einen hybriden Ansatz, bei dem photonische Qubits aus künstlichen Atomen, genauer gesagt aus Halbleiter-Quantenemittern, erzeugt werden. Im Februar 2025 meldete Quandela eine Methode, die das Potenzial hat, die Anzahl der für fehlertolerante Berechnungen benötigten Komponenten um den Faktor 100.000 zu reduzieren, von etwa einer Million auf nur noch 12 Komponenten pro logischem Qubit \cite{QuandelaAnnounces100000fold2025}. Sollte sich diese massive Reduktion als validierbar und breit anwendbar erweisen, könnte sie die Ressourcenskalierung und Wirtschaftlichkeit der photonischen Quantenberechnung dramatisch verändern. Dies könnte zu einer Divergenz innerhalb des photonischen Paradigmas selbst führen, bei der sich Ansätze mit hoher Komponentenintegration von solchen abheben, die auf einer großen Anzahl konventioneller Bauteile beruhen. Bereits im November 2024 ermöglichte Quandela der europäischen Forschungsgemeinschaft einen ersten Fernzugriff auf eine 6-Qubit-Maschine \cite{QuandelaAnnounces100000fold2025}.
Parallel zu diesen industriellen Entwicklungen unterstreichen grundlegende Forschungsdemonstrationen die wachsende Leistungsfähigkeit photonischer Systeme. Experimente zum Boson Sampling mit über 100 Photonen haben in einigen Fällen die Grenzen klassischer Simulationen überschritten \cite{LinearOpticsScalable}. Übergreifend zeigen die Roadmaps der führenden Unternehmen eine klare Ausrichtung auf die zentralen Herausforderungen: die Realisierung von Fehlertoleranz und den Aufbau großskaliger Systeme \cite{QuandelaAnnounces100000fold2025}.

\subsubsection{Engineering- und Skalierungslösungen}
Die Skalierung photonischer Quantencomputer erfordert erhebliche ingenieurtechnische Fortschritte. Eine Schlüsseltechnologie ist die integrierte Photonik, bei der photonische integrierte Schaltkreise (PICs) auf Plattformen wie Silizium, Siliziumnitrid (SiN) und Lithiumniobat ($LiNbO_3$) für Miniaturisierung, Stabilität und Massenfertigung sorgen \cite{abughanemPhotonicQuantumComputers2024}. Ein Beispiel hierfür ist die Errichtung einer eigenen Fabrik für Dünnschicht-Lithiumniobat (TFLN) durch QCi \cite{QuantumComputingInc}. Dies geht Hand in Hand mit der Entwicklung fortschrittlicher Fertigungstechniken, um Wellenleiter mit geringen Verlusten und effiziente Koppeltechniken zu realisieren \cite{LinearOpticsScalable}. Um die Effizienz probabilistischer Quellen zu steigern, werden zudem
Multiplexing-Strategien eingesetzt, die Photonen aus mehreren Quellen räumlich oder zeitlich bündeln \cite{salavrakosPhotonnativeQuantumAlgorithms2025}. Schließlich ist eine schnelle
Feedforward-Kontrolle unerlässlich, bei der Detektorsignale genutzt werden, um photonische Schaltkreise adaptiv neu zu konfigurieren. Dies ist eine Grundvoraussetzung für adaptive Protokolle und das messungsbasierte Quantencomputing (MBQC) \cite{salavrakosPhotonnativeQuantumAlgorithms2025}.

\subsubsection{Quantenfehlerkorrektur}
Aufgrund von Photonenverlust und probabilistischen Gattern ist die Quantenfehlerkorrektur (QEC) für die Realisierung fehlertoleranter photonischer Quantencomputer von fundamentaler Bedeutung \cite{swayneChineseScientistsOvercome2025}. Die Forschung konzentriert sich auf verschiedene Arten von Fehlerkorrekturcodes. Dazu gehören bosonische Codes wie die Gottesman-Kitaev-Preskill (GKP)-Zustände, die die kontinuierliche Natur des Lichts ausnutzen, sowie an photonische Systeme angepasste topologische Codes und Oberflächencodes \cite{LinearOpticsScalable}. Ein wichtiger experimenteller Fortschritt wurde 2025 von Forschern der University of Science and Technology of China (USTC) in
Nature berichtet: Sie demonstrierten eine Einzelphotonenquelle mit einer Effizienz von 71,2\%, die damit die theoretische Schwelle von zwei Dritteln übertrifft, welche für die Funktionsfähigkeit von QEC in der photonischen Quantenberechnung als kritisch gilt \cite{swayneChineseScientistsOvercome2025}. Parallel dazu zielt der Ansatz von Quandela darauf ab, die für die Konstruktion logischer Qubits benötigten physikalischen Ressourcen drastisch zu reduzieren, was die Implementierung von QEC erheblich vereinfachen könnte \cite{QuandelaAnnounces100000fold2025}.
Der photonische Ansatz befindet sich in einer dynamischen Entwicklungsphase, mit vielversprechenden Fortschritten sowohl bei den grundlegenden Komponenten als auch bei der Systemintegration und den Strategien zur Fehlertoleranz.

\subsection{Halbleiter basierte Qubits / Spin Qubits}
Halbleiter-Spin-Qubits nutzen den Spin von Elektronen oder Atomkernen, die in Nanostrukturen aus Halbleitermaterialien eingeschlossen sind. Ihre Hauptattraktivität liegt in der potenziellen Kompatibilität mit der hoch entwickelten CMOS-Fertigungstechnologie (Complementary Metal-Oxide-Semiconductor), die eine Skalierung zu großen Qubit-Zahlen verspricht.

\subsubsection{Physische Realisierung}
Die physikalische Realisierung von Spin-Qubits basiert auf mehreren Kernprinzipien. Die Qubit-Kodierung erfolgt typischerweise durch die Spin-Ausrichtung („Spin-up“ und „Spin-down“) eines einzelnen Elektrons oder Lochs, das elektrostatisch in einem Quantenpunkt (QD) gefangen ist. Eine spezielle Variante sind die Exchange-Only (EO) Qubits, bei denen der Qubit-Zustand in einem dekohärenzfreien Unterraum von drei Elektronenspins auf drei benachbarten Quantenpunkten definiert und ausschließlich über die Austauschwechselwirkung gesteuert wird (\cite{chadwickShortTwoqubitPulse2025}). Als
Materialien dominiert Silizium, oft in Form von Si/SiGe-Heterostrukturen oder Si-MOS-Bauelementen (Metall-Oxid-Halbleiter). Silizium bietet den Vorteil einer ausgereiften Prozesstechnologie und, insbesondere bei Verwendung von isotopenreinem $28Si$ ohne Kernspin, das Potenzial für sehr lange Kohärenzzeiten. Ein entscheidender Faktor ist die
CMOS-Kompatibilität. Die strukturelle Ähnlichkeit von Quantenpunkten mit Transistoren legt nahe, dass die etablierte Infrastruktur der Halbleiterindustrie für die Massenproduktion von Spin-Qubit-Prozessoren genutzt werden könnte (\cite{stuyckCMOSCompatibilitySemiconductor2024}). Diese Kompatibilität ist jedoch nicht trivial, da Quantenanforderungen wie extrem rauscharme Umgebungen, spezifische Materialreinheit und Tieftemperaturbetrieb eine signifikante Ko-Entwicklung und Anpassung der Standard-CMOS-Prozesse erfordern. Dies könnte die Geschwindigkeit begrenzen, mit der Spin-Qubits die Skalierung klassischer Chips nachahmen können (\cite{chadwickShortTwoqubitPulse2025}).

\subsubsection{Auslesen der Spin-Qubits}
Die Kontrolle der Spin-Qubits erfolgt auf verschiedene Weisen. Einzel-Qubit-Rotationen werden typischerweise durch resonante Mikrowellenpulse erzeugt, entweder über Elektronenspinresonanz (ESR) mit einer On-Chip-Antenne oder über elektrische Dipolspinresonanz (EDSR), die elektrische Felder zur Spinmanipulation nutzt (\cite{stuyckCMOSCompatibilitySemiconductor2024}). Zwei-Qubit-Gatter werden meist durch die Austauschwechselwirkung zwischen benachbarten Spins realisiert, deren Stärke durch Anpassen der Gatespannungen und der damit verbundenen Tunnelbarriere zwischen den Quantenpunkten gesteuert wird. EO-Qubits werden, wie ihr Name andeutet, ausschließlich über die Austauschwechselwirkung gesteuert, was die Notwendigkeit präziser lokaler Magnetfelder oder Mikrowellenkontrolle eliminiert (\cite{chadwickShortTwoqubitPulse2025}).
Das Auslesen des Spinzustands erfolgt üblicherweise über einen Prozess der Spin-zu-Ladungs-Konversion. Dabei wird der Spinzustand in einen messbaren Ladungszustand überführt, beispielsweise ob ein Elektron einen Quantenpunkt besetzen kann oder nicht. Dieser Ladungszustand wird dann mit hochempfindlichen Elektrometern wie Einzel-Elektronen-Transistoren (SETs) oder benachbarten Quantenpunkten detektiert (\cite{stuyckCMOSCompatibilitySemiconductor2024}).

\subsubsection{Vorteile und spezifische Herauforderungen}
Spin-Qubits bieten eine Reihe attraktiver Vorteile. Ihre kleine Qubit-Größe ermöglicht potenziell sehr hohe Integrationsdichten auf einem Chip (\cite{stuyckCMOSCompatibilitySemiconductor2024}).  Insbesondere in isotopenreinem Silizium wurden
lange Kohärenzzeiten demonstriert, was für die Durchführung komplexer Algorithmen entscheidend ist (\cite{stuyckCMOSCompatibilitySemiconductor2024}).  Der wohl größte Vorteil ist die
Kompatibilität mit industrieller Fertigung, die das Versprechen einer kostengünstigen Skalierung auf Millionen von Qubits birgt (\cite{stuyckCMOSCompatibilitySemiconductor2024}).  Jüngste Fortschritte zeigen zudem eine
Betriebsfähigkeit bei Temperaturen über 1 Kelvin, was die anspruchsvollen kryogenen Anforderungen im Vergleich zum Millikelvin-Bereich potenziell erleichtert (\cite{stuyckCMOSCompatibilitySemiconductor2024}). 
Diesen Vorteilen stehen jedoch erhebliche Herausforderungen gegenüber. Eine Hauptquelle für Dekohärenz und Gatterfehler ist Ladungsrauschen, also Fluktuationen im elektrostatischen Umfeld, die oft durch Defekte an Grenzflächen oder in Oxidschichten verursacht werden (\cite{stuyckCMOSCompatibilitySemiconductor2024}).  Dieses Rauschen kann zudem stark zwischen benachbarten Qubits korreliert sein, was die Fehlerkorrektur erschwert. Die Materialwissenschaft und Fertigungspräzision im Nanobereich sind daher für den Erfolg von Spin-Qubits von überragender Bedeutung. Ein weiteres Problem ist die
Qubit-Variabilität: Kleinste Abweichungen in der Nanofabrikation führen zu unterschiedlichen Qubit-Eigenschaften, was das Einstellen der Betriebsparameter großer Qubit-Arrays erheblich erschwert. In dichten Arrays führen zudem Restkopplungen und unbeabsichtigte Wechselwirkungen zu Fehlern, bekannt als
Konnektivität und Crosstalk. Schließlich erfordern Spin-Qubits typischerweise einen
kryogenen Betrieb im Millikelvin-Bereich. Obwohl Fortschritte beim Betrieb über 1K erzielt wurden, bleibt die Entwicklung von Kryo-CMOS-Elektronik zur Steuerung der Qubits ein kritischer paralleler Entwicklungspfad, um den "Verdrahtungsengpass" bei der Skalierung zu überwinden (\cite{stuyckCMOSCompatibilitySemiconductor2024}). 

% \input{content/jakob/neutralatom}
\subsection{Topologische Qubits}
Die in den vorangegangenen Kapiteln beschriebenen Ansätze zur Realisierung von Quantencomputern – seien es supraleitende Schaltkreise, gefangene Ionen oder Photonen – teilen eine grundlegende Eigenschaft: Die Quanteninformation wird in lokalen physikalischen Zuständen gespeichert. Diese lokale Kodierung macht die Qubits anfällig für Dekohärenz, also den Verlust ihrer fragilen Quanteneigenschaften durch Wechselwirkungen mit der Umgebung \cite{bolgarMicrosoftsMajorana1}. Die gängige Strategie zur Bewältigung dieses Problems ist die Quantenfehlerkorrektur (QEC), bei der die Information eines einzelnen „logischen“ Qubits redundant auf viele fehleranfällige „physische“ Qubits verteilt wird. Dieser Ansatz erfordert jedoch einen enormen Hardware-Aufwand, der Schätzungen zufolge Tausende von physischen Qubits für ein einziges robustes logisches Qubit umfassen kann \cite{PDFMicrosoftsMajorana2025}.

Das topologische Quantencomputing verfolgt einen radikal anderen Weg. Anstatt Fehler nachträglich durch komplexe Kodierungsschemata zu korrigieren, zielt dieser Ansatz darauf ab, den Fehlerschutz direkt in die physikalische Struktur der Hardware zu integrieren \cite{lutchynRealizingMajoranaZero2017}. Die Kernidee ist, Quanteninformation nicht lokal, sondern in den globalen, topologischen Eigenschaften eines Materials zu speichern. Solche Eigenschaften sind von Natur aus robust gegenüber lokalen Störungen, ähnlich wie die Anzahl der Löcher in einem Donut unverändert bleibt, solange man ihn nicht zerreißt.

Die physikalische Grundlage für diesen Ansatz bilden exotische Quasiteilchen, die als nicht-abelsche Anyonen bezeichnet werden. Die vielversprechendsten Kandidaten für deren Realisierung in Festkörpersystemen sind die sogenannten Majorana-Nullmoden (MZMs) \cite{dougfinkeDeeperDiveMicrosofts2023}. Ein Paar dieser räumlich getrennten MZMs kann ein einzelnes, nicht-lokal geschütztes Qubit kodieren \cite{lutchynRealizingMajoranaZero2017}. Dieses Kapitel konzentriert sich auf die immense Herausforderung der Hardware-Realisierung solcher Majorana-basierten Qubits und beleuchtet die jüngsten Fortschritte, insbesondere im Kontext von Microsofts „Majorana 1“-Chip, der als wichtiger Meilenstein in diesem „High-Risk, High-Reward“-Forschungsfeld gilt.

\subsubsection{Physikalische Realisierung}
Die theoretische Grundlage für die experimentelle Suche nach MZMs in eindimensionalen Systemen lieferte das Kitaev-Ketten-Modell, das die wesentlichen physikalischen Zutaten für ihre Erzeugung beschreibt \cite{PDFMicrosoftsMajorana2025}. Da die im Modell geforderte p-Wellen-Supraleitung in der Natur kaum vorkommt, konzentriert sich die Forschung auf die künstliche Herstellung eines äquivalenten Systems durch die Kombination bekannter Materialien. Die heute kanonische Plattform dafür ist eine Heterostruktur aus einem Halbleiter-Nanodraht und einem konventionellen Supraleiter.

Die Umsetzung erfordert ein präzises Zusammenspiel von drei Komponenten. Als Basismaterial dienen Nanodrähte aus Halbleitern wie Indiumarsenid (InAs), die eine sehr starke Spin-Bahn-Kopplung aufweisen. Diese Eigenschaft koppelt den Spin eines Elektrons an seine Bewegung. Zweitens wird ein konventioneller s-Wellen-Supraleiter, typischerweise Aluminium (Al), in unmittelbarer Nähe zum Nanodraht aufgebracht, oft als dünne, epitaktisch aufgewachsene Hülle. Durch den supraleitenden Proximity-Effekt werden Cooper-Paar-Korrelationen in den Halbleiter induziert, wodurch dieser sich wie ein Supraleiter verhält. Als dritte Zutat wird ein externes Magnetfeld parallel zur Achse des Nanodrahts angelegt. Dieses Feld bewirkt eine Zeeman-Aufspaltung der Energieniveaus. Erreicht das Magnetfeld eine kritische Stärke, durchläuft das System einen topologischen Phasenübergang, und an den Enden des Nanodrahts entstehen die gepaarten Majorana-Nullmoden. \cite{amorimMajoranaBraidingDynamics2015}

Obwohl dieses Rezept konzeptionell klar ist, stellt seine praktische Umsetzung enorme Anforderungen an die Materialwissenschaft und Nanofabrikation. Die Qualität der Grenzfläche zwischen Halbleiter und Supraleiter ist von entscheidender Bedeutung. Sie muss atomar rein und scharf sein, um einen effizienten Proximity-Effekt und eine robuste, sogenannte „harte“ induzierte supraleitende Lücke zu gewährleisten. Jegliche Unordnung oder Verunreinigungen an dieser Grenzfläche können die topologische Phase zerstören. Ebenso muss der Nanodraht selbst von höchster kristalliner Qualität sein, da Defekte die Signaturen von MZMs überdecken oder fälschlicherweise imitieren können. Diese extremen Anforderungen verdeutlichen, warum die Herstellung funktionierender Bauteile eine der größten Hürden auf dem Weg zum topologischen Quantencomputer darstellt.


\subsubsection{Fallstudie: Microsofts ``Majorana 1''-Chip}

Nach jahrelanger Forschung, die von erheblicher Skepsis in der Fachwelt begleitet wurde, präsentierte Microsoft Anfang 2025 mit dem „Majorana 1“-Prozessor einen wichtigen experimentellen Fortschritt. Die technologische Basis dieses Chips sind die beschriebenen InAs-Al-Heterostrukturen, die Microsoft als „Topoconductor“ bezeichnet – ein Begriff, der ein ganzes materialwissenschaftliches Programm zur Optimierung der topologischen Supraleitung beschreibt. Die Architektur des Chips hat sich von einfachen linearen Nanodrähten zu komplexeren, skalierbaren Designs weiterentwickelt. Ein zentrales Element ist das „Tetron“-Qubit, das vier MZMs zur Kodierung eines logischen Qubits nutzt, die auf H-förmigen Nanodrahtstrukturen realisiert werden \cite{bolgarMicrosoftsMajorana1}.

Die zentrale Errungenschaft, die in der Fachzeitschrift Nature veröffentlicht wurde, war jedoch nicht die erneute Beobachtung einer mehrdeutigen Signatur, sondern die Demonstration einer neuen Messtechnik: die zuverlässige Einzelschuss-Messung der Fermionenparität \cite{PDFMicrosoftsMajorana2025}. Die Fermionenparität gibt an, ob sich eine gerade oder ungerade Anzahl von Elektronen in einem System befindet und ist die Eigenschaft, in der die Qubit-Information ($\ket{0}$ oder $\ket{1}$) kodiert ist.
Für diese Messung wird der topologische Nanodraht an einen Quantenpunkt gekoppelt, der als extrem empfindliches Elektrometer fungiert. Die Fähigkeit des Quantenpunkts, Ladung zu speichern, ändert sich messbar in Abhängigkeit von der Parität des MZM-Paares. Diese winzige Kapazitätsänderung wird dann mit hoher Präzision durch Mikrowellen in einer einzigen Messung detektiert, mit einer beeindruckend niedrigen Fehlerrate von nur 1\%. Diese Fähigkeit, eine fundamentale nicht-lokale Quanteneigenschaft direkt und zuverlässig zu messen, ist ein entscheidender Schritt, da sie die Grundlage für die Durchführung von Quantenoperationen bildet. \cite{PDFMicrosoftsMajorana2025}


\subsubsection{Quantenoperationen in topologischen Quantencomputern}

Die ursprüngliche Vision des topologischen Quantencomputings basierte auf dem physischen „Flechten“ (Braiding) von MZMs im Raum, um Gatteroperationen durchzuführen. Dies würde komplexe Nanodraht-Netzwerke mit T-Verzweigungen erfordern, deren Herstellung bei der geforderten Materialqualität eine gewaltige Herausforderung darstellt.6 Angesichts dieser Hürden hat sich das Feld zunehmend einem pragmatischeren Ansatz zugewandt: dem messungsbasierten Braiding \cite{amorimMajoranaBraidingDynamics2015}. Die zentrale Idee hierbei ist, dass eine Sequenz von Paritätsmessungen an Gruppen von stationären MZMs mathematisch äquivalent zu einer physischen Flechtoperation sein kann. Dieser Ansatz eliminiert die Notwendigkeit für komplexe T-Verzweigungen und erlaubt den Aufbau von Prozessoren aus einfachen, kachelartigen Architekturen wie dem Tetron-Design, was die Skalierbarkeit erheblich erleichtert \cite{bolgarMicrosoftsMajorana1}. Die von Microsoft demonstrierte Paritätsmessung ist somit ein fundamentaler Baustein für diese Strategie.

Trotz dieses Fortschritts bleibt eine zentrale wissenschaftliche Debatte bestehen: die eindeutige Unterscheidung von echten topologischen MZMs und trivialen, nicht-topologischen Zuständen, die ähnliche Signaturen erzeugen können. Insbesondere sogenannte Andreev-gebundene Zustände (ABS) können unter bestimmten Bedingungen ebenfalls zu Null-Energie-Zuständen an den Enden eines Drahtes führen und in Experimenten Signaturen erzeugen, die von denen echter MZMs kaum zu unterscheiden sind. Diese Möglichkeit, dass triviale Zustände die topologischen Signaturen „imitieren“, ist das wissenschaftliche Kernproblem des Feldes. Die Entwicklung von robusteren Messprotokollen wie der Paritätsmessung ist eine direkte Antwort auf diese Herausforderung. Dennoch wird auch in den jüngsten Veröffentlichungen eingeräumt, dass selbst diese fortgeschrittenen Techniken unter bestimmten, fein abgestimmten Bedingungen von nicht-topologischen Systemen nachgeahmt werden könnten, was die Notwendigkeit weiterer, unzweideutiger Experimente unterstreicht \cite{amorimMajoranaBraidingDynamics2015}.

\subsubsection{Fehlerquellen und Ausblick}

Neben der fundamentalen Herausforderung der Verifikation ist das sogenannte Quasiteilchen-Poisoning (QP) die dominante dynamische Fehlerquelle für Majorana-Qubits \cite{svetogorovQuasiparticlePoisoningTrivial2021}. Dieser Prozess beschreibt, wie ein unkontrolliertes Quasiteilchen – typischerweise ein Elektron aus einem aufgebrochenen Cooper-Paar – vom Qubit-System absorbiert wird. Ein solches Ereignis schaltet die Fermionenparität des Qubits zufällig um und führt damit zu einem logischen Bit-Flip-Fehler. Die Rate dieser Ereignisse, die durch thermische Anregungen oder Hintergrundstrahlung verursacht werden können, begrenzt direkt die Kohärenzzeit des topologischen Qubits. Die Reduzierung dieser Raten durch bessere Abschirmung und Materialdesigns ist daher ein zentrales Forschungsfeld \cite{svetogorovQuasiparticlePoisoningTrivial2021}.

Zusammenfassend lässt sich sagen, dass die Hardware-Entwicklung für topologisches Quantencomputing ein Niveau erreicht hat, auf dem die Demonstration entscheidender Bausteine wie der zuverlässigen Paritätsmessung auf dem „Majorana 1“-Chip möglich ist. Der endgültige, unumstößliche Nachweis der nicht-abelschen Flechtstatistik von MZMs steht jedoch noch aus und bleibt der „heilige Gral“ des Feldes.

Die nächsten Schritte auf dem Weg zu einem funktionierenden topologischen Quantencomputer sind klar definiert: der Übergang von einzelnen Bauteilen zu Multi-Qubit-Systemen, die Demonstration von messungsbasiertem Braiding zwischen ihnen und schließlich der Bau eines ersten logischen Qubits.Der Weg dorthin bleibt fundamental und lang, doch ein Durchbruch hätte das Potenzial, die Landschaft des Quantencomputings durch die Realisierung einer inhärent fehlertoleranten Hardware grundlegend zu verändern.

\subsection{Abschließender Vergleich}
\begin{table}[ht]
\centering
\caption{Vergleichende Analyse der Quantencomputing-Hardware-Modalitäten (Stand 2025)}
\label{tab:quantum_comparison_simple}
\begin{tabular}{lllll}
\toprule
\textbf{Ansatz} & \textbf{Stärken} & \textbf{Schwächen} & \textbf{Herausforderungen \& Risiken} & \textbf{Chancen \& Ausblick} \\
\midrule

Topologisch & Inhärenter Fehlerschutz, hohe Dichte & Physik unbewiesen, Materialanforderungen & \textbf{Ambivalenz} bei Nachweis, \textbf{Quasiteilchen-Vergiftung} & Hohes Risiko, direkter Weg zu Fehlertoleranz \\
\addlinespace % Adds a little extra space between rows

Photonisch & Hohe Kohärenz, Raumtemperatur-Betrieb & Photon-Verlust, probabilistische Gatter & \textbf{Fertigungsintegration}, \textbf{Kryotechnik-Dilemma} & Skalierbare, vernetzte Architekturen \\
\addlinespace

Neutrale Atome & Höchste Qubit-Zahl, hohe Konnektivität & Langsame Gatter, ungelöste Interconnects & \textbf{Skalierungs-Mauer} bei Modulverbindung & Ideal für Simulation, effiziente Fehlerkorrektur \\
\addlinespace

Supraleitend & Technologisch reif, schnelle Gatter & Begrenzte Konnektivität, Rauschanfälligkeit & \textbf{Konnektivitäts-Limitierung}, Qubit-Variabilität & Führend bei NISQ-Anwendungen, Cloud-Integration \\
\addlinespace

Silizium-Spin & Sehr hohe Dichte, CMOS-kompatibel & Weniger reif, Gatter-Fidelität & \textbf{Material-Paradox} (Si vs. $^{28}$Si), Kontrollintegration & Langfristig vielversprechend für Millionen Qubits \\

\bottomrule
\end{tabular}
\end{table}


\section{Quantencomputer-Architekturen und Vernetzung}
Dieses Kapitel widmet sich den physischen Architekturen von Quantenprozessoren und des aufstrebenden Feldes der Quantennetzwerken. Es wird auf den Aufbau von Qubit-Arrays und die Mechanismen ihrer Kopplung eingegangen, die für kohärente Operationen unerlässlich sind. Darüber hinaus werden die strategischen Ansätze zur Skalierung von Quantensystemen durch modulare und verteilte Architekturen beleuchtet, wobei die entscheidende Rolle von Quanten-Interkonnektoren im Mittelpunkt steht. Ein weiterer Schwerpunkt liegt auf der unterstützenden Infrastruktur, einschließlich Kryo- und Hochfrequenzelektronik sowie fortschrittlichen Rauschunterdrückungstechniken, die für den stabilen Betrieb von Quantenhardware unerlässlich sind. Abschließend werden die Konzepte und Technologien erster Quantennetzwerke erörtert, die die Grundlage für ein zukünftiges Quanteninternet bilden. Dieses Kapitel verdeutlicht, dass die Überwindung von Skalierungsbeschränkungen für die Beherrschung der Quantenkommunikation benötigt wird.
\subsection{Aufbau eines Quantenprozessors: Qubit-Array, Kopplungsmechanismen}
Dieser Abschnitt befasst sich mit den grundlegenden physikalischen Realisierungen von Quantenprozessoren, wobei ihre räumliche Anordnung in Arrays und die  Mechanismen, die ihre kohärente Interaktion ermöglichen, beschrieben werden.
\subsubsection{Qubit-Array-Architekturen: Vielfalt und Skalierungsherausforderungen}
Quantenprozessoren arbeiten grundlegend anders als klassische Systeme, indem sie Qubits nutzen, die Superposition und Verschränkung für die Berechnung verwenden. (Quelle 1) Die Architektur muss diese Quantenprinzipien berücksichtigen, und verschiedene physikalische Implementierungen von Qubits werden aktiv erforscht, wobei jede einzigartige Eigenschaften und Skalierungsherausforderungen aufweist. (Quelle 1)
Derzeit werden Supraleitende Qubits, Ionenfallen Qubits, Silizium Spin Qubits, und Topologische Qubits im Bezug auf die Skalierbarkeit erforscht. Diese physikalischen Implementierungen wurden bereits vorangegangen in Kapitel 3 erläutert. Die Diskussion der verschiedenen Qubit-Plattformen – supraleitend, Ionenfallen, Silizium-Spin und topologisch – offenbart einen grundlegenden Aspekt im Design von Quantenprozessoren. Es gibt Kompromisse bei der Optimierung für Skalierbarkeit. Jede Plattform bietet spezifische Vorteile, wie die hohe Qubit-Anzahl bei Supraleitern, die langen Kohärenzzeiten bei Ionenfallen, die CMOS-Kompatibilität von Silizium-Qubits oder die  Fehlertoleranz topologischer Qubits. Gleichzeitig wird für jede Plattform ein spezifisches Skalierungsproblem deutlich. Bei supraleitenden Systemen sind dies beispielsweise kryogene Beschränkungen und die Verdrahtungsdichte. Bei Ionenfallen die individuelle Adressierung und der Aufwand durch Ionen-Shuttling. Bei Silizium-Qubits Rauschen und bei topologischen Qubits der Reifegrad der experimentellen Umsetzung.1 Dies verdeutlicht, dass die Verbesserung einer wünschenswerten Eigenschaft, wie etwa der Kohärenzzeit, oft zu neuen oder verstärkten Herausforderungen in anderen Bereichen führt, wie der Skalierbarkeit, der Komplexität der Steuerung oder der thermischen Last. Die Wahl einer Architektur ist daher keine Suche nach dem "besten" Qubit-Typ, sondern erfordert eine sorgfältige Abwägung dieser Kompromisse im Hinblick auf die angestrebte Skalierungsstrategie und die erforderliche Fehlertoleranz. Dies führt zu der Erkenntnis, dass die Zukunft der Quantenprozessoren wahrscheinlich nicht von einem einzigen dominierenden Qubit-Typ geprägt sein wird, sondern von hybriden Ansätzen oder spezialisierten Architekturen, die auf spezifische rechnerische Aufgaben zugeschnitten sind. Der Fokus verschiebt sich von der bloßen Erhöhung der Qubit-Anzahl hin zur Schaffung von qualitativ hochwertigen, miteinander verbundenen Qubits. Dies erfordert eine ganzheitliche Betrachtung des Systems, bei der die Interaktionen zwischen den Qubits und ihrer Umgebung von Anfang an im Design berücksichtigt werden.
Die folgende Tabelle fasst die Merkmale und Herausforderungen der verschiedenen Qubit-Architekturen zusammen:
\begin{table}[h]
  \centering % Zentriert die Tabelle innerhalb der table-Umgebung
\begin{tabular}{ |p{2,5cm}|p{3,5cm}|p{3cm}|p{3cm}|p{1cm}|  }
 \hline
 \textbf{Qubit Typ}& \textbf{Kopplungsmechanismus} & \textbf{Hauptvorteil} & \textbf{Herausforderung} & \textbf{Quelle}\\
 \hline
 \textbf{Supraleitend}   & Abstimmbar/Parametrisch, Fest    & Hohe Qubit-Anzahl, etablierte Fertigungsprozesse, schnelle Gatter & Kryogene Anforderungen, Verdrahtungsdichte, thermische Last, Übersprechen & 1\\
 \hline
  \textbf{Ionenfallen}   & Laser-vermittelt, Ionen-Shuttling    & Hohe Fidelity, lange Kohärenzzeiten, universelles Gatter-Set & Skalierbarkeit (Einzel-Falle), Ionen-Shuttling-Overhead, Adressierung in großen Arrays & 1\\
 \hline
  \textbf{Silizium Spin}   & Elektrisch (Gate-Steuerung)    & Kompatibilität mit Halbleiterindustrie, Potenzial für Millionen Qubits & ntegration von Auslese/Steuerung, Rauschen, Übersprechen, Resonator-Ringing & 3\\
 \hline
   \textbf{Topologisch}   & Topologische Eigenschaften (z.B. Majorana-Nullmoden)    & Intrinsische Fehlertoleranz, Robustheit gegenüber Umgebungsrauschen & Experimenteller Reifegrad, Komplexität der Realisierung & 1\\
 \hline

 \hline
\end{tabular}
\end{table}
\subsubsection{Kopplungsmechanismen: Von fest zu parametrisch und Sonstige}
Effiziente und hochfidele Kopplungsmechanismen sind von entscheidender Bedeutung, um Wechselwirkungen zwischen Qubits zu vermitteln, Zwei-Qubit-Gatteroperationen zu ermöglichen und Dekohärenzfehler zu minimieren.15
Feste Kopplung: In frühen Experimenten mit supraleitenden Qubits wurde häufig eine feste Kopplung verwendet, die typischerweise durch Kondensatoren vermittelt wurde.18 Obwohl für kleine Systeme effektiv, erweist sich diese Strategie als schwierig auf eine große Anzahl von Qubits zu skalieren, da sie zu unerwünschtem Übersprechen und eingeschränkter Flexibilität führt.18
Abstimmbare und Parametrische Kopplung: Moderne Architekturen setzen zunehmend auf abstimmbare oder parametrisch angesteuerte Koppler.5 Diese Koppler ermöglichen eine selektive Aktivierung von Wechselwirkungen zwischen Qubits, wodurch Übersprechen und umgebungsbedingte Dekohärenz reduziert werden können.16 Parametrische Wechselwirkungen können eine Reihe von Operationen ausführen, darunter Zwei-Qubit-Gatter (z.B. Controlled-Z-Gatter mit einer Fidelity von 98,3), Reset-Operationen, Leckage-Wiederherstellung und das Auslesen von Qubits, alles mit einem einzigen abstimmbaren Koppler. Dies reduziert die Systemkomplexität und den Steueraufwand in skalierbaren Quantenprozessoren erheblich.15 Sie ermöglichen zudem schnelle Zwei-Qubit-Operationen und können mit hoher Geschwindigkeit und einem hohen Ein/Aus-Verhältnis ein- und ausgeschaltet werden.18 Eine Herausforderung besteht jedoch darin, dass das Erreichen einer starken Kopplung zu Qubits unbeabsichtigt die Kohärenz reduzieren kann, was einen Kompromiss darstellt.17 Resonante Einzel-Qubit-Operationen können langsam sein und erfordern große Ansteueramplituden, was in dichteren Schaltungen zu übermäßigem Übersprechen führen kann.20
Metasurfaces für polychromatische Kopplung: Ein innovativer Ansatz beinhaltet kryogen-kompatible Metasurfaces, die eine einzelne Eingangsfrequenz in mehrere gezielte Frequenzen umwandeln und so eine polychromatische Qubit-Kopplung ermöglichen.16 Diese Technologie mindert Übersprechen und umgebungsbedingte Dekohärenz, verlängert die Kohärenzzeiten und bewahrt die Fidelity des Quantenzustands, wodurch komplexe Operationen wie Multi-Qubit-Verschränkung und Fehlerkorrektur unterstützt werden.16
Steuerung von Kopplung und Rauschen: Das komplexe Zusammenspiel von Gerätearchitektur und Leistung mit Qubit-Steuerungsschemata, Erwärmung und Übersprecheffekten ist eine große Herausforderung.10
Die Entwicklung von Kopplungsmechanismen in Quantenprozessoren wird von einem grundlegenden Trilemma geprägt: dem "Steuerung-Kohärenz-Übersprechen"-Problem. Feste Kopplungsmechanismen, obwohl einfach zu implementieren, sind für große Qubit-Anzahlen nicht skalierbar, da sie zu unkontrollierbarem Übersprechen führen.18 Abstimmbare und parametrische Kopplung bieten hier eine deutlich verbesserte Kontrolle und Flexibilität, indem sie selektive Wechselwirkungen ermöglichen und den Steueraufwand reduzieren.15 Die Notwendigkeit schneller Gatteroperationen erfordert jedoch oft starke Ansteuerungen, die paradoxerweise die Qubit-Kohärenz reduzieren können 17 und in dichten Qubit-Arrays zu unerwünschtem Mikrowellen-Übersprechen führen.20 Die Einführung von Lösungen wie kryogen-kompatiblen Metasurfaces, die polychromatische Kopplung ermöglichen, oder die präzise Gestaltung von Steuerpulsen, um Resonanzüberlappungen zu vermeiden, sind Versuche, dieses Trilemma zu navigieren.10 Es geht hierbei nicht nur um die Verbesserung einzelner Komponenten, sondern um die Optimierung der systemweiten Interaktion, um sicherzustellen, dass Verbesserungen in einem Bereich nicht zu einer katastrophalen Verschlechterung in einem anderen führen. Dies erfordert ein tiefes Verständnis der physikalischen Wechselwirkungen und eine präzise technische Umsetzung. Diese Entwicklung deutet darauf hin, dass die zukünftige Entwicklung von Quantenprozessoren stark auf einem integrierten Co-Design von Qubit-Arrays und ihren Steuerungs- und Kopplungsmechanismen basieren wird, das über die isolierte Komponentenoptimierung hinausgeht. Dies erfordert ausgefeilte Simulations- und Fertigungstechniken, um komplexe Interaktionseffekte vorherzusagen und zu mindern. Die Fähigkeit, diese Kompromisse effektiv zu managen, wird die Skalierbarkeit und Leistungsfähigkeit zukünftiger Quantencomputer maßgeblich bestimmen.

\subsection{Skalierungsstrategien: Modulare Systeme}
\subsection{Unterstützende Infrastruktur: Kryo-Elektronik, Hochfrequenzelektronik, Steuerungseinheiten, Filter gegen thermisches Rauschen}
\subsection{Erste Netzwerke: Konzepte des Quanteninternets (Architekturmodell), Quantenrepeater, Quantenrouter, Verschränkungsverteilung}

\section{Praxisbeispiel(e): Im Inneren eines IBM-Quantencomputers}
Die Entwicklung kommerziell nutzbarer Quantencomputer stellt eine der größten wissenschaftlichen und ingenieurtechnischen Herausforderungen unserer Zeit dar. Während die physikalischen Grundlagen der Quantenmechanik, wie Qubit-Logik, Superposition und Verschränkung in den vorangegangenen Kapiteln dieses Buches bereits detailliert erläutert wurden, fokussiert sich dieses Kapitel auf die konkrete Hardwarearchitektur und Systemintegration eines der ersten kommerziellen Quantencomputersysteme: das IBM Q System One. Dieses System wurde erstmals 2019 vorgestellt und markiert einen wichtigen Meilenstein im Quantencomputing. IBM Q System One zielt darauf ab, Quantencomputing aus dem reinen Forschungslabor heraus und in eine zuverlässigere, wartungsärmere und industriell einsetzbare Form zu überführen (Gambetta et al., 2019; IBM News Room, 2019).


\subsection{Aufbaus eines kommerziellen Quantencomputers - IBM Q System One}
Beschreibung des Aufbaus eines kommerziellen Quantencomputers - IBM Q System One
Dieser Abschnitt beschreibt die physikalische Struktur eines typischen supraleitenden Quantencomputers. Im Fokus steht der Qubit-Chip, der auf einer stark heruntergekühlten Plattform montiert ist – einer sogenannten Verdünnungskühlstufe mit Temperaturen im Millikelvin-Bereich. Verschachtelte Abschirmungen und Vakuumkammern sorgen für eine minimale Störung durch Wärme, Strahlung oder elektromagnetische Einflüsse von außen.

\subsection{Foto-Illustration}
Foto-Illustration: Kaltes Verdünnungskryostat mit hängender Chip-Ebene (Gold-Coax-Kabel zu Qubits)
Hier wird mithilfe eines Bildes gezeigt, wie ein realer Kryostat aufgebaut ist, in dem der Qubit-Chip „hängt“. Die goldfarbenen Koaxialkabel, die an den Chip führen, dienen der Steuerung und Auslesung der Qubits mit hochfrequenten Mikrowellensignalen. Das Bild veranschaulicht die aufwendige technische Infrastruktur, die notwendig ist, um Quantenoperationen durchzuführen.


\subsection{Erläuterung eines einzelnen supraleitenden Qubits (Transmon)}
Erläuterung eines einzelnen supraleitenden Qubits (Transmon) und wie ein Zwei-Qubit-Gatter durch kapazitive Kopplung realisiert wird
Ein Transmon-Qubit ist ein spezieller supraleitender Schaltkreis, der zur Stabilisierung gegen Ladungsrauschen designt wurde. In diesem Teil wird erklärt, wie durch gezielte Mikrowellenpulse Zustände manipuliert und gelesen werden können. Zusätzlich wird gezeigt, wie zwei Transmon-Qubits über kapazitive Kopplung ein kontrolliertes Quantenlogikgatter bilden – ein zentrales Element zur Realisierung von Quantenalgorithmen.

\subsection{IBM Quantum System Two – Auf dem Weg zur Quanten-zentrierten Supercomputation}
Die Einführung des IBM Quantum System Two, dessen Details erstmals im Dezember 2023 umfassend vorgestellt wurden (Gambetta, 2023; IBM News Room, 2023), signalisiert einen signifikanten Fortschritt durch IBM in der Entwicklung kommerziell verfügbarer Quantencomputersysteme. Dieses System stellt nicht lediglich eine Weiterentwicklung des IBM Q System One dar, sondern verkörpert einen Paradigmenwechsel hin zu einer Architektur, die durch Modularität, Skalierbarkeit und Vernetzbarkeit charakterisiert ist. Diese Architektur soll als Fundament für das Konzept des „Quantum-centric Supercomputing“ dienen.

\subsubsection{Designphilosophie und Architekturziele}
Im Unterschied zum eher monolithischen Aufbau des IBM Q System One, das primär für den Betrieb eines einzelnen Quantenprozessors (QPU) konzipiert war, wurde das IBM Quantum System Two von Grund auf mit Blick auf mehrere Kernziele entwickelt. 

Zu diesen Zielen zählt erstens die Modularität: Das System ist so strukturiert, dass es aus multiplen, potenziell miteinander verbundenen Modulen aufgebaut werden kann. Dieser Ansatz ermöglicht eine flexible Systemkonfiguration und eine schrittweise Erweiterung der Gesamtanlage. Zweitens steht die Skalierbarkeit im Fokus, mit dem Ziel, ein Wachstum auf Tausende von Qubits und darüber hinaus zu ermöglichen. Dies soll durch die Kooperation mehrerer QPUs innerhalb eines oder mehrerer vernetzter Systeme realisiert werden. Drittens ist die Konnektivität von essenzieller Bedeutung für die Skalierbarkeit. Darunter wird die Fähigkeit verstanden, QPUs sowohl innerhalb eines Systems als auch systemübergreifend zu verbinden, um die Bearbeitung umfangreicherer und komplexerer Problemstellungen zu ermöglichen. Viertens wurden Aspekte der Servicefreundlichkeit und Aufrüstbarkeit berücksichtigt; das modulare Design ist darauf ausgelegt, Wartungsarbeiten sowie die Implementierung von Hardware-Upgrades – beispielsweise neue QPU-Generationen oder verbesserte Steuerungselektronik – zu vereinfachen. Fünftens bildet die Hybridität, also die nahtlose Integration mit klassischer Hochleistungsrechner-Infrastruktur, einen zentralen Bestandteil der Vision des Quantum-centric Supercomputing.Kernbestandteil der Vision des Quantum-centric Supercomputing.

\subsubsection{Systemarchitektur und Gehäuse}
Das IBM Quantum System Two weist eine deutlich veränderte äußere Erscheinungsform auf. Die für das System One charakteristische Glaskuppel wurde durch eine umfangreichere, hexagonale Struktur ersetzt. Diese besteht aus hexagonal geformten Basiseinheiten, welche jeweils einen Kryostaten mit Quantenprozessoren sowie die zugehörige unterstützende Infrastruktur aufnehmen können. Die Wahl der hexagonalen Form ist nicht rein ästhetisch motiviert, sondern dient auch funktionalen Zwecken, indem sie die Verbindung mehrerer solcher Einheiten zu größeren Clustern erlaubt (Red Dot Design Award, 2024). Jede dieser Einheiten weist Abmessungen von etwa 4,6 Metern in der Höhe und 6,7 Metern in der Breite auf. Die modulare Erweiterung wird dadurch realisiert, dass an die Seitenflächen dieser hexagonalen Einheiten weitere Module angedockt werden können. Diese zusätzlichen Module können entweder weitere QPUs oder klassische Steuer- und Peripherieelektronik enthalten, was ein physisches Wachstum des Systems parallel zur Steigerung der Rechenleistung ermöglicht. Bei den Materialien und dem Design des Gehäuses kommen eloxiertes, poliertes Aluminium sowie Glaselemente zum Einsatz. Die sichtbaren Fugen zwischen den Modulen akzentuieren den modularen Charakter des Systems (iF Design Award, 2024). Das Gesamtdesign zielt darauf ab, Prinzipien der Offenheit und Zugänglichkeit zu vermitteln, während gleichzeitig der Schutz der sensitiven Technologie gewährleistet wird.

\subsubsection{Kryogene Infrastruktur und QPU-Umgebung}
Die Notwendigkeit, eine steigende Anzahl von Qubits bei extrem tiefen Temperaturen – typischerweise im Bereich von 10 bis 20 Millikelvin für supraleitende Qubits – zu betreiben, stellt hohe Anforderungen an die kryogene Infrastruktur. Im Hinblick auf eine skalierbare Kühlung müssen, obgleich die fundamentalen Prinzipien der Dilutionsrefrigeration beibehalten werden, die Kühlleistung und die interne Kapazität der Kryostaten für das Quantum System Two signifikant erhöht werden. Dies ist erforderlich, um multiple oder größere QPUs sowie die damit assoziierte Verkabelung und Komponenten, wie beispielsweise Verstärker, adäquat zu versorgen. In diesem Kontext hat IBM auch den „Goldeneye“-Kryostaten erwähnt, einen Dilutionsrefrigerator mit einer besonders großen Kühlkapazität, welcher für die Kühlung zukünftiger Generationen von Multi-Chip-Prozessoren ausgelegt ist (SpinQ, 2025). Die Integration von Prozessoren ist ein weiteres Kernelement; Quantum System Two ist dafür konzipiert, mehrere Quantenprozessoren, beispielsweise drei IBM Heron Prozessoren, in einem einzelnen Kryostaten zu beherbergen und zu betreiben (arXiv, 2024). Eine solche Integration erfordert eine präzise Planung der internen Verkabelung, der thermischen Anbindung und der elektromagnetischen Abschirmung. Das Vibrationsmanagement und die Stabilität gewinnen in einem modularen und potenziell sehr ausgedehnten System zusätzlich an Komplexität und Kritikalität im Vergleich zum System One, um die Kohärenz der Quantenzustände sicherzustellen.

\includepdf[pages=1, fitpaper=true]{GrafikKühlungDeutsch.pdf}


\subsubsection{Signalübertragung, Steuerungselektronik und Konnektivität}
Mit der Zunahme der Anzahl von Qubits und Prozessoren steigen die Anforderungen an die Steuerungs- und Ausleseelektronik sowie an die Verbindungen zwischen den Systemkomponenten exponentiell. IBM hat eine Steuerungselektronik der dritten Generation entwickelt, die sich durch eine höhere Kompaktheit, gesteigerte Leistungsfähigkeit und eine engere Integration mit den QPUs auszeichnet. Diese Elektronik ist darauf ausgelegt, eine größere Anzahl von Qubits effizient zu steuern und auszulesen (IBM News Room, 2023). Die Erhöhung der Qubit-Dichte und -Anzahl macht zudem innovative Lösungen für eine hochdichte kryogene Verkabelung notwendig, um die Wärmelast gering zu halten und gleichzeitig eine hohe Signalintegrität zu gewährleisten. Flexible kryogene Hochfrequenzleitungen (CryoFlex) spielen hierbei eine wichtige Rolle. Für die Koordination und das sogenannte „Circuit Knitting“ – ein Verfahren, das große Quantenschaltkreise auf mehrere QPUs aufteilt und klassische Kommunikation von Zwischenergebnissen erfordert – sind leistungsfähige klassische Kommunikationslinks zwischen den Prozessoren und den Steuerungseinheiten unerlässlich. Langfristig verfolgt IBM das Ziel, auch direkte quantenkohärente Verbindungen (Quantum Links) zwischen QPUs zu realisieren, um echte verteilte Quantenberechnungen über mehrere Chips hinweg zu ermöglichen (IBM Quantum Roadmap).

\subsubsection{Integration klassischer und quantenmechanischer Komponenten}
Das IBM Quantum System Two ist als zentrales Element einer „Quantum-centric Supercomputing“-Architektur konzipiert. Dies impliziert, dass das System für eine enge Kooperation mit klassischen Supercomputern und Cloud-Ressourcen ausgelegt ist, um hybride Workflows zu unterstützen. Middleware und Software-Werkzeuge wie Qiskit werden kontinuierlich weiterentwickelt, um solche hybriden Quanten-Klassik-Workflows effizient zu orchestrieren, Rechenaufgaben dynamisch zu verteilen und Ergebnisse zu konsolidieren (The Quantum Insider, 2024). Der Ansatz des Quantum Serverless zielt darauf ab, Quanten- und klassische Berechnungen nahtlos in unterschiedlichen Umgebungen, sei es in der Cloud oder on-premises, zu integrieren und auszuführen.

\subsubsection{Wartung, Skalierbarkeit und Modularität in der Praxis}
Die praktische Umsetzung der Prinzipien von Modularität und Skalierbarkeit stellt einen Hauptfokus bei der Entwicklung des IBM Quantum System Two dar. Das System ist für eine phasenweise Inbetriebnahme (Phased Deployment) und für Upgrades konzipiert. Dies bedeutet, dass es schrittweise ausgebaut und mit neueren Prozessorgenerationen (wie den in der IBM Roadmap genannten zukünftigen Prozessoren Flamingo oder Kookaburra) oder verbesserter Steuerungshardware aufgerüstet werden kann, ohne dass ein Austausch des Gesamtsystems notwendig wird. Das Design berücksichtigt ferner die Notwendigkeit regelmäßiger Wartung durch entsprechend gestaltete Servicezugänge, die den Zugriff auf kritische Komponenten erleichtern. Über spezifische Wartungsprozeduren sind, ähnlich wie beim System One, öffentlich meist nur allgemeine Informationen verfügbar.

\subsubsection{Relevante Designentscheidungen aus Ingenieursperspektive}
Mehrere zentrale Designentscheidungen prägen die ingenieurtechnische Ausrichtung des IBM Quantum System Two. Die Priorisierung der Modularität stellt hierbei die fundamentalste und weitreichendste Entscheidung dar, welche die Skalierbarkeit und Zukunftsfähigkeit des Systems gewährleisten soll. Parallel dazu wurde ein starker Fokus auf die Interkonnektivität gelegt; die Fähigkeit, Prozessoren und Systeme miteinander zu verbinden, ist entscheidend, um die Limitierungen einzelner QPUs zu überwinden. Die Entwicklung des Systems folgt einem integrierten Ansatz, bei dem Hardware, Software (einschließlich Qiskit und Middleware), Steuerung und klassische Computerressourcen als eine kohärente Einheit betrachtet und entwickelt werden. Schließlich schafft die Architektur des Quantum System Two, obwohl sie selbst noch nicht die Ära der vollständigen Fehlertoleranz einläutet, die notwendige Skalierbarkeit und Komplexität, um fortgeschrittene Fehlerminderungs- und Quantenfehlerkorrekturcodes zu implementieren und zu testen.
\\\\
Das IBM Quantum System Two stellt somit einen entscheidenden Evolutionsschritt dar. Es legt die technologischen Grundlagen für Quantencomputer, die potenziell in der Lage sein werden, Probleme von praktischer Relevanz zu lösen, welche für klassische Supercomputer als unlösbar gelten. Die Realisierung dieser Vision ist maßgeblich von der erfolgreichen Bewältigung der komplexen ingenieurtechnischen Herausforderungen in den Bereichen Skalierung, Konnektivität und Systemintegration abhängig.


\printbibliography
