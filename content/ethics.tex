%\motto{Use the template \emph{chapter.tex} to style the various elements of your chapter content.}
\chapter{Ethische Aspekte}
\label{ethics} % Always give a unique label
% use \chaptermark{}
% to alter or adjust the chapter heading in the running head

\chapterauthor{Karin Mustermann, Max Mustermann}

\abstract{Kurzfassung des Kapitels}

\section{Chancen und Risiken}

\subsection{Chancen}

\begin{itemize}
\item Grundlagen
\item Gesundheit \& Medizin
\item Umwelt \& Klimaschutz
\item Soziale Gerechtigkeit \& Teilhabe
\item Infrastruktur \& Mobilität
\item Rahmenbedingungen \& Good Practices
\end{itemize}

\subsection{Risiken}
Datenschutz \& Informationssicherheit
\begin{itemize}
    \item Der Missbrauch von Technologien, einschließlich unbefugtem Zugriff auf sensible Daten durch Quantenhacking oder die Schaffung von Systemen, denen es an Transparenz und Rechenschaftspflicht mangelt, wird ebenfalls als potenzielle ethische und gesellschaftliche Probleme genannt, insbesondere im Zusammenhang mit fehlenden regulatorischen Rahmenbedingungen \cite{umbrello_quantum_2024}
\end{itemize}
\begin{itemize}
    \item Machtasymmetrien und Monopolisierung
    \item Gesellschaftliche Ungleichheit
    \item \item Unvorhersehbarkeit \& Kontrollverlust
    \item \item Dual-Use-Problematik
    \item \item Inkompatibilität
    \item \end{itemize}
\end{itemize}

Allgemeine ethische Herausforderungen und Risiken umfassen Bedenken hinsichtlich:

\subsubsection{\textbf{1. Technische Risiken}}
\begin{itemize}
    \item Systemzuverlässigkeit und Stabilitätsprobleme
    \item Gefahr von Quantum Hacking
    \item \textbf{Technologischer Wandel} bzw. schnelle \textbf{Obsoleszenz} (Überalterung neuer Technologien)
    \item Beeinträchtigung der \textbf{Funktionalität} und \textbf{Sicherheit} von Quanten-Technologien
\end{itemize}

\subsubsection{\textbf{2. Ethische und gesellschaftliche Risiken}}
\begin{itemize}
    \item \textbf{Verletzung der Privatsphäre} durch neue Möglichkeiten der Datenanalyse
    \item \textbf{Bias (Voreingenommenheit)} in Entscheidungsalgorithmen
    \item \textbf{Mangel an Transparenz} und \textbf{fehlende Rechenschaftspflicht}
    \item Konflikt mit gesellschaftlichen Normen und \textbf{ethischen Prinzipien}
\end{itemize}

\subsubsection{\textbf{3. Ökonomische Risiken}}
\begin{itemize}
    \item \textbf{Ungleicher Zugang} zu Quanten-Technologien (Digital Divide)
    \item Verstärkung bestehender \textbf{gesellschaftlicher oder wirtschaftlicher Ungleichheiten}
    \item \textbf{Monopolisierung} oder Machtkonzentration bei wenigen Akteuren
    \item Risiken für eine faire \textbf{Verteilung wirtschaftlicher Vorteile}
\end{itemize}

\subsubsection{\textbf{4. Umweltbezogene Risiken}}
\begin{itemize}
    \item \textbf{Hoher Energieverbrauch} von Quantencomputern
    \item \textbf{Ökologische Auswirkungen} durch Herstellung und Betrieb
    \item Risiken für die \textbf{Nachhaltigkeit} und \textbf{Umweltverantwortung} beim Einsatz dieser Technologien
\end{itemize}

\cite{umbrello_quantum_2024}

\section{Governance \& Verantwortung}
\begin{itemize}
\item Rolle der Wissenschaft, Politik und Industrie
\item Globale ethische Standards vs. nationale Interessen
\item Braucht es ethische Leitplanken – und wenn ja, wie?
\item Internationale Regulierung: Lessons learned von der KI-Debatte
\end{itemize}


    \begin{itemize}
        \item aktuell noch keine regulatorischen Vorgaben für Nutzung und die Entwicklung von Quantentechnologien. Das begünstigt ethische und gesellschaftliche Herausforderungen. \cite{umbrello_quantum_2024} \cite{1-s2.0-S0370157324001078}

        \item Notwendigkeit betont, den Dialog anzustoßen, Kooperationen zu fördern und Strukturen zu schaffen, die nicht nur technologische Innovation vorantreiben, sondern zugleich gewährleisten, dass diese im Einklang mit unseren gemeinsamen menschlichen Werten gestaltet und gelenkt wird. \cite{umbrello_quantum_2024}
        \item  Es wird betont, wie wichtig es ist, ethische Diskussionen möglichst frühzeitig zu führen – auch als Lehre aus dem Umgang mit Künstlicher Intelligenz, bei der viele Ethikinitiativen als „zu spät“ kamen, um die Designkultur noch maßgeblich zu prägen. Daraus ergibt sich die Forderung, Governance-Strukturen und normative Rahmenwerke frühzeitig zu etablieren, um später nicht mühsam versuchen zu müssen, gewinnorientierte, aber potenziell schädliche Entwicklungen einzuhegen.\cite{1-s2.0-S0370157324001078}
        \item Die Methodik des Value Sensitive Design (VSD), ein Ansatz im Sinne von „Ethics-by-Design“, bezieht explizit unterschiedliche Stakeholder ein – darunter auch politische Entscheidungsträger. Dies verdeutlicht, dass die Perspektiven von Regulierungsinstanzen sowie der Aufbau geeigneter Governance-Strukturen als zentrale Elemente einer ethisch verantwortungsvollen Gestaltung von Quantentechnologien angesehen werden – auch in Abwesenheit formeller regulatorischer Rahmenbedingungen. \cite{umbrello_quantum_2024}
        \item Die Quantum Ethics Project (QEP) baut auf Responsible Innovation auf, versucht jedoch, Perspektiven der am stärksten marginalisierten Gruppen in den Mittelpunkt zu stelle \cite{1-s2.0-S0370157324001078}
        \item Das QEP bezieht auch gezielt politische Entscheidungsträger mit ein, um sicherzustellen, dass die künftige Quantenarbeitswelt gesellschaftlich breit aufgestellt ist. Zwar handelt es sich hierbei nicht um eine direkte Regulierung der Technologie, doch stellt dieses politische Engagement einen wichtigen Schritt zur Mitgestaltung des übergeordneten Governance-Rahmens dar. \cite{1-s2.0-S0370157324001078}

\section{Freier Wille für Maschinen?}



\section{Diskussion}

\begin{itemize}
\item Wie können Chancen maximiert, Risiken minimiert werden?
\item Was bedeutet „verantwortungsvolle Quantenforschung“?
\item Ethik als Designprinzip
\end{itemize}

\printbibliography
