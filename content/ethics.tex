%\motto{Use the template \emph{chapter.tex} to style the various elements of your chapter content.}
\chapter{Ethische Aspekte}
\label{ethics} % Always give a unique label
% use \chaptermark{}
% to alter or adjust the chapter heading in the running head

\chapterauthor{Halima Albader Alhusini, Gizem Bulut, Adrian Hußfeldt Vazquez, Tayebeh Nazari}

\abstract{Kurzfassung des Kapitels}

\section{Ethik und Grundlagen}
\subsection{Ethik‑Grundlagen \& Bewertungsrahmen}

Die schnelle Entwicklung des Quantencomputings wirft weitreichende Fragen auf, die rein über technische oder ökonomische Fragestellungen hinausgehen. Als disruptive Technologie hat QC das Potenzial, die globale Kommunikation, Sicherheitsmechanismen, politische und gesellschaftliche Machtverhältnisse sowie viele weitere Aspekte grundlegend zu verändern. Die potenziell weitreichenden Auswirkungen erfordern eine sorgfältige Auseinandersetzung auch mit ethischen Prinzipien und Maßstäben, um den Einsatz von QC auch mit gesellschaftlichen Werten in Einklang zu bringen. Ethik fungiert dabei als normativer Kompass: Sie zeigt nicht nur technische Potenziale auf, sondern auch, was moralisch erstrebenswert ist, und wo moralisch Risiken liegen. Gerade im Kontext des Quantencomputings, dessen langfristige Folgen aktuell noch schwer abzuschätzen sind, ist es von Bedeutung, geeignete ethische Ansätze zu identifizieren und diese bereits in der Entwicklung solcher Technologien anzuwenden. Ein Beispiel für die frühe Entwicklung ethischer Leitlinien und deren Bedeutung wird beim Thema künstliche Intelligenz deutlich. So hat die EU beispielsweise bereits im Jahr 2019 durch Fachexperten der Technikethik ethische Leitlinien für vertrauenswürdige KI entwickeln lassen. \cite{}
(Quelle: Ethics guidelines for trustworthy AI)

\subsection{Überblick über zentrale Ethiktheorien}

Es existieren unterschiedliche ethische Ansätze, die als Bewertungsrahmen für technologische Innovation herangezogen werden können. Dieser Abschnitt soll einen Überblick über ethische Grundmodelle im Bereich der Technikethik bieten, die zu unserer Thematik Anwendung finden können. In dieser Betrachtung werden die Kernelemente der jeweiligen Ethik betrachtet.

\subsubsection{Utilitarismus}
Der Utilitarismus ist Teil der konsequentialistischen Ethikansätze. Er geht zurück auf Philosophen wie Jeremy Bentham und John Stuart Mill und setzt das Prinzip der kollektiven Nutzenmaximierung aller Betroffenen ins Zentrum. Das Ziel ist demnach das größtmögliche Glück für die größtmögliche Zahl zu erzeugen. Die Folgen einer Handlung oder Innovation werden demnach bilanziell aufgestellt und aggregiert. Das Ziel besteht darin, das subjektive Wohlbefinden zu steigern. Der Utilitarismus fordert dabei auch ein Miteinbeziehen zukünftiger Generationen bei langfristigen Entscheidungen. Im Kontext von Technologien und Innovationen ist der Utilitarismus passend, da auch diese häufig die Charakteristik der Maximierung der Nützlichkeit aufweisen. Ebenfalls zeigt sich eine Ähnlichkeit in der utilitaristischen Ansicht, dass ein Eingriff in die Vorgabe der Natur zulässig ist, wenn eine technische Optimierung stattfindet. Es wird dabei rein auf die Folgen der "Optimierung" geachtet.  Jedoch existieren ebenfalls hier der Natur nähere "grüne" Interpretationen des Utilitarismus, welche Demut, Achtsamkeit und Kooperativität zur Erhaltung und Sicherung der Lebensgrundlagen als Kernelemente identifizieren. Bei der utilitaristischen Betrachtung technischer Entwicklungen müssen neben Chancen und Risiken auch Unsicherheit und Katastrophenpotenzial berücksichtigt werden. \cite{}
Quelle: Handbuch Technikethik, Seite 160 - 164

\subsubsection{Tugendethik}
Im Gegensatz zu den anderen großen Ethiktheorien (Deontologie und Konsequentialismus), bei welchen eher Handlungen im Mittelpunkt stehen, wendet sich die Tugendethik nicht den einzelnen Handlungen, sondern der Persönlichkeit der handelnden Person zu, also dem jeweiligen Träger der moralischen Verantwortung. Im Zentrum stehen dabei Tugenden, also bestimmte Haltungen eines moralischen Charakters, wie beispielsweise Klugheit (phronesis oder Verstandestugend), Mut und Gerechtigkeit, an welchen ein Mensch sein Handeln und Denken ausrichtet. Moralisches Handeln wird hierbei nicht als Pflicht angesehen, sondern als Ausdruck eines guten Charakters. Dieser entwickelt sich aus dem Studium des richtigen Handelns. Historisch hatte die Tugendethik, welche auf Aristoteles zurückgeht, ein tugendhaftes, also erfülltes Leben, zum Ziel. Im Kontext der Technologieentwicklung sind Ansätze erforderlich, um die Tugendethik in das Curriculum der Entwickler zu integrieren und technisches Wissen mit verantwortungsvollem Handeln in komplexen und unsicheren Situationen zu verknüpfen. Verbinden lässt sich dies, da technische Innovation meist ebenfalls eine Verbesserung des menschlichen Lebens zum Ziel hat. \cite{}
Quelle: Handbuch Technikethik, Seite 165 - 169


\subsubsection{Deontologie}
Die Deontologie gehört zu den bekanntesten und am weitesten verbreiteten ethischen Denkschulen. Einer ihrer bedeutendsten Vertreter und Begründer war Immanuel Kant. Über die Jahre hinweg haben sich in Bezug auf die Deontologie sowohl verschiedene Interpretationen als auch Definitionen ergeben, diese stehen zwar meist eng in Verbindung, sind allerdings nicht deckungsgleich. Im Vergleich zur Tugendethik steht bei der deontologischen Ethik nicht der Charakter des Handelnden, sondern die moralische Beurteilung der Handlung selbst im Vordergrund. Zentral ist die Vorstellung, dass bestimmte Handlungen aus Prinzip verboten oder geboten sind, unabhängig von den Konsequenzen. In der Ethik nach Kant bildet der kategorische Imperativ dabei den Kern: Jede Handlung muss so beschaffen sein, dass ihre zugrunde liegende Maxime als allgemeines Gesetz gelten kann. Damit weist Kant allen rationalen Subjekten universelle Pflichten zu, unabhängig davon, welchen Nutzen oder Schaden das konkrete Ergebnis dieser Handlung haben mag. Im Laufe des 20. Jahrhunderts wurde die kantische Pflichtethik in zahlreichen Varianten und Ergänzungen weiterentwickelt. In der Technikethik markiert die deontologische Perspektive einen grundsätzlichen Bruch mit rein zweck-mittelorientierten Entscheidungslogiken. Gemäß der Standarddefinition deontologischer Ethiken wird die moralische Richtigkeit technischer Handlungsoptionen nicht allein an ihrer Effizienz bei der Zielerreichung gemessen, sondern auch an Kriterien, die sich jeder reinen Nutzen- bzw. Risikoabwägung entziehen. So lassen sich Gerechtigkeitsüberlegungen zur Risikoverteilung oder Rechte wie beispielsweise das Recht auf Privatsphäre nicht einfach als Variable in eine Optimierung einbauen. Sie müssen vielmehr als vorrangige Normen gelten, die technische Lösungen einschränken. \cite{}
Quelle: Quelle: Handbuch Technikethik, Seite 171 - 174
pfleger, das gute Leben, Seite 94 - 101
Neuhäuser, handbuch angewandte ethik, Seite 67 - 73

\subsection{Technikethische Ansätze}



\subsection{Verantwortung}
In der Angewandten Ethik ist die Verantwortung ein Grundbegriff. Einerseits dient er dazu, zu klären, welche Akteure überhaupt als moralische Subjekte gelten können. Andererseits beschreibt er, wie abstrakte Rechte und Pflichten konkret auf Handelnde verteilt werden. Gerade in Anwendungsfeldern wie der Umwelt- oder Medizinethik oder im Bereich grundlegender technologischer Innovationen geht es immer wieder darum, wer welche Pflichten trägt und wofür diesbezüglich Rechenschaft abzulegen ist. Im Bereich der Ethik lässt sich die Verantwortung meist als dreistellige Relation darstellen mit den zentralen Fragestellungen: (1) Wer ist verantwortlich? (2) Wofür ist jemand verantwortlich? (3) Wem gegenüber wird die Verantwortung getragen? Diese Struktur ermöglicht es, die Komplexität von Verantwortungszuschreibungen abzubilden, während zugleich die Übersichtlichkeit erhalten bleibt. Zur ersten Frage „Wer ist verantwortlich?” – wird zwischen allgemeiner Verantwortungsfähigkeit und konkreter Verantwortungszuschreibung unterschieden. Verantwortungsfähig sind nur Akteure, die \textbf{willensfrei}, \textbf{handlungsfähig} sind und einen \textbf{moralischen Standpunkt einnehmen} können. Erst auf dieser Grundlage kann in einer konkreten Situation Verantwortung zugeschrieben werden. Bezüglich konkreter Verantwortung wird diese zugeschrieben, wenn eine Person sich der Verantwortung freiwillig bekennt, ihr die Verantwortung automatisch zufällt (z. B. wenn der diensthabende Notarzt in der Notaufnahme automatisch die Verantwortung für die Versorgung eines bewusstlosen Patienten übernimmt) oder diese zugewiesen wird. Die zweite zentrale Fragestellung nach dem "Wofür jemand verantwortlich ist" differenziert zwischen \textbf{haftender Verantwortung} für bereits eingetretene Folgen eigener Handlungen und \textbf{sorgender Verantwortung} für künftige Ereignisse oder ihre Abschätzung. Haftende Verantwortung bezieht sich auf die eigenen Handlungen und deren Folgen. Die sorgende Verantwortung hingegen fordert präventives, aktives Eingreifen, um mögliche Schäden abzuwenden oder zu mindern, auch wenn der Schaden nicht auf das eigene Handeln zurückgeht, sondern auf andere Personen oder natürliche Vorgänge. Zur dritten zentralen Fragestellung bezüglich wem gegenüber die Verantwortung getragen wird. Hierbei kann eine Verantwortung einerseits gegenüber anderen Personen entstehen oder gegenüber bestimmter normativer (Werte-)Maßstäbe (Recht, moralische Verantwortung). Die Verantwortung gegenüber Personen oder spezifischen Gruppen umfasst beispielsweise Patientinnen und Patienten, deren Angehörige, betroffene Bürgerinnen und Bürger sowie spezifische Stakeholder, wie etwa Anwohner bei einem Industrieprojekt. In diesem Sinne richtet sich die Rechenschaftspflicht direkt an diejenigen, die unter einer Handlung leiden oder profitieren. Es besteht die normative Erwartung, dass der Verantwortungsträger die Interessen und Rechte dieser konkreten Adressaten angemessen berücksichtigt und zum Ausdruck bringt - etwa durch Partizipation an einem Entscheidungsprozess. Die zweite Dimension bezieht sich auf die Verantwortung gegenüber normativen Maßstäben. Dabei steht nicht eine konkrete Person im Zentrum, sondern die Einhaltung allgemein akzeptierter Regeln und Prinzipien, wie beispielsweise rechtlicher Vorgaben, ethischer Leitlinien oder wissenschaftlicher Standards. Der Verantwortungsträger muss demnach nicht nur seinen Mitmenschen Rede und Antwort stehen, sondern auch sicherstellen, dass sein Handeln mit übergeordneten Normen im Einklang steht. \cite{}
Quelle: Neuhäuser, Handbuch angwandte Ethik, seite: 215 - 221


\section{Stakeholder und Verantwortung}
Die rasante Entwicklung von Quatencomputing verspricht transformative Fortschritte in zahlreichen Bereichen, wie Chemie, Finanzwesen, oder Logistik.\textbf{ Quelle: https://www.mckinsey.de/news/presse/quantum-technology-monitor-2024}
Doch dieses Innovationspotenzial bringt auch komplexe ethische und gesellschaftliche Fragen mit sich. Wer trägt Verantwortung, wenn Quantencomputing unsere bestehenden Systeme grundlegend verändert?

\subsection{2.1 Identifikation relevanter Stakeholder im Quantencomputing-Ökosystem}
Eine fundierte ethische Bewertung von Quantencomputing setzt ein umfassendes Verständnis aller Akteur*innen voraus, die entweder an der Entwicklung und Anwendung beteiligt sind oder von deren Folgen betroffen sein könnten. 

In diesem Abschnitt werden zentrale Stakeholder-Gruppen systematisch identifiziert, ihre Rollen, Interessen und potenziellen Einflussbereiche innerhalb des Quantencomputing-Ökosystems analysiert. Dabei stehen folgende Leitfragen im Fokus:
\begin{itemize}
    \item Wer hat Einfluss auf die Entwicklung, Verbreitung oder Nutzung von Quantencomputing?
    \item  Wer ist direkt oder indirekt betroffen?
\end{itemize}

\subsubsection{2.1.1 Individuen: Nutzer, Betroffene und die Frage der Privatsphäre und Autonomie}
\begin{itemize}

Individuen sind die Nutzer und Betroffenen von Quantentechnologien. Ihre Erfahrungen, Werte und Bedürfnisse sind entscheidend für eine nützliche und verantwortungsvolle Technologieentwicklung. Ihre Autonomie und ihr Recht auf informierte Entscheidungen müssen geschützt werden, insbesondere angesichts der Fähigkeit von QKI, menschliches Verhalten zu modellieren und potenziell zu beeinflussen. Sie sind direkt oder indirekt von den Auswirkungen von Quantentechnologien betroffen und tragen die Werte, die in der Entwicklung berücksichtigt werden müssen.


\subsubsection{2.1.2 Unternehmen: Rolle in Entwicklung, Kommerzialisierung und ethischer Verantwortung}


\subsubsection{2.1.3 Forschung und Wissenschaft: Treiber der Innovation und ethische Reflexion}
    \item Forschung und Wissenschaft: entscheidender Akteur im Dialog über die Governance von Quantencomputing. Ihre Rolle umfasst die Bereitstellung von Forschungsergebnissen und kritischen Analysen, um die komplexen Implikationen von Quantentechnologien zu verstehen. Sie ist unerlässlich für den Aufbau einer qualifizierten Arbeitskraft, die die ethischen, rechtlichen und gesellschaftlichen Aspekte von Quantentechnologien versteht. Ihre primäre Rolle ist die direkte Schaffung und Weiterentwicklung von Quantentechnologien, wobei sie sich für verantwortungsvolles Design, ethische Forschungspraktiken und die Vermeidung von Missbrauch einsetzen müssen.

\subsubsection{2.1.4 Staaten und Internationale Organisationen: Geopolitik, Regulierung und globale Zusammenarbeit}
    \begin{itemize}
    \item \textbf{Staaten (Regierungen, Gesetzgeber und Regulierungsbehörden):} Regierungen sind dafür verantwortlich, klare politische Ziele zu setzen, die Innovation mit dem öffentlichen Interesse in Einklang bringen. Sie schaffen legislative Grundlagen, gewährleisten Aufsicht und Durchsetzung und legen Rechenschaftspflicht und Haftung fest. Darüber hinaus fördern sie das öffentliche Vertrauen, investieren in Forschung und Entwicklung und leiten die internationale Zusammenarbeit. Im Kontext der Quanten-KI tragen sie auch die Verantwortung für nationale Sicherheitsimplikationen und die Planung der Migration zur Post-Quanten-Kryptographie (PQC). Regionale Beispiele umfassen die Europäische Union (EU AI Act , vorgeschlagener European Quantum Act ), die Vereinigten Staaten (National Quantum Initiative Act ) und China (Comprehensive Quantum Law ). 

\subsubsection{2.1.5 Zivilgesellschaft und NGOs: Anwaltschaft, Sensibilisierung und Partizipation}
    \item Internationale Organisationen:} Organisationen wie das World Economic Forum (WEF) , die OECD  und die Vereinten Nationen (UN)  spielen eine entscheidende Rolle bei der Formulierung globaler Prinzipien, der Förderung der Zusammenarbeit und der Schaffung von Plattformen für den Dialog. Sie initiieren Vorschläge wie den "Quantum Acquis Planétaire"  und eine "Atomagentur für Quanten-KI" (IAEA-Q) , um globale Standards und Nichtverbreitungsmechanismen zu etablieren.
    \end{itemize}
    \item Zivilgesellschaft und NGOs: Organisationen der Zivilgesellschaft sind aktiv an der Bewältigung von Governance-Herausforderungen beteiligt. Sie setzen sich für ethische Überlegungen, Menschenrechte und verantwortungsvolle Entwicklung ein. Sie spielen eine entscheidende Rolle bei der Gestaltung des öffentlichen Diskurses und der Sicherstellung, dass Quanten-KI der Gesellschaft als Ganzes zugutekommt. Sie repräsentieren oft die Interessen der breiten Öffentlichkeit und der betroffenen Gemeinschaften.
    \item \textbf{Weitere Akteure: }
    \begin{itemize}
        \item \textbf{Investoren:} deren Finanzentscheidungen beeinflussen die Richtung der Entwicklung
        \item \textbf{Journalisten und Medien:} prägen das öffentliche Bewusstsein und erleichtern den Dialog
        \item \textbf{Ethik und Gesellschaftsexperten: }helfen bei der Identifizierung und Analyse etischen Implikationen
\cite{Communities of Quantum Technologies: Stakeholder Identification, Legitimation, and Interaction}
(DOI:\href{http://dx.doi.org/10.13140/RG.2.2.25656.42240}{10.13140/RG.2.2.25656.42240})    \end{itemize}
\end{itemize}


\subsection{2.2 Verantwortungsdimensionen}
\subsubsection{2.2.1 Entscheidungsfindung und Machtasymmetrien: Wer entscheidet für wen?}
\subsubsection{2.2.2 Haftung für Folgen: Systemversagen, Black-Box-Algorithmen und Zurechenbarkeit?}
\subsubsection{2.2.3 Repräsentationsfragen: Inklusion und die Gefahr einer "Quanten-Elite"}

\subsection{2.3 Governance \& Steuerung ethischer Herausforderungen im Quantencomputing}


\section{3. Normative Spannungsfelder  }

3.1 \textbf{Zugang \& Gerechtigkeit} 
Die Vision eines breit zugänglichen Quantencomputings weckt die Erwartung, technologische Teilhabe grundlegend neu zu gestalten. Offene Cloud-Plattformen, frei verfügbare Software-Frameworks sowie internationale Bildungskooperationen könnten Studierenden, Start-ups und Forschungseinrichtungen ohne eigene Kryo-Labore den direkten Zugriff auf reale Quantenprozessoren erlauben. In einer idealen Ausprägung senken solche Open-Access-Wege nicht nur Eintrittsbarrieren, sondern schaffen neue Lernökosysteme, in denen Programmier-AGs an Schulen mit Nobelpreislaboren vernetzt sind; ganze Regionen könnten so erstmals eigenständig Innovationspfade beschreiten.

Doch genau hier öffnet sich ein tiefes normatives Spannungsfeld. Virtueller Zugang allein garantiert keine strukturelle Gerechtigkeit, solange die physischen Schlüsselressourcen – supraleitende Chip-Fertigung, isotopenreine Silizium-28-Wafer oder energiefressende Dilution-Kühlsysteme – in den Händen weniger ökonomisch mächtiger Akteure verbleiben. Entsteht eine Quanten-Elite, kann sie Forschungsagenden, Standardisierungsprozesse und Lizenzmodelle diktieren. Für Staaten mit begrenzter Wirtschaftskraft droht eine neue technologische Abhängigkeit: Rechenzeit wird zwar „on demand“ angeboten, echte Souveränität über kritische Infrastruktur jedoch verwehrt.

Hinzu kommt das Phänomen eines potenziellen „Braindrain“. Hochqualifizierte Talente aus dem Globalen Süden erhalten Stipendien an hardwareführenden Universitäten, bleiben dort jedoch häufig dauerhaft, weil heimische Institutionen weder Infrastruktur noch Karrierepfade bieten. So vergrößert sich die Wissenskluft, während die Herkunftsländer weiterhin auf externen Quanten-Dienstleistungen angewiesen sind. Selbst umfangreiche Open-Source-Lehrpläne können diese Lücke nicht schließen, wenn vor Ort Laborkapazitäten und Wartungskompetenz fehlen.

Auch die ökonomische Gestaltung von Freemium-Tarifen für Cloud-Rechenzeit wirft Gerechtigkeitsfragen auf: Wer sich längere und komplexere Programme leisten will, muss zahlen – oft in Währungen und mit Zahlungsmitteln, die in Schwellen- und Entwicklungsländern schwer zugänglich sind. Dadurch entsteht eine digitale Schichtstruktur: kostenfreie Einstiegskurse für alle, fortgeschrittene Quantenressourcen für zahlungskräftige Kundschaft.

Ein weiterer Aspekt betrifft das geistige Eigentum. Selbst wenn Quantenalgorithmen quelloffen publiziert werden, sind sie häufig auf proprietäre Hardware zugeschnitten. Ohne plattformunabhängige Schnittstellen drohen Code-Lock-ins, die kollaborative Innovation einschränken. Regulatorische Ansätze können hier ansetzen, indem sie offene Hardware-Abstraktionsebenen und portierbare Compiler vorschreiben, sodass Wissenstransfer nicht an Unternehmensgrenzen scheitert.

Die normative Kernfrage verschiebt sich somit: Nicht ob Quantencomputing offen oder exklusiv sein wird, sondern welche Tiefenebenen des Zugangs – von Bildungsinhalten über Testzugänge bis hin zu Fertigungskompetenz – erforderlich sind, um echte Chancengleichheit zu erreichen. Gelingt es, Hardware-Kapazitäten, Ausbildungsförderung und wirtschaftliche Teilhaberechte global neu zu verteilen, kann Quantencomputing Innovationslücken schließen und wissenschaftliche Diversität stärken. Misslingt dieser Balanceakt, droht die Technologie zum Verstärker bestehender Ungleichheiten zu werden – und damit genau jene digitale Hierarchie zu reproduzieren, die sie eigentlich überwinden sollte. (Coenen et al., 2022)

3.2 \textbf{Sicherheit und Kontrolle}

3.3 \textbf{Nachhaltigkeit und Umwelt}

3.4 \textbf{Autonomie und Verantwortlichkeit}
3.4.1 \textbf{”Black-Box“-Entscheidungen durch Quantenalgorithmen}
Die fortschreitende Entwicklung von Quantenalgorithmen eröffnet vielfältige neue Anwendungsmöglichkeiten, bringt aber auch erhebliche ethische Herausforderungen mit sich – insbesondere im Hinblick auf Entscheidungsprozesse, die sich unserer Kontrolle oder unserem Verständnis entziehen. Quantenalgorithmen basieren oft auf probabilistischen Prinzipien, die es erschweren, einzelne Ergebnisse eindeutig zu erklären oder zurückzuverfolgen. Wenn Systeme autonome Entscheidungen treffen, deren Logik für Nutzer oder Entwickler nicht transparent ist, spricht man von sogenannten „Black-Box“-Entscheidungen.

In sensiblen Bereichen wie Justiz, Medizin oder Sicherheit können solche intransparenten Entscheidungen gravierende Folgen für Individuen und Gesellschaft haben. Die Schwierigkeit besteht darin, Verantwortung klar zuzuweisen, wenn weder Entwickler noch Anwender genau nachvollziehen können, wie ein Ergebnis zustande kam (Stahl, 2021). Hier stellt sich die Frage, wie viel Autonomie ein technisches System überhaupt haben darf – und wer letztlich die Kontrolle behält.

 3.4.2 \textbf{Verantwortungslücken bei Systemversagen}
 Mit wachsender Komplexität von Quantencomputersystemen steigt auch das Risiko von Verantwortungslücken. Anders als bei klassischen IT-Systemen, bei denen Fehlerquellen oft identifizierbar und zuordenbar sind, kann bei quantenbasierten Systemen ein Versagen durch viele Ebenen der Technologie bedingt sein – von physikalischen Fehlern im Qubit-Handling bis zu unvorhersehbaren Algorithmusverläufen.

Diese komplexe Verantwortungsdiffusion erschwert die juristische und ethische Zurechenbarkeit. Wer haftet, wenn ein quantenbasiertes Entscheidungssystem fehlerhaft urteilt – der Entwickler, der Betreiber, der Datenlieferant? Die bestehenden rechtlichen Rahmenwerke reichen oft nicht aus, um solche Situationen angemessen zu regeln. Daher bedarf es eines ethisch fundierten Verständnisses von „geteilter Verantwortung“, das neue Kooperations- und Kontrollmodelle zwischen verschiedenen Akteuren vorsieht (Dignum, 2019).

3.4.3 \textbf{Intransparente Systeme }
Ein zentrales Problem im Kontext von Verantwortung ist die mangelnde Transparenz quantenbasierter Systeme. Während in klassischen Systemen bereits große Anstrengungen zur Erklärbarkeit („Explainability“) unternommen werden, ist dies bei Quantenalgorithmen besonders herausfordernd. Ihre auf Überlagerung und Verschränkung basierenden Prozesse lassen sich nur schwer in klassische Denkstrukturen übersetzen.

Diese Intransparenz gefährdet nicht nur das Vertrauen der Nutzer, sondern auch die Möglichkeit, Fehlverhalten oder Diskriminierung aufzudecken. Besonders kritisch ist dies bei öffentlich relevanten Entscheidungen, etwa in sozialen Diensten oder im Finanzwesen. Deshalb wird die Entwicklung von „Quantum Explainability“-Ansätzen entscheidend sein, um Akzeptanz und Kontrolle zu ermöglichen – auch wenn diese Bemühungen derzeit noch am Anfang stehen.

 
\paragraph{Fazit:Die ethischen Herausforderungen im Bereich \textbf{Autonomie und Verantwortlichkeit} zeigen, dass die Entwicklung von Quantencomputing nicht nur eine technische, sondern vor allem eine gesellschaftliche Aufgabe ist. Entscheidungen müssen nachvollziehbar, Verantwortlichkeiten klar verteilt und Systeme transparent gestaltet sein – sonst drohen technologische Macht ohne demokratische Kontrolle.
 In Zukunft wird sich Quantencomputing nur dann breit durchsetzen können, wenn es gelingt, Vertrauen aufzubauen – durch internationale ethische Standards, neue Formen der Governance und eine frühzeitige Einbettung von Verantwortung in die Technologieentwicklung selbst. Bildung, Regulierung und Kooperation werden hierbei Schlüsselrollen spielen.}

 

 
 

 

\section{4. Gestaltungsoptionen und Empfehlungen (Optional)}

\section{5. Fazit und Ausblick}





















\section{Chancen und Risiken}

\subsection{Chancen}

\item Grundlagen
\begin{itemize}
    \item Die ethischen Chancen des Quantencomputings umfassen nicht nur technische Leistungssteigerungen, sondern insbesondere die positiven sozialen, wirtschaftlichen und politischen Auswirkungen dieser Technologien, sofern sie unter Fairness- und Verantwortungsaspekten eingesetzt werden. \cite{arrow_holistic_2023}

In diesem Kontext ist nicht nur der unmittelbare Einsatz der Technologie von Relevanz, sondern auch die Entwicklung, die anhand der vier Dimensionen Antizipation, Reflexivität, Inklusion und Reaktionsfähigkeit – mit denen technologische Entwicklungen systematisch auf ihre ethischen Chancen hin bewertet und bereits im Design integriert werden – erfolgt. Quelle: https://doi.org/10.1016/j.respol.2013.05.008

\item \item Gesundheit \& Medizin
Im Bereich der Gesundheit und Medizin bietet Quantencomputing ein Potenzial hinsichtlich molekularer Simulationen, Präzisionsmedizin, Diagnostik, Radiotherapie, Arzneimittelforschung und Preisstrategien. Quelle: \href{https://doi.org/10.3390/fi15030094}{\textbf{https://doi.org/10.3390/fi15030094}}

\item Umwelt \& Klimaschutz
Quantencomputing hat das Potential, die Berechnung von Differentialgleichungen in Erdsystemmodellen signifikant zu beschleunigen und maschinelles Lernen auf Quantenhardware zu ermöglichen. Dadurch können bislang unzureichend abgebildete lokale Prozesse, wie beispielsweise Wolkenbildung und Turbulenzen, feiner und genauer dargestellt werden. Quelle: https://doi.org/10.48550/arXiv.2502.10488

\item Soziale Gerechtigkeit \& Teilhabe
\begin{itemize}
    \item Open-Source-Software im Bereich des Quantencomputings ermöglicht die Entwicklung und den Test von Algorithmen, ohne dass eine proprietäre Quantenmaschine erforderlich ist. Dies ist ein wesentlicher Aspekt, um einen gleichberechtigten Zugang zu Forschung und Innovation zu gewährleisten. Quelle: Fingerhuth et al
    \item Moderne Quanten-Cloud-Plattformen wie IBM Quantum Experience oder Amazon Braket offerieren nicht nur die Bereitstellung von Rechenzeit, sondern auch interaktive Lernumgebungen und Tutorials. Es wird den Einsteiger:innen somit die Möglichkeit geboten, weltweit praxisnah zu programmieren und zu experimentieren. Quelle: Nguyen et al.

\end{itemize}
\item Dititale Souveränität & Sicherheit
Quantencomputing eröffnet neue Möglichkeiten, um kritische Infrastrukturen vor Cyberangriffen zu schützen und somit die digitale Souveränität von Individuen und Gemeinschaften zu stärken. Durch den Einsatz von quantensicheren Kryptografietechnologien und innovativen Abwehrmechanismen können die Datenintegrität und die Privatsphäre auf ein bislang unerreichbares Niveau gehoben werden.  Quellen: Faruk et al, & https://doi.org/10.48009/1_iis_2024_125
\item 


\end{itemize}
\item Infrastruktur \& Mobilität
Quantenalgorithmen haben das Potential, die Effizienz von Verkehrsflüssen, Logistiknetzwerken und Routenplanungen zu steigern. Dies wird durch die Lösung von Optimierungsproblemen in Parallelität erreicht, was wiederum zur Reduzierung von Staus beiträgt und die Auslastung von Verkehrswegen sowie autonom fahrenden Fahrzeugen erhöht. Quelle: zhuang et al

Im Rahmen des Q-GRID-Projekts wird der Frage nachgegangen, inwiefern Quantenoptimierung die Effizienz und Ausfallsicherheit dezentraler Energieerzeugung und -übertragung sowie neuartiger Energiemarkt-Modelle (beispielsweise Peer-to-Peer-Handel, Mikrogrids) optimieren kann. Quelle: Blenninger et al. 


\item Fazit


\subsection{Risiken}
Datenschutz \& Informationssicherheit
\begin{itemize}
    \item Der Missbrauch von Technologien, einschließlich unbefugtem Zugriff auf sensible Daten durch Quantenhacking oder die Schaffung von Systemen, denen es an Transparenz und Rechenschaftspflicht mangelt, wird ebenfalls als potenzielle ethische und gesellschaftliche Probleme genannt, insbesondere im Zusammenhang mit fehlenden regulatorischen Rahmenbedingungen \cite{umbrello_quantum_2024}
\end{itemize}
\begin{itemize}
    \item Machtasymmetrien und Monopolisierung
    \item Gesellschaftliche Ungleichheit
    \item Unvorhersehbarkeit \& Kontrollverlust
    \item Dual-Use-Problematik
    \item Inkompatibilität
\end{itemize}
Allgemeine ethische Herausforderungen und Risiken umfassen Bedenken hinsichtlich:

\subsubsection{\textbf{1. Technische Risiken}}
\begin{itemize}
    \item Systemzuverlässigkeit und Stabilitätsprobleme
    \item Gefahr von Quantum Hacking
    \item \textbf{Technologischer Wandel} bzw. schnelle \textbf{Obsoleszenz} (Überalterung neuer Technologien)
    \item Beeinträchtigung der \textbf{Funktionalität} und \textbf{Sicherheit} von Quanten-Technologien
\end{itemize}

\subsubsection{\textbf{2. Ethische und gesellschaftliche Risiken}}
\begin{itemize}
    \item \textbf{Verletzung der Privatsphäre} durch neue Möglichkeiten der Datenanalyse
    \item \textbf{Bias (Voreingenommenheit)} in Entscheidungsalgorithmen
    \item \textbf{Mangel an Transparenz} und \textbf{fehlende Rechenschaftspflicht}
    \item Konflikt mit gesellschaftlichen Normen und \textbf{ethischen Prinzipien}
\end{itemize}

\subsubsection{\textbf{3. Ökonomische Risiken}}
\begin{itemize}
    \item \textbf{Ungleicher Zugang} zu Quanten-Technologien (Digital Divide)
    \item Verstärkung bestehender \textbf{gesellschaftlicher oder wirtschaftlicher Ungleichheiten}
    \item \textbf{Monopolisierung} oder Machtkonzentration bei wenigen Akteuren
    \item Risiken für eine faire \textbf{Verteilung wirtschaftlicher Vorteile}
\end{itemize}

\subsubsection{\textbf{4. Umweltbezogene Risiken}}
\begin{itemize}
    \item \textbf{Hoher Energieverbrauch} von Quantencomputern
    \item \textbf{Ökologische Auswirkungen} durch Herstellung und Betrieb
    \item Risiken für die \textbf{Nachhaltigkeit} und \textbf{Umweltverantwortung} beim Einsatz dieser Technologien
\end{itemize}

\cite{umbrello_quantum_2024}


\section{Freier Wille für Maschinen?}



\section{Diskussion}

\textbf{Wie können Chancen maximiert, Risiken minimiert werden?}
Klarheit und Transparenz hinsichtlich der Richtlinien und Protokolle zur Datenverwaltung sowie systematische Prüfungen der Quanten-Dienstleister sind entscheidend, um größtmögliche Flexibilität bei der Bewertung quantentechnologischer Chancen zu ermöglichen, Risiken zu minimieren und verantwortungsvolle Forschung und Innovation zu fördern .
quelle{https://www.ey.com/en_uk/insights/emerging-technologies/why-innovation-leaders-must-consider-quantum-ethics}


\textbf{Was bedeutet „verantwortungsvolle Quantenforschung“?
Ethik als Designprinzip}

\printbibliography
