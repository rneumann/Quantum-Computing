%\motto{Use the template \emph{chapter.tex} to style the various elements of your chapter content.}
\chapter{Ethische Aspekte}
\label{ethics} % Always give a unique label
% use \chaptermark{}
% to alter or adjust the chapter heading in the running head

\chapterauthor{Karin Mustermann, Max Mustermann}

\abstract{Kurzfassung des Kapitels}

\section{Chancen und Risiken}

\subsection{Chancen}

\begin{itemize}
\item Grundlagen
\item Gesundheit \& Medizin
\item Umwelt \& Klimaschutz
\item Soziale Gerechtigkeit \& Teilhabe
\item Infrastruktur \& Mobilität
\item Rahmenbedingungen \& Good Practices
\end{itemize}

\subsection{Risiken}

\begin{itemize}
\item Datenschutz \& Informationssicherheit
\item Machtasymmetrien und Monopolisierung
\item Gesellschaftliche Ungleichheit
\item Unvorhersehbarkeit \& Kontrollverlust
\item Dual-Use-Problematik
\end{itemize}

\section{Governance \& Verantwortung}

\begin{itemize}
\item Rolle der Wissenschaft, Politik und Industrie
\item Globale ethische Standards vs. nationale Interessen
\item Braucht es ethische Leitplanken – und wenn ja, wie?
\item Internationale Regulierung: Lessons learned von der KI-Debatte
\end{itemize}

\section{Freier Wille für Maschinen?}


\section{Diskussion}

\begin{itemize}
\item Wie können Chancen maximiert, Risiken minimiert   werden?
\item Was bedeutet „verantwortungsvolle Quantenforschung“?
\item Ethik als Designprinzip
\end{itemize}

\printbibliography
