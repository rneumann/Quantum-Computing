%\motto{Use the template \emph{chapter.tex} to style the various elements of your chapter content.}
\chapter{Ethische Aspekte}
\label{ethics} % Always give a unique label
% use \chaptermark{}
% to alter or adjust the chapter heading in the running head

\chapterauthor{Halima Albader Alhusini, Gizem Bulut, Adrian Hußfeldt Vazquez, Tayebeh Nazari}

\abstract{Kurzfassung des Kapitels}

\section{Ethik und Grundlagen}
\subsection{Ethik‑Grundlagen \& Bewertungsrahmen}
\subsection{Begriffsklärung „ethische Chancen“ \& Risiken}

\section{Stakeholder und Verantwortung}
\subsection{2.1 Stakeholder identifizieren}
\begin{itemize}
    \item Individuen (Nutzer, Betroffene): Individuen sind die Nutzer und Betroffenen von Quantentechnologien. Ihre Erfahrungen, Werte und Bedürfnisse sind entscheidend für eine nützliche und verantwortungsvolle Technologieentwicklung. Ihre Autonomie und ihr Recht auf informierte Entscheidungen müssen geschützt werden, insbesondere angesichts der Fähigkeit von QKI, menschliches Verhalten zu modellieren und potenziell zu beeinflussen. Sie sind direkt oder indirekt von den Auswirkungen von Quantentechnologien betroffen und tragen die Werte, die in der Entwicklung berücksichtigt werden müssen.
    \item Unternehmen
    \item Forschung und Wissenschaft: entscheidender Akteur im Dialog über die Governance von Quantencomputing. Ihre Rolle umfasst die Bereitstellung von Forschungsergebnissen und kritischen Analysen, um die komplexen Implikationen von Quantentechnologien zu verstehen. Sie ist unerlässlich für den Aufbau einer qualifizierten Arbeitskraft, die die ethischen, rechtlichen und gesellschaftlichen Aspekte von Quantentechnologien versteht. Ihre primäre Rolle ist die direkte Schaffung und Weiterentwicklung von Quantentechnologien, wobei sie sich für verantwortungsvolles Design, ethische Forschungspraktiken und die Vermeidung von Missbrauch einsetzen müssen.
    \item Staaten und internationale Organisationen
    \begin{itemize}
    \item \textbf{Staaten (Regierungen, Gesetzgeber und Regulierungsbehörden):} Regierungen sind dafür verantwortlich, klare politische Ziele zu setzen, die Innovation mit dem öffentlichen Interesse in Einklang bringen. Sie schaffen legislative Grundlagen, gewährleisten Aufsicht und Durchsetzung und legen Rechenschaftspflicht und Haftung fest. Darüber hinaus fördern sie das öffentliche Vertrauen, investieren in Forschung und Entwicklung und leiten die internationale Zusammenarbeit. Im Kontext der Quanten-KI tragen sie auch die Verantwortung für nationale Sicherheitsimplikationen und die Planung der Migration zur Post-Quanten-Kryptographie (PQC). Regionale Beispiele umfassen die Europäische Union (EU AI Act , vorgeschlagener European Quantum Act ), die Vereinigten Staaten (National Quantum Initiative Act ) und China (Comprehensive Quantum Law ). 
    \item \textbf{Internationale Organisationen:} Organisationen wie das World Economic Forum (WEF) , die OECD  und die Vereinten Nationen (UN)  spielen eine entscheidende Rolle bei der Formulierung globaler Prinzipien, der Förderung der Zusammenarbeit und der Schaffung von Plattformen für den Dialog. Sie initiieren Vorschläge wie den "Quantum Acquis Planétaire"  und eine "Atomagentur für Quanten-KI" (IAEA-Q) , um globale Standards und Nichtverbreitungsmechanismen zu etablieren.
    \end{itemize}
    \item Zivilgesellschaft und NGOs: Organisationen der Zivilgesellschaft sind aktiv an der Bewältigung von Governance-Herausforderungen beteiligt. Sie setzen sich für ethische Überlegungen, Menschenrechte und verantwortungsvolle Entwicklung ein. Sie spielen eine entscheidende Rolle bei der Gestaltung des öffentlichen Diskurses und der Sicherstellung, dass Quanten-KI der Gesellschaft als Ganzes zugutekommt. Sie repräsentieren oft die Interessen der breiten Öffentlichkeit und der betroffenen Gemeinschaften.
    \item \textbf{Weitere Akteure: }
    \begin{itemize}
        \item \textbf{Investoren:} deren Finanzentscheidungen beeinflussen die Richtung der Entwicklung
        \item \textbf{Journalisten und Medien:} prägen das öffentliche Bewusstsein und erleichtern den Dialog
        \item \textbf{Ethik und Gesellschaftsexperten: }helfen bei der Identifizierung und Analyse etischen Implikationen
\cite{Communities of Quantum Technologies: Stakeholder Identification, Legitimation, and Interaction}
(DOI:\href{http://dx.doi.org/10.13140/RG.2.2.25656.42240}{10.13140/RG.2.2.25656.42240})    \end{itemize}
\end{itemize}


\subsection{2.2 Verantwortungsdimensionen}


\subsection{2.3 Governance und Steuerung}



\section{3. Normative Spannungsfelder }

\section{4. Gestaltungsoptionen und Empfehlungen (Optional)}

\section{5. Fazit und Ausblick}





















\section{\textbf{Ethik‑Grundlagen \& Bewertungsrahmen}

\subsection{\textbf{Ethik‑Grundlagen \& Bewertungsrahmen} }
\textbf{• Überblick über Ethiktheorien: Utilitarismus (Konsequenzialismus), eventuell weitere

• Technikethik: ELSI, RRI, Ethics‑by‑Design

• Bewertungskriterien: Schaden/Nutzen, Fairness, Verantwortung, Nachhaltigkeit}


\subsection{\textbf{Begriffsklärung „ethische Chancen“ \& Risiken}}

\textbf{• Definition ethischer Chancen vs. technologische Potenziale

• Abgrenzung zu bloßen Fortschritts‑ oder Anwendungserfolgen}

\textbf{Definition der Risiken}

\subsection{Verantwortung}


\section{Freier Wille für Maschinen?}



\section{Diskussion}

\textbf{Wie können Chancen maximiert, Risiken minimiert werden?}
Klarheit und Transparenz hinsichtlich der Richtlinien und Protokolle zur Datenverwaltung sowie systematische Prüfungen der Quanten-Dienstleister sind entscheidend, um größtmögliche Flexibilität bei der Bewertung quantentechnologischer Chancen zu ermöglichen, Risiken zu minimieren und verantwortungsvolle Forschung und Innovation zu fördern .
quelle{https://www.ey.com/en_uk/insights/emerging-technologies/why-innovation-leaders-must-consider-quantum-ethics}


\textbf{Was bedeutet „verantwortungsvolle Quantenforschung“?
Ethik als Designprinzip}

\printbibliography
