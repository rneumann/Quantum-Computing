%\motto{Use the template \emph{chapter.tex} to style the various elements of your chapter content.}
\chapter{Ethische Aspekte}
\label{ethics} % Always give a unique label
% use \chaptermark{}
% to alter or adjust the chapter heading in the running head

\chapterauthor{Halima Albader Alhusini, Gizem Bulut, Adrian Hußfeldt Vazquez, Tayebeh Nazari}

\abstract{Kurzfassung des Kapitels}

\section{Ethik und Grundlagen}
\subsection{Ethik‑Grundlagen \& Bewertungsrahmen}
\subsection{Begriffsklärung „ethische Chancen“ \& Risiken}

\section{Stakeholder und Verantwortung}
\subsection{2.1 Stakeholder identifizieren}
\begin{itemize}
    \item Individuen (Nutzer, Betroffene): Individuen sind die Nutzer und Betroffenen von Quantentechnologien. Ihre Erfahrungen, Werte und Bedürfnisse sind entscheidend für eine nützliche und verantwortungsvolle Technologieentwicklung. Ihre Autonomie und ihr Recht auf informierte Entscheidungen müssen geschützt werden, insbesondere angesichts der Fähigkeit von QKI, menschliches Verhalten zu modellieren und potenziell zu beeinflussen. Sie sind direkt oder indirekt von den Auswirkungen von Quantentechnologien betroffen und tragen die Werte, die in der Entwicklung berücksichtigt werden müssen.
    \item Unternehmen
    \item Forschung und Wissenschaft: entscheidender Akteur im Dialog über die Governance von Quantencomputing. Ihre Rolle umfasst die Bereitstellung von Forschungsergebnissen und kritischen Analysen, um die komplexen Implikationen von Quantentechnologien zu verstehen. Sie ist unerlässlich für den Aufbau einer qualifizierten Arbeitskraft, die die ethischen, rechtlichen und gesellschaftlichen Aspekte von Quantentechnologien versteht. Ihre primäre Rolle ist die direkte Schaffung und Weiterentwicklung von Quantentechnologien, wobei sie sich für verantwortungsvolles Design, ethische Forschungspraktiken und die Vermeidung von Missbrauch einsetzen müssen.
    \item Staaten und internationale Organisationen
    \begin{itemize}
    \item \textbf{Staaten (Regierungen, Gesetzgeber und Regulierungsbehörden):} Regierungen sind dafür verantwortlich, klare politische Ziele zu setzen, die Innovation mit dem öffentlichen Interesse in Einklang bringen. Sie schaffen legislative Grundlagen, gewährleisten Aufsicht und Durchsetzung und legen Rechenschaftspflicht und Haftung fest. Darüber hinaus fördern sie das öffentliche Vertrauen, investieren in Forschung und Entwicklung und leiten die internationale Zusammenarbeit. Im Kontext der Quanten-KI tragen sie auch die Verantwortung für nationale Sicherheitsimplikationen und die Planung der Migration zur Post-Quanten-Kryptographie (PQC). Regionale Beispiele umfassen die Europäische Union (EU AI Act , vorgeschlagener European Quantum Act ), die Vereinigten Staaten (National Quantum Initiative Act ) und China (Comprehensive Quantum Law ). 
    \item \textbf{Internationale Organisationen:} Organisationen wie das World Economic Forum (WEF) , die OECD  und die Vereinten Nationen (UN)  spielen eine entscheidende Rolle bei der Formulierung globaler Prinzipien, der Förderung der Zusammenarbeit und der Schaffung von Plattformen für den Dialog. Sie initiieren Vorschläge wie den "Quantum Acquis Planétaire"  und eine "Atomagentur für Quanten-KI" (IAEA-Q) , um globale Standards und Nichtverbreitungsmechanismen zu etablieren.
    \end{itemize}
    \item Zivilgesellschaft und NGOs: Organisationen der Zivilgesellschaft sind aktiv an der Bewältigung von Governance-Herausforderungen beteiligt. Sie setzen sich für ethische Überlegungen, Menschenrechte und verantwortungsvolle Entwicklung ein. Sie spielen eine entscheidende Rolle bei der Gestaltung des öffentlichen Diskurses und der Sicherstellung, dass Quanten-KI der Gesellschaft als Ganzes zugutekommt. Sie repräsentieren oft die Interessen der breiten Öffentlichkeit und der betroffenen Gemeinschaften.
    \item \textbf{Weitere Akteure: }
    \begin{itemize}
        \item \textbf{Investoren:} deren Finanzentscheidungen beeinflussen die Richtung der Entwicklung
        \item \textbf{Journalisten und Medien:} prägen das öffentliche Bewusstsein und erleichtern den Dialog
        \item \textbf{Ethik und Gesellschaftsexperten: }helfen bei der Identifizierung und Analyse etischen Implikationen
\cite{Communities of Quantum Technologies: Stakeholder Identification, Legitimation, and Interaction}
(DOI:\href{http://dx.doi.org/10.13140/RG.2.2.25656.42240}{10.13140/RG.2.2.25656.42240})    \end{itemize}
\end{itemize}


\subsection{2.2 Verantwortungsdimensionen}


\subsection{2.3 Governance und Steuerung}



\section{3. Normative Spannungsfelder }

\section{4. Gestaltungsoptionen und Empfehlungen (Optional)}

\section{5. Fazit und Ausblick}





















\section{Chancen und Risiken}

\subsection{Chancen}

\item Grundlagen
\begin{itemize}
    \item Die ethischen Chancen des Quantencomputings umfassen nicht nur technische Leistungssteigerungen, sondern insbesondere die positiven sozialen, wirtschaftlichen und politischen Auswirkungen dieser Technologien, sofern sie unter Fairness- und Verantwortungsaspekten eingesetzt werden. \cite{arrow_holistic_2023}

In diesem Kontext ist nicht nur der unmittelbare Einsatz der Technologie von Relevanz, sondern auch die Entwicklung, die anhand der vier Dimensionen Antizipation, Reflexivität, Inklusion und Reaktionsfähigkeit – mit denen technologische Entwicklungen systematisch auf ihre ethischen Chancen hin bewertet und bereits im Design integriert werden – erfolgt. Quelle: https://doi.org/10.1016/j.respol.2013.05.008

\item \item Gesundheit \& Medizin
Im Bereich der Gesundheit und Medizin bietet Quantencomputing ein Potenzial hinsichtlich molekularer Simulationen, Präzisionsmedizin, Diagnostik, Radiotherapie, Arzneimittelforschung und Preisstrategien. Quelle: \href{https://doi.org/10.3390/fi15030094}{\textbf{https://doi.org/10.3390/fi15030094}}

\item Umwelt \& Klimaschutz
Quantencomputing hat das Potential, die Berechnung von Differentialgleichungen in Erdsystemmodellen signifikant zu beschleunigen und maschinelles Lernen auf Quantenhardware zu ermöglichen. Dadurch können bislang unzureichend abgebildete lokale Prozesse, wie beispielsweise Wolkenbildung und Turbulenzen, feiner und genauer dargestellt werden. Quelle: https://doi.org/10.48550/arXiv.2502.10488

\item Soziale Gerechtigkeit \& Teilhabe
\begin{itemize}
    \item Open-Source-Software im Bereich des Quantencomputings ermöglicht die Entwicklung und den Test von Algorithmen, ohne dass eine proprietäre Quantenmaschine erforderlich ist. Dies ist ein wesentlicher Aspekt, um einen gleichberechtigten Zugang zu Forschung und Innovation zu gewährleisten. Quelle: Fingerhuth et al
    \item Moderne Quanten-Cloud-Plattformen wie IBM Quantum Experience oder Amazon Braket offerieren nicht nur die Bereitstellung von Rechenzeit, sondern auch interaktive Lernumgebungen und Tutorials. Es wird den Einsteiger:innen somit die Möglichkeit geboten, weltweit praxisnah zu programmieren und zu experimentieren. Quelle: Nguyen et al.

\end{itemize}
\item Dititale Souveränität & Sicherheit
Quantencomputing eröffnet neue Möglichkeiten, um kritische Infrastrukturen vor Cyberangriffen zu schützen und somit die digitale Souveränität von Individuen und Gemeinschaften zu stärken. Durch den Einsatz von quantensicheren Kryptografietechnologien und innovativen Abwehrmechanismen können die Datenintegrität und die Privatsphäre auf ein bislang unerreichbares Niveau gehoben werden.  Quellen: Faruk et al, & https://doi.org/10.48009/1_iis_2024_125
\item 


\end{itemize}
\item Infrastruktur \& Mobilität
Quantenalgorithmen haben das Potential, die Effizienz von Verkehrsflüssen, Logistiknetzwerken und Routenplanungen zu steigern. Dies wird durch die Lösung von Optimierungsproblemen in Parallelität erreicht, was wiederum zur Reduzierung von Staus beiträgt und die Auslastung von Verkehrswegen sowie autonom fahrenden Fahrzeugen erhöht. Quelle: zhuang et al

Im Rahmen des Q-GRID-Projekts wird der Frage nachgegangen, inwiefern Quantenoptimierung die Effizienz und Ausfallsicherheit dezentraler Energieerzeugung und -übertragung sowie neuartiger Energiemarkt-Modelle (beispielsweise Peer-to-Peer-Handel, Mikrogrids) optimieren kann. Quelle: Blenninger et al. 


\item Fazit


\subsection{Risiken}
Datenschutz \& Informationssicherheit
\begin{itemize}
    \item Der Missbrauch von Technologien, einschließlich unbefugtem Zugriff auf sensible Daten durch Quantenhacking oder die Schaffung von Systemen, denen es an Transparenz und Rechenschaftspflicht mangelt, wird ebenfalls als potenzielle ethische und gesellschaftliche Probleme genannt, insbesondere im Zusammenhang mit fehlenden regulatorischen Rahmenbedingungen \cite{umbrello_quantum_2024}
\end{itemize}
\begin{itemize}
    \item Machtasymmetrien und Monopolisierung
    \item Gesellschaftliche Ungleichheit
    \item Unvorhersehbarkeit \& Kontrollverlust
    \item Dual-Use-Problematik
    \item Inkompatibilität
\end{itemize}
Allgemeine ethische Herausforderungen und Risiken umfassen Bedenken hinsichtlich:

\subsubsection{\textbf{1. Technische Risiken}}
\begin{itemize}
    \item Systemzuverlässigkeit und Stabilitätsprobleme
    \item Gefahr von Quantum Hacking
    \item \textbf{Technologischer Wandel} bzw. schnelle \textbf{Obsoleszenz} (Überalterung neuer Technologien)
    \item Beeinträchtigung der \textbf{Funktionalität} und \textbf{Sicherheit} von Quanten-Technologien
\end{itemize}

\subsubsection{\textbf{2. Ethische und gesellschaftliche Risiken}}
\begin{itemize}
    \item \textbf{Verletzung der Privatsphäre} durch neue Möglichkeiten der Datenanalyse
    \item \textbf{Bias (Voreingenommenheit)} in Entscheidungsalgorithmen
    \item \textbf{Mangel an Transparenz} und \textbf{fehlende Rechenschaftspflicht}
    \item Konflikt mit gesellschaftlichen Normen und \textbf{ethischen Prinzipien}
\end{itemize}

\subsubsection{\textbf{3. Ökonomische Risiken}}
\begin{itemize}
    \item \textbf{Ungleicher Zugang} zu Quanten-Technologien (Digital Divide)
    \item Verstärkung bestehender \textbf{gesellschaftlicher oder wirtschaftlicher Ungleichheiten}
    \item \textbf{Monopolisierung} oder Machtkonzentration bei wenigen Akteuren
    \item Risiken für eine faire \textbf{Verteilung wirtschaftlicher Vorteile}
\end{itemize}

\subsubsection{\textbf{4. Umweltbezogene Risiken}}
\begin{itemize}
    \item \textbf{Hoher Energieverbrauch} von Quantencomputern
    \item \textbf{Ökologische Auswirkungen} durch Herstellung und Betrieb
    \item Risiken für die \textbf{Nachhaltigkeit} und \textbf{Umweltverantwortung} beim Einsatz dieser Technologien
\end{itemize}

\cite{umbrello_quantum_2024}


\section{Freier Wille für Maschinen?}



\section{Diskussion}

\textbf{Wie können Chancen maximiert, Risiken minimiert werden?}
Klarheit und Transparenz hinsichtlich der Richtlinien und Protokolle zur Datenverwaltung sowie systematische Prüfungen der Quanten-Dienstleister sind entscheidend, um größtmögliche Flexibilität bei der Bewertung quantentechnologischer Chancen zu ermöglichen, Risiken zu minimieren und verantwortungsvolle Forschung und Innovation zu fördern .
quelle{https://www.ey.com/en_uk/insights/emerging-technologies/why-innovation-leaders-must-consider-quantum-ethics}


\textbf{Was bedeutet „verantwortungsvolle Quantenforschung“?
Ethik als Designprinzip}

\printbibliography
