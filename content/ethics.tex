%\motto{Use the template \emph{chapter.tex} to style the various elements of your chapter content.}
\chapter{Ethische Aspekte}

\label{ethics} % Always give a unique label
% use \chaptermark{}
% to alter or adjust the chapter heading in the running head

\chapterauthor{Halima Albader Alhusini, Gizem Bulut, Adrian Hußfeldt Vazquez, Tayebeh Nazari}

\abstract{Kurzfassung des Kapitels}

\section{\textbf{Ethik‑Grundlagen \& Bewertungsrahmen}}

Die schnelle Entwicklung des Quantencomputings wirft weitreichende Fragen auf, die rein über technische oder ökonomische Fragestellungen hinausgehen. Als disruptive Technologie hat QC das Potenzial, die globale Kommunikation, Sicherheitsmechanismen, politische und gesellschaftliche Machtverhältnisse sowie viele weitere Aspekte grundlegend zu verändern. Die potenziell weitreichenden Auswirkungen erfordern eine sorgfältige Auseinandersetzung auch mit ethischen Prinzipien und Maßstäben, um den Einsatz von QC auch mit gesellschaftlichen Werten in Einklang zu bringen. Ethik fungiert dabei als normativer Kompass: Sie zeigt nicht nur technische Potenziale auf, sondern auch, was moralisch erstrebenswert ist, und wo moralisch Risiken liegen. Gerade im Kontext des Quantencomputings, dessen langfristige Folgen aktuell noch schwer abzuschätzen sind, ist es von Bedeutung, geeignete ethische Ansätze zu identifizieren und diese bereits in der Entwicklung solcher Technologien anzuwenden. Ein Beispiel für die frühe Entwicklung ethischer Leitlinien und deren Bedeutung wird beim Thema künstliche Intelligenz deutlich. So hat die EU beispielsweise bereits im Jahr 2019 durch Fachexperten der Technikethik ethische Leitlinien für vertrauenswürdige KI entwickeln lassen. \cite{}
(Quelle: Ethics guidelines for trustworthy AI)

\subsection{Überblick über zentrale Ethiktheorien}

Es existieren unterschiedliche ethische Ansätze, die als Bewertungsrahmen für technologische Innovation herangezogen werden können. 


\textbf{• Definition ethischer Chancen vs. technologische Potenziale

• Abgrenzung zu bloßen Fortschritts‑ oder Anwendungserfolgen}

\textbf{Definition der Risiken}

\subsection{Verantwortung}

\section{Governance \& Verantwortung}
\begin{itemize}
\item Rolle der Wissenschaft, Politik und Industrie
\item Globale ethische Standards vs. nationale Interessen
\item Braucht es ethische Leitplanken – und wenn ja, wie?
\item Internationale Regulierung: Lessons learned von der KI-Debatte
\end{itemize}


    aktuell noch keine regulatorischen Vorgaben für Nutzung und die Entwicklung von Quantentechnologien. Das begünstigt ethische und gesellschaftliche Herausforderungen. \cite{umbrello_quantum_2024} \cite{1-s2.0-S0370157324001078}
\begin{itemize}
    \item Die Dimensionen der Verantwortung: Ethische, gesellschaftliche und rechtliche Imperative
    \begin{itemize}
        \item Ethische Verantwortung
        \item Gesellschaftliche Verantwortung
        \item Rechtliche Verantwortung
    \end{itemize}
\item Kernethische Prinzipien für verantwortungsvolle Quanten-KI
\item Initiativen von Unternehmen und Organisationen
\begin{itemize}
    \item IBM https://www.ibm.com/quantum/blog/responsible-quantum
\end{itemize}
\item Nationale und regionale Regulierungslandschaften
\begin{itemize}
    \item EU
    \item USA
    \item OECD
\end{itemize}
\end{itemize}
    Notwendigkeit betont, den Dialog anzustoßen, Kooperationen zu fördern und Strukturen zu schaffen, die nicht nur technologische Innovation vorantreiben, sondern zugleich gewährleisten, dass diese im Einklang mit unseren gemeinsamen menschlichen Werten gestaltet und gelenkt wird. \cite{umbrello_quantum_2024}
    Es wird betont, wie wichtig es ist, ethische Diskussionen möglichst frühzeitig zu führen – auch als Lehre aus dem Umgang mit Künstlicher Intelligenz, bei der viele Ethikinitiativen als „zu spät“ kamen, um die Designkultur noch maßgeblich zu prägen. Daraus ergibt sich die Forderung, Governance-Strukturen und normative Rahmenwerke frühzeitig zu etablieren, um später nicht mühsam versuchen zu müssen, gewinnorientierte, aber potenziell schädliche Entwicklungen einzuhegen.\cite{1-s2.0-S0370157324001078}
    Die Methodik des Value Sensitive Design (VSD), ein Ansatz im Sinne von „Ethics-by-Design“, bezieht explizit unterschiedliche Stakeholder ein – darunter auch politische Entscheidungsträger. Dies verdeutlicht, dass die Perspektiven von Regulierungsinstanzen sowie der Aufbau geeigneter Governance-Strukturen als zentrale Elemente einer ethisch verantwortungsvollen Gestaltung von Quantentechnologien angesehen werden – auch in Abwesenheit formeller regulatorischer Rahmenbedingungen. \cite{umbrello_quantum_2024}
    Die Quantum Ethics Project (QEP) baut auf Responsible Innovation auf, versucht jedoch, Perspektiven der am stärksten marginalisierten Gruppen in den Mittelpunkt zu stelle \cite{1-s2.0-S0370157324001078}
    Das QEP bezieht auch gezielt politische Entscheidungsträger mit ein, um sicherzustellen, dass die künftige Quantenarbeitswelt gesellschaftlich breit aufgestellt ist. Zwar handelt es sich hierbei nicht um eine direkte Regulierung der Technologie, doch stellt dieses politische Engagement einen wichtigen Schritt zur Mitgestaltung des übergeordneten Governance-Rahmens dar. \cite{1-s2.0-S0370157324001078}


\section{Freier Wille für Maschinen?}



\section{Diskussion}

\textbf{Wie können Chancen maximiert, Risiken minimiert werden?}
Klarheit und Transparenz hinsichtlich der Richtlinien und Protokolle zur Datenverwaltung sowie systematische Prüfungen der Quanten-Dienstleister sind entscheidend, um größtmögliche Flexibilität bei der Bewertung quantentechnologischer Chancen zu ermöglichen, Risiken zu minimieren und verantwortungsvolle Forschung und Innovation zu fördern .
quelle{https://www.ey.com/en_uk/insights/emerging-technologies/why-innovation-leaders-must-consider-quantum-ethics}


\textbf{Was bedeutet „verantwortungsvolle Quantenforschung“?
Ethik als Designprinzip}





\section{Ethik‑Grundlagen und Bewertungsrahmen}

\subsection{Ethik‑Grundlagen und Bewertungsrahmen}
\paragraph{Welche normativen Maßstäbe nutzen wir zur ethischen Bewertung von Quantencomputing?}

\subsubsection{Überblick über Ethiktheorien: Utilitarismus (Konsequenzialismus), eventuell weitere}
\subsubsection{Technikethik: ELSI, RRI, Ethics‑by‑Design}
\subsubsection{Bewertungskriterien: Schaden/Nutzen, Fairness, Verantwortung, Nachhaltigkeit}


\subsection{Begriffsklärung „ethische Chancen“ und Risiken}
\paragraph{Was macht eine Entwicklung ethisch wertvoll – über ihre Funktionalität und Profitabilität hinaus? /Wodurch entsteht ein ethisches Risiko}

\subsubsection{Definition ethischer Chancen vs. technologische Potenziale}
\subsubsection{Abgrenzung zu bloßen Fortschritts‑ oder Anwendungserfolgen}
\subsubsection{Definition der Risiken}



\section{Stakeholder und Verantwortung}

\subsection{Stakeholder identifizieren}
\paragraph{Wer hat Einfluss auf die Entwicklung, Verbreitung oder Nutzung von QC? 
Wer ist betroffen – direkt oder indirekt?
(Unterteilung in gewisse Regionen)}

\subsubsection{Individuen (Nutzer, Betroffene)}
\subsubsection{Unternehmen}
\subsubsection{Forschung und Wissenschaft}
\subsubsection{Staaten und internationale Organisationen}
\subsubsection{Zivilgesellschaft und NGOs}


\subsection{Verantwortungsdimensionen}
\paragraph{Wie ist Verantwortung verteilt – politisch, wirtschaftlich, ethisch? 
Wer kann, darf oder sollte ethische Rahmen setzen?}

\begin{enumerate}
    \item EU
    \item China
    \item USA
    \item Einzelpersonen
    \item internationale Organisationen (UN etc.)
    \item Unternehmen (Beispiele)
\end{enumerate}
 

\subsubsection{Wer trifft welche Entscheidungen für wen?}
\subsubsection{Wer haftet für Folgen?}
\subsubsection{Machtasymmetrien und Repräsentationsfragen}


\subsection{Governance und Steuerung}
\paragraph{Welche Steuerungs‑ und Kontrollmechanismen sind ethisch geboten? 
Wie lassen sich faire, partizipative und globale Prozesse institutionalieren?
}
\subsubsection{Ethik‑Boards und Multi‑Stakeholder‑Panels}
\subsubsection{Internationale Regulierungsansätze (z.B. KI‑Analogien)}
\subsubsection{Offene Standards vs. proprietäre Entwicklung}

\section{Normative Spannungsfelder (Chancen und Risiken)}

\subsection{Zugang und Gerechtigkeit}
\paragraph{Wer bekommt Zugang zu QC – und zu welchen Bedingungen? 
Führt QC zu mehr Chancengleichheit oder verschärft es bestehende Ungleichheiten?
}

\subsubsection{Demokratisierung (Open Access, QaaS, Bildung)}
\subsubsection{Gefahr einer „Quanten‑Elite“}
\subsubsection{Globale Ressourcen‑Verteilung}

\subsection{Sicherheit und Kontrolle}
\paragraph{Welche Chancen bietet QC für mehr digitale Sicherheit – und wo entstehen neue Risiken für Freiheit, Privatsphäre und Demokratie?}

\subsubsection{Quantensichere Verschlüsselung (QKD, PQC) als Chance}
\subsubsection{Überwachungspotenzial / Missbrauch durch Staaten oder Konzerne}

\subsection{Nachhaltigkeit und Umwelt}
\paragraph{Trägt QC zur Lösung ökologischer Probleme bei – oder schafft es neue? 
Wie kann ökologische Verantwortung technologisch mitgedacht werden?
}

\subsubsection{Energie‑ und Materialverbrauch von QC}
\subsubsection{Effizienzsteigerung (Klimamodelle, Energiemanagement)}
\subsubsection{Ökologischer Fußabdruck}

\subsection{Autonomie und Verantwortlichkeit}
\paragraph{Wie sichern wir Transparenz, Kontrolle und Zurechenbarkeit bei hochkomplexen Systemen?}

\subsubsection{ „Black‑Box“‑Entscheidungen durch Quantenalgorithmen}
\subsubsection{Verantwortungslücken bei Systemversagen}
\subsubsection{Intransparente Systeme}

\subsection{Dual‑Use und militärisches Potenzial}
\paragraph{Wie lassen sich missbräuchliche Anwendungen erkennen und begrenzen? 
Wer trägt ethisch Verantwortung für Dual‑Use‑Forschung?}

\subsubsection{Friedliche vs. militärische Nutzung (z.B. Quanten‑Cyberwaffen)}
\subsubsection{Wettlauf und Kontrollverlust bei globaler Technologiekonkurrenz}

\section{(Optional): Gestaltungsoptionen und Empfehlungen}

\subsection{Gestaltungsoptionen und Empfehlungen}
\paragraph{Was können Politik, Forschung und Industrie konkret tun, um ethische Chancen zu realisieren? 
Wie sieht „verantwortliche QC“ praktisch aus?}

\subsubsection{Ethik‑by‑Design als Leitprinzip}
\subsubsection{Förderprogramme für gemeinwohlorientierte Anwendungen}
\subsubsection{Ethikbildung in Forschung und Lehre}

\section{Fazit und Ausblick}
\paragraph{}







\printbibliography








