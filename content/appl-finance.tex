%\motto{Use the template \emph{chapter.tex} to style the various elements of your chapter content.}
\chapter{Anwendungsgebiete im Finanzbereich}
\label{trends} % Always give a unique label
% use \chaptermark{}
% to alter or adjust the chapter heading in the running head

\chapterauthor{Lola Bankai, Felix Goos}

\abstract{keine Endversion}

\section{Einleitung}
Der Finanzsektor gilt als eines der vielversprechendsten Anwendungsgebiete für Quantencomputing. Der Einsatz von Quantencomputern ist besonders geeignet für Bereiche, in denen klassische Systeme an ihre Leistungsgrenzen stoßen, da die Datenintensität und Rechenkomplexität hoch sind. Quantencomputer eröffnen mit ihrer Fähigkeit, bestimmte mathematische Probleme exponentiell schneller zu lösen, neue Perspektiven – insbesondere in Bereichen, die von großen Datenmengen, Unsicherheiten oder nichtlinearen Zusammenhängen geprägt sind.

Derzeit können drei wesentliche Anwendungsbereiche im Finanzwesen ausgemacht werden, die in Forschung und Praxis besonders intensiv erörtert werden:

• Simulation (Monte-Carlo-Simulation)
• Optimierung (Portfoliooptimierung)
• Maschinelles Lernen
 
Bei der Bewertung komplexer Finanzinstrumente und in der Risikomodellierung sind Monte-Carlo-Simulationen von großer Bedeutung. Quantenalgorithmen bieten in diesem Zusammenhang die Aussicht auf eine erhebliche Beschleunigung der Berechnungen und eine genauere Modellierung von Zufallsprozessen.
Die Portfoliooptimierung befasst sich mit der Auswahl und Gewichtung von Anlageklassen wie Aktien, Anleihen oder Rohstoffen. Es soll ein Portfolio erstellt werden, das entweder bei vorgegebenem Risiko die maximale Rendite oder bei gegebener Rendite das minimale Risiko aufweist. Quantencomputer bieten einen vielversprechenden Ansatz zur Effizienzsteigerung, da die zugrunde liegenden mathematischen Optimierungsprobleme mit zunehmender Anzahl von Anlageklassen schnell an Komplexität zunehmen.

Zusätzlich bieten sich durch Quantum Machine Learning innovative Ansätze in der Mustererkennung und Voraussage an – zum Beispiel für die Beurteilung der Kreditwürdigkeit, Marktanalysen oder die Identifizierung von Betrugsfällen. Obwohl sich viele dieser Anwendungen noch in der Experimentierphase befinden, demonstrieren erste Prototypen und Pilotprojekte bereits ein zunehmendes Interesse von Seiten der Banken und Technologieunternehmen.
Die wesentlichen Anwendungsfelder, Technologien und Akteure der Quantum Finance werden in den kommenden Kapiteln näher untersucht und auf ihr Potenzial hin bewertet.


\section{Relevanz und Problemstellung}
Zu den datenintensivsten und rechenaufwendigsten Bereichen der modernen Wirtschaft zählt das Finanzwesen. Jeden Tag müssen enorme Datenmengen ausgewertet, komplizierte Modelle beurteilt und Entscheidungen mit Risiko getroffen werden – häufig unter Bedingungen von Unsicherheit und in dynamischen Märkten. Bei der Bewältigung solcher Aufgaben geraten herkömmliche Computer zunehmend an ihre Grenzen. Der Rechenaufwand in Bereichen wie Optimierung, Simulation oder probabilistischer Risikobewertung wächst insbesondere mit der Komplexität von Finanzprodukten und -märkten exponentiell an \cite{springer2025,plos2024}.

Quantencomputer stellen ein neuartiges Paradigma dar: Sie verarbeiten Informationen auf der Grundlage quantenmechanischer Zustände, was die parallele Bearbeitung bestimmter mathematischer Probleme mit einem erheblichen Geschwindigkeitsvorteil ermöglicht. Neue Lösungsräume eröffnen sich insbesondere bei kombinatorischen Optimierungsproblemen, wie sie beispielsweise in der Portfoliozusammensetzung oder im Optionspricing vorkommen, sowie bei stochastischen Simulationen. Erste Untersuchungen haben ergeben, dass Quantenalgorithmen diese Aufgaben mit einer deutlich höheren Effizienz bewältigen können als klassische Verfahren \cite{quantumjournal2020,orus2019}.

Zugleich bringt die Verwendung von Quantencomputern im Finanzsektor neue Herausforderungen mit sich. Vor allem die Gefahr, dass klassische kryptographische Verfahren brechen könnten, steht neben der noch nicht voll ausgereiften Hardware im Raum. Es geht dabei nicht nur um den Schutz sensibler Finanzdaten, sondern auch um die Stabilität ganzer Systeme. Das sogenannte „Harvest Now, Decrypt Later“-Szenario ist besonders kritisch. Dabei werden heute verschlüsselte Daten abgefangen, um sie später mit leistungsfähigen Quantencomputern zu entschlüsseln \cite{finance21net}.

Damit befindet sich der Finanzsektor an einem Wendepunkt: Auf der einen Seite bieten Quantencomputer enorme Möglichkeiten zur Effizienzsteigerung und zur Verbesserung bestehender Verfahren – wie bei Risikomodellen, Investmentstrategien oder der Integration mit künstlicher Intelligenz \cite{finance21net}. Auf der anderen Seite ist es notwendig, dass bestehende Infrastrukturen rechtzeitig gegen neue Bedrohungen geschützt werden. Die Chance-Risiko-Dualität macht Quantencomputing für den Finanzsektor äußerst relevant und zeigt die Notwendigkeit praxisnaher Forschung und strategischer Vorausschau auf \cite{springer2025,orus2019}.




\section{Top 3 Anwendungsfelder (Praxis \& Theorie)}
Die konkrete Anwendbarkeit von Quantencomputern im Finanzsektor kann anhand von drei zentralen Bereichen veranschaulicht werden: Portfoliooptimierung, Monte-Carlo-Simulation und Risikomessung. Aufgrund ihrer hohen rechnerischen Komplexität und datenintensiven Anforderungen gelten diese Felder als besonders geeignet für quantenbasierte Verfahren. Im nachfolgenden Abschnitt wird untersucht, in welchem Ausmaß Quantenalgorithmen klassische Methoden in diesen Bereichen ergänzen oder ersetzen können.

\subsection{Portfoliooptimierung}
Die Portfoliotheorie ist ein Konzept der Finanzwirtschaft und beschreibt die Bestimmung eines optimalen Portfolios durch die Zusammensetzung von mehreren Kapitalanlagen. Kapitalanlagen beschreiben Investitionen in zahlreiche, langfristig orientierte Anlagen, die in Form von Aktien, Anleihen, Währungen, Immobilien oder Rohstoffen über Jahre gehalten und nicht verkauft werden. 
\cite{orus2019,sakuler2025}

    "Efficient Frontier Schaubild"
 
Harry Markowitz widmete sich der Frage in welcher Form ein optimales Portfolio, unter berücksichtigung der Marktbedingungen zusammengestellt werden sollte. Die Effizienzkurve repräsentiert das zentrale Konzept und stellt jene Portfolios dar, die die maximale Rendite zu einem gegebenen Risikoniveau, beziehungsweise umgekehrt das minimale Risiko für eine Zielrendite darstellt. (Markowitz 1959) modelliert daher ein Portfolio mittels des Erwartungswert der zukünftigen Renditen, sowie der Varianz (bzw. Standardabweichung) als Maß für das Risiko. Er entdeckte, dass die Varianz des Portfolios durch Diversifikation, also der Gewichtung unterschiedlicher Anlageklassen mit geringer Korrelation, minimiert werden kann. Hierdurch entsteht die gekrümmte vorm der Effizienzkurve. Ein rational handelnder Investor würde nun ein Portfolio auf der Kurve wählen. Portfolios rechts oder unterhalb der Effizienzkurve weisen entweder eine geringere Rendite, zu dem selben Risiko oder eine identische Rendite zu mehr Risiko auf. 
Die Optimierung nach Markowitz lässt sich in einem kontinuierlichem, quadratischen Optimierungsproblem schreiben das mithilfe von quadratischer Programmierung lösbar ist. 
\cite{Sakuler et al., 2025, Orús u. a. 2019}.
In der Praxis müssen jedoch oft diskrete Kauf- oder Nicht Kauf Entscheidungen getroffen werden, die das Problem zu einem kombinatorischen Optimierungsproblem machen. Solche Probleme lassen sich dann in einem Quadratic Unconstrained Binary Optimization (QUBO) Formulierung überführen. Die daraus Entstehende Lösungsmenge wächst dann exponentiell mit der Anzahl der hinzugefügten Assets. 
Trotz der theoretischen Stärke der Portfoliotheorie stößt sie schnell bei solch wachsender Komplexität an Ihre Grenzen.
\cite{mondello2017,markowitz1959, Sakuler et al. 2025}.

\subsubsection*{Hybride Quantenalgorithmen}
An dieser Stelle kommen Quantencomputer ins Spiel. Die Idee dahinter ist, das kombinatorische Problem direkt auf Quantenbits abzubilden und damit die Suche nach einer effizienten Lösung zu optimieren. Zwei Technologien stechen in den letzten Jahren der Forschung besonders heraus: Quantum Annealing und gate-basierte Qantenalgorithmen. \cite{Mugel et al., 2022,Orus et al., 2019} 

Zahlreiche kombinatorische Optimierungsprobleme wie eine Portfolioauswahl von 0/1-Entscheidungen gehören zu den sehr schwer lösbaren Problemen (NP-schwere Probleme). Diese lassen sich in eine QUBO-Form überführen exakt wie in der statistischen Physik bei dem Ising-Modell. Quantum Annealer hingegen sind spezialisierte Quantencomputer, die QUBO-Probleme oder auch Ising-Probleme durch quantenmechanische Tunnel zur Minimumslösung finden sollen.  
\cite{Mugel et al., 2022}
Die Übersetzung auf die Portfoliotheorie ist wie folgt: Jede mögliche Portfolioentscheidung wird als Qubit abgebildet, während die Markowitzfunktion, also die Kombination aus Rendite und Risiko als effektives Hamiltonian formuliert wird. Hamiltonian ist die mathematische Form der Energie des Modells. Diese dient zur Identifikation des optimalen Portfolios, also den Zustand mit der geringstmöglichen Energie durch Herunterkühlens des Quantenannealers. \cite{Mugel et al., 2022}

In ersten Studien wird deutlich, dass kleine Portfolioinstanzen bereits mithilfe eines D-Wave-2000Q-Annealer gelöst werden kann. Rosenberg (2016) verzeichnete hier Erfolge in der Nutzung um Arbitrage-Möglichkeiten und optimale Portfolios zu finden. 
In ersten Praxis versuchen von (Sakuler et. Al), Raiffeisen Bank International und dem Tech-Dienstleister Reply wurde reale Portfoliooptimierungsprobleme (9-11 Assets aus Aktien, Anleihen, Geldmarkt) mit Budget- und Volatilitätsrestriktionen ins QUBO-Modell mit einem D-Wave Quantenanealler getestet. Es gelang Ergebnisse in Sekunden bis Minuten zu erzielen, die mit der klassischen Lösung im Einklang standen. Herausforderungen stellen die Vernetzung der Quantenbits im Annealer dar sowie die einspeisung des Problems in die Hardware.\cite{Sakuler et al., 2025}

Die bisherigen Erfahrungen mit Quantum Annealing zeigen, dass es bisher noch keine deutlich Verbesserungen erziehlt werden können die über die klassischen Verfahren hinaus gehen. Jedoch sind Quantum-Annealer für kleinere bis mittelgroße Probleme von Vorteil und stehen bereits zur Verfügung. 

Gate-basierte Quantencomputer arbeiten mit Schaltkreisen. Auf diesen Systemen können Variationsalgorithmen implementiert werden, die Optimierungsprobleme lösen. Die zwei bekanntesten Methoden sind hierbei Quantum Approximate Optimization Algorithm (QAQO) und Variational Quantum Eigensolver (VQE). In beiden Methoden kommt ein parametischer Quantenschaltkreis zum Einsatz, dessen Parameter iterativ von einem klassischen
Optimierer so eingestellt werden, dass das gemessene Ausgabe-Stadium die Zielfunktion minimiert. Für die Portfoliooptimierung muss zuerst das eigentliche Problem abgeleitet und in einen Hamiltonoperator übersetzt werden.
\cite{Buonaiuto et al., 2023, Brandhofer et al., 2023}

Für den Quantum Approximate Optimization Algorithm (QAOA) Ansatz muss durch die Umwandlung in ein Ising-Modell zunächst das Optimierungsproblem binär formuliert werden. Dieses Modell bildet mit Wechselwirkungsparametern zwischen den Qubits das Ertrags-Risiko-profil des Portfolios ab. (Brandhofer et al., 2023)
Der QAO-Algorythmus verwendet abwechselnd zwei Arten von Gate-Operationen. Eine Operation auf dem Problem Hamiltonian basierenden Phasenschicht und eine auf einer Mixer-Hamiltonian Rotationsschicht. ¬Die Anzahl der Alternierungen zwischen Misch- und Phasen-Gates ist bestimmt durch die Tiefe. Ziel ist es, die Messwahrscheinlichkeit für eine optimale Lösung zu verbessern. In der Studie von Brandhofer et al. (2023), zeigen die Autoren, dass der Einsatz von diesen spezialisierten Mixern, die Güte und die Umsetzbarkeit der Lösung deutlich verbessert. 

Der Variational Quantum Eigensolver (VQE) verfolgt zwar vom Grundsatz her ein ähnliches Ziel, verwendet jedoch einen anderen Ansatz. Statt des festgelegten Schematas bei QAOA, ist die Wahl des Schaltkreises beliebig und mit festen Parametern versehen. Der Erwartungswert des Problem-Hamiltonians wird durch wiederholtes Messen geschätzt, und ein klassischer Optimierer passt die Parameter an, um den Energieerwartungswert zu minimieren. Die Ansätze können beispielsweise Hardware-effizient oder Problem-spezifisch sein.
\cite{Buonaiuto et al., 2023}
Buonaiuto et al. (2023) testeten in einer Studie die Wirksamkeit der VQE-Pipeline. Die Ergebnisse lieferten, dass der VQE auf einem 5-Qubit-Gerät eine fast identische Lösung zum klassischen Optimum generieren konnte. Die Qualität der Lösung variierte jedoch mit der Qualität des Quantenchips. Je größer und weniger von Rausch behaftete Geräte lieferten verbesserte als kleinere. Somit schmälern Rauschen und die begrenzte Qubit-Zahlen zwar noch den Nutzen, obwohl die Algorithmen prinzipiell eine optimale Lösung finden. 
\cite{Buonaiuto et al., 2023}

Der Einsatz von QAOQ mit der Fixierung der Anzahl der enthaltenen Assets wurde von Brandhofer et al. (2023) untersucht. Die Resultate zeigten, dass die Anzahl benötigter Quantengatter und Messungen die für mittlere Problemgrößen benötigt werden, die NISQ-Geräte an ihre Grenzen bringt. Für kleinere Portfolios konnten jedoch annähernd optimale Ergebnisse erziehlt werden. 

Beide Ansätze QAOA/VQE glänzen in Ihrer Flexibilität. Es können leichter zusätzliche Nebenbedingungen eingebaut werden und dynamische Portfolio-Probleme formulieren. Nachteile spiegeln sich in der in der limitation durch Rauschen und Decoherence, die ohne notwendige Fehlerkorrektur nicht behoben werden können. Aufgrund dieser Einschränkungen bleiben die Anwedungen der gate-basierten Ansätze noch auf rein Simulierter Ebene oder auf kleinen Hardware-Experimenten beschränkt. 


\subsection{ Monte Carlo Simulation}

\subsubsection*{Grundlagen der Monte-Carlo-Simulation}
Die Bewertung von Anlagen und Portfolios bedingt eine enorme Komplexität durch die unvollkommene Verteilung von Informationen auf Finanzmärkten. Entscheidungen über Risikostrategien, Bewertungen und Investitionen müssen häufig in Situationen getroffen werden, in denen wirtschaftliche Verläufe nicht direkt einsichtig sind. Hierbei greifen Marktteilnehmer auf Schätzungen, Wahrscheinlichkeiten und historische Daten zurück, um zukünftige Entwicklungen zu bewerten. 
 
Als grundlegendes Instrument wird in diesem Bereich die Monte-Carlo-Simulation in Betracht gezogen. Mithilfe der Monte-Carlo-Simulation können die Lösungen von analytisch komplexen und schwer lösbaren mathematischen Problemen geschätzt werden. Es nutzt also statistische Simulierung und Zufallsstichproben, um wiederholt verschiedene Szenarien eines Systems zu simulieren, um Erwartungswerte oder Wahrscheinlichkeiten approximativ zu bestimmen

Dieser stochastische Ansatz wird in der Finanzwelt besonders bei der Bewertung von Portfolios, Optionen und Aktien verwendet \cite{orus2019}.




\subsection{Risikomessung und Value at Risk}
Eine der zentralen Herausforderungen im modernen Finanzwesen ist die Risikomessung. Um drohende Verluste rechtzeitig zu erkennen und gezielt Gegenmaßnahmen zu ergreifen, sind Banken, Investoren und andere Marktteilnehmer in einem hochvolatilen Marktumfeld darauf angewiesen, Risiken genau zu quantifizieren und zuverlässig einschätzen zu können. Der „Value at Risk“ (VaR) ist ein anerkanntes und gängiges Instrument zur Quantifizierung von Risiko. Der VaR zeigt den maximalen Verlust an, den ein Portfolio oder eine einzelne Anlageposition innerhalb eines bestimmten Zeitraums und mit einer festgelegten Wahrscheinlichkeit (in der Regel 95\,\% oder 99\,\%) nicht überschreiten wird \cite{springer2025,plos2024}.

Die Berechnung des VaR bringt jedoch bedeutende Herausforderungen mit sich. Die historischen Simulationen, die Varianz-Kovarianz-Methode und Monte-Carlo-Simulationen stoßen schnell an ihre Grenzen, insbesondere mit zunehmender Anzahl der einbezogenen Anlageklassen, Risikofaktoren oder komplexen Finanzinstrumente. In diesen Fällen wächst die erforderliche Rechenleistung exponentiell an, was nicht nur zu erheblichen zeitlichen Verzögerungen bei der Risikomessung führt, sondern auch eine kontinuierliche Echtzeit-Überwachung nahezu unmöglich macht \cite{plos2024,bouland2020}. In Krisenzeiten können diese Verzögerungen besonders ernsthafte Folgen haben, weil es für die Begrenzung von Verlusten und die optimale Nutzung von Marktchancen entscheidend ist, in dynamischen Märkten schnelle Entscheidungen zu treffen \cite{orus2018}.

In dieser kritischen Situation bietet das Quantencomputing vielversprechende Lösungsansätze. Vor allem der Quantum-Amplitude-Estimation-Algorithmus (QAE) kann klassische Verfahren erheblich verbessern. Der QAE-Algorithmus bietet eine quadratische Geschwindigkeitssteigerung im Vergleich zu herkömmlichen Monte-Carlo-Simulationen \cite{quantumjournal2020,rebentrost2018}. Der Hauptvorteil des QAE besteht darin, dass er quantenmechanische Prinzipien verwendet, um Erwartungswerte deutlich effizienter zu approximieren, was weniger Simulationsdurchläufe erfordert, um eine vergleichbare Genauigkeit zu erreichen \cite{quantumjournal2020,martin2022}.

Diese quantenmechanische Beschleunigung eröffnet neue Perspektiven in der Risikomodellierung. Die erhöhten Rechenkapazitäten ermöglichen es, Modelle von deutlich höherer Komplexität und Realitätsnähe zu entwickeln, die umfassende Risikofaktoren und komplexe Abhängigkeiten wesentlich besser berücksichtigen. Insbesondere umfasst dies die genauere Modellierung von Extremrisiken („Tail-Risks“), die mit herkömmlichen Methoden aufgrund begrenzter Rechenkapazität oft nur unzureichend erfasst werden konnten \cite{orus2019,orus2018}. Die Durchführbarkeit und die erheblichen Vorteile solcher quantenbasierter Risikomodelle werden bereits durch aktuelle Forschungsergebnisse und Pilotprojekte bestätigt. Quantenalgorithmen haben beispielsweise in der Bewertung komplexer Finanzderivate und in der Analyse von Kreditrisiken effizientere Berechnungen und genauere Risikoeinschätzungen ermöglicht \cite{egger2020,rebentrost2018}.

Die praktische Umsetzung quantenmechanischer Verfahren im Finanzsektor ist jedoch derzeit mit erheblichen Herausforderungen verbunden. Vor allem in Bezug auf die verfügbare Quantenhardware bestehen noch Einschränkungen wie Stabilitätsprobleme (Qubit-Dekohärenz), begrenzte Skalierbarkeit und hohe Fehleranfälligkeit \cite{bouland2020,martin2022}. Obwohl diese technischen Hürden noch nicht vollständig überwunden sind, zeigen laufende Forschungsinitiativen und Kooperationen zwischen Banken und Technologieunternehmen wie IBM und JPMorgan klare Fortschritte. Die genannten Initiativen demonstrieren eindrücklich, dass die quantenmechanische Risikomessung, vor allem durch Quantum-Amplitude-Estimation, nicht nur theoretische Möglichkeiten bietet, sondern realistische Chancen hat, sich in der nahen Zukunft als wesentlicher Wettbewerbsvorteil zu etablieren \cite{orus2019,martin2022}.

Abschließend erkennt man, dass das Quantum Computing, vor allem durch den Quantum-Amplitude-Estimation-Algorithmus, eine vielversprechende und neuartige Herangehensweise bietet, um die gegenwärtigen Herausforderungen der Risikomessung im Finanzwesen nachhaltig zu bewältigen. Obwohl derzeit noch technische Schwierigkeiten bestehen, wird deutlich, dass das Quantencomputing die Risikomessung langfristig revolutionieren und damit die Stabilität und Effizienz des gesamten Finanzsystems erheblich verbessern könnte.

\section{Top Technologien und Algorithmen}

Die Komplexität im Finanzwesen nimmt rasant zu, ebenso wie die Menge der Daten – dies stellt traditionelle rechnergestützte Methoden vor große Herausforderungen. Bei komplexen Optimierungs- und Simulationsproblemen stoßen klassische Algorithmen und Berechnungsmethoden zunehmend an ihre Grenzen, sowohl hinsichtlich der benötigten Rechenzeit als auch der erforderlichen Genauigkeit \cite{plos2024,bouland2020}. Daher kommt quantenbasierten Technologien im Finanzsektor zunehmend Bedeutung zu.

Quantencomputer verwenden grundlegende quantenmechanische Prinzipien wie Superposition und Verschränkung, um komplexe Probleme effizienter zu lösen. Klassische Computer verwenden Bits (0 und 1), während Quantencomputer mit Quantenbits (Qubits) in der Lage sind, viele Berechnungen gleichzeitig durchzuführen. Quantencomputing bietet insbesondere bei spezifischen finanziellen Herausforderungen wie der Portfoliooptimierung, Optionsbewertung und komplexen Simulationen Vorteile im Vergleich zu klassischen Methoden \cite{orus2019,bouland2020,martin2022}.

Die derzeit marktführenden quantenbasierten Technologien im Finanzsektor umfassen insbesondere:
Quantum Amplitude Estimation (QAE), Quantum Approximate Optimization Algorithm (QAOA), Quantum Annealing (QA), Variational Quantum Algorithms (VQE, VQA) sowie Quantum Machine Learning (QML). Im Folgenden werden diese Technologien im Detail beschrieben, wobei ihre Stärken und Schwächen sowie mögliche Anwendungen im Finanzbereich behandelt werden.

\subsection{Quantum Amplitude Estimation (QAE)}

Der Quantenalgorithmus Quantum Amplitude Estimation (QAE), der von Brassard et al. entwickelt wurde, ist speziell für Monte-Carlo-basierte Verfahren relevant \cite{quantumjournal2020,rebentrost2018}.

Im Finanzbereich wird die Monte-Carlo-Simulation zur Bewertung komplexer Finanzinstrumente, zur Risikomessung (wie Value at Risk) und zu stochastischen Modellierungen eingesetzt. Da klassische Monte-Carlo-Verfahren auf zahlreichen Zufallsstichproben basieren, ist der Rechenaufwand groß. Der QAE-Algorithmus verringert die erforderlichen Stichproben durch quantenmechanische Effekte wie Superposition und Interferenz. Es ergibt sich im Vergleich zu herkömmlichen Verfahren eine quadratische Beschleunigung \cite{quantumjournal2020,rebentrost2018,martin2022}.

Der QAE findet Anwendung bei der Bewertung von Finanzderivaten, der Modellierung von Kreditrisiken und der Evaluierung komplexer pfadabhängiger Optionen. QAE bietet nicht nur die Möglichkeit einer schnelleren Berechnung, sondern auch die Unterstützung für komplexere Modelle, die auf traditionelle Weise schwer zu realisieren wären \cite{orus2018,egger2020,rebentrost2018}.

Die praktische Umsetzung von QAE steht derzeit noch vor Herausforderungen. Vor allem die Einschränkungen der Quantenhardware, wie die Stabilität der Qubits und die begrenzte Skalierbarkeit, machen eine großflächige Implementierung schwierig \cite{bouland2020,martin2022}. Dennoch deuten Forschungsprojekte und Partnerschaften zwischen Banken und Technologieunternehmen (wie etwa IBM und JPMorgan) darauf hin, dass eine realistische Anwendung von QAE in der nahen Zukunft möglich ist \cite{egger2020}.

\subsection{Quantum Approximate Optimization Algorithm (QAOA)}

Um kombinatorische Probleme zu lösen \cite{orus2019,bouland2020}, kombiniert der Quantum Approximate Optimization Algorithm (QAOA) quantenbasierte Berechnungen mit klassischen Optimierungsschritten in einem hybriden Ansatz.

Im Finanzbereich könnten beispielsweise Portfoliooptimierung, Asset Allocation und das Clustering finanzieller Daten zum Einsatz kommen. Der QAOA hat das Potenzial, schnellere und genauere Ergebnisse als rein klassische Methoden zu liefern \cite{orus2019}.

\subsection{Quantum Annealing (QA)}

Quantum Annealing, repräsentiert durch D-Wave-Systeme, stellt eine quantenbasierte Methode zur Optimierung dar. Es ist besonders geeignet für quadratische Optimierungsprobleme (QUBO – Quadratic Unconstrained Binary Optimization). Im Finanzsektor kann Quantum Annealing für die Portfolioauswahl, Arbitrage oder Optimierungsaufgaben eingesetzt werden \cite{orus2019,bouland2020}.

\subsection{Variational Quantum Algorithms (VQE, VQA)}

Hybride Algorithmen, die sowohl auf klassischen als auch auf quantenbasierten Schritten basieren, sind Variational Quantum Algorithms, insbesondere der Variational Quantum Eigensolver (VQE). Anwendungsfälle im Finanzbereich umfassen komplexe Bewertungen von Optionen sowie Risikoanalysen, für die herkömmliche Methoden nicht effizient genug sind \cite{bouland2020,martin2022}.

\subsection{Quantum Machine Learning (QML)}

Quantenalgorithmen werden im Quantum Machine Learning (QML) für Mustererkennung und Vorhersagen verwendet. Im Finanzsektor kann es bei der Bewertung von Krediten, der Aufdeckung von Betrug und der Analyse des Marktsentiments genutzt werden. QML kann dazu beitragen, die Verarbeitung komplexer Daten effizienter zu gestalten \cite{bouland2020,martin2022}.


\section{Wichtige Unternehmen und Akteure}
Der Einsatz von Quantencomputing im Finanzwesen wird derzeit durch Partnerschaften zwischen Technologieanbietern und Finanzinstituten bestimmt. Während Technologiefirmen den Zugang zu Quantenhardware und Softwareinfrastruktur ermöglichen, tragen Finanzunternehmen mit ihrer Expertise zu Pilotprojekten bei. In diesem Kontext spielen einige Akteure eine besonders markante Rolle, sowohl im Forschungsbereich als auch bei der praktischen Erprobung quantenbasierter Verfahren.
IBM gehört zu den Spitzenreitern im Bereich des Quantencomputings. Das Unternehmen stellt mit der Plattform „IBM Quantum“ Cloud-Zugänge bereit, die eine praktische Nutzung von Quantenhardware ermöglichen. Zudem führt IBM unter dem Namen „IBM Research“ eigene anwendungsbezogene Forschungsprojekte in Zusammenarbeit mit Finanzinstitutionen durch. Ein Fokus liegt auf der Entwicklung und dem Test quantenbasierter Algorithmen zur Portfoliooptimierung, Risikomodellierung und Generierung zertifizierter Zufallszahlen, wie sie zur Absicherung stochastischer Verfahren im Risikomanagement benötigt werden [\cite{jpmorgan_certified}]. Daniel Egger [\cite{ibm_egger}] ist einer der forschenden Wissenschaftler, die in diesem Gebiet publizieren.
Eines der ersten Finanzinstitute, das systematisch mit IBM an quantenbezogener Forschung arbeitet, ist JPMorgan Chase. Das Unternehmen gehört zum IBM-Quantum-Netzwerk und erforscht unter anderem die Nutzung der Quantum Random Number Generation (QRNG) zur Verfeinerung von Simulationsmethoden. Die Arbeiten werden in die „Applied Research“-Initiative von JPMorgan integriert, die Fortschritte in Bereichen wie Simulation, Pricing und Risikoanalyse dokumentiert [\cite{jpmorgan_certified}, \cite{jpmorgan_applied}]. Ein Schwerpunkt der Forschung ist die Anwendung von Quantum-Amplitude-Estimation, um den Rechenaufwand bei Monte-Carlo-Simulationen zu verringern.
Mit „Quantum AI“ verfolgt Alphabet (Google) einen forschungsorientierten Ansatz, der sich hauptsächlich auf die Weiterentwicklung supraleitender Quantenprozessoren konzentriert. Bisher wurden keine Projekte mit finanziellem Bezug veröffentlicht. Die Infrastruktur von Google bietet potenzielle Anknüpfungspunkte für zukünftige Anwendungen im Finanzbereich, wie der Bewertung komplexer Derivate oder dem maschinellen Lernen.
Mit „Azure Quantum“ arbeitet Microsoft an einer cloudbasierten Plattform, die den Zugang zu unterschiedlichen Quantenhardware-Backends bietet. Microsoft verfolgt neben eigenen Entwicklungsansätzen auch einen hybriden Architekturansatz, der es ermöglicht, klassische und quantenbasierte Berechnungsschritte zu kombinieren. Zwar wurden im Finanzbereich Anwendungsbeispiele vorgeführt, jedoch sind konkrete Zusammenarbeiten mit Finanzinstituten bisher nicht ausführlich dokumentiert.
Die jüngsten Entwicklungen verdeutlichen, dass die Realisierung quantenbasierter Finanzanwendungen in hohem Maße von der interdisziplinären Kooperation zwischen Technologieanbietern und Finanzinstitutionen abhängt. Während einige Firmen bereits öffentlich dokumentierte Pilotprojekte haben, befinden sich andere Akteure noch in der Erprobungs- oder Infrastrukturphase.


\section{Top 3 Zukunftsprojekte und Forschungsinitiativen}

\vspace{0.5em}
\begin{table}[h]
\centering
\renewcommand{\arraystretch}{1.3}
\begin{tabular}{|p{3.5cm}|p{3.5cm}|p{3cm}|p{4.5cm}|}
\hline
\textbf{Projekt / Akteur} & \textbf{Innovationshöhe} & \textbf{Anwendungsbezug} & \textbf{Potenzial im Finanzwesen} \\
\hline
\textbf{IBM Quantum (USA)} & Führend in Hardware (1121-Qubit, Roadmap zu 100k+ Qubits); neuartige Algorithmen (z.\,B. Amplitude Estimation) & Hoch – direkte Partner wie HSBC, JPMorgan; Fokus auf Pricing, VaR, Optimierung & Plattform für zukünftige Quanteninfrastruktur im Finanzwesen; starker Einfluss auf Forschung und Industrialisierung \\
\hline
\textbf{JPMorgan Chase (USA)} & Pionier unter Banken; eigene Forschungsabteilung; hybride Algorithmen 1000x schneller; Quantensicherheit & Sehr hoch – speziell auf finanzmathematische Probleme (VaR, Derivate, Kryptografie) ausgerichtet & Strategischer Vorsprung als Early Adopter; hohe Relevanz für Risiko- und Portfolioanalyse \\
\hline
\textbf{NEASQC (EU)} & EU-gefördertes Forschungsprojekt; Fokus auf praxistaugliche NISQ-Algorithmen; Open-Source-Bibliotheken & Hoch – Monte-Carlo, PCA, Portfolioanalyse in realen Finance-Use-Cases & Frühzeitige Anwendungsdemonstration in Europa; Stärkung europäischer Souveränität \\
\hline
\end{tabular}
\caption{Top 3 Zukunftsprojekte im Quantencomputing für das Finanzwesen – Vergleich nach Innovationsgrad, Anwendungsnähe und Potenzial (eigene Darstellung)}
\end{table}


\section{Bewertung anhand der Kriterien}

\section{Teilfazit}

Hier noch ein Beispiel für eine Literaturquelle 


\cite{feri_cognitive_finance_institute_quantenzeitalter_2024}

\printbibliography



