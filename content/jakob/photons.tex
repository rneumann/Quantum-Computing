\subsection{Photonische Quantencomputer}
Photonische Quantencomputer nutzen Lichtquanten (Photonen) als Träger von Quanteninformation. Dieser Ansatz bietet eine Reihe einzigartiger Vorteile, steht aber auch vor spezifischen Herausforderungen, die intensive Forschungs- und Entwicklungsanstrengungen erfordern.

\subsubsection{Grundlegende Prinzipien und Qubit-Kodierung}
Photonen werden oft als „fliegende Qubits“ bezeichnet, da sie sich naturgemäß mit Lichtgeschwindigkeit bewegen, eine geringe Wechselwirkung mit ihrer Umgebung aufweisen (was zu langer Kohärenz führt) und potenziell bei Raumtemperatur betrieben werden können (\cite{abughanemPhotonicQuantumComputers2024}). Die Kodierung von Qubits in Photonen kann auf verschiedene Weisen erfolgen. Eine verbreitete Methode ist die Dual-Rail-Kodierung, bei der die Zustände $\ket{0}$ und $\ket{1}$ durch die Anwesenheit eines Photons in einem von zwei räumlichen Moden (z. B. zwei Wellenleitern) oder zwei orthogonalen Polarisationen repräsentiert werden (\cite{slussarenkoPhotonicQuantumInformation2019}). Alternativ werden kontinuierliche Variablen (Continuous Variable, CV) Ansätze verfolgt, bei denen Qubits beispielsweise in gequetschten Lichtzuständen (Squeezed States) kodiert werden, wie es von Unternehmen wie Xanadu erforscht wird (\cite{QuantumComputingArchitecture}). Auch die Zeitmultiplex-Kodierung (Time-Bin Encoding), bei der die Ankunftszeit eines Photons relativ zu einem Referenzpuls das Qubit definiert, ist eine gängige Methode \cite{LinearOpticsScalable}.

\subsubsection{Schlüsselkomponenten}
Die Realisierung photonischer Quantencomputer hängt von drei Schlüsselkomponenten ab:
\par Photonenquellen: Eine Herausforderung ist die Erzeugung einzelner Photonen bei Bedarf (deterministisch). Probabilistische Quellen, basierend auf spontaner parametrischer Fluoreszenz (Spontaneous Parametric Downconversion, SPDC), sind zwar etabliert, aber für skalierbare Systeme ineffizient (\cite{slussarenkoPhotonicQuantumInformation2019}). Daher liegt ein starker Fokus auf der Entwicklung deterministischer oder angekündigter (heralded) Einzelphotonenquellen, beispielsweise basierend auf Quantenpunkten (\cite{LinearOpticsScalable}) oder Halbleiter-Quantenemittern, wie sie Quandela einsetzt (\cite{QuandelaAnnounces100000fold2025}). Kürzlich wurde über Quantenpunktquellen mit Helligkeiten über 106 Photonenpaaren/s/mW und angekündigte Einzelphotonenquellen mit Reinheiten über 99\% berichtet (\cite{LinearOpticsScalable}).
\par Manipulation: Die Manipulation photonischer Qubits erfolgt durch lineare optische Elemente wie Strahlteiler und Phasenschieber (\cite{slussarenkoPhotonicQuantumInformation2019}). Für erhöhte Stabilität, Miniaturisierung und Skalierbarkeit werden zunehmend integrierte photonische Schaltkreise (Photonic Integrated Circuits, PICs) eingesetzt (\cite{abughanemPhotonicQuantumComputers2024}). Diese PICs werden auf verschiedenen Materialplattformen wie Silizium, Siliziumnitrid oder Lithiumniobat realisiert.
\par Detektion: Hocheffiziente Einzelphotonendetektoren sind unerlässlich. Supraleitende Nanodraht-Einzelphotonendetektoren (Superconducting Nanowire Single-Photon Detectors, SNSPDs) erreichen Detektionseffizienzen von über 98\% (\cite{LinearOpticsScalable}), stellen aber hohe Anforderungen an die Kühlung. Eine Herausforderung bleibt die photonenzahlauflösende Detektion (\cite{slussarenkoPhotonicQuantumInformation2019}).

\subsubsection{Vorteile und spezifische Herausforderungen}

Photonische Quantencomputer bieten eine Reihe signifikanter Vorteile. Ein wesentlicher Pluspunkt ist ihr Potenzial für den Raumtemperaturbetrieb bei einigen Komponenten, auch wenn Quellen und Detektoren oft eine Kühlung erfordern (\cite{abughanemPhotonicQuantumComputers2024}). Ferner zeichnen sich photonische Qubits durch eine geringe Dekohärenz aus, bedingt durch die schwache Wechselwirkung von Photonen mit ihrer Umgebung, was zu langen Kohärenzzeiten führt (\cite{abughanemPhotonicQuantumComputers2024}). Darüber hinaus eignen sich Photonen hervorragend für die Quantenkommunikation und -netzwerke, da sie als natürliche Informationsträger über lange Distanzen fungieren (\cite{abughanemPhotonicQuantumComputers2024}). Diese Eigenschaft positioniert die Photonik einzigartig für modulare Architekturen, bei denen spezialisierte photonische Module andere Arten von Quantenprozessoren miteinander verbinden und somit als „Quanten-Internet-Backbone“ dienen könnten.
\newline
\newline
Dem gegenüber stehen die spezifischen Herausforderungen Photonenverlust, probabilistische Qubit-Gatter und die Skalierbarkeit von Quellen, sowie Detektoren.
\par 
Photonenverlust: 
Dies ist ein Hauptproblem, das alle Komponenten betrifft – von der Quelle über die Manipulation bis zur Detektion – und die Fidelität stark beeinträchtigt (\cite{LinearOpticsScalable}). Strategien zur Minderung umfassen die Entwicklung von Materialien mit extrem geringen Verlusten (z. B. Siliziumnitrid) und die Verbesserung der Effizienz von Komponenten und Kopplungen (\cite{salavrakosPhotonnativeQuantumAlgorithms2025}).

\par Probabilistische Zwei-Qubit-Gatter:
Die schwache native Wechselwirkung zwischen Photonen macht deterministische Zwei-Qubit-Gatter sehr schwierig (\cite{slussarenkoPhotonicQuantumInformation2019}). Übliche Ansätze basieren auf messungsinduzierter Nichtlinearität unter Verwendung von Hilfsphotonen und Postselektion, was zu einem hohen Ressourcenaufwand führt. Alternative Ansätze wie Cluster-Zustands-basiertes oder fusionsbasiertes Quantencomputing werden intensiv erforscht, um diese Hürde zu überwinden (\cite{salavrakosPhotonnativeQuantumAlgorithms2025}).

\par Skalierbarkeit von Quellen und Detektoren: 
Die zuverlässige Erzeugung und Detektion einer großen Anzahl von Photonen mit hoher Qualität bleibt eine technische Herausforderung (\cite{LinearOpticsScalable}).

\subsubsection{Aktueller Stand der Technik}
Das Feld der photonischen Quantencomputer verzeichnet rasante Fortschritte, die maßgeblich von akademischer Forschung und industriellen Akteuren vorangetrieben werden. Führende Unternehmen verfolgen dabei unterschiedliche, aber gleichermaßen ambitionierte Strategien.
Das kanadische Unternehmen Xanadu stellte Anfang 2025 seinen Prototyp „Aurora“ vor, ein modulares System, das aus 35 photonischen Chips und 13 Kilometern Glasfaser besteht. Eine strategische Partnerschaft mit dem U.S. Air Force Research Laboratory (AFRL) unterstreicht das Ziel, fehlertolerante Prozessoren zu entwickeln und die Quellen für verschränkte Photonen sowie für gequetschtes Licht zu optimieren (\cite{abdel-kareemXanaduUSAir2025}).
Das Unternehmen PsiQuantum verfolgt einen Ansatz, der auf eine Skalierung im Millionen-Qubit-Bereich abzielt und entwickelt dafür das „Omega“-Chipset. In Partnerschaft mit Regierungen plant das Unternehmen den Bau dedizierter Quantenrechenzentren in Brisbane, Australien, und Chicago, USA. Für die Fertigung der Chips nutzt PsiQuantum die Kapazitäten des etablierten Halbleiterherstellers GlobalFoundries (\cite{DARPAEyesCompanies}). Diese „Fabless“-Strategie, bei der die Produktion ausgelagert wird, steht im Kontrast zum Vorgehen von Unternehmen wie Quantum Computing Inc. (QCi), die eine eigene Fabrik für Dünnschicht-Lithiumniobat-Chips (TFLN) schaffen. (\cite{QuantumComputingInc}). Diese unterschiedlichen Geschäftsmodelle – „Fabless“ versus „integrierter Hersteller“ – spiegeln Entwicklungen wider, die bereits in der klassischen Halbleiterindustrie zu beobachten waren, und dürften langfristige Auswirkungen auf Kostenstrukturen, Innovationsgeschwindigkeit und die Verbreitung der Technologie haben. \newline
Das französische Unternehmen Quandela verfolgt einen hybriden Ansatz, bei dem photonische Qubits aus künstlichen Atomen erzeugt werden. Im Februar 2025 meldete Quandela eine Methode, die das Potenzial hat, die Anzahl der für fehlertolerante Berechnungen benötigten Komponenten um den Faktor 100.000 zu reduzieren, von etwa einer Million auf nur noch 12 Komponenten pro logischem Qubit (\cite{QuandelaAnnounces100000fold2025}). Sollte sich diese massive Reduktion als validierbar und breit anwendbar erweisen, könnte sie die Ressourcenskalierung und Wirtschaftlichkeit der photonischen Quantenberechnung dramatisch verändern. Dies könnte zu einer Veränderung innerhalb des photonischen Paradigmas selbst führen, bei der sich Ansätze mit hoher Komponentenintegration von solchen abheben, die auf einer großen Anzahl konventioneller Bauteile beruhen. Bereits im November 2024 ermöglichte Quandela der europäischen Forschungsgemeinschaft einen ersten Fernzugriff auf eine 6-Qubit-Maschine (\cite{QuandelaAnnounces100000fold2025}). \newline
Parallel zu diesen industriellen Entwicklungen unterstreichen grundlegende Forschungsdemonstrationen die wachsende Leistungsfähigkeit photonischer Systeme. Experimente zum Boson Sampling mit über 100 Photonen haben in einigen Fällen die Grenzen klassischer Simulationen überschritten (\cite{LinearOpticsScalable}). Übergreifend zeigen die Roadmaps der führenden Unternehmen eine klare Ausrichtung auf die zentralen Herausforderungen: die Realisierung von Fehlertoleranz und den Aufbau großskaliger Systeme (\cite{QuandelaAnnounces100000fold2025}).

\subsubsection{Engineering- und Skalierungslösungen}
Die Skalierung photonischer Quantencomputer erfordert erhebliche ingenieurtechnische Fortschritte. Eine Schlüsseltechnologie ist die integrierte Photonik, bei der photonische integrierte Schaltkreise (PICs) auf Plattformen wie Silizium, Siliziumnitrid (SiN) und Lithiumniobat ($LiNbO_3$) für Miniaturisierung, Stabilität und Massenfertigung sorgen (\cite{abughanemPhotonicQuantumComputers2024}). Ein Beispiel hierfür ist die Errichtung einer eigenen Fabrik für Dünnschicht-Lithiumniobat (TFLN) durch QCi (\cite{QuantumComputingInc}). Dies geht Hand in Hand mit der Entwicklung fortschrittlicher Fertigungstechniken, um Leiter mit geringen Verlusten und effiziente Koppeltechniken zu realisieren (\cite{LinearOpticsScalable}). Um die Effizienz probabilistischer Quellen zu steigern, werden zudem
Multiplexing-Strategien eingesetzt, die Photonen aus mehreren Quellen räumlich oder zeitlich bündeln (\cite{salavrakosPhotonnativeQuantumAlgorithms2025}). Schließlich ist eine schnelle
Feedforward-Kontrolle unerlässlich, bei der Detektorsignale genutzt werden, um photonische Schaltkreise adaptiv neu zu konfigurieren. Dies ist eine Grundvoraussetzung für adaptive Protokolle und das messungsbasierte Quantencomputing (MBQC) (\cite{salavrakosPhotonnativeQuantumAlgorithms2025}).

\subsubsection{Quantenfehlerkorrektur}
Aufgrund von Photonenverlust und probabilistischen Gattern ist die Quantenfehlerkorrektur (QEC) für die Realisierung fehlertoleranter photonischer Quantencomputer von fundamentaler Bedeutung (\cite{swayneChineseScientistsOvercome2025}). Die Forschung konzentriert sich auf verschiedene Arten von Fehlerkorrekturcodes. Dazu gehören bosonische Codes wie die Gottesman-Kitaev-Preskill (GKP)-Zustände, die die kontinuierliche Natur des Lichts ausnutzen, sowie an photonische Systeme angepasste topologische Codes und Oberflächencodes (\cite{LinearOpticsScalable}). 
\newline Ein wichtiger experimenteller Fortschritt wurde 2025 von Forschern der University of Science and Technology of China (USTC) in
Nature berichtet: Sie demonstrierten eine Einzelphotonenquelle mit einer Effizienz von 71,2\%, die damit die theoretische Schwelle von zwei Dritteln übertrifft, welche für die Funktionsfähigkeit von QEC in der photonischen Quantenberechnung als kritisch gilt (\cite{swayneChineseScientistsOvercome2025}). Parallel dazu zielt der Ansatz von Quandela darauf ab, die für die Konstruktion logischer Qubits benötigten physikalischen Ressourcen drastisch zu reduzieren, was die Implementierung von QEC erheblich vereinfachen könnte (\cite{QuandelaAnnounces100000fold2025}).
Der photonische Ansatz befindet sich in einer dynamischen Entwicklungsphase, mit vielversprechenden Fortschritten sowohl bei den grundlegenden Komponenten als auch bei der Systemintegration und den Strategien zur Fehlertoleranz.
