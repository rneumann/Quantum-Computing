\subsection{Abschließender Vergleich}
\begin{table}[ht]
\centering
\caption{Vergleichende Analyse der Quantencomputing-Hardware-Modalitäten (Stand 2025)}
\label{tab:quantum_comparison_simple}
\begin{tabular}{lllll}
\toprule
\textbf{Ansatz} & \textbf{Stärken} & \textbf{Schwächen} & \textbf{Herausforderungen \& Risiken} & \textbf{Chancen \& Ausblick} \\
\midrule

Topologisch & Inhärenter Fehlerschutz, hohe Dichte & Physik unbewiesen, Materialanforderungen & \textbf{Ambivalenz} bei Nachweis, \textbf{Quasiteilchen-Vergiftung} & Hohes Risiko, direkter Weg zu Fehlertoleranz \\
\addlinespace % Adds a little extra space between rows

Photonisch & Hohe Kohärenz, Raumtemperatur-Betrieb & Photon-Verlust, probabilistische Gatter & \textbf{Fertigungsintegration}, \textbf{Kryotechnik-Dilemma} & Skalierbare, vernetzte Architekturen \\
\addlinespace

Neutrale Atome & Höchste Qubit-Zahl, hohe Konnektivität & Langsame Gatter, ungelöste Interconnects & \textbf{Skalierungs-Mauer} bei Modulverbindung & Ideal für Simulation, effiziente Fehlerkorrektur \\
\addlinespace

Supraleitend & Technologisch reif, schnelle Gatter & Begrenzte Konnektivität, Rauschanfälligkeit & \textbf{Konnektivitäts-Limitierung}, Qubit-Variabilität & Führend bei NISQ-Anwendungen, Cloud-Integration \\
\addlinespace

Silizium-Spin & Sehr hohe Dichte, CMOS-kompatibel & Weniger reif, Gatter-Fidelität & \textbf{Material-Paradox} (Si vs. $^{28}$Si), Kontrollintegration & Langfristig vielversprechend für Millionen Qubits \\

\bottomrule
\end{tabular}
\end{table}