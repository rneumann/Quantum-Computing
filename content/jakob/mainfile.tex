% \addbibressource{content/jakob/citations.bib}

\section{Photonen und andere Ansätze}
Die Realisierung eines skalierbaren, fehlertoleranten Quantencomputers stellt eine der größten wissenschaftlichen und technischen Herausforderungen unserer Zeit dar \cite{QuantumHardwareExplained}. Weltweit werden diverse physikalische Systeme intensiv erforscht, um die komplexen Anforderungen an Quantenhardware zu erfüllen \cite{chengNoisyIntermediatescaleQuantum2023}. Diese Vielfalt an Ansätzen spiegelt die Tatsache wider, dass noch kein einzelnes System alle notwendigen Kriterien optimal erfüllt und der Weg zu universellen Quantencomputern nicht singulär erscheint. Vielmehr deutet die aktuelle Forschungslandschaft auf eine längere Phase paralleler Entwicklungen und potenzieller Hybridisierungen hin, in der verschiedene Technologien koexistieren und möglicherweise integriert werden, anstatt dass eine einzelne Technologie die anderen schnell verdrängt \cite{QuantumHardwareExplained}.
\newline \newline
Ein etablierter Rahmen zur Bewertung der verschiedenen Hardwareplattformen sind die DiVincenzo-Kriterien \cite{QuantumComputingArchitecture}. Diese umfassen typischerweise die Existenz gut charakterisierter Qubits, die Möglichkeit zur Initialisierung der Qubits in einen definierten Zustand, lange Kohärenzzeiten, die Fähigkeit zur Durchführung universeller Quantengatter sowie eine zuverlässige Auslesemethode für Qubits. Die Erfüllung dieser Kriterien unterliegt signifikanten ingenieurtechnischen Hürden, die Expertise aus Physik, Kryotechnik, Materialwissenschaften und Elektronik erfordern \cite{QuantumHardwareExplained}.
\newline \newline
Dieses Kapitel fokussiert auf photonische Quantencomputer als einen prominenten alternativen Ansatz und beleuchtet darüber hinaus weitere vielversprechende Technologien wie Halbleiter-Spin-Qubits, Neutralatom-Qubits und topologische Qubits. Es wird anerkannt, dass etabliertere Plattformen wie supraleitende Qubits \cite{QuantumComputingArchitecture} und gefangene Ionen \cite{sruthisomarouthuQuantumComputingDigital2025} bereits einen höheren Reifegrad erreicht haben und vermutlich in anderen Kapiteln dieses Buches ausführlich behandelt werden. Die hier diskutierten Ansätze zeichnen sich durch spezifische Vorteile und innovative Lösungsstrategien für die Kernprobleme der Quanteninformationsverarbeitung aus.
\newline \newline
Die aktuelle Ära der Quantencomputerentwicklung wird häufig als "Noisy \linebreak Intermediate-Scale Quantum" (NISQ) bezeichnet \cite{chengNoisyIntermediatescaleQuantum2023}. In dieser Phase sind die verfügbaren Quantenprozessoren noch fehleranfällig und die Anzahl der Qubits begrenzt. Ein zentrales Forschungsziel ist daher die Entwicklung und Implementierung von Strategien zur Fehlerminderung und -korrektur sowie die Realisierung sogenannter logischer Qubits, die eine höhere Robustheit gegenüber physikalischen Fehlern aufweisen \cite{QuantumHardwareExplained}. Dieser Trend zur Fehlertoleranz ist ein kritisches Moment in der Entwicklung und signalisiert eine Verschiebung von der reinen Demonstration quantenmechanischer Phänomene hin zur Bewältigung der ingenieurtechnischen Herausforderungen für praktische Quantenvorteile. Jede Bewertung der hier vorgestellten "anderen Ansätze" muss daher nicht nur die rohe Qubit-Anzahl, sondern auch den Fortschritt und die Strategien auf dem Weg zur Fehlertoleranz berücksichtigen.

\subsection{Photonische Quantencomputer}
Photonische Quantencomputer nutzen Lichtquanten (Photonen) als Träger von Quanteninformation. Dieser Ansatz bietet eine Reihe einzigartiger Vorteile, steht aber auch vor spezifischen Herausforderungen, die intensive Forschungs- und Entwicklungsanstrengungen erfordern.

\subsubsection{Grundlegende Prinzipien und Qubit-Kodierung}
Photonen werden oft als „fliegende Qubits“ bezeichnet, da sie sich naturgemäß mit Lichtgeschwindigkeit bewegen, eine geringe Wechselwirkung mit ihrer Umgebung aufweisen (was zu langer Kohärenz führt) und potenziell bei Raumtemperatur betrieben werden können (\cite{abughanemPhotonicQuantumComputers2024}). Die Kodierung von Qubits in Photonen kann auf verschiedene Weisen erfolgen. Eine verbreitete Methode ist die Dual-Rail-Kodierung, bei der die Zustände $\ket{0}$ und $\ket{1}$ durch die Anwesenheit eines Photons in einem von zwei räumlichen Moden (z. B. zwei Wellenleitern) oder zwei orthogonalen Polarisationen repräsentiert werden (\cite{slussarenkoPhotonicQuantumInformation2019}). Alternativ werden kontinuierliche Variablen (Continuous Variable, CV) Ansätze verfolgt, bei denen Qubits beispielsweise in gequetschten Lichtzuständen (Squeezed States) kodiert werden, wie es von Unternehmen wie Xanadu erforscht wird (\cite{QuantumComputingArchitecture}). Auch die Zeitmultiplex-Kodierung (Time-Bin Encoding), bei der die Ankunftszeit eines Photons relativ zu einem Referenzpuls das Qubit definiert, ist eine gängige Methode \cite{LinearOpticsScalable}.

\subsubsection{Schlüsselkomponenten}
Die Realisierung photonischer Quantencomputer hängt von drei Schlüsselkomponenten ab:
\par Photonenquellen: Eine Herausforderung ist die Erzeugung einzelner Photonen bei Bedarf (deterministisch). Probabilistische Quellen, basierend auf spontaner parametrischer Fluoreszenz (Spontaneous Parametric Downconversion, SPDC), sind zwar etabliert, aber für skalierbare Systeme ineffizient (\cite{slussarenkoPhotonicQuantumInformation2019}). Daher liegt ein starker Fokus auf der Entwicklung deterministischer oder angekündigter (heralded) Einzelphotonenquellen, beispielsweise basierend auf Quantenpunkten (\cite{LinearOpticsScalable}) oder Halbleiter-Quantenemittern, wie sie Quandela einsetzt (\cite{QuandelaAnnounces100000fold2025}). Kürzlich wurde über Quantenpunktquellen mit Helligkeiten über 106 Photonenpaaren/s/mW und angekündigte Einzelphotonenquellen mit Reinheiten über 99\% berichtet (\cite{LinearOpticsScalable}).
\par Manipulation: Die Manipulation photonischer Qubits erfolgt durch lineare optische Elemente wie Strahlteiler und Phasenschieber (\cite{slussarenkoPhotonicQuantumInformation2019}). Für erhöhte Stabilität, Miniaturisierung und Skalierbarkeit werden zunehmend integrierte photonische Schaltkreise (Photonic Integrated Circuits, PICs) eingesetzt (\cite{abughanemPhotonicQuantumComputers2024}). Diese PICs werden auf verschiedenen Materialplattformen wie Silizium, Siliziumnitrid oder Lithiumniobat realisiert.
\par Detektion: Hocheffiziente Einzelphotonendetektoren sind unerlässlich. Supraleitende Nanodraht-Einzelphotonendetektoren (Superconducting Nanowire Single-Photon Detectors, SNSPDs) erreichen Detektionseffizienzen von über 98\% (\cite{LinearOpticsScalable}), stellen aber hohe Anforderungen an die Kühlung. Eine Herausforderung bleibt die photonenzahlauflösende Detektion (\cite{slussarenkoPhotonicQuantumInformation2019}).

\subsubsection{Vorteile und spezifische Herausforderungen}

Photonische Quantencomputer bieten eine Reihe signifikanter Vorteile. Ein wesentlicher Pluspunkt ist ihr Potenzial für den Raumtemperaturbetrieb bei einigen Komponenten, auch wenn Quellen und Detektoren oft eine Kühlung erfordern (\cite{abughanemPhotonicQuantumComputers2024}). Ferner zeichnen sich photonische Qubits durch eine geringe Dekohärenz aus, bedingt durch die schwache Wechselwirkung von Photonen mit ihrer Umgebung, was zu langen Kohärenzzeiten führt (\cite{abughanemPhotonicQuantumComputers2024}). Darüber hinaus eignen sich Photonen hervorragend für die Quantenkommunikation und -netzwerke, da sie als natürliche Informationsträger über lange Distanzen fungieren (\cite{abughanemPhotonicQuantumComputers2024}). Diese Eigenschaft positioniert die Photonik einzigartig für modulare Architekturen, bei denen spezialisierte photonische Module andere Arten von Quantenprozessoren miteinander verbinden und somit als „Quanten-Internet-Backbone“ dienen könnten.
\newline
\newline
Dem gegenüber stehen die spezifischen Herausforderungen Photonenverlust, probabilistische Qubit-Gatter und die Skalierbarkeit von Quellen, sowie Detektoren.
\par 
Photonenverlust: 
Dies ist ein Hauptproblem, das alle Komponenten betrifft – von der Quelle über die Manipulation bis zur Detektion – und die Fidelität stark beeinträchtigt (\cite{LinearOpticsScalable}). Strategien zur Minderung umfassen die Entwicklung von Materialien mit extrem geringen Verlusten (z. B. Siliziumnitrid) und die Verbesserung der Effizienz von Komponenten und Kopplungen (\cite{salavrakosPhotonnativeQuantumAlgorithms2025}).

\par Probabilistische Zwei-Qubit-Gatter:
Die schwache native Wechselwirkung zwischen Photonen macht deterministische Zwei-Qubit-Gatter sehr schwierig (\cite{slussarenkoPhotonicQuantumInformation2019}). Übliche Ansätze basieren auf messungsinduzierter Nichtlinearität unter Verwendung von Hilfsphotonen und Postselektion, was zu einem hohen Ressourcenaufwand führt. Alternative Ansätze wie Cluster-Zustands-basiertes oder fusionsbasiertes Quantencomputing werden intensiv erforscht, um diese Hürde zu überwinden (\cite{salavrakosPhotonnativeQuantumAlgorithms2025}).

\par Skalierbarkeit von Quellen und Detektoren: 
Die zuverlässige Erzeugung und Detektion einer großen Anzahl von Photonen mit hoher Qualität bleibt eine technische Herausforderung (\cite{LinearOpticsScalable}).

\subsubsection{Aktueller Stand der Technik}
Das Feld der photonischen Quantencomputer verzeichnet rasante Fortschritte, die maßgeblich von akademischer Forschung und industriellen Akteuren vorangetrieben werden. Führende Unternehmen verfolgen dabei unterschiedliche, aber gleichermaßen ambitionierte Strategien.
Das kanadische Unternehmen Xanadu stellte Anfang 2025 seinen Prototyp „Aurora“ vor, ein modulares System, das aus 35 photonischen Chips und 13 Kilometern Glasfaser besteht. Eine strategische Partnerschaft mit dem U.S. Air Force Research Laboratory (AFRL) unterstreicht das Ziel, fehlertolerante Prozessoren zu entwickeln und die Quellen für verschränkte Photonen sowie für gequetschtes Licht zu optimieren (\cite{abdel-kareemXanaduUSAir2025}).
Das Unternehmen PsiQuantum verfolgt einen Ansatz, der auf eine Skalierung im Millionen-Qubit-Bereich abzielt und entwickelt dafür das „Omega“-Chipset. In Partnerschaft mit Regierungen plant das Unternehmen den Bau dedizierter Quantenrechenzentren in Brisbane, Australien, und Chicago, USA. Für die Fertigung der Chips nutzt PsiQuantum die Kapazitäten des etablierten Halbleiterherstellers GlobalFoundries (\cite{DARPAEyesCompanies}). Diese „Fabless“-Strategie, bei der die Produktion ausgelagert wird, steht im Kontrast zum Vorgehen von Unternehmen wie Quantum Computing Inc. (QCi), die eine eigene Fabrik für Dünnschicht-Lithiumniobat-Chips (TFLN) schaffen. (\cite{QuantumComputingInc}). Diese unterschiedlichen Geschäftsmodelle – „Fabless“ versus „integrierter Hersteller“ – spiegeln Entwicklungen wider, die bereits in der klassischen Halbleiterindustrie zu beobachten waren, und dürften langfristige Auswirkungen auf Kostenstrukturen, Innovationsgeschwindigkeit und die Verbreitung der Technologie haben. \newline
Das französische Unternehmen Quandela verfolgt einen hybriden Ansatz, bei dem photonische Qubits aus künstlichen Atomen erzeugt werden. Im Februar 2025 meldete Quandela eine Methode, die das Potenzial hat, die Anzahl der für fehlertolerante Berechnungen benötigten Komponenten um den Faktor 100.000 zu reduzieren, von etwa einer Million auf nur noch 12 Komponenten pro logischem Qubit (\cite{QuandelaAnnounces100000fold2025}). Sollte sich diese massive Reduktion als validierbar und breit anwendbar erweisen, könnte sie die Ressourcenskalierung und Wirtschaftlichkeit der photonischen Quantenberechnung dramatisch verändern. Dies könnte zu einer Veränderung innerhalb des photonischen Paradigmas selbst führen, bei der sich Ansätze mit hoher Komponentenintegration von solchen abheben, die auf einer großen Anzahl konventioneller Bauteile beruhen. Bereits im November 2024 ermöglichte Quandela der europäischen Forschungsgemeinschaft einen ersten Fernzugriff auf eine 6-Qubit-Maschine (\cite{QuandelaAnnounces100000fold2025}). \newline
Parallel zu diesen industriellen Entwicklungen unterstreichen grundlegende Forschungsdemonstrationen die wachsende Leistungsfähigkeit photonischer Systeme. Experimente zum Boson Sampling mit über 100 Photonen haben in einigen Fällen die Grenzen klassischer Simulationen überschritten (\cite{LinearOpticsScalable}). Übergreifend zeigen die Roadmaps der führenden Unternehmen eine klare Ausrichtung auf die zentralen Herausforderungen: die Realisierung von Fehlertoleranz und den Aufbau großskaliger Systeme (\cite{QuandelaAnnounces100000fold2025}).

\subsubsection{Engineering- und Skalierungslösungen}
Die Skalierung photonischer Quantencomputer erfordert erhebliche ingenieurtechnische Fortschritte. Eine Schlüsseltechnologie ist die integrierte Photonik, bei der photonische integrierte Schaltkreise (PICs) auf Plattformen wie Silizium, Siliziumnitrid (SiN) und Lithiumniobat ($LiNbO_3$) für Miniaturisierung, Stabilität und Massenfertigung sorgen (\cite{abughanemPhotonicQuantumComputers2024}). Ein Beispiel hierfür ist die Errichtung einer eigenen Fabrik für Dünnschicht-Lithiumniobat (TFLN) durch QCi (\cite{QuantumComputingInc}). Dies geht Hand in Hand mit der Entwicklung fortschrittlicher Fertigungstechniken, um Leiter mit geringen Verlusten und effiziente Koppeltechniken zu realisieren (\cite{LinearOpticsScalable}). Um die Effizienz probabilistischer Quellen zu steigern, werden zudem
Multiplexing-Strategien eingesetzt, die Photonen aus mehreren Quellen räumlich oder zeitlich bündeln (\cite{salavrakosPhotonnativeQuantumAlgorithms2025}). Schließlich ist eine schnelle
Feedforward-Kontrolle unerlässlich, bei der Detektorsignale genutzt werden, um photonische Schaltkreise adaptiv neu zu konfigurieren. Dies ist eine Grundvoraussetzung für adaptive Protokolle und das messungsbasierte Quantencomputing (MBQC) (\cite{salavrakosPhotonnativeQuantumAlgorithms2025}).

\subsubsection{Quantenfehlerkorrektur}
Aufgrund von Photonenverlust und probabilistischen Gattern ist die Quantenfehlerkorrektur (QEC) für die Realisierung fehlertoleranter photonischer Quantencomputer von fundamentaler Bedeutung (\cite{swayneChineseScientistsOvercome2025}). Die Forschung konzentriert sich auf verschiedene Arten von Fehlerkorrekturcodes. Dazu gehören bosonische Codes wie die Gottesman-Kitaev-Preskill (GKP)-Zustände, die die kontinuierliche Natur des Lichts ausnutzen, sowie an photonische Systeme angepasste topologische Codes und Oberflächencodes (\cite{LinearOpticsScalable}). 
\newline Ein wichtiger experimenteller Fortschritt wurde 2025 von Forschern der University of Science and Technology of China (USTC) in
Nature berichtet: Sie demonstrierten eine Einzelphotonenquelle mit einer Effizienz von 71,2\%, die damit die theoretische Schwelle von zwei Dritteln übertrifft, welche für die Funktionsfähigkeit von QEC in der photonischen Quantenberechnung als kritisch gilt (\cite{swayneChineseScientistsOvercome2025}). Parallel dazu zielt der Ansatz von Quandela darauf ab, die für die Konstruktion logischer Qubits benötigten physikalischen Ressourcen drastisch zu reduzieren, was die Implementierung von QEC erheblich vereinfachen könnte (\cite{QuandelaAnnounces100000fold2025}).
Der photonische Ansatz befindet sich in einer dynamischen Entwicklungsphase, mit vielversprechenden Fortschritten sowohl bei den grundlegenden Komponenten als auch bei der Systemintegration und den Strategien zur Fehlertoleranz.

\subsection{Halbleiter basierte Qubits / Spin Qubits}
Halbleiter-Spin-Qubits nutzen den Spin von Elektronen oder Atomkernen, die in Nanostrukturen aus Halbleitermaterialien eingeschlossen sind. Ihre Hauptattraktivität liegt in der potenziellen Kompatibilität mit der hoch entwickelten CMOS-Fertigungstechnologie (Complementary Metal-Oxide-Semiconductor), die eine Skalierung zu großen Qubit-Zahlen verspricht.

\subsubsection{Physische Realisierung}
Die physikalische Realisierung von Spin-Qubits basiert auf mehreren Kernprinzipien. Die Qubit-Kodierung erfolgt typischerweise durch die Spin-Ausrichtung („Spin-up“ und „Spin-down“) eines einzelnen Elektrons oder Lochs, das elektrostatisch in einem Quantenpunkt (QD) gefangen ist. Eine spezielle Variante sind die Exchange-Only (EO) Qubits, bei denen der Qubit-Zustand in einem dekohärenzfreien Unterraum von drei Elektronenspins auf drei benachbarten Quantenpunkten definiert und ausschließlich über die Austauschwechselwirkung gesteuert wird (\cite{chadwickShortTwoqubitPulse2025}). Als
Materialien dominiert Silizium, oft in Form von Si/SiGe-Heterostrukturen oder Si-MOS-Bauelementen (Metall-Oxid-Halbleiter). Silizium bietet den Vorteil einer ausgereiften Prozesstechnologie und, insbesondere bei Verwendung von isotopenreinem $28Si$ ohne Kernspin, das Potenzial für sehr lange Kohärenzzeiten. Ein entscheidender Faktor ist die
CMOS-Kompatibilität. Die strukturelle Ähnlichkeit von Quantenpunkten mit Transistoren legt nahe, dass die etablierte Infrastruktur der Halbleiterindustrie für die Massenproduktion von Spin-Qubit-Prozessoren genutzt werden könnte (\cite{stuyckCMOSCompatibilitySemiconductor2024}). Diese Kompatibilität ist jedoch nicht trivial, da Quantenanforderungen wie extrem rauscharme Umgebungen, spezifische Materialreinheit und Tieftemperaturbetrieb eine signifikante Ko-Entwicklung und Anpassung der Standard-CMOS-Prozesse erfordern. Dies könnte die Geschwindigkeit begrenzen, mit der Spin-Qubits die Skalierung klassischer Chips nachahmen können (\cite{chadwickShortTwoqubitPulse2025}).

\subsubsection{Auslesen der Spin-Qubits}
Die Kontrolle der Spin-Qubits erfolgt auf verschiedene Weisen. Einzel-Qubit-Rotationen werden typischerweise durch resonante Mikrowellenpulse erzeugt, entweder über Elektronenspinresonanz (ESR) mit einer On-Chip-Antenne oder über elektrische Dipolspinresonanz (EDSR), die elektrische Felder zur Spinmanipulation nutzt (\cite{stuyckCMOSCompatibilitySemiconductor2024}). Zwei-Qubit-Gatter werden meist durch die Austauschwechselwirkung zwischen benachbarten Spins realisiert, deren Stärke durch Anpassen der Gatespannungen und der damit verbundenen Tunnelbarriere zwischen den Quantenpunkten gesteuert wird. EO-Qubits werden, wie ihr Name andeutet, ausschließlich über die Austauschwechselwirkung gesteuert, was die Notwendigkeit präziser lokaler Magnetfelder oder Mikrowellenkontrolle eliminiert (\cite{chadwickShortTwoqubitPulse2025}).
Das Auslesen des Spinzustands erfolgt üblicherweise über einen Prozess der Spin-zu-Ladungs-Konversion. Dabei wird der Spinzustand in einen messbaren Ladungszustand überführt, beispielsweise ob ein Elektron einen Quantenpunkt besetzen kann oder nicht. Dieser Ladungszustand wird dann mit hochempfindlichen Elektrometern wie Einzel-Elektronen-Transistoren (SETs) oder benachbarten Quantenpunkten detektiert (\cite{stuyckCMOSCompatibilitySemiconductor2024}).

\subsubsection{Vorteile und spezifische Herauforderungen}
Spin-Qubits bieten eine Reihe attraktiver Vorteile. Ihre kleine Qubit-Größe ermöglicht potenziell sehr hohe Integrationsdichten auf einem Chip (\cite{stuyckCMOSCompatibilitySemiconductor2024}).  Insbesondere in isotopenreinem Silizium wurden
lange Kohärenzzeiten demonstriert, was für die Durchführung komplexer Algorithmen entscheidend ist (\cite{stuyckCMOSCompatibilitySemiconductor2024}).  Der wohl größte Vorteil ist die
Kompatibilität mit industrieller Fertigung, die das Versprechen einer kostengünstigen Skalierung auf Millionen von Qubits birgt (\cite{stuyckCMOSCompatibilitySemiconductor2024}).  Jüngste Fortschritte zeigen zudem eine
Betriebsfähigkeit bei Temperaturen über 1 Kelvin, was die anspruchsvollen kryogenen Anforderungen im Vergleich zum Millikelvin-Bereich potenziell erleichtert (\cite{stuyckCMOSCompatibilitySemiconductor2024}). 
Diesen Vorteilen stehen jedoch erhebliche Herausforderungen gegenüber. Eine Hauptquelle für Dekohärenz und Gatterfehler ist Ladungsrauschen, also Fluktuationen im elektrostatischen Umfeld, die oft durch Defekte an Grenzflächen oder in Oxidschichten verursacht werden (\cite{stuyckCMOSCompatibilitySemiconductor2024}).  Dieses Rauschen kann zudem stark zwischen benachbarten Qubits korreliert sein, was die Fehlerkorrektur erschwert. Die Materialwissenschaft und Fertigungspräzision im Nanobereich sind daher für den Erfolg von Spin-Qubits von überragender Bedeutung. Ein weiteres Problem ist die
Qubit-Variabilität: Kleinste Abweichungen in der Nanofabrikation führen zu unterschiedlichen Qubit-Eigenschaften, was das Einstellen der Betriebsparameter großer Qubit-Arrays erheblich erschwert. In dichten Arrays führen zudem Restkopplungen und unbeabsichtigte Wechselwirkungen zu Fehlern, bekannt als
Konnektivität und Crosstalk. Schließlich erfordern Spin-Qubits typischerweise einen
kryogenen Betrieb im Millikelvin-Bereich. Obwohl Fortschritte beim Betrieb über 1K erzielt wurden, bleibt die Entwicklung von Kryo-CMOS-Elektronik zur Steuerung der Qubits ein kritischer paralleler Entwicklungspfad, um den "Verdrahtungsengpass" bei der Skalierung zu überwinden (\cite{stuyckCMOSCompatibilitySemiconductor2024}). 

\subsection{Neutralatom-Quantencomputer}
\subsection{Topologische Qubits}
Die in den vorangegangenen Kapiteln beschriebenen Ansätze zur Realisierung von Quantencomputern – seien es supraleitende Schaltkreise, gefangene Ionen oder Photonen – teilen eine grundlegende Eigenschaft: Die Quanteninformation wird in lokalen physikalischen Zuständen gespeichert. Diese lokale Kodierung macht die Qubits anfällig für Dekohärenz, also den Verlust ihrer fragilen Quanteneigenschaften durch Wechselwirkungen mit der Umgebung (\cite{bolgarMicrosoftsMajorana1}). Die gängige Strategie zur Bewältigung dieses Problems ist die Quantenfehlerkorrektur (QEC), bei der die Information eines einzelnen „logischen“ Qubits redundant auf viele fehleranfällige „physische“ Qubits verteilt wird. Dieser Ansatz erfordert jedoch einen enormen Hardware-Aufwand, der Schätzungen zufolge Tausende von physischen Qubits für ein einziges robustes logisches Qubit umfassen kann (\cite{PDFMicrosoftsMajorana2025}).

Das topologische Quantencomputing verfolgt einen radikal anderen Weg. Anstatt Fehler nachträglich durch komplexe Kodierungsschemata zu korrigieren, zielt dieser Ansatz darauf ab, den Fehlerschutz direkt in die physikalische Struktur der Hardware zu integrieren (\cite{lutchynRealizingMajoranaZero2017}). Die Kernidee ist, Quanteninformation nicht lokal, sondern in den globalen, topologischen Eigenschaften eines Materials zu speichern. Solche Eigenschaften sind von Natur aus robust gegenüber lokalen Störungen, ähnlich wie die Anzahl der Löcher in einem Donut unverändert bleibt, solange man ihn nicht zerreißt.

Die physikalische Grundlage für diesen Ansatz bilden exotische Quasiteilchen, die als nicht-abelsche Anyonen bezeichnet werden. Die vielversprechendsten Kandidaten für deren Realisierung in Festkörpersystemen sind die sogenannten Majorana-Nullmoden (MZMs) (\cite{dougfinkeDeeperDiveMicrosofts2023}). Ein Paar dieser räumlich getrennten MZMs kann ein einzelnes, nicht-lokal geschütztes Qubit kodieren (\cite{lutchynRealizingMajoranaZero2017}). Dieses Kapitel konzentriert sich auf die immense Herausforderung der Hardware-Realisierung solcher Majorana-basierten Qubits und beleuchtet die jüngsten Fortschritte, insbesondere im Kontext von Microsofts „Majorana 1“-Chip, der als wichtiger Meilenstein in diesem „High-Risk, High-Reward“-Forschungsfeld gilt.

\subsubsection{Physikalische Realisierung}
Die theoretische Grundlage für die experimentelle Suche nach MZMs in eindimensionalen Systemen lieferte das Kitaev-Ketten-Modell, das die wesentlichen physikalischen Zutaten für ihre Erzeugung beschreibt (\cite{PDFMicrosoftsMajorana2025}). Da die im Modell geforderte p-Wellen-Supraleitung in der Natur kaum vorkommt, konzentriert sich die Forschung auf die künstliche Herstellung eines äquivalenten Systems durch die Kombination bekannter Materialien. Die heute kanonische Plattform dafür ist eine Heterostruktur aus einem Halbleiter-Nanodraht und einem konventionellen Supraleiter.

Die Umsetzung erfordert ein präzises Zusammenspiel von drei Komponenten. Als Basismaterial dienen Nanodrähte aus Halbleitern wie Indiumarsenid (InAs), die eine sehr starke Spin-Bahn-Kopplung aufweisen. Diese Eigenschaft koppelt den Spin eines Elektrons an seine Bewegung. Zweitens wird ein konventioneller s-Wellen-Supraleiter, typischerweise Aluminium (Al), in unmittelbarer Nähe zum Nanodraht aufgebracht, oft als dünne, epitaktisch aufgewachsene Hülle. Durch den supraleitenden Proximity-Effekt werden Cooper-Paar-Korrelationen in den Halbleiter induziert, wodurch dieser sich wie ein Supraleiter verhält. Als dritte Zutat wird ein externes Magnetfeld parallel zur Achse des Nanodrahts angelegt. Dieses Feld bewirkt eine Zeeman-Aufspaltung der Energieniveaus. Erreicht das Magnetfeld eine kritische Stärke, durchläuft das System einen topologischen Phasenübergang, und an den Enden des Nanodrahts entstehen die gepaarten Majorana-Nullmoden. (\cite{amorimMajoranaBraidingDynamics2015})

Obwohl dieses Rezept konzeptionell klar ist, stellt seine praktische Umsetzung enorme Anforderungen an die Materialwissenschaft und Nanofabrikation. Die Qualität der Grenzfläche zwischen Halbleiter und Supraleiter ist von entscheidender Bedeutung. Sie muss atomar rein und scharf sein, um einen effizienten Proximity-Effekt und eine robuste, sogenannte „harte“ induzierte supraleitende Lücke zu gewährleisten. Jegliche Unordnung oder Verunreinigungen an dieser Grenzfläche können die topologische Phase zerstören. Ebenso muss der Nanodraht selbst von höchster kristalliner Qualität sein, da Defekte die Signaturen von MZMs überdecken oder fälschlicherweise imitieren können. Diese extremen Anforderungen verdeutlichen, warum die Herstellung funktionierender Bauteile eine der größten Hürden auf dem Weg zum topologischen Quantencomputer darstellt.


\subsubsection{Fallstudie: Microsofts ``Majorana 1''-Chip}

Nach jahrelanger Forschung, die von erheblicher Skepsis in der Fachwelt begleitet wurde, präsentierte Microsoft Anfang 2025 mit dem „Majorana 1“-Prozessor einen wichtigen experimentellen Fortschritt. Die technologische Basis dieses Chips sind die beschriebenen InAs-Al-Heterostrukturen, die Microsoft als „Topoconductor“ bezeichnet – ein Begriff, der ein ganzes materialwissenschaftliches Programm zur Optimierung der topologischen Supraleitung beschreibt. Die Architektur des Chips hat sich von einfachen linearen Nanodrähten zu komplexeren, skalierbaren Designs weiterentwickelt. Ein zentrales Element ist das „Tetron“-Qubit, das vier MZMs zur Kodierung eines logischen Qubits nutzt, die auf H-förmigen Nanodrahtstrukturen realisiert werden (\cite{bolgarMicrosoftsMajorana1}).

Die zentrale Errungenschaft, die in der Fachzeitschrift Nature veröffentlicht wurde, war jedoch nicht die erneute Beobachtung einer mehrdeutigen Signatur, sondern die Demonstration einer neuen Messtechnik: die zuverlässige Einzelschuss-Messung der Fermionenparität (\cite{PDFMicrosoftsMajorana2025}). Die Fermionenparität gibt an, ob sich eine gerade oder ungerade Anzahl von Elektronen in einem System befindet und ist die Eigenschaft, in der die Qubit-Information ($\ket{0}$ oder $\ket{1}$) kodiert ist.
Für diese Messung wird der topologische Nanodraht an einen Quantenpunkt gekoppelt, der als extrem empfindliches Elektrometer fungiert. Die Fähigkeit des Quantenpunkts, Ladung zu speichern, ändert sich messbar in Abhängigkeit von der Parität des MZM-Paares. Diese winzige Kapazitätsänderung wird dann mit hoher Präzision durch Mikrowellen in einer einzigen Messung detektiert, mit einer beeindruckend niedrigen Fehlerrate von nur 1\%. Diese Fähigkeit, eine fundamentale nicht-lokale Quanteneigenschaft direkt und zuverlässig zu messen, ist ein entscheidender Schritt, da sie die Grundlage für die Durchführung von Quantenoperationen bildet. (\cite{PDFMicrosoftsMajorana2025})


\subsubsection{Quantenoperationen in topologischen Quantencomputern}

Die ursprüngliche Vision des topologischen Quantencomputings basierte auf dem physischen „Flechten“ (Braiding) von MZMs im Raum, um Gatteroperationen durchzuführen. Dies würde komplexe Nanodraht-Netzwerke mit T-Verzweigungen erfordern, deren Herstellung bei der geforderten Materialqualität eine gewaltige Herausforderung darstellt (\cite{amorimMajoranaBraidingDynamics2015}). Angesichts dieser Hürden hat sich das Feld zunehmend einem pragmatischeren Ansatz zugewandt: dem messungsbasierten Braiding (\cite{amorimMajoranaBraidingDynamics2015}). Die zentrale Idee hierbei ist, dass eine Sequenz von Paritätsmessungen an Gruppen von stationären MZMs mathematisch äquivalent zu einer physischen Flechtoperation sein kann. Dieser Ansatz eliminiert die Notwendigkeit für komplexe T-Verzweigungen und erlaubt den Aufbau von Prozessoren aus einfachen, kachelartigen Architekturen wie dem Tetron-Design, was die Skalierbarkeit erheblich erleichtert (\cite{bolgarMicrosoftsMajorana1}). Die von Microsoft demonstrierte Paritätsmessung ist somit ein fundamentaler Baustein für diese Strategie.

Trotz dieses Fortschritts bleibt eine zentrale wissenschaftliche Debatte bestehen: die eindeutige Unterscheidung von echten topologischen MZMs und trivialen, nicht-topologischen Zuständen, die ähnliche Signaturen erzeugen können. Insbesondere sogenannte Andreev-gebundene Zustände (ABS) können unter bestimmten Bedingungen ebenfalls zu Null-Energie-Zuständen an den Enden eines Drahtes führen und in Experimenten Signaturen erzeugen, die von denen echter MZMs kaum zu unterscheiden sind. Diese Möglichkeit, dass triviale Zustände die topologischen Signaturen „imitieren“, ist das wissenschaftliche Kernproblem des Feldes. Die Entwicklung von robusteren Messprotokollen wie der Paritätsmessung ist eine direkte Antwort auf diese Herausforderung. Dennoch wird auch in den jüngsten Veröffentlichungen eingeräumt, dass selbst diese fortgeschrittenen Techniken unter bestimmten, fein abgestimmten Bedingungen von nicht-topologischen Systemen nachgeahmt werden könnten, was die Notwendigkeit weiterer, unzweideutiger Experimente unterstreicht (\cite{amorimMajoranaBraidingDynamics2015}).

\subsubsection{Fehlerquellen und Ausblick}

Neben der fundamentalen Herausforderung der Verifikation ist das sogenannte Quasiteilchen-Poisoning (QP) die dominante dynamische Fehlerquelle für Majorana-Qubits (\cite{svetogorovQuasiparticlePoisoningTrivial2021}). Dieser Prozess beschreibt, wie ein unkontrolliertes Quasiteilchen – typischerweise ein Elektron aus einem aufgebrochenen Cooper-Paar – vom Qubit-System absorbiert wird. Ein solches Ereignis schaltet die Fermionenparität des Qubits zufällig um und führt damit zu einem logischen Bit-Flip-Fehler. Die Rate dieser Ereignisse, die durch thermische Anregungen oder Hintergrundstrahlung verursacht werden können, begrenzt direkt die Kohärenzzeit des topologischen Qubits. Die Reduzierung dieser Raten durch bessere Abschirmung und Materialdesigns ist daher ein zentrales Forschungsfeld (\cite{svetogorovQuasiparticlePoisoningTrivial2021}).

Zusammenfassend lässt sich sagen, dass die Hardware-Entwicklung für topologisches Quantencomputing ein Niveau erreicht hat, auf dem die Demonstration entscheidender Bausteine wie der zuverlässigen Paritätsmessung auf dem „Majorana 1“-Chip möglich ist. Der endgültige, unumstößliche Nachweis der nicht-abelschen Flechtstatistik von MZMs steht jedoch noch aus und bleibt der „heilige Gral“ des Feldes.

Die nächsten Schritte auf dem Weg zu einem funktionierenden topologischen Quantencomputer sind klar definiert: der Übergang von einzelnen Bauteilen zu Multi-Qubit-Systemen, die Demonstration von messungsbasiertem Braiding zwischen ihnen und schließlich der Bau eines ersten logischen Qubits.Der Weg dorthin bleibt fundamental und lang, doch ein Durchbruch hätte das Potenzial, die Landschaft des Quantencomputings durch die Realisierung einer inhärent fehlertoleranten Hardware grundlegend zu verändern.
