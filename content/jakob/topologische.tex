\subsection{Topologische Qubits}
Die in den vorangegangenen Kapiteln beschriebenen Ansätze zur Realisierung von Quantencomputern – seien es supraleitende Schaltkreise, gefangene Ionen oder Photonen – teilen eine grundlegende Eigenschaft: Die Quanteninformation wird in lokalen physikalischen Zuständen gespeichert. Diese lokale Kodierung macht die Qubits anfällig für Dekohärenz, also den Verlust ihrer fragilen Quanteneigenschaften durch Wechselwirkungen mit der Umgebung \cite{bolgarMicrosoftsMajorana1}. Die gängige Strategie zur Bewältigung dieses Problems ist die Quantenfehlerkorrektur (QEC), bei der die Information eines einzelnen „logischen“ Qubits redundant auf viele fehleranfällige „physische“ Qubits verteilt wird. Dieser Ansatz erfordert jedoch einen enormen Hardware-Aufwand, der Schätzungen zufolge Tausende von physischen Qubits für ein einziges robustes logisches Qubit umfassen kann \cite{PDFMicrosoftsMajorana2025}.

Das topologische Quantencomputing verfolgt einen radikal anderen Weg. Anstatt Fehler nachträglich durch komplexe Kodierungsschemata zu korrigieren, zielt dieser Ansatz darauf ab, den Fehlerschutz direkt in die physikalische Struktur der Hardware zu integrieren \cite{lutchynRealizingMajoranaZero2017}. Die Kernidee ist, Quanteninformation nicht lokal, sondern in den globalen, topologischen Eigenschaften eines Materials zu speichern. Solche Eigenschaften sind von Natur aus robust gegenüber lokalen Störungen, ähnlich wie die Anzahl der Löcher in einem Donut unverändert bleibt, solange man ihn nicht zerreißt.

Die physikalische Grundlage für diesen Ansatz bilden exotische Quasiteilchen, die als nicht-abelsche Anyonen bezeichnet werden. Die vielversprechendsten Kandidaten für deren Realisierung in Festkörpersystemen sind die sogenannten Majorana-Nullmoden (MZMs) \cite{dougfinkeDeeperDiveMicrosofts2023}. Ein Paar dieser räumlich getrennten MZMs kann ein einzelnes, nicht-lokal geschütztes Qubit kodieren \cite{lutchynRealizingMajoranaZero2017}. Dieses Kapitel konzentriert sich auf die immense Herausforderung der Hardware-Realisierung solcher Majorana-basierten Qubits und beleuchtet die jüngsten Fortschritte, insbesondere im Kontext von Microsofts „Majorana 1“-Chip, der als wichtiger Meilenstein in diesem „High-Risk, High-Reward“-Forschungsfeld gilt.

\subsubsection{Physikalische Realisierung}
Die theoretische Grundlage für die experimentelle Suche nach MZMs in eindimensionalen Systemen lieferte das Kitaev-Ketten-Modell, das die wesentlichen physikalischen Zutaten für ihre Erzeugung beschreibt \cite{PDFMicrosoftsMajorana2025}. Da die im Modell geforderte p-Wellen-Supraleitung in der Natur kaum vorkommt, konzentriert sich die Forschung auf die künstliche Herstellung eines äquivalenten Systems durch die Kombination bekannter Materialien. Die heute kanonische Plattform dafür ist eine Heterostruktur aus einem Halbleiter-Nanodraht und einem konventionellen Supraleiter.

Die Umsetzung erfordert ein präzises Zusammenspiel von drei Komponenten. Als Basismaterial dienen Nanodrähte aus Halbleitern wie Indiumarsenid (InAs), die eine sehr starke Spin-Bahn-Kopplung aufweisen. Diese Eigenschaft koppelt den Spin eines Elektrons an seine Bewegung. Zweitens wird ein konventioneller s-Wellen-Supraleiter, typischerweise Aluminium (Al), in unmittelbarer Nähe zum Nanodraht aufgebracht, oft als dünne, epitaktisch aufgewachsene Hülle. Durch den supraleitenden Proximity-Effekt werden Cooper-Paar-Korrelationen in den Halbleiter induziert, wodurch dieser sich wie ein Supraleiter verhält. Als dritte Zutat wird ein externes Magnetfeld parallel zur Achse des Nanodrahts angelegt. Dieses Feld bewirkt eine Zeeman-Aufspaltung der Energieniveaus. Erreicht das Magnetfeld eine kritische Stärke, durchläuft das System einen topologischen Phasenübergang, und an den Enden des Nanodrahts entstehen die gepaarten Majorana-Nullmoden. \cite{amorimMajoranaBraidingDynamics2015}

Obwohl dieses Rezept konzeptionell klar ist, stellt seine praktische Umsetzung enorme Anforderungen an die Materialwissenschaft und Nanofabrikation. Die Qualität der Grenzfläche zwischen Halbleiter und Supraleiter ist von entscheidender Bedeutung. Sie muss atomar rein und scharf sein, um einen effizienten Proximity-Effekt und eine robuste, sogenannte „harte“ induzierte supraleitende Lücke zu gewährleisten. Jegliche Unordnung oder Verunreinigungen an dieser Grenzfläche können die topologische Phase zerstören. Ebenso muss der Nanodraht selbst von höchster kristalliner Qualität sein, da Defekte die Signaturen von MZMs überdecken oder fälschlicherweise imitieren können. Diese extremen Anforderungen verdeutlichen, warum die Herstellung funktionierender Bauteile eine der größten Hürden auf dem Weg zum topologischen Quantencomputer darstellt.


\subsubsection{Fallstudie: Microsofts ``Majorana 1''-Chip}

Nach jahrelanger Forschung, die von erheblicher Skepsis in der Fachwelt begleitet wurde, präsentierte Microsoft Anfang 2025 mit dem „Majorana 1“-Prozessor einen wichtigen experimentellen Fortschritt. Die technologische Basis dieses Chips sind die beschriebenen InAs-Al-Heterostrukturen, die Microsoft als „Topoconductor“ bezeichnet – ein Begriff, der ein ganzes materialwissenschaftliches Programm zur Optimierung der topologischen Supraleitung beschreibt. Die Architektur des Chips hat sich von einfachen linearen Nanodrähten zu komplexeren, skalierbaren Designs weiterentwickelt. Ein zentrales Element ist das „Tetron“-Qubit, das vier MZMs zur Kodierung eines logischen Qubits nutzt, die auf H-förmigen Nanodrahtstrukturen realisiert werden \cite{bolgarMicrosoftsMajorana1}.

Die zentrale Errungenschaft, die in der Fachzeitschrift Nature veröffentlicht wurde, war jedoch nicht die erneute Beobachtung einer mehrdeutigen Signatur, sondern die Demonstration einer neuen Messtechnik: die zuverlässige Einzelschuss-Messung der Fermionenparität \cite{PDFMicrosoftsMajorana2025}. Die Fermionenparität gibt an, ob sich eine gerade oder ungerade Anzahl von Elektronen in einem System befindet und ist die Eigenschaft, in der die Qubit-Information ($\ket{0}$ oder $\ket{1}$) kodiert ist.
Für diese Messung wird der topologische Nanodraht an einen Quantenpunkt gekoppelt, der als extrem empfindliches Elektrometer fungiert. Die Fähigkeit des Quantenpunkts, Ladung zu speichern, ändert sich messbar in Abhängigkeit von der Parität des MZM-Paares. Diese winzige Kapazitätsänderung wird dann mit hoher Präzision durch Mikrowellen in einer einzigen Messung detektiert, mit einer beeindruckend niedrigen Fehlerrate von nur 1\%. Diese Fähigkeit, eine fundamentale nicht-lokale Quanteneigenschaft direkt und zuverlässig zu messen, ist ein entscheidender Schritt, da sie die Grundlage für die Durchführung von Quantenoperationen bildet. \cite{PDFMicrosoftsMajorana2025}


\subsubsection{Quantenoperationen in topologischen Quantencomputern}

Die ursprüngliche Vision des topologischen Quantencomputings basierte auf dem physischen „Flechten“ (Braiding) von MZMs im Raum, um Gatteroperationen durchzuführen. Dies würde komplexe Nanodraht-Netzwerke mit T-Verzweigungen erfordern, deren Herstellung bei der geforderten Materialqualität eine gewaltige Herausforderung darstellt.6 Angesichts dieser Hürden hat sich das Feld zunehmend einem pragmatischeren Ansatz zugewandt: dem messungsbasierten Braiding \cite{amorimMajoranaBraidingDynamics2015}. Die zentrale Idee hierbei ist, dass eine Sequenz von Paritätsmessungen an Gruppen von stationären MZMs mathematisch äquivalent zu einer physischen Flechtoperation sein kann. Dieser Ansatz eliminiert die Notwendigkeit für komplexe T-Verzweigungen und erlaubt den Aufbau von Prozessoren aus einfachen, kachelartigen Architekturen wie dem Tetron-Design, was die Skalierbarkeit erheblich erleichtert \cite{bolgarMicrosoftsMajorana1}. Die von Microsoft demonstrierte Paritätsmessung ist somit ein fundamentaler Baustein für diese Strategie.

Trotz dieses Fortschritts bleibt eine zentrale wissenschaftliche Debatte bestehen: die eindeutige Unterscheidung von echten topologischen MZMs und trivialen, nicht-topologischen Zuständen, die ähnliche Signaturen erzeugen können. Insbesondere sogenannte Andreev-gebundene Zustände (ABS) können unter bestimmten Bedingungen ebenfalls zu Null-Energie-Zuständen an den Enden eines Drahtes führen und in Experimenten Signaturen erzeugen, die von denen echter MZMs kaum zu unterscheiden sind. Diese Möglichkeit, dass triviale Zustände die topologischen Signaturen „imitieren“, ist das wissenschaftliche Kernproblem des Feldes. Die Entwicklung von robusteren Messprotokollen wie der Paritätsmessung ist eine direkte Antwort auf diese Herausforderung. Dennoch wird auch in den jüngsten Veröffentlichungen eingeräumt, dass selbst diese fortgeschrittenen Techniken unter bestimmten, fein abgestimmten Bedingungen von nicht-topologischen Systemen nachgeahmt werden könnten, was die Notwendigkeit weiterer, unzweideutiger Experimente unterstreicht \cite{amorimMajoranaBraidingDynamics2015}.

\subsubsection{Fehlerquellen und Ausblick}

Neben der fundamentalen Herausforderung der Verifikation ist das sogenannte Quasiteilchen-Poisoning (QP) die dominante dynamische Fehlerquelle für Majorana-Qubits \cite{svetogorovQuasiparticlePoisoningTrivial2021}. Dieser Prozess beschreibt, wie ein unkontrolliertes Quasiteilchen – typischerweise ein Elektron aus einem aufgebrochenen Cooper-Paar – vom Qubit-System absorbiert wird. Ein solches Ereignis schaltet die Fermionenparität des Qubits zufällig um und führt damit zu einem logischen Bit-Flip-Fehler. Die Rate dieser Ereignisse, die durch thermische Anregungen oder Hintergrundstrahlung verursacht werden können, begrenzt direkt die Kohärenzzeit des topologischen Qubits. Die Reduzierung dieser Raten durch bessere Abschirmung und Materialdesigns ist daher ein zentrales Forschungsfeld \cite{svetogorovQuasiparticlePoisoningTrivial2021}.

Zusammenfassend lässt sich sagen, dass die Hardware-Entwicklung für topologisches Quantencomputing ein Niveau erreicht hat, auf dem die Demonstration entscheidender Bausteine wie der zuverlässigen Paritätsmessung auf dem „Majorana 1“-Chip möglich ist. Der endgültige, unumstößliche Nachweis der nicht-abelschen Flechtstatistik von MZMs steht jedoch noch aus und bleibt der „heilige Gral“ des Feldes.

Die nächsten Schritte auf dem Weg zu einem funktionierenden topologischen Quantencomputer sind klar definiert: der Übergang von einzelnen Bauteilen zu Multi-Qubit-Systemen, die Demonstration von messungsbasiertem Braiding zwischen ihnen und schließlich der Bau eines ersten logischen Qubits.Der Weg dorthin bleibt fundamental und lang, doch ein Durchbruch hätte das Potenzial, die Landschaft des Quantencomputings durch die Realisierung einer inhärent fehlertoleranten Hardware grundlegend zu verändern.
