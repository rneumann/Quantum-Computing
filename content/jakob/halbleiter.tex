\subsection{Halbleiter basierte Qubits / Spin Qubits}
Halbleiter-Spin-Qubits nutzen den Spin von Elektronen oder Atomkernen, die in Nanostrukturen aus Halbleitermaterialien eingeschlossen sind. Ihre Hauptattraktivität liegt in der potenziellen Kompatibilität mit der hochentwickelten CMOS-Fertigungstechnologie (Complementary Metal-Oxide-Semiconductor), die eine Skalierung zu großen Qubit-Zahlen verspricht.

\subsubsection{Physische Realisierung}
Die physikalische Realisierung von Spin-Qubits basiert auf mehreren Kernprinzipien. Die Qubit-Kodierung erfolgt typischerweise durch die Spin-Ausrichtung ("Spin-up" und "Spin-down") eines einzelnen Elektrons oder Lochs, das elektrostatisch in einem Quantenpunkt (QD) gefangen ist. Eine spezielle Variante sind die Exchange-Only (EO) Qubits, bei denen der Qubit-Zustand in einem dekohärenzfreien Unterraum von drei Elektronenspins auf drei benachbarten Quantenpunkten definiert und ausschließlich über die Austauschwechselwirkung gesteuert wird \cite{chadwickShortTwoqubitPulse2025}. Als
Materialien dominiert Silizium, oft in Form von Si/SiGe-Heterostrukturen oder Si-MOS-Bauelementen (Metall-Oxid-Halbleiter). Silizium bietet den Vorteil einer ausgereiften Prozesstechnologie und, insbesondere bei Verwendung von isotopenreinem $28Si$ ohne Kernspin, das Potenzial für sehr lange Kohärenzzeiten. Ein entscheidender Faktor ist die
CMOS-Kompatibilität. Die strukturelle Ähnlichkeit von Quantenpunkten mit Transistoren legt nahe, dass die etablierte Infrastruktur der Halbleiterindustrie für die Massenproduktion von Spin-Qubit-Prozessoren genutzt werden könnte \cite{stuyckCMOSCompatibilitySemiconductor2024}. Diese Kompatibilität ist jedoch nicht trivial, da Quantenanforderungen wie extrem rauscharme Umgebungen, spezifische Materialreinheit und Tieftemperaturbetrieb eine signifikante Ko-Entwicklung und Anpassung der Standard-CMOS-Prozesse erfordern. Dies könnte die Geschwindigkeit begrenzen, mit der Spin-Qubits die Skalierung klassischer Chips nachahmen können \cite{chadwickShortTwoqubitPulse2025}.

\subsubsection{Auslesen der Spin-Qubits}
Die Kontrolle der Spin-Qubits erfolgt auf verschiedene Weisen. Einzel-Qubit-Rotationen werden typischerweise durch resonante Mikrowellenpulse erzeugt, entweder über Elektronenspinresonanz (ESR) mit einer On-Chip-Antenne oder über elektrische Dipolspinresonanz (EDSR), die elektrische Felder zur Spinmanipulation nutzt \cite{stuyckCMOSCompatibilitySemiconductor2024}. Zwei-Qubit-Gatter werden meist durch die Austauschwechselwirkung zwischen benachbarten Spins realisiert, deren Stärke durch Anpassen der Gatespannungen und der damit verbundenen Tunnelbarriere zwischen den Quantenpunkten gesteuert wird. EO-Qubits werden, wie ihr Name andeutet, ausschließlich über die Austauschwechselwirkung gesteuert, was die Notwendigkeit präziser lokaler Magnetfelder oder Mikrowellenkontrolle eliminiert \cite{chadwickShortTwoqubitPulse2025}.
Das Auslesen des Spinzustands erfolgt üblicherweise über einen Prozess der Spin-zu-Ladungs-Konversion. Dabei wird der Spinzustand in einen messbaren Ladungszustand überführt, beispielsweise ob ein Elektron einen Quantenpunkt besetzen kann oder nicht. Dieser Ladungszustand wird dann mit hochempfindlichen Elektrometern wie Einzel-Elektronen-Transistoren (SETs) oder benachbarten Quantenpunkten detektiert \cite{stuyckCMOSCompatibilitySemiconductor2024}.

\subsubsection{Vorteile und spezifische Herauforderungen}
Spin-Qubits bieten eine Reihe attraktiver Vorteile. Ihre kleine Qubit-Größe ermöglicht potenziell sehr hohe Integrationsdichten auf einem Chip\cite{stuyckCMOSCompatibilitySemiconductor2024}.  Insbesondere in isotopenreinem Silizium wurden
lange Kohärenzzeiten demonstriert, was für die Durchführung komplexer Algorithmen entscheidend ist\cite{stuyckCMOSCompatibilitySemiconductor2024}.  Der wohl größte Vorteil ist die
Kompatibilität mit industrieller Fertigung, die das Versprechen einer kostengünstigen Skalierung auf Millionen von Qubits birgt\cite{stuyckCMOSCompatibilitySemiconductor2024}.  Jüngste Fortschritte zeigen zudem eine
Betriebsfähigkeit bei Temperaturen über 1 Kelvin, was die anspruchsvollen kryogenen Anforderungen im Vergleich zum Millikelvin-Bereich potenziell erleichtert\cite{stuyckCMOSCompatibilitySemiconductor2024}. 
Diesen Vorteilen stehen jedoch erhebliche Herausforderungen gegenüber. Eine Hauptquelle für Dekohärenz und Gatterfehler ist Ladungsrauschen, also Fluktuationen im elektrostatischen Umfeld, die oft durch Defekte an Grenzflächen oder in Oxidschichten verursacht werden\cite{stuyckCMOSCompatibilitySemiconductor2024}.  Dieses Rauschen kann zudem stark zwischen benachbarten Qubits korreliert sein, was die Fehlerkorrektur erschwert. Die Materialwissenschaft und Fertigungspräzision im Nanobereich sind daher für den Erfolg von Spin-Qubits von überragender Bedeutung. Ein weiteres Problem ist die
Qubit-Variabilität: Kleinste Abweichungen in der Nanofabrikation führen zu unterschiedlichen Qubit-Eigenschaften, was das Einstellen der Betriebsparameter großer Qubit-Arrays erheblich erschwert. In dichten Arrays führen zudem Restkopplungen und unbeabsichtigte Wechselwirkungen zu Fehlern, bekannt als
Konnektivität und Crosstalk. Schließlich erfordern Spin-Qubits typischerweise einen
kryogenen Betrieb im Millikelvin-Bereich. Obwohl Fortschritte beim Betrieb über 1K erzielt wurden, bleibt die Entwicklung von Kryo-CMOS-Elektronik zur Steuerung der Qubits ein kritischer paralleler Entwicklungspfad, um den "Verdrahtungsengpass" bei der Skalierung zu überwinden\cite{stuyckCMOSCompatibilitySemiconductor2024}. 
