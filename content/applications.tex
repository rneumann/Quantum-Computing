%\motto{Use the template \emph{chapter.tex} to style the various elements of your chapter content.}
\chapter{Anwendungsgebiete}
\label{applications} % Always give a unique label
% use \chaptermark{}
% to alter or adjust the chapter heading in the running head

\abstract*{some abstract}

\abstract{some abstract}


%
%  RN: Ich habe nach der Diskussion vom Mittwoch das Kapitel ausgeblendet und dafür die
%  Teilkapitel appl-XXX angelegt
%  Solltet ihr die Reihenfolge ändern wollen oder in zusätzliches Kapitel, dann sagt mir
%  kurz Bescheid, oder ändert das direkt selbst.
%


\section{Einleitung}
\begin{itemize}
    \item Ziel des Kapitels: Überblick über aktuelle und zukünftige Anwendungen von Quantencomputing
    \item Warum: Anwendungen sind der Schlüssel zur gesellschaftlichen und wirtschaftlichen Relevanz
    \item Überblick: Welche Anwendungsbereiche werden betrachtet?
\end{itemize}


\section{Bewertungsrahmen \& Methodik}
\textbf{Vorstellung eines einheitlichen Kriterienkatalogs für die spätere Bewertung:}
\begin{itemize}
    \item Technologischer Reifegrad
    \item Marktrelevanz / Wirtschaftliche Nutzbarkeit
    \item Gesellschaftlicher Nutzen
    \item Forschungspotenzial / Innovationsgrad
    \item Risiken und ethische Implikationen
\end{itemize}

\textbf{Abgrenzung Theorie \& Praxis:}
\begin{itemize}
    \item \textbf{Theorie:} Visionäre Szenarien, Forschungskonzepte, wissenschaftliche Paper
    \item \textbf{Praxis:} Marktnahe Technologien, Unternehmen, Anwendungen in der Pilotierung oder Anwendung
\end{itemize}


\section{Analyse ausgewählter Anwendungsbereiche}

% ----------------------------------------------------------------
% >>>>> Abschnitt: Chemie & Materialwissenschaften
% ----------------------------------------------------------------
\subsection{Chemie \& Materialwissenschaften}

\subsubsection*{Relevanz \& Problemstellung}
\subsubsection*{Top 3 Anwendungsfelder (Praxis \& Theorie)}
\begin{itemize}
    \item Simulation von Molekülen und Reaktionen
    \item Materialentwicklung für Batterien \& Quantenmaterialien
    \item Wirkstofferforschung \& Drug Discovery
\end{itemize}
\subsubsection*{Top Technologien \& Algorithmen}
\begin{itemize}
    \item Variational Quantum Eigensolver (VQE)
    \item Quantum Phase Estimation (QPE)
    \item Quantum Simulation (not sure yet)
\end{itemize}

\subsubsection*{Wichtige Unternehmen \& Akteure}
\begin{itemize}
    \item IBM: Entwicklung von Qiskit Chemistry, Simulationsframework für Moleküle.
    \item Google Quantum AI: Erfolgreiche Simulation von H\₂ und Forschung zu Quantenchemie.
    \item QC Ware: Kooperationen mit Chemieunternehmen (z.B. Roche) zur Molekülsimulation.
\end{itemize}
\subsubsection*{Top 3 Zukunftsprojekte \& Forschungsinitiativen}
\subsubsection*{Bewertung anhand der Kriterien}
\subsubsection*{Teilfazit}


\subsection{Finanzen}

% ----------------------------------------------------------------
% >>>>> Abschnitt: Quanten-KI
% ----------------------------------------------------------------
\subsection{Quanten-KI}

% ----------------------------------------------------------------
% >>>>> Abschnitt: Kryptographie & Sicherheit
% ----------------------------------------------------------------
\subsection{Kryptographie \& Sicherheit}

% ----------------------------------------------------------------
% >>>>> Abschnitt: Medizin & Pharmazie
% ----------------------------------------------------------------
\subsection{Medizin \& Pharmazie}

\subsubsection*{Relevanz \& Problemstellung}
\begin{itemize}
    \item Datenexplosion in Medizin: Genomik, Bildgebung, EHRs, Wearables
    \item Komplexität biologischer Systeme (z.\,B. Proteinstruktur, Geninteraktionen)
    \item Grenzen klassischer Computer: lineare Skalierung, geringe Effizienz bei Simulation
    \item Quantencomputing als Lösung:
    \begin{itemize}
        \item Superposition und Verschränkung
        \item Parallele Datenverarbeitung, exponentielle Beschleunigung
    \end{itemize}
    \item Technologische Hürden: NISQ-Geräte, Fehlerkorrektur, Hardwareinstabilität
    \item Regulatorische und ethische Fragen: Datenschutz, Informed Consent, Normmangel
\end{itemize}

\subsubsection*{Top 3 Anwendungsfelder (Praxis \& Theorie)}
\begin{itemize}
    \item Bildgebungsdiagnostik:
    \begin{itemize}
        \item QML-Modelle für Klassifikation (QSVM, QCNN)
        \item Anwendung auf MRT, CT, Röntgen
    \end{itemize}
    \item Arzneimittelforschung:
    \begin{itemize}
        \item Molekülsimulation mit VQE
        \item Struktur- und Wirkstoffdesign, generative Chemie
    \end{itemize}
    \item Personalisierte Medizin:
    \begin{itemize}
        \item QNNs für Therapieprognose
        \item Clustering genetischer Daten, patientenspezifische Behandlung
    \end{itemize}
\end{itemize}

\subsubsection*{Top Technologien \& Algorithmen}
\begin{itemize}
    \item QSVM: Klassifikation medizinischer Bilddaten
    \item QCNN: Feature-Extraktion bei kleinen Bilddatensätzen
    \item QNN/QBM: Vorhersagemodelle für personalisierte Therapie
    % und weitere wahrscheinlich
\end{itemize}

\subsubsection*{Wichtige Unternehmen \& Akteure}
% Inhalte hier ergänzen
...

\subsubsection*{Top 3 Zukunftsprojekte \& Forschungsinitiativen}
% Inhalte hier ergänzen
...

\subsubsection*{Bewertung anhand der Kriterien}
% Inhalte hier ergänzen
...

\subsubsection*{Teilfazit}
% Inhalte hier ergänzen
...


\section{Vergleich der Anwendungsbereiche}
Synoptische Gegenüberstellung aller Bereiche anhand der Bewertungskriterien\\

\subsubsection*{Darstellung:}
Vergleichstabelle / Matrix / Balken- oder Spider-Diagramm \\

\subsubsection*{Diskussion:}
\begin{itemize}
    \item Wer ist Vorreiter? Wo ist der Reifegrad am höchsten?
    \item Wo liegt das größte langfristige Potenzial?
    \item Wie stehen Risiko \& Nutzen im Verhältnis?
\end{itemize}


\section{Abschlussfazit}
\begin{itemize}
    \item Zusammenfassung der übergreifenden Erkenntnisse
    \item Reflexion: Was sagen die Anwendungen über den Status quo des Quantencomputings aus?
    \item Ausblick: Welche Anwendungen könnten zum „Gamechanger“ werden?
\end{itemize}


\printbibliography
